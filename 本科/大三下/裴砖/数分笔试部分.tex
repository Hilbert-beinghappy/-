\documentclass[lang=cn,10pt,thmcnt=section]{elegantbook}
\usepackage{graphicx}
\usepackage{float}
\usepackage{esint}
\usepackage{mathtools}
\usepackage{tikz}
\title{数学分析}



\author{Huang}
\date{\today}




\setcounter{tocdepth}{3}


\cover{cover.jpg}

% 本文档命令
\usepackage{array}
\newcommand{\ccr}[1]{\makecell{{\color{#1}\rule{1cm}{1cm}}}}

% 修改标题页的橙色带
% \definecolor{customcolor}{RGB}{32,178,170}
% \colorlet{coverlinecolor}{customcolor}

\begin{document}
	
	\maketitle
	\frontmatter
	
	\tableofcontents
	
	\mainmatter
	\chapter{数列}
	\section{极限的计算}
	\subsection{stolz定理}
	\begin{theorem}[stolz]
		\begin{itemize}
			\item $(\frac{0}{0})$型,$\{ a_n\} ,\{ b_n\}$是无穷小量,$\{ a_n\}$单调递减,$\lim_{n \to \infty}  \frac{b_{n+1}-b_n}{a_{n+1}-a_n}=l$,	则$\lim_{n \to \infty}  \frac{b_n}{a_n}=l$
			\item $(\frac{*}{\infty})$型,$\{ a_n\} $是严格单调递增无穷大量,$\lim_{n \to \infty}  \frac{b_{n+1}-b_n}{a_{n+1}-a_n}=l$,则$\lim_{n \to \infty}  
			\frac{b_n}{a_n}=l$
		\end{itemize}
	\end{theorem}

	\begin{example}
		设 $a_1 \in (0,1), a_{n+1} = \sin a_n, n = 1,2,\cdots$,试计算  
\[ \lim_{n \to \infty} \sqrt[n]{n a_n}. \]
	\end{example}
	\begin{example}
		设 $a_n > 0$ 且  
\[ \lim_{n \to \infty} \frac{a_{n+1}}{a_n} \]  
存在或者为确定符号的 $\infty$。

(1) 求证:  
\[ \lim_{n \to \infty} \sqrt[n]{a_n} = \lim_{n \to \infty} \frac{a_{n+1}}{a_n}. \]

(2) 进一步,若  
\[ \lim_{n \to \infty} a_n = a, \]  
计算  
\[ \lim_{n \to \infty} \left( \frac{\sqrt[n]{a_1} + \sqrt[n]{a_2} + \cdots + \sqrt[n]{a_n}}{n} \right)^n. \]  
(2023中科院夏令营)
	\end{example}
\begin{example}
	\begin{itemize}
		\item 设  
		\[ S_n = \sum_{k=0}^n \frac{\ln C_n^k}{n^2}, \]  
		求  
		\[ \lim_{n \to \infty} S_n. \]
		
		\item 计算  
		\[ \lim_{n \to \infty} \frac{\ln n}{\sum_{k=1}^n \frac{1}{k} - \ln n}. \]  
		(第十二届全国大学生数学竞赛)
	\end{itemize}
\end{example}

\subsection{abel变换}
\begin{theorem}
	\[
\sum_{k=1}^n a_k b_k = \sum_{k=1}^{n-1} (a_k - a_{k+1}) B_k + a_n B_n, \quad \text{其中 } B_k = \sum_{i=1}^k b_i.
\]
\end{theorem}
\begin{example}
	设 $\lim_{n \to \infty} \sum_{k=1}^n a_k$ 存在,试计算 $\lim_{n \to \infty} \frac{1}{n} \sum_{k=1}^n k a_k$.
\end{example}
\begin{example}
	(2023 中科大考研压轴) 设有实数列 $\{a_n\}$, 令 $S_n = \sum\limits_{k=1}^n a_k$, $\sigma_n = \frac{1}{n}\sum\limits_{k=1}^n S_k$。

(1) 证明:若 $\{S_n\}$ 有极限 $S$, 则 $\lim\limits_{n \to \infty} \sigma_n = S$。

(2) 若 $\{\sigma_n\}$ 收敛, 且 $a_n = o\left(\frac{1}{n}\right)$, 则 $\{S_n\}$ 收敛。

\end{example}
\begin{example}
	(2023 中科院提前批) 设 $\lim\limits_{n \to \infty} a_n = a$, $\lim\limits_{n \to \infty} b_n = b$, 试求:

\[
\lim_{n \to \infty} \frac{a_1 b_n + a_2 b_{n-1} + \cdots + a_n b_1}{n}
\]


\end{example}
\subsection{拟合法}
拟合法主要是一种思想,在于抓住问题的关键部分
\begin{example}
	(2023 北师大夏令营) 设 $f \in C[0,1]$,求证:
\[
\lim_{h \to 0^+} \int_0^1 \frac{h}{h^2 + x^2} f(x) dx = \frac{\pi}{2} f(0).
\]

\end{example}
\begin{example}
	(2022 浙大直博压轴) 设 $f(x) \in R[0,1]$,求证:
\[
\lim_{n \to \infty} \int_0^1 f(x) |\sin nx| dx = \frac{2}{\pi} \int_0^1 f(x) dx.
\]

\end{example}
\begin{example}
	\begin{itemize}
		\item 求证:
		\[
		\lim_{n \to \infty} \int_0^{\frac{\pi}{2}} \sin^n x \, dx = 0.
		\]
		
		\item (2022 吉大夏令营改编) 设 $f \in R[0,1]$,且 $f(x)$ 在 $x=1$ 处连续,求证:
		\[
		\lim_{n \to \infty} n \int_0^1 f(x) x^n \, dx = f(1).
		\]
	\end{itemize}
\end{example}
\section{渐进展开}
\subsection{初等方法}
反复利用Stolz定理即可
\begin{example}
	(2021 电子科大考研) 设 $x_{n+1} = \ln(1+x_n)$, $n=1,2,\cdots$, $x_1>0$。试求 
\[
\lim_{n\to\infty}\frac{n(nx_n-2)}{\ln n}。
\]

\end{example}
\begin{example}
	设 $x_{n+1}=\sin x_n$, $n=1,2,\cdots$, $x_1\in (0,\pi)$。计算
\[
\lim_{n\to\infty}\frac{n}{\ln n}\left(1-\sqrt{\frac{n}{3}}x_n\right)。
\]
\end{example}
\begin{example}
	(2023 南大夏令营) 已知方程 $x^n - 2023x = 2023$ 在 $(0,+\infty)$ 上有唯一解 $x_n$,试求极限
\[
\lim_{n\to\infty}\frac{nx_n^n}{\ln n}。
\]
\end{example}
\begin{example}
	\begin{enumerate}
		\item (2020 电子科大考研) 设 $0 < a_n < 1$, $a_{n+1} = a_n (1 - a_n)$,求证:
		\[
		\lim_{n \to \infty} \frac{n a_n}{\ln n} = 1.
		\]
		
		\item (2024 浙大考研) 设 $x_1 = 1$, $x_{n+1} = \sqrt{\frac{2x_n^2}{2 + x_n^2}}$, $n = 1, 2, \cdots$。求证:
		\[
		\lim_{n \to \infty} \frac{n(x_n - x_{n+1})}{\ln (1 + x_n)} = 1.
		\]
	\end{enumerate}
\end{example}
\subsection{迭代方法}
\begin{example}
	设 $x_n > 0$ 且满足 $x_n e^{x_n} = n$, $n = 1, 2, \cdots$,求证:
\[
x_n = \ln n - \ln \ln n + \frac{\ln \ln n}{\ln n} + o\left(\frac{\ln \ln n}{\ln n}\right), \quad n \to \infty.
\]
\end{example}
\begin{example}
	(2024 上交考研) 设 $f \in C(0, +\infty)$,且满足 $x = f(x)5^{f(x)}$,求证:
\begin{enumerate}
    \item $f(x)$ 严格单调递增,
    \item $\lim\limits_{x \to +\infty} f(x) = +\infty$,
    \item $\lim\limits_{x \to +\infty} \frac{f(x)}{\ln x} = \frac{1}{\ln 5}$。
\end{enumerate}
\end{example}
\begin{example}
	(2021 北师夏令营压轴) 给定方程 $x^n(x-1) = 1$,求证:

\begin{enumerate}
    \item 上述方程在 $[1, +\infty)$ 上有唯一根 $x_n$。
    \item 成立不等式:
    \[
    x_{n+1} > 1 + \frac{\ln n}{n} - \frac{\ln \ln n}{n}
    \]
\end{enumerate}
\end{example}
\subsection{Laplace方法}
大体上,Laplace 方法适用于 $\int_{a}^{b} f^n(x)g(x)dx$ 型积分的渐近估计。可以通过变形和换元法转化为标准形式。这种方法的整体思想就是抓极值部分和所谓的局部化原理。
\begin{example}
	(Wallis 公式) 求证:
\[
\frac{(2n)!!}{(2n-1)!!} \sim \sqrt{\pi n}, \quad n \to \infty.
\]
\end{example}
\begin{example}
	(Stirling 公式) 求证:
\[
n! \sim \sqrt{2\pi n}\left(\frac{n}{e}\right)^n, \quad n \to \infty.
\]

\end{example}
\begin{example}
	\begin{itemize}
		\item (2023 吉大夏令营) 已知函数 $f(x)$ 一阶导存在且 $f(1) = 0$,求极限
		\[
		\lim_{n \to \infty} n \int_{0}^{1} x^{n} f(x) \, dx.
		\]
	\end{itemize}
\end{example}

\section{递推数列的敛散性判断}
\subsection{单调性分析方法}
单调性分析方法仅仅适用于
\[
x_{n+1} = f(x_n), \quad n \in \mathbb{N}.
\]
$f$ 是单调递增或是单调递减的情形。

\subsubsection*{结论 1}
设 $f$ 是单调递增函数,则上述递推式确定的 $x_n$ 一定单调,且和不动点的大小关系固定。



\subsubsection*{结论 2}
设 $f$ 是单调递减函数,则上述递推式确定的 $x_n$ 一定不单调,且和不动点的大小关系交错。

\begin{example}
	设 $x_1 > -1$, $x_{n+1} = \frac{1}{1+x_n}$, $n = 1, 2, \cdots$,求极限
\[
\lim_{n \to \infty} x_n.
\]
\end{example}
\begin{example}
	(第十五届全国大学生数学竞赛赛制)设 $x_1 = a = \sqrt[3]{3}$, $x_{n+1} = a^{x_n}$, $n = 1, 2, 3, \ldots$,求证:
\[
\lim_{n \to \infty} x_n
\]
存在但不等于3。
\end{example}
\begin{example}
	\begin{itemize}
		\item 设 $x_1 > -6$, $x_{n+1} = \sqrt{6 + x_n}$, $n = 1, 2, \ldots$, 计算 $\lim\limits_{n \to \infty} x_n$.
		
		\item 设 $x_1 = 2$, $x_n + (x_n - 4)x_{n-1} = 3$, $n = 2, 3, \ldots$, 计算 $\lim\limits_{n \to \infty} x_n$.
	\end{itemize}
\end{example}
\subsection{压缩映像方法}
压缩映像方法是一种非常重要的处理模型。其思想内核有两种。
一种是找到不动点 \( x_0 \),然后得到某个 \( L \in (0,1) \)。s.t.

\[|x_n - x_0| \leq L |x_{n-1} - x_0| \leq \cdots \leq L^{n-1} |x_1 - x_0|\]

另一种是得到某个 \( L \in (0,1) \)。s.t.

\[|x_n - x_{n-1}| \leq L |x_{n-1} - x_{n-2}| \leq \cdots \leq L^{n-2} |x_2 - x_1|\]

当数列的递推式 \( x_{n+1} = f(x_n) \) 确定时,我们有:

\[|x_n - x_0| = |f(x_{n-1}) - f(x)|, \quad |x_n - x_{n-1}| = |f(x_{n-1}) - f(x_{n-2})|\]

因此往往可以通过中值定理或直接放缩来得到 \( L \in (0,1) \)。


\begin{example}
	(2023 中科院提前批)对于给定实数 \( x \),不断将余弦函数作用在之前的数上,得到的序列 \(\{a_n\}\) 如下:\( a_0 = x, a_1 = \cos(x), a_2 = \cos(\cos(x)) \),\(\cdots\),试问当 \( n \to \infty \) 时,这一序列会有怎样的趋势?
\end{example}
\begin{example}
	给定方程 \(x^n + 2023x = 2023\),求证:方程在 \((0,+\infty)\) 只有唯一解 \(x_n\)。且 \(\lim\limits_{n \to \infty} x_n = 1\)。

\end{example}
\begin{example}
	设 \(x_1 > -1, x_{n+1} = \frac{1}{1+x_n}, n = 1, 2, \cdots\),求极限 \(\lim\limits_{n \to \infty} x_n\)。
\end{example}
\begin{example}
	求数列 \(\sqrt{7}, \sqrt{7-\sqrt{7}}, \sqrt{7-\sqrt{7}+\sqrt{7}}, \ldots\) 的极限。
\end{example}
\subsection{蛛网工作法}
\begin{figure}[h]
	\centering
	\includegraphics[width=0.5\textwidth]{figure/1.PNG}
\end{figure}
先看图2.4(a)。在其中的曲线代表函数 \( y = f(x) \)。它同直线 \( y = x \) 的交点的横坐标 \( a \) 就是 \( f \) 的不动点。从图中的 \( x \) 轴上代表初始值 \( a_1 \) 的点出发作平行于 \( y \) 轴的直线,它与曲线 \( y = f(x) \) 的交点的纵坐标就是 \( a_2 = f(a_1) \)。在这里的一个技巧是从上述交点作平行于 \( x \) 轴的直线与直线 \( y = x \) 相交。这个交点的横坐标当然也是 \( a_2 \)。在图中从这个交点作一条虚线与纵轴平行,并将它与 \( x \) 轴的交点标为 \( a_2 \)。这就完成了蛛网工作法的第一步。

在图2.4(a)上将这个方法继续做几步:可以看出,所得的数列是单调增加的

\begin{example}
	设 \( u_1 = b \), \( u_{n+1} = u_n^2 + (1 - 2a)u_n + a^2 \),试判断 \( u_n \) 的敛散性。
\end{example}

\begin{example}
	设 \( x_{n+1}(2 - x_n) = 1 \), \( n = 1, 2, \cdots \)。试判断 \( x_n \) 的敛散性。
\end{example}

\begin{example}
	定义数列 \( a_0 = x \), \( a_{n+1} = \frac{a_n^2 + y^2}{2} \), \( n = 0, 1, 2, \cdots \),记 \( D = \{(x, y) \in \mathbb{R}^2 : \text{数列 } a_n \text{ 收敛}\} \)。
\end{example}
\chapter{函数}
\section{连续性和可微性}
\subsection{定义法}
\begin{example}
	(2023 复旦应统夏令营) 若 \( f'(x_0) \) 存在,试求 \(\lim_{h \to 0} \frac{f(x_0 + ah) - f(x_0 - bh)}{h}\);反之,若 \(\lim_{h \to 0} \frac{f(x_0 + h) - f(x_0) - h}{h}\) 存在,试问 \( f(x) \) 在 \( x_0 \) 处是否可导?
\end{example}
\begin{example}
	(2021 中科院考研) 若 \( f(x) \) 在 \( x = 0 \) 处连续,且 \(\lim_{x \to 0} \frac{f(2x) - f(x)}{x} = 0\),求证:\( f'(0) = 0 \)。
\end{example}
\begin{example}
	(2024 上交夏令营) 设函数 \( f(x) \) 在 \((-\infty, +\infty)\) 上连续,且 \(\lim_{x \to 0} \frac{f(x)}{x}\) 存在,令  
\[g(x) = \int_0^1 f(xt) \, dt,\]

(1) 求 \( g'(x) \);  

(2) 讨论 \( g'(x) \) 在 \( x=0 \) 处的连续性。
\end{example}
\subsection{级数法}
\begin{example}
	(2024 同济夏令营) 设 \(\{x_n\} \subset (0,1)\), \(x_j \neq x_i\) (\(i \neq j\)), 定义函数
\[ f(x) = \sum_{n=1}^{\infty} \frac{\operatorname{sgn}(x - x_n)}{2^n}, \]
试判断 \(f(x)\) 的连续性并给出证明。

\end{example}
\begin{example}
	(2023 复旦数科院夏令营) 设数列 \(\{r_n\}\) 为 \([0,1]\) 中的所有有理点的一个排列,证明函数
\[ f(x) = \sum_{n=1}^{\infty} \frac{|x - r_n|}{3^n}, \quad x \in [0,1] \]
具有以下性质:
\begin{enumerate}
    \item[(1)] 处处连续;
    \item[(2)] 在无理点处可微,有理点处不可微。
\end{enumerate}
\end{example}
\begin{example}
	(Cantor 函数) 设 \( C \) 为 \([0,1]\) 上的 Cantor 集,对于 \( x = 2 \sum_{i=1}^{\infty} \frac{a_i}{3^i} \in C \), \( a_i \in \{0,1\} \),令  
\[ \phi(x) = \phi\left(2 \sum_{i=1}^{\infty} \frac{a_i}{3^i}\right) = \sum_{i=1}^{\infty} \frac{a_i}{2^i}. \]  
Cantor 函数 \(\phi(x)\) 定义为:\(\forall x \in [0,1]\),  
\[ \phi(x) = \sup\{\phi(y) \mid y \in C, y \leq x\}. \]  
求证:\(\phi\) 为 \([0,1]\) 上的连续函数。
\end{example}
\begin{example}
	设 \( Q = \{x_1, x_2, \cdots\} \) 为有理数集合,令  
	\[ f(x) = \sum_{x_n \leq x} \frac{1}{2^n}, \]  
	证明:\( f(x) \) 仅在有理点处不连续。
\end{example}
\subsection{Schwarz导数}
\begin{definition}
	设有定义在开集 \( A \subset \mathbb{R} \) 上的实函数 \( f(x) \),若对于 \( a \in A \),
\[ \lim_{h \to 0} \frac{f(a+h)-f(a-h)}{2h} = f^s(a) \]
存在,则称 \( f(x) \) 在点 \( a \) Schwarz 可导,称 \( f^s(a) \) 为 Schwarz 导数。

\end{definition}
\begin{proposition}
	设 \( f(x) \in C[a,b] \) 且在 \( (a,b) \) 上 Schwarz 可导,那么:
\begin{itemize}
    \item 若 \( f(b) > f(a) \),则 \( \exists\, c \in (a,b) \),使得 \( f^s(c) \geq 0 \)。
    \item 若 \( f(a) > f(b) \),则 \( \exists\, c \in (a,b) \),使得 \( f^s(c) \leq 0 \)。
\end{itemize}

\end{proposition}
\begin{theorem}[Schwarz 导数的 Rolle 中值定理]
	设 \( f(x) \in C[a, b] \) 且在 \( (a, b) \) 上 Schwarz 可导,若 \( f(a) = f(b) \),则  
\(\exists\, r, t \in (a, b) \),使得 \( f^s(r) \geq 0 \) 且 \( f^s(t) \leq 0 \)。

\end{theorem}
\begin{theorem}[Schwarz 导数的 Lagrange 中值定理]
	设 \( f(x) \in C[a, b] \) 且在 \( (a, b) \) 上 Schwarz 可导,则存在 \( r, t \in (a, b) \),使得
\[ f^s(r) \leq \frac{f(b) - f(a)}{b - a} \leq f^s(t). \]
\end{theorem}
\begin{theorem}[Schwarz 导数与一般导数的联系]
	设 \( f(x) \in C[a, b] \) 且在 \( (a, b) \) 上 Schwarz 可导,若 \( f^s(x) \) 在 \( (a, b) \) 上连续,则 \( f(x) \) 在 \( (a, b) \) 上可导且
\[ f'(x) = f^s(x), \quad \forall x \in (a, b). \]

\end{theorem}
\begin{theorem}[利用 Schwarz 导数判断函数单调性]
	设函数 \( f(x) \) 在开区间 \( I \) 上连续且 Schwarz 可导,若 \( f^s(x) \geq 0 \) 对所有 \( x \in I \) 成立,则 \( f(x) \) 在 \( I \) 上单调递增。
\end{theorem}
\section{一致连续性}
\begin{example}
	设 \( f(x) \) 在 \([a, +\infty)\) 上一致连续,\( g(x) \) 在 \([a, +\infty)\) 上连续,且
\[ \lim_{x \to +\infty} |f(x) - g(x)| = 0. \]
求证:\( g(x) \) 在 \([a, +\infty)\) 上一致连续。
\end{example}
\begin{example}
	设 \( f(x) \) 在 \([1, +\infty)\) 上一致连续。求证:存在 \( M > 0 \),使得
\[ \frac{|f(x)|}{x} \leq M, \quad \forall x \in [1, +\infty). \]
\end{example}
\begin{example}
	设 \( f(x) \) 在 \([0, +\infty)\) 上一致连续,且对任意 \( x \geq 0 \) 有
\[ \lim_{n \to +\infty} f(x+n) = 0, \quad n \in \mathbb{N}. \]
求证:
\[ \lim_{x \to +\infty} f(x) = 0. \]
\end{example}
\begin{example}
	设 \( f(x) \) 在 \([0, +\infty)\) 上连续,且满足
\[ \lim_{n \to \infty} f(\sqrt{n}) = 0. \]
求证:\(\lim_{x \to +\infty} f(x)\) 存在当且仅当 \( f(x) \) 在 \([0, +\infty)\) 上一致连续。

\end{example}
\begin{example}
	若 \( f(x) \) 在 \([0,+\infty)\) 上可导,且满足:
\begin{enumerate}
    \item \( f'(x) \) 在 \([0,+\infty)\) 上一致连续;
    \item \( \lim_{x \to +\infty} f(x) \) 存在。
\end{enumerate}
求证:
\[ \lim_{x \to +\infty} f'(x) = 0. \]
\end{example}
\begin{example}
	(2024 国防科大考研) 设函数 \( f(x) \) 在 \( (0,1] \) 上连续且可导,且满足
    \[ \lim_{x \to 0^+} \sqrt{x}f'(x) = a. \]
    求证:\( f(x) \) 在 \( (0,1] \) 上一致连续。
\end{example}
\begin{example}
	(2024 哈工大考研) 设 \( (a,b) \) 为有界开区间,求证:\( f(x) \) 在 \( (a,b) \) 上一致连续的充要条件是对于任意 Cauchy 列 \( \{x_n\} \subset (a,b) \),像列 \( \{f(x_n)\} \) 也是 Cauchy 列。
\end{example}
\begin{example}
	(2023 吉大夏令营) 设 \( f(x) \in C[1,+\infty) \),且满足
    \[ \lim_{x \to +\infty} \frac{f(x)}{x^2} = 1. \]
    求证:\( f(x) \) 在 \( [1,+\infty) \) 上非一致连续。
\end{example}
\section{函数方程}
\subsection{柯西方程}
\begin{definition}
	我们称函数 \( f \colon \mathbb{R} \to \mathbb{R} \) 满足的函数方程
\[ f(x + y) = f(x) + f(y) \]
为 \textbf{Cauchy 方程}。
\end{definition}
\begin{example}
	设 \( f \colon \mathbb{R} \to \mathbb{R} \) 是 Cauchy 方程的解,则:
\begin{enumerate}
    \item[(1)] 对任意有理数 \( r \in \mathbb{Q} \),有
    \[ f(rx) = r f(x); \]
    \item[(2)] 进一步,若 \( f \) 连续,则存在常数 \( c = f(1) \) 使得
    \[ f(x) = c x, \quad \forall x \in \mathbb{R}. \]
\end{enumerate}
\end{example}
\begin{example}
	求证:\(\mathbb{R}\) 上既凸又凹的连续函数必为线性函数,即存在常数 \( a, b \in \mathbb{R} \) 使得
\[ f(x) = a x + b, \quad \forall x \in \mathbb{R}. \]
\end{example}
\begin{example}
	设 \( f(x) \) 在 \( (0, +\infty) \) 上连续,且满足函数方程:
\[ f(xy) = x f(y) + y f(x), \quad \forall x, y \in (0, +\infty). \]
求证:\( f(x) \) 在 \( (0, +\infty) \) 上可微。
\end{example}
\begin{example}
	(2024 中科院夏令营) 证明:在 \( \mathbb{R} \) 上满足函数方程
\[ f(x+y) = f(x) f(y) \]
的唯一不恒等于零的连续解是指数函数 \( f(x) = a^x \),其中 \( a > 0 \) 为常数。
\end{example}
\subsection{迭代法}
\textbf{基本思想}:通过构造递推关系,将函数方程转化为可求和的形式。
\begin{example}
	求函数方程 
\[ 2f(2x) = f(x) + x \]
在 $\mathbb{R}$ 上且满足 $f$ 在 $x=0$ 处连续的所有解。
\end{example}
\begin{example}
	求函数方程
\[ f(x+y) - f(x-y) = f(x)f(y) \]
在 $\mathbb{R}$ 上且满足 $f$ 在 $x=0$ 处连续的所有解。
\end{example}
\begin{example}
	求函数方程 
    \[ f(\log_2 x) = f(\log_3 x) + \log_5 x \]
    在 \( \mathbb{R}^+ \) 上的所有连续解。
\end{example}

\chapter{一元函数微分学}
\section{微分中值定理}
\subsection{插值法}
有一类中值定理习题中往往会给我们很多关于函数 $f(x)$ 的信息,进而去证明一个等式。如何利用好这些已知信息?这就涉及到数值分析中的 Lagrange 插值与 Newton 插值。
\begin{theorem}[Lagrange 插值]
	已知插值节点为 $(x_i, y_i), i = 0, 1, 2, \cdots, n$,那么

\[ y = f(x) = \sum_{0 \leq i < n} \frac{(x - x_0) \cdots (x - x_{i-1})(x - x_{i+1}) \cdots (x - x_n)}{(x_i - x_0) \cdots (x_i - x_{i-1})(x_i - x_{i+1}) \cdots (x_i - x_n)} y_i \]

对应的插值余项为:

\[ R_n(x) = f(x) - L_n(x) = \frac{f^{(n+1)}(\xi)}{(n+1)!}(x - x_0)(x - x_1) \cdots (x - x_n) \]

\end{theorem}
\begin{theorem}[Newton 插值]
	已知插值节点为 $(x_i, y_i), i = 0, 1, 2, \cdots, n$,那么

\[ f(x) = a_0 + a_1(x - x_0) + a_2(x - x_0)(x - x_1) + \cdots + a_n(x - x_0)(x - x_1)(x - x_2) \cdots (x - x_{n-1}), \]

其中 $a_n = f[x_0, x_1, \cdots, x_k]$,对应的插值余项为:

\[ R_n(x) = f(x) - L_n(x) = \frac{f^{(n+1)}(\xi)}{(n+1)!}(x - x_0)(x - x_1) \cdots (x - x_n) \]

\textbf{Tip:} Newton 插值可以带重节点,进而反映出导数信息。
\end{theorem}
\begin{example}
	设 $f \in D^3[-1, 1]$,$f(-1) = 0$,$f'(0) = 0$,$f(1) = 1$。求证:$\exists \theta \in (-1, 1)$,s.t. $f'''(\theta) = 3$。
\end{example}
\begin{example}
	设 \(f \in C[0, 2] \cap D(0, 2)\) 满足 \(f(0) = f(2) = 0\), \(|f'(x)| \leq M\), \(\forall x \in (0, 2)\)。求证:
\[ \left| \int_{0}^{2} f(x) \, dx \right| \leq M. \]
\end{example}
\begin{example}
	设 \( f \in C^3[0,2] \) 满足
\[ f(0) = f'(0) = 0, \quad \int_{0}^{2} f(x) \, dx = 8 \int_{0}^{1} f(x) \, dx. \]
证明:存在 \( \theta \in (0,2) \),使得 \( f''(\theta) = 0 \)。
\end{example}
下面我们来介绍插值法的积分型余项,这种余项往往蕴含着更多的信息。
\begin{theorem}[积分型余项]
	设 \( f(x) \) 是 \([a,b]\) 上的二阶可导函数且 \( f''(x) \in R[a,b] \),则有:
\[ f(x) = \frac{b-x}{b-a}f(a) + \frac{x-a}{b-a}f(b) + \int_a^b f''(y)k(x,y)dy, \]
其中,
\[ k(x,y) = 
\begin{cases} 
\frac{(x-a)(y-b)}{b-a}, & a \leq y \leq x \leq b; \\ 
\frac{(b-x)(a-y)}{b-a}, & a \leq x \leq y \leq b.
\end{cases} \]
\end{theorem}
\begin{example}
	(2021华东师范考研压轴) 设 \( f \in C^2[0,1] \) 满足 \( f(0) = f(1) = 0 \),证明:
\[ \int_0^1 \left| \frac{f''(x)}{f(x)} \right| dx \leq 4. \]
\end{example}
\begin{example}
	设 \( f(x) \in D^2[0, 1] \), \( f(0) = f(1) = 0 \), \( |f''(x)| \leq M \), 求证:
    \[ \int_{0}^{1} f(x) \, dx \leq \frac{M}{12} \]
    
\end{example}
\begin{example}
	已知 \( f(x) \in C^2[a, b] \), \( f(a) = f(b) = 0 \), 求证:
    \[ |f(x)| \leq \frac{(x - a)(b - x)}{b - a} \int_{a}^{b} |f''(y)| \, dy \]
\end{example}
\subsection{K值法}
看例题操作,有手就行
\begin{example}
	(2021 天津大学考研) 设函数 \( f(x) \in C[a,b] \cap D^2(a,b) \),求证:\(\exists \xi \in (a,b)\),使得
\[ f(b) - 2f\left(\frac{a+b}{2}\right) + f(a) = \frac{(b-a)^2}{4}f''(\xi). \]
\end{example}
\begin{example}
	(2024 大连理工考研) 设函数 \( f(x) \) 在 \([0,2]\) 上有二阶连续导数,且 \( f(0) = f(1) = f(2) = 0 \),求证:\(\forall x \in (0,2)\),\(\exists c \in (0,2)\),使得
\[ f(x) = \frac{1}{6}x(x-1)(x-2)f'''(c). \]
\end{example}
\begin{example}
	设 $f$ 在 $[a,b]$ 上二阶可微,$f(a) = f(b) = 0$。证明:对每个 $x \in (a,b)$,存在 $\xi \in (a,b)$,使得
    \[ f(x) = \frac{f''(\xi)}{2}(x-a)(x-b). \]
\end{example}
\begin{example}
	设 $f \in D^3[0,1]$ 满足 $f(0) = -1$, $f'(0) = 0$, $f(1) = 0$。证明对任何 $x \in [0,1]$,存在 $\theta \in (0,1)$,使得
    \[ f(x) = -1 + x^2 + \frac{x^2(x-1)}{6}f'''(\theta). \]
\end{example}
\subsection{微分方程法}
有一类中值定理习题的解决需要我们构造合适的函数,我们可以通过解微分方程来得到。下面我们结合具体的例子来说明。
\begin{example}
	设函数 \( f(x) \in C[0,2] \cap D(0,2) \),且 \( f(0) = f(2) = 0 \),\(\lim\limits_{x \to 1} \frac{f(x) - 2}{x - 1} = 5\)。求证:
\[ \exists \xi \in (0,2), \text{ s.t. } f'(\xi) = \frac{2\xi - f(\xi)}{\xi}. \]
\end{example}
\begin{example}
	(2024 复旦夏令营) 设函数 \( f(x) \in C[0,1] \cap D(0,1) \),且 \( f'(1) = 0 \)。求证:
\[ \exists \xi \in (0,1), \text{ s.t. } f'(\xi) = 2024(f(\xi) - f(0)). \]
\end{example}
\begin{example}
	(2024 上海夏令营) 设 \( f(x) \) 在 \([a,b]\) 上可导,且 \(\exists c \in [a,b]\), s.t. \( f'(c) = 0 \). 求证:
\[ \exists \xi \in [a,b], \text{ s.t. } f'(\xi) = \frac{f(\xi) - f(a)}{b - a}. \]
\end{example}
\begin{example}
	设函数 \( f(x) \in C[0,1] \cap D(0,1) \),且 \( f(0) = 0 \)。求证:
    \[ \exists u \in (0,1), \text{ s.t. } f'(u) = \frac{u f(u)}{1 - u}. \]
\end{example}
\begin{example}
	设函数 \( f(x) \in C[-1,2] \cap D(-1,2) \), \( f(-1) = f(2) = -\frac{1}{2} \), \( f\left(\frac{1}{2}\right) = 1 \)。求证:
    \[ \forall \lambda \in \mathbb{R}, \exists \xi \in (-1,2), \text{ s.t. } f'(\xi) = \lambda \left|f(\xi) - \frac{\xi}{2}\right| + \frac{1}{2}. \]
\end{example}
\begin{example}
	设函数 \( f(x) \in D^2\left[0,\frac{\pi}{4}\right] \), \( f(0) = 0 \), \( f'(0) = 1 \), \( f\left(\frac{\pi}{4}\right) = 1 \)。求证:
    \[ \exists \xi \in \left(0,\frac{\pi}{4}\right), \text{ s.t. } f''(\xi) = 2f(\xi)f'(\xi). \]
\end{example}
\section{函数性态分析}
\subsection{常用结论}
\begin{theorem}
	导数大于0则函数的趋于无穷设 $f$ 在 $[0,+\infty)$ 上可微且 $\lim_{x \to +\infty} f(x) = x > 0$。求证:$\lim_{x \to +\infty} f(x) = +\infty$。
\end{theorem}
\begin{theorem}
	函数值在 $[0,+\infty)$ 处必然有趋于0的子列,设 $k \in \mathbb{N}, a \in \mathbb{R}$ 且 $f \in D^k[a,-\infty]$。若 $\lim_{x \to +\infty} |f|(x) \neq +\infty$,那么存在趋于正无穷的数列 $\{x_n\}_{n=1}^\infty \subset [0,+\infty), \forall \lim_{x \to +\infty} f^{(k)}(x_n) = 0$。
\end{theorem}
\begin{theorem}[导数极限定理]
	设 $f(x) \in C^{m,n} \cup D^m[a,b]$,且 $\lim_{x \to a_+} f'(x) = c$,存在。求证:$f(x)$ 在 $[0,+\infty)$ 处可导且 $f_+'(a) = c$。
\end{theorem}
\begin{theorem}[低阶导数控制高阶导数]
	设 $f(x)$ 在 $[0,+\infty)$ 上 $n$ 阶可微,且存在有限极限 $\lim_{x \to +\infty} f(x)$ 和 $\lim_{x \to +\infty} f^{(k)}(x)$。求证:$\forall k = 1,2,\cdots,m$,有:$\lim_{x \to +\infty} f^{(k)}(x) = 0$。
\end{theorem}
\begin{theorem}[低阶导数控制高阶导数]
	设 $f(x)$ 在 $[0,+\infty)$ 上二阶可微,且$
	|f(x)|,|f'(x)|$的上确界 $A,B$。求证:$|f(x)| \leq \sqrt{2AB}$。
\end{theorem}
\begin{theorem}[低阶导数控制高阶导数]
	设 $f(x)$ 在 $[0,+\infty)$ 上二阶可微,$\lim_{x \to \infty} f(x)=0 $,且$\lambda\in\mathbb{R} s.t.f''(x)+\lambda f'(x)$有上界,求证$\lim_{x \to \infty} f'(x)=0 $
\end{theorem}
\subsection{综合运用}
\begin{example}
	设 \( f(x) \in C(\mathbb{R}) \), \( g(x) = f(x)\int_{0}^{x} f(t) dt \) 单调递减, 求证: \( f(x) \equiv 0 \).
\end{example}
\begin{example}
	设函数 \( f(x) \) 在 \((0, +\infty)\) 上可微, 极限 \(\lim_{x \to +\infty} f(x)\) 和 \(\lim_{x \to +\infty} f'(x)\) 均存在, 求证: \(\lim_{x \to +\infty} f'(x) = 0\).
\end{example}
\begin{example}
	若 \( f'(x) \in C^2[0,1] \),\( f'(0) = 0 \),\( |f''(x)| \leq |f(x) - f(0)| \)。求证:\( f(x) \) 在 \([0,1]\) 上为常值函数。
\end{example}
\begin{example}
	设 \( f(x) \in C^2[0,1] \),\( f(0) = f(1) \),且 \( |f''(x)| \leq 2 \),\(\forall x \in [0,1]\)。  
证明:\(\forall x \in [0,1],|f'(x)| \leq 1\)。
\end{example}
\begin{example}
	设 \( f(x) \in C[0,1] \cap D(0,1) \), \( f(0) = f(1) \), 且 \( |f'(x)| < 1 \).  
求证:\( \forall x_1, x_2 \in [0,1] \), \( |f(x_1) - f(x_2)| < \frac{1}{2} \).
\end{example}
\section{函数逼近问题}
\subsection{连续函数的逼近}
\begin{theorem}[Weierstrass 第一逼近定理]
	对于闭区间 $[a, b]$ 上的任意连续函数 $f(x)$,存在多项式序列 $\{P_n\}$ 在 $[a, b]$ 上一致收敛于 $f(x)$。
\end{theorem}
\begin{theorem}[Weierstrass 第二逼近定理]
	$\mathbb{R}$上周期为 $2\pi$ 的连续函数可被三角多项式一致逼近。
\end{theorem}


\begin{example}
	设 $f(x) \in C[0, 1]$,$\forall n \in \mathbb{N}$,$\int_0^1 f(x) x^n dx = 0$,$\forall n = 0, 1, 2, \cdots$,求证:$f(x) \equiv 0, \forall x \in [0, 1]$。
\end{example}
\begin{example}[Riemann 引理]
	设函数 $f(x)$ 在 $[a, b]$ 上可积,那么有:
\[
\lim_{\lambda \to \infty} \int_a^b f(x) \sin \lambda x dx = 0, \quad \lim_{\lambda \to \infty} \int_a^b f(x) \cos \lambda x dx = 0.
\]
\end{example}
\begin{example}
	设 $f(x) \in R[a, b], g(x)$ 以 $T$ 为周期且在 $[0, T]$ 上可积,则有:
    \[
    \lim_{n \to \infty} \int_a^b f(x) g(nx) dx = \frac{1}{T} \int_0^T g(x) dx \int_a^b f(x) dx.
    \]
\end{example}
\begin{example}
	设 $f(x) \in R[a, b]$,求证:$\lim_{n \to \infty} \int_a^b f(x) \sin nx dx = \frac{2}{\pi} \int_a^b f(x) dx$。
\end{example}
\subsection{可积函数的逼近}
\begin{example}[阶梯逼近]
	设 \( f(x) \in R[a,b] \), \(\forall \epsilon > 0\), 存在 \([a,b]\) 上的阶梯函数 \( g(x) \), 使得 
\[ \int_{a}^{b} |f(x)-g(x)|dx \leq \epsilon. \]
\end{example}
\begin{example}[连续逼近]
	设 \( f(x) \in R[a,b] \), \(\forall \epsilon > 0\), 存在 \( g(x) \in C[a,b] \), 使得 
\[ \int_{a}^{b} |f(x)-g(x)|dx < \epsilon. \]
\end{example}
\begin{example}[可微逼近]
	设 \( f(x) \in R[a,b] \), \(\forall \epsilon > 0\), 存在 \( g(x) \in C^{1}[a,b] \), 使得 
\[ \int_{a}^{b} |f(x)-g(x)|dx < \epsilon. \]
\end{example}
\begin{example}[绝对连续性]
	设 \( f(x) \) 在任意有限区间可积,证明:\(\forall [a,b]\), 有:
\[ \lim_{h \to 0} \int_{a}^{b} |f(x+h)-f(x)|dx = 0. \]
\end{example}
\begin{example}
	设 \( f(x) \in R[a,b] \), \( F(x) = \int_{a}^{x} f(t)dt \). 求证:
\[ \int_{a}^{b} F(x)dx = \int_{a}^{b} (b-x)f(x)dx. \]
\end{example}

\chapter{一元函数积分学}
\section{积分的计算}
\subsection{区间再现公式}
\begin{example}[区间再现公式]
	求 \( \int_{a}^{b} f(x) dx - \int_{a}^{b} f(a+b-x) dx = \frac{1}{2} \int_{a}^{b} [f(x)+f(a+b-x)] dx \)。
	\end{example}
	
	\begin{example}
	计算:
	\[(1) \int_{0}^{1} \ln a \ln x dx, \quad (2) \int_{0}^{1} \frac{\ln(1+x)}{1+x^2} dx.\]
	\end{example}
	
\begin{example}
	计算:
	\[(1) \int_{0}^{\infty} \frac{dx}{(1+x^2)(1+x^a)} \ (a>0), \quad (2) \int_{0}^{\infty} \frac{\ln x}{x^2 + x + 1 }dx.\]
\end{example}
\begin{example}
	计算
	\[\int_{0}^{\frac{\pi}{2}}\frac{e^{\sin x}}{e^{\sin x}+e^{\cos x}}  \,dx 
	\]
\end{example}
\begin{example}
	对$n\in\mathbb{N} $计算
	\[\int_{0}^{2\pi}  \sin(\sin x+nx )\,dx 
	\]
\end{example}
\subsection{Froullani积分}
\begin{example}[Froullani积分]
	设 \( f \in C(0,+\infty) \),若存在极限  
	\[\lim_{x \to 0^+} f(x), \quad \lim_{x \to +\infty} f(x)\]  
	则有:  
	\[\int_0^\infty \frac{f(ax) - f(bx)}{x} dx = \left[ \lim_{x \to 0^+} f(x) - \lim_{x \to +\infty} f(x) \right] \ln \frac{b}{a}\]
\end{example}
	
\begin{example}
	计算  
	\[\int_0^{+\infty} \left( \frac{\sin 3x}{3x^2} - \frac{\sin 2x}{2x^2} \right) dx\]
\end{example}
	
\begin{example}
	计算 \[ \int_{0}^{\infty} \frac{\cos ax - \cos bx}{x} dx, \quad b > a > 0. \]
\end{example}
\begin{example}
	\begin{enumerate}
		\item 若存在极限和积分 \[ \lim_{x \to 0} f(x) = 0, \quad \int_{a}^{b} \frac{f(x)}{x} dx \],求证:
		\[ \forall a, b > 0, \int_{0}^{a} \frac{f(ax) - f(bx)}{x} dx = a \ln \frac{b}{a};\]
		\item 若存在极限和积分 \[ \lim_{x \to \infty} f(x) = 0, \quad \int_{0}^{a} \frac{f(x)}{x} dx \],求证:
		\[ \forall a, b > 0, \int_{0}^{a} \frac{f(ax) - f(bx)}{x} dx = a \ln \frac{b}{a}.\]
	\end{enumerate}
\end{example}
\subsection{化为含参积分处理}
\begin{example}
	计算 \[ \int_{0}^{1} \frac{\arctan x}{x\sqrt{1-x^2}} \, dx \]
	\end{example}
	
	\begin{example}
	计算 \( I(y) = \int_{0}^{\infty} e^{-x^2} \cos 2xy \, dx, \quad y \in \mathbb{R} \).
	\end{example}
	
	\begin{example}
	计算 \( \int_{0}^{\infty} \frac{\arctan ax}{x(1+x^2)} \, dx \).
	\end{example}
\begin{example}
	计算
	\[\int_{0}^{+\infty}\frac{\cos x-\cos 2x}{x}e^{-x}  \,dx 
	\]
\end{example}
\begin{example}
	计算
	\[\int_{0}^{+\infty}\frac{\sin bx-\sin ax}{x}e^{-px}  \,dx ,p>0,b>a
	\]
\end{example}

\subsection{级数方法}
为了换序 \(\sum_{n=1}^{\infty} \int_{a}^{b} f_n(x) dx = \int_{a}^{b} \sum_{n=1}^{\infty} f_n(x) dx\),我们需要:

\[\lim_{m \to \infty} \sum_{n=1}^{m} \int_{a}^{b} f_n(x) dx = \int_{a}^{b} \sum_{n=1}^{\infty} f_n(x) dx.\]

有限和随意交换,我们需要:

\[\lim_{m \to \infty} \int_{a}^{b} \sum_{n=1}^{m} f_n(x) dx = \int_{a}^{b} \sum_{n=1}^{\infty} f_n(x) dx.\]

于是需要:

\[\lim_{m \to \infty} \int_{a}^{b} \sum_{n=m+1}^{\infty} f_n(x) dx = 0.\]

\begin{example}
计算:\(\int_{0}^{\infty} \frac{x}{1+e^x} dx\).
\end{example}

\begin{example}
计算:\(\int_{0}^{\infty} \frac{\ln x}{1-x^2} dx\).
\end{example}

\begin{example}
计算:  
\[\int_{0}^{1} \ln x \ln (1 - x) dx \]
\end{example}
\begin{example}
	计算积分:\[\int_{0}^{1} \frac{\ln (1 + x + x^2)}{x} dx.\]
\end{example}
\begin{example}
	计算积分:\[\int_{0}^{+\infty} \frac{x - [x] - \frac{1}{2}}{x} dx.\]
\end{example}
\section{积分的渐进展开}
\subsection{定积分定义}
\begin{example}
	设 \( f \) 在 \([0,1]\) 上可微,\(\left|f'(x)\right| < M\),证明:
\[
\left|\int_0^1 f(x) \, dx - \frac{1}{n} \sum_{k=1}^n f\left(\frac{k}{n}\right)\right| \leq \frac{M}{2n}.\]
\end{example}
\begin{example}
	设 \( f(x) \) 在 \([0,1]\) 可微且导数在 \([0,1]\) 上黎曼可积,则有:
\[
\lim_{n \to \infty} n \left( \frac{1}{n} \sum_{k=1}^n f\left(\frac{k}{n}\right) - \int_0^1 f(x) \, dx \right) = \frac{f(1) - f(0)}{2}.
\]
\end{example}
\begin{example}
	设 \( f \) 在区间 \([0,1]\) 上二阶可微,且 \( f'' \in R[0,1] \),求证:
\[
\lim_{n \to \infty} n^2 \left( \int_0^1 f(x) \, dx - \frac{1}{n} \sum_{k=1}^n f\left(\frac{2k-1}{2n}\right) \right) = \frac{1}{24} [f'(1) - f'(0)].
\]
\end{example}
\begin{example}
	设 \( f(x) \) 在 \([0,1]\) 上 \(2m\) 阶可微且 \(2m\) 阶导数在 \([0,1]\) 上勒贝格可积,则有:
\[
\frac{1}{n} \sum_{k=1}^n f\left(\frac{k}{n}\right) = \frac{f(1) - f(0)}{2n} + \int_0^1 f(x) \, dx + \sum_{k=1}^m \frac{\left(f^{(2k-1)}(1) - f^{(2k-1)}(0)\right)}{n^{2k}} b_{2k}(0) + o\left(\frac{1}{n^{2m}}\right).
\]
\end{example}
\begin{example}
	设 \( f(x) = \arctan x \),\( A \) 为常数,若
	\[
	B = \lim_{n \to \infty} \left( \sum_{k=1}^n f\left(\frac{k}{n}\right) - A n \right)
	\]
	存在,求 \( A \) 和 \( B \)。
\end{example}
\subsection{Euler-Maclaurin公式}
\begin{theorem}[0 阶情形]
	设 \( a, b \in \mathbb{Z} \),\( f(x) \in D^1[a, b] \),\( f'(x) \in L^1[a, b] \),则有:
\[
\sum_{k=a}^b f(k) = \int_a^b f(x) \, dx + \frac{f(a) + f(b)}{2} + \int_a^b \left( [x] - \frac{1}{2} \right) f'(x) \, dx.
\]

\end{theorem}
\begin{theorem}[一般情形(了解即可,本质上是分部积分)]
	设 \( a, b, m \in \mathbb{Z} \),\( m \geq 2 \),\( f(x) \in D^m[a, b] \),\( f^{(m)}(x) \in L^1[a, b] \),则有:
	\[
	\sum_{k=a}^b f(k) = \int_a^b f(x) \, dx + \frac{f(b) + f(a)}{2} + \sum_{k=2}^m \frac{f^{(k-1)}(b) - f^{(k-1)}(a)}{k!} (b - a)^k + (-1)^{m+1} \int_a^b B_m(x) f^{(m)}(x) \, dx.
	\]
\end{theorem}
\begin{example}
	设 \( f \in C^2[0, h] \),则存在 \( \xi \in [0, h] \),使得:
\[
\int_0^h f(x) \, dx = \frac{h}{2} [f(0) + f(h)] - \frac{1}{12} f''(\xi) h^3.
\]
\end{example}
\begin{example}
	设 \( f \in C^4[0, h] \),则存在 \( \xi \in [0, h] \),使得:
\[
\int_0^h f(x) \, dx = \frac{h}{2} [f(0) + f(h)] - \frac{h^2}{12} [f'(h) - f'(0)] + \frac{1}{720} f^{(4)}(\xi) h^5.
\]
\end{example}
\textbf{利用 Euler-Maclaurin 公式,我们可以导出很多渐近展开:}
\begin{itemize}
    \item \(\sum_{k=1}^n \frac{1}{k} \sim \ln n + \frac{1}{2n} - \frac{1}{12n^2} + \frac{1}{120n^4} + \cdots\)
    \item \(\ln(n!) \approx n \ln n - n + \frac{1}{2} \ln(2\pi n) + \frac{1}{12n} - \frac{1}{360n^3} + \cdots\)
\end{itemize}
\section{积分不等式}
\subsection{Cauchy不等式}
\begin{example}[Cauchy不等式]
	设 \( f(x), g(x) \in R[a, b] \),则有:
\[
\left( \int_a^b f(x) g(x) \, dx \right)^2 \leq \left( \int_a^b f^2(x) \, dx \right) \left( \int_a^b g^2(x) \, dx \right).
\]
\end{example}
\begin{example}
	(第十一届全国大学生数学竞赛) 设 \( f(x) \in C[0,1] \),且 \( 1 \leq f(x) \leq 3 \),求证:
\[
1 \leq \int_0^1 f(x) \, dx \int_0^1 \frac{1}{f(x)} \, dx \leq \frac{4}{3}.
\]
\end{example}
\begin{example}
	设 \( f(x) \in C^1[a, b] \),\( f(a) = 0 \),求证:
\[
\int_a^b f^2(x) \, dx \leq \frac{(b - a)^2}{2} \int_a^b f'^2(x) \, dx.
\]
\end{example}
\begin{example}
	设 \( f(x): [0,1] \to \mathbb{R} \),且 \(\int_0^1 x f(x) \, dx = 0\),求证:
\[
\int_0^1 f^2(x) \, dx \geq 4 \left( \int_0^1 f(x) \, dx \right)^2.
\]
\end{example}
\begin{example}
	(2024 厦门大学数学夏令营) 设 \( f(x) \in C[a, b] \),\( f(a) = 0 \),求证:
\[
\int_a^b f^2(x) \, dx \leq \frac{(b - a)^2}{2} \int_a^b [f'(x)]^2 \, dx - \frac{1}{2} \int_a^b [f'(x)]^2 (x - a)^2 \, dx.
\]
\end{example}
\begin{example}
	已知 \( f(x) \geq 0 \),\( f(x) \in C[a, b] \),\(\int_a^b f(x) \, dx = 1\),\( k \) 为任意实数,求证:
    \[
    \left( \int_a^b f(x) \cos kx \, dx \right)^2 + \left( \int_a^b f(x) \sin kx \, dx \right)^2 \leq 1.
    \]
\end{example}
\begin{example}
	设 \( f(x) \in C^1[a, b] \),\( f(a) = f(b) = 0 \),求证:
    \[
    \int_a^b f^2(x) \, dx \leq \frac{(b - a)^2}{4} \int_a^b f'^2(x) \, dx.
    \]
\end{example}
\begin{example}
	设 \( f(x, y) \in C[a, b] \),求证:
    \[
    \iint_{D} e^{f(x) - f(y)} \, dx \, dy \geq (a - b)^2, \quad D = [a, b] \times [a, b].
    \]
\end{example}
\subsection{Jensen不等式}
\begin{example}[Jensen不等式]
	设 \( f(x) \in R[a, b] \),且 \( m \leq f(x) \leq M \),\(\phi(x)\) 为 \([m, M]\) 上的连续下凸函数,则有:
\[
\phi\left( \frac{1}{b - a} \int_a^b f(x) \, dx \right) \leq \frac{1}{b - a} \int_a^b \phi(f(x)) \, dx.
\]
\end{example}
\begin{example}
	设 \( f(x) \in C[0,1] \),\(\forall x, y \in [0,1]\),\( f\left( \frac{x + y}{2} \right) \leq \frac{f(x) + f(y)}{2} \),求证:
\[
\int_0^1 f(x) \, dx \leq f\left( \frac{1}{2} \right).
\]
\end{example}
\begin{example}
	(2023 中科院提前批) 设函数 \( f(x) \) 在 \([a, b]\) 上二阶可导,且 \( f''(x) > 0 \),求证:
\begin{itemize}
    \item[(1)] \( f\left( \frac{a + b}{2} \right) \leq \frac{1}{b - a} \int_a^b f(x) \, dx \);
    \item[(2)] 若 \( f(x) \leq 0 \),\( x \in [a, b] \),则有:
    \[
    f(x) \geq \frac{2}{b - a} \int_a^b f(x) \, dx.
    \]
\end{itemize}
\end{example}
\begin{example}
	(Hardmard 不等式) 设 \( f(x) \in C^2[a, b] \),\( f''(x) > 0 \),求证:
\[
f\left( \frac{a + b}{2} \right) \leq \frac{1}{b - a} \int_a^b f(x) \, dx \leq \frac{f(a) + f(b)}{2}.
\]
\end{example}
\begin{example}
	求证:设 \( f(x) \in C[0,1] \),\( f(x) > 0 \),则有:
    \[
    \ln \int_0^1 f(x) \, dx \geq \int_0^1 \ln f(x) \, dx.
    \]
\end{example}
\begin{example}
	设 \( f(x) \) 是 \([0,1]\) 上非负连续的凹函数,且 \( f(0) = 1 \),求证:
    \[
    2 \int_0^1 f(x) \, dx + \frac{1}{12} \leq \left( \int_0^1 f(x) \, dx \right)^2.
    \]
\end{example}
\subsection{Chebyshev不等式}
\begin{example}[Chebyshev不等式]
	设 \( f(x), g(x) \in C[a, b] \),且 \( f(x), g(x) \) 在 \([a, b]\) 上单调性一致,求证:
\[
\int_a^b f(x) \, dx \int_a^b g(x) \, dx \leq (b - a) \int_a^b f(x) g(x) \, dx.
\]
\end{example}
\begin{remark}
	若 \( f(x), g(x) \) 单调性不一致,则上述不等式反号。
\end{remark}
\begin{example}
	求证:
\[
\int_0^{\frac{1}{2}} \frac{\sin x}{1 + x^2} \, dx \leq \int_{\frac{1}{2}}^1 \frac{\cos x}{1 + x^2} \, dx.
\]
\end{example}
\begin{example}
	证明:
\[
\int_0^1 \frac{\sin x}{1 + x^2} \, dx \leq \int_0^1 \frac{\cos x}{1 + x^2} \, dx.
\]
\end{example}
\begin{example}
	证明 Chebyshev 的一般形式,即:若 \( f(x), g(x), p(x) \in C[a, b] \),且 \(\forall x \in [a, b]\),\( p(x) \geq 0 \),\( f(x), g(x) \) 的单调性一致,则有:
\[
\int_a^b p(x) f(x) \, dx \int_a^b p(x) g(x) \, dx \leq \int_a^b p(x) \, dx \int_a^b p(x) f(x) g(x) \, dx.
\]
\end{example}
\begin{example}
	设 \( f(x) \in C[0,1] \) 且单调递增,求证:
\[
\frac{\int_0^1 x f^2(x) \, dx}{\int_0^1 x f(x) \, dx} \geq \frac{\int_0^1 f^2(x) \, dx}{\int_0^1 f(x) \, dx}.
\]
\end{example}
\subsection{Opial不等式}
\begin{example}[Opial不等式]
	设 \( f(x) \in C^1[0, a] \) 且 \( f(0) = 0 \),求证:
\[
\int_0^a |f(x) f'(x)| \, dx \leq \frac{a}{2} \int_0^a f'^2(x) \, dx.
\]
\end{example}
\begin{example}
	设 \( f(x) \in C^1[0, a] \) 且 \( f(0) = f(a) = 0 \),求证:
\[
\int_0^a |f(x) f'(x)| \, dx \leq \frac{a}{4} \int_0^a f'^2(x) \, dx.
\]
\end{example}
\subsection{Young不等式}
\begin{example}
	设 \( f(x) \in C[0, c] \) (\( c > 0 \)) 且严格递增,若 \( f(0) = 0 \) 且 \( a \in [0, c] \),\( b \in [0, f(c)] \),则:
\[
\int_0^a f(x) \, dx + \int_0^b f^{-1}(x) \, dx \geq ab.
\]
\end{example}
\subsection{单调性方法}
\begin{example}
	设 \( f(x) \in C[0,1] \) 且单调递减,求证:\(\forall a \in (0,1)\),
\[
\int_0^a f(x) \, dx \geq a \int_0^1 f(x) \, dx.
\]
\end{example}
\begin{example}
	设 \( f(x) \in C[0,1] \),且 \( 0 \leq f(x) \leq x \),求证:
\[
\int_0^1 x^2 f(x) \, dx \geq \left( \int_0^1 f(x) \, dx \right)^2.
\]
\end{example}
\subsection{中值定理}
\begin{example}
	设 \( f(x) \in C^2[0,1] \),求证:
\[
\max_{x \in [0,1]} |f'(x)| \leq |f(1) - f(0)| + \int_0^1 |f''(x)| \, dx.
\]
\end{example}
\begin{example}
	设 \( f(x) \in C^1[0,2] \),\(|f'(x)| \leq 1\),\( f(0) = f(2) = 1 \),求证:
\[
1 \leq \int_0^2 f(x) \, dx \leq 3.
\]
\end{example}
\chapter{反常积分}
\section{反常积分收散性判断}
\begin{example}
	设 \( f > 0 \) 内可积,若  
\[
\lim_{x \to +\infty} \frac{\ln f(x)}{\ln x} = p, \quad \text{则:}
\]

\[
\int f(x) dx 
\begin{cases} 
\text{收敛, } & -\infty < p < -1 \\ 
\text{发散, } & -1 < p < +\infty.
\end{cases}
\]
\end{example}
\begin{example}
	若 \( f \in C^1[0,+\infty) \) 且 \( f(0) > 0 \), \( f'(x) > 0 \). 若  
\[
\int_0^\infty \frac{1}{f(x)+f'(x)} dx < \infty, \quad \text{证明:}
\]

\[
\int_0^\infty \frac{1}{f(x)} dx < \infty.
\]
\end{example}
\begin{example}
	设 \( f \in C[0,+\infty) \), \( f(x) > 0 \),  
\[
\int_{0}^{\infty} \frac{1}{f(x)} dx < \infty,
\]
求证:  
\[
\lim_{X \to \infty} \frac{1}{X} \int_{0}^{X} f(x) dx = +\infty.
\]
\end{example}
\begin{example}
	设 \( f \in D[n, +\infty) \),\(\lim_{x \to +\infty} f'(x) = +\infty\) 且 \( f \) 严格递增,求证:  
\[
\int_0^{+\infty} \sin f(x) dx
\]  
收敛。
\end{example}
\begin{example}
	设 \( a > 0 \),\( f \) 在 \([a, +\infty)\) 上平方可积,证明:  
\[
\int_a^{+\infty} \frac{f(x)}{x} dx
\]  
收敛。
\end{example}
\begin{example}
	设 \( f \) 在 \([a, +\infty)\) (\( a > 1 \)) 上内闭可积,且已知  
\[
\int_a^{+\infty} x f(x) dx
\]  
收敛,求证:  
\[
\int_a^{+\infty} f(x) dx
\]  
收敛。
\end{example}
\begin{example}
	讨论广义积分 
\[
\int_{0}^{\infty} \frac{\sin x}{x^p} dx \quad ( p > 0 )
\] 
的敛散性。对于收敛的情况还要判断是条件收敛还是绝对收敛。

设 \( p > 0 \),证明广义积分 
\[
\int_{0}^{\infty} \frac{\sin x}{x^p + \sin x} dx
\] 
在 \( 0 < p \leq \frac{1}{2} \) 时发散,在 \(\frac{1}{2} < p \leq 1\) 时条件收敛,在 \( p > 1 \) 时绝对收敛。
\end{example}
\begin{example}
	已知 \( f(x) \) 在 \([0, +\infty)\) 单调、导数连续,且  
\[
\lim_{x \to +\infty} f(x) = 0,
\] 
证明:
\[
\int_{0}^{\infty} f(x) \sin x dx
\] 
绝对收敛。
\end{example}
\section{反常积分特殊性质}
对于正则函数 \(\sum_{n=0}^{\infty}\) 而言,若级数收敛,则 \(\lim_{n \to \infty} a_n = 0\)。但是这对于反常积分未必成立,常常需要增加适当的条件。否则,\(\lim_{n \to \infty} f(n)\) 可能不存在任何非绝对的平方项!
\begin{example}
	判断“求积分 \(\int_{-\infty}^{+\infty} \frac{dx}{1 + x^2 + y^2}\)”的函数性质。
\end{example}
\begin{example}
	设反常积分 \(\int_{a}^{\infty} f(x) dx\) 收敛,且 \(\lim_{x \to +\infty} f(x)\) 有意义,则它一定为 0。
\end{example}
\begin{example}
	
	若无穷限积分 \(\int_{a}^{\infty} f(x) dx\) 收敛,且 \(f\) 单调,则有 \(\lim_{x \to +\infty} x f(x) = 0\)。

\end{example}
\begin{example}
	反常积分 \(\int_a^{+\infty} f(x) dx\) 收敛,且被积函数 \(f\) 在 \([a, +\infty)\) 上一致连续,则 \(\lim_{x \to +\infty} f(x) = 0\)。

\end{example}
\begin{remark}
	在无穷限积分 \( \int f(x) dx \) 收敛时,若 \( f(x) \) 不连续,则 \( f(x) \) 不成立。若 \( f(x) \) 不成立,则 \( f(x) \) 不成立。若 \( f(x) \) 不成立,则 \( f(x) \) 不成立。
	\end{remark}
\begin{example}
	设 \( f(x) \) 为正态的整数,且被积函数 \( f \) 在 \([a,+\infty)\) 上一致连续,则
\[
f(x) = \frac{1}{2} \int_0^a f(t) dt
\]
在 \([a,+\infty)\) 上一致连续。
\end{example}
\begin{example}
	已知无穷积分 
\[
\int_{a}^{\infty} f(x) \, dx
\] 
收敛且 \( xf \) 单调,证明 
\[
\lim_{x \to +\infty} xf(x)\ln x = 0.
\]
\end{example}
\begin{example}
	设函数 \( f(x) \) 连续且 
\[
\int_{0}^{+\infty} f(x) \, dx
\] 
收敛,则一定存在一个趋于 \( +\infty \) 的数列 \(\{x_n\}_{n=1}^{+\infty} \subset [0, +\infty)\),使得 \(\lim_{n \to +\infty} f(x_n) = 0\).

\end{example}
\begin{example}
	(2021 人大预推免) 若 \( f(x) \in C[0, +\infty) \),且 \( f(x) \) 的广义积分 
\[
\int_0^\infty f(x) dx
\] 
收敛。

\begin{enumerate}
\item $\lim_{x \to \infty} f(x)$ 是否一定存在?不存在。请举反例:存在,请给出理由。
\item 若 (1) 中的极限存在,是多少?此时 \( f(x) \) 在 \((0, +\infty)\) 上是否一致收敛?
\end{enumerate}
\end{example}
\begin{example}
	(2021 厦大夏令营) 设函数 \( f \in L^1(\mathbb{R}) \),正项级数 
\[
\sum_{n=1}^\infty a_n
\] 
收敛。求证: 
\[
\lim_{n \to \infty} f\left(\frac{x}{a_n}\right) = 0, \text{ a.e. } x \in \mathbb{R}.
\]
\end{example}
\chapter{级数}
\section{级数的计算}
\subsection{凑已知函数}
\begin{example}
	对 $|x| < 1$,计算
\[
\sum_{n=0}^{\infty} \frac{4n^2 + 4n + 3}{2n+1}x^n.
\]
\end{example}
\begin{example}
	(2024 人大预推免) 判断
\[
\sum_{n=1}^{\infty} \left( \frac{1}{n^2} - (-1)^n \frac{1}{n} \right)
\]
是否收敛?若收敛,求其和。
\end{example}
\begin{example}
	令 $a_n = 1 - \frac{1}{2} + \cdots + \frac{(-1)^{n-1}}{n} - \ln 2$, $n = 1, 2, \cdots$,求
\[
\sum_{n=1}^{\infty} a_n
\]
的和。
\end{example}
\begin{example}
	解决以下问题:
\begin{enumerate}
    \item 设 $a_1 \in (0, 1)$, $a_{n+1} = \sqrt{\frac{1 + a_n}{2}}$, $n = 1, 2, \cdots$,求 $\lim_{n \to \infty} a_1 a_2 \cdots a_n$。
    \item 对 $|x| < 1$, 计算 $\sum_{k=1}^{\infty} \frac{1}{2^k} \tan \frac{x}{2^k}$。
\end{enumerate}
\end{example}
\subsection{利用幂级数}
\begin{example}
	计算 $\sum_{n=1}^{\infty} \left(1 + \frac{1}{2} + \cdots + \frac{1}{n}\right) x^n$.
	\end{example}
	
	\begin{example}
	(2023 北师大夏令营) 确定级数 $S(x) = \sum_{n=1}^{\infty} \frac{x^n}{n}$ 的收敛域,并计算:
	\[
	S(x) + S(1-x) + \ln x \cdot \ln(1-x)
	\]
	\end{example}
	
	\begin{example}
	(2023 浙大夏令营) 计算 $\sum_{n=0}^{\infty} \frac{1}{(2n)!}$.
	\end{example}

	\begin{example}
		利用幂级数的性质,求下列级数的和:
		\begin{enumerate}
			\item $\sum_{n=1}^{\infty} \frac{n(n+2)}{4^{n+1}}$;
			\item $\sum_{n=0}^{\infty} \frac{(n+1)^2}{2^n}$;
			\item $\sum_{n=0}^{\infty} (-1)^n \frac{1}{3^n (2n+1)}$;
			\item $\sum_{n=2}^{\infty} (-1)^n \frac{1}{2^n (n^2 - 1)}$;
			\item $\sum_{n=0}^{\infty} (-1)^n \frac{2^{n+1}}{n!}$.
		\end{enumerate}
	\end{example}
\subsection{特殊方法}
\begin{example}
	求证:
\[
\sum_{n=1}^{\infty} \frac{1}{(2n-1)^2} = \frac{\pi^2}{8}.
\]
\end{example}
\begin{example}
	设 $f \in C^1[0,1]$, $f(x) \geq 0$,证明下述级数收敛且求值
\[
\sum_{n=1}^{\infty} (-1)^{n-1} \int_0^1 x^n f(x) dx.
\]
\end{example}
\begin{example}
	计算
\[
\sum_{n=1}^{\infty} \left( \ln 2 - \sum_{k=1}^n \frac{(-1)^{k-1}}{k} \right).
\]
\end{example}
\begin{example}
	计算
\[
\sum_{n=1}^{\infty} \left( \ln 2 - \frac{1}{n+1} - \frac{1}{n+2} - \cdots - \frac{1}{2n} \right).
\]          
\end{example}
\section{敛散性判断}
\subsection{数项级数}
\begin{example}
	设正项级数 $\sum_{n=1}^{\infty} a_n$ 收敛,$0 < \alpha, \beta < 1$,$\alpha + \beta > 1$,求证:级数 $\sum_{n=1}^{\infty} \frac{a_n^{\alpha}}{n^{\beta}}$ 收敛。
	\end{example}
	
	\begin{example}
	设正项数列 $\{a_n\}$ 单调减少,则 $\lim_{n \to \infty} a_n = 0$ 的充要条件是正项级数 $\sum_{n=1}^{\infty} \left(1 - \frac{a_{n+1}}{a_n}\right)$ 发散。
	\end{example}
	
	该命题有如下等价形式:
	
	\begin{itemize}
		\item 设 $\{a_n\}$ 为单调增加的正数数列,则该数列与级数 $\sum_{n=1}^{\infty} \left(1 - \frac{a_n}{a_{n+1}}\right)$ 同敛散。(2023 华东师大夏令营)
		\item 若正项级数 $\sum_{n=1}^{\infty} a_n$ 的部分和数列为 $\{S_n\}$,则 $\sum_{n=1}^{\infty} \frac{a_n}{S_n}$ 与 $\sum_{n=1}^{\infty} \frac{a_n}{S_n}$ 同敛散。
	\end{itemize}
	
	\begin{example}
	(Kummer 判别法) 证明以下命题:
	\begin{itemize}
		\item 正项级数 $\sum_{n=1}^{\infty} a_n$ 收敛的充分必要条件是存在正数数列 $\{b_n\}$ 和正数 $\delta$,使得当 $n$ 充分大时有 $b_n \cdot \frac{a_n}{a_{n+1}} - b_{n+1} \geq \delta > 0$;
		\item 正项级数 $\sum_{n=1}^{\infty} a_n$ 发散的充分必要条件是存在发散的正项级数 $\sum_{n=1}^{\infty} \frac{1}{b_n}$,当 $n$ 充分大时有 $b_n \cdot \frac{a_n}{a_{n+1}} - b_{n+1} \leq 0$。
	\end{itemize}
	\end{example}
	
	\begin{example}
	证明以下两个命题:
	\begin{itemize}
		\item 对于给定的收敛正项级数 $\sum_{n=1}^{\infty} a_n$,一定存在收敛正项级数 $\sum_{n=1}^{\infty} b_n$,使得 $\lim_{n \to \infty} \frac{a_n}{b_n} = 0$;
		\item 对于给定的发散正项级数 $\sum_{n=1}^{\infty} a_n$,一定存在发散正项级数 $\sum_{n=1}^{\infty} b_n$,使得 $\lim_{n \to \infty} \frac{b_n}{a_n} = 0$。
	\end{itemize}
	\end{example}
	
	注:上述这两个命题表明:对于比较判别法而言,不论是判别出收敛还是发散,都不可能存在万能的比较级数!
	
	\begin{example}
	正项级数 $\sum_{n=1}^{\infty} \frac{1}{a_n}$ 收敛,证明:级数 $\sum_{n=1}^{\infty} \frac{n}{a_1 + a_2 + \cdots + a_n}$ 也收敛。
	\end{example}
	
	\begin{example}
	(2024 北师大夏令营) 设 $x > 0$,证明:级数 $\sum_{n=1}^{\infty} \frac{x^n}{(1+x)(1+x^2)\cdots(1+x^n)}$ 收敛。
	\end{example}
	
	\begin{example}
	(2023 华东师范夏令营) 设 $\{a_n\}$ 单调增加,讨论 $\sum_{n=1}^{\infty} \left(1 - \frac{a_n}{a_{n+1}}\right)$ 的收敛性。
	\end{example}
	
	\begin{example}
	(2023 中山夏令营) 设正项级数 $\sum_{n=1}^{\infty} u_n$ 收敛,证明:级数 $\sum_{n=1}^{\infty} (e^{u_n} - 1)$ 也收敛。若 $\sum_{n=1}^{\infty} u_n$ 不是正项级数,问能否由它收敛推知级数 $\sum_{n=1}^{\infty} (e^{u_n} - 1)$ 也收敛?
	\end{example}
	
	\begin{example}
	(2023 中科院提前批) 证明:函数项级数 $\sum_{n=1}^{\infty} (-1)^n x^n (1-x)$ 在 $[0,1]$ 上绝对收敛且一致收敛,但不绝对一致收敛。
	\end{example}
	
	\begin{example}
	(2023 山大夏令营) 设 $S_n(x) = nx(1-x^2)^n, x \in [0,1]$,证明 $S_n(x)$ 收敛于 $S(x) = 0$,但不一致收敛于 $S(x) = 0$。
	\end{example}
	
	\begin{example}
	(2023 浙大直博) 求证:$f(x) = \sum_{n=1}^{+\infty} \frac{1}{n^x}$ 在 $(1, +\infty)$ 上连续可微。
	\end{example}
	
	\begin{example}
	(2024 上交夏令营) 设函数项级数 $S(x) = \sum_{n=0}^{\infty} \frac{(-1)^n e^{-nx}}{n+x^2}$,试讨论以下问题:
	\begin{itemize}
		\item 求函数 $S(x)$ 的定义域。
		\item 讨论 $S(x)$ 在定义域内的连续性。
		\item $\lim_{x \to +\infty} S(x)$ 是否存在?给出证明。
	\end{itemize}
	\end{example}
	
\subsection{函数项级数}
对于函数项级数来说,主要的一致收敛性判别法有:Cauchy 一致收敛准则 Weierstrass 判别法(也称为 M 判别法、强级数判别法和优级数判别法),以及通过 Abel 变换得到的 Abel 判别法和 Dirichlet 判别法。

Cauchy 一致收敛准则是充分必要条件,但应用时往往需要较复杂的技巧。

Weierstrass 判别法只对绝对一致收敛的情况有效,但是可举例说明,存在绝对一致收敛的函数项级数的例子,使得 Weierstrass 判别法失效,但由于它将问题归结为正项级数的收敛性判别,使用方便,因此有广泛的应用。

Abel 判别法和 Dirichlet 判别法都是函数项级数一致收敛的充分必要条件,其证明与广义积分和数项级数的同名判别法类似。

在讨论函数项级数 $\sum_{n=1}^{\infty} u_n(x)$ 的一致收敛性时比较容易的一类情况是能够得到其部分和函数列 $\{S_n(x)\}$ 的紧凑表达式,这时问题就转化为函数列的一致收敛性问题(不难看出可以将已经列举的各种判别法改述为关于函数列的一致收敛性判别法)。如果同时还能得到级数的和函数,即函数列 $\{S_n(x)\}$ 的极限函数 $S(x)$ 的表达式,则就有上确界判别法:函数列 $\{S_n(x)\}$ 在数集 $E$ 上一致收敛于 $S(x)$ 的充分必要条件是
\[
\lim_{n \to \infty} \sup_{x \in E} \{|S_n(x) - S(x)|\} = 0.
\]

\begin{example}
设连续函数列 $\{f_n(x)\}$ 在 $[a,b]$ 上一致收敛于 $f(x)$,而 $g(x) \in C(-\infty, +\infty)$,求证:$\{g(f_n(x))\}$ 在 $[a,b]$ 上一致收敛于 $g(f(x))$。
\end{example}

\begin{example}
(2023 中科院提前批) 证明:函数项级数 $\sum_{n=1}^{\infty} (-1)^n x^n (1-x)$ 在 $[0,1]$ 上绝对收敛且一致收敛,但不绝对一致收敛。
\end{example}

\begin{example}
(2024 浙大夏令营) 设可微函数列 $\{f_n(x)\}$ 在 $[0,1]$ 上处处收敛,且 $\{f'_n(x)\}$ 在 $[0,1]$ 上一致有界。证明 $\{f_n(x)\}$ 在 $[0,1]$ 上一致收敛。
\end{example}

\begin{example}
(2023 山大夏令营) 设 $S_n(x) = nx(1-x^2)^n, x \in [0,1]$,证明 $S_n(x)$ 收敛于 $S(x) = 0$,但不一致收敛于 $S(x) = 0$。
\end{example}

\begin{example}
(2023 浙大直博) 求证:$f(x) = \sum_{n=1}^{+\infty} \frac{1}{n^x}$ 在 $(1, +\infty)$ 上连续可微。
\end{example}

\begin{example}
(2024 上交夏令营) 设函数项级数 $S(x) = \sum_{n=0}^{\infty} \frac{(-1)^n e^{-nx}}{n+x^2}$,试讨论以下问题:
\begin{itemize}
    \item 求函数 $S(x)$ 的定义域。
    \item 讨论 $S(x)$ 在定义域内的连续性。
    \item $\lim_{x \to +\infty} S(x)$ 是否存在?给出证明。
\end{itemize}
\end{example}

\begin{example}
(2023 首师夏令营) 设 $u_n(x) \geq 0$ 在 $[a,b]$ 上连续,而 $\sum_{n=1}^{\infty} u_n(x)$ 在 $[a,b]$ 上收敛于连续函数 $f(x)$,则 $\sum_{n=1}^{\infty} u_n(x)$ 在 $[a,b]$ 上一致收敛于 $f(x)$。
\end{example}

\begin{example}
(2024 浙大夏令营) 设可微函数列 $\{f_n(x)\}$ 在 $[0,1]$ 上处处收敛,且 $\{f'_n(x)\}$ 在 $[0,1]$ 上一致有界,证明 $\{f_n(x)\}$ 在 $[0,1]$ 上一致收敛。
\end{example}
\section{综合运用}
\subsection{级数证明}
\begin{example}
	设 $a_n > -1$ 且 $\sum_{n=1}^{\infty} a_n$ 收敛,则 $\prod_{n=1}^{\infty} (1 + a_n)$ 收敛当且仅当 $\sum_{n=1}^{\infty} a_n^2$ 收敛。
	\end{example}
	
	\begin{example}
	已知正项级数 $\sum_{n=1}^{\infty} a_n$ 收敛,证明 $\sum_{n=1}^{\infty} \sqrt{a_n a_{n+1}}$ 也收敛,反之正确吗?
	\end{example}
	
	\begin{example}
	设 $a_n > 0$,且满足 $\{a_n - a_{n+1}\}$ 递减,$\sum_{n=1}^{\infty} a_n$ 收敛,求证:$\lim_{n \to \infty} \left( \frac{1}{a_{n+1}} - \frac{1}{a_n} \right) = +\infty$。
	\end{example}
	
	\begin{example}
	设 $a_n > 0$,$\sum_{n=1}^{\infty} |b_n| < \infty$ 且 $\frac{a_n}{a_{n+1}} \leq 1 + \frac{1}{n} + \frac{1}{n \ln n} + b_n$,$n = 1, 2, \ldots$,证明:$\sum_{n=1}^{\infty} a_n$ 发散。
	\end{example}
	
	\begin{example}
	设 $\{a_n\}$ 递减到 0,证明:$\sum_{n=1}^{\infty} n(a_n - a_{n+1})$ 收敛的充要条件是 $\sum_{n=1}^{\infty} a_n$ 收敛。
	\end{example}
\subsection{幂级数的阶}
下面讨论幂级数系数的阶与和函数的阶之间的关系。

\begin{example}
证明以下命题:
\begin{itemize}
    \item 设 $f(x) = \sum_{n=0}^{\infty} a_n x^n$,$g(x) = \sum_{n=0}^{\infty} b_n x^n$,$x \in (-1, 1)$,满足 $b_n > 0$,$\lim_{n \to \infty} \frac{a_n}{b_n} = 0$,$\lim_{x \to 1^-} g(x) = +\infty$,则 $\lim_{x \to 1^-} \frac{f(x)}{g(x)} = 0$。
    \item 设 $f(x) = \sum_{n=0}^{\infty} a_n x^n$,$g(x) = \sum_{n=0}^{\infty} b_n x^n$,$x \in (-1, 1)$,满足 $b_n > 0$,$\lim_{n \to \infty} \frac{a_n}{b_n} = 1$,$\lim_{x \to 1^-} g(x) = +\infty$,则 $\lim_{x \to 1^-} \frac{f(x)}{g(x)} = 1$。
    \item 设 $f(x) = \sum_{n=0}^{\infty} a_n x^n$,$g(x) = \sum_{n=0}^{\infty} b_n x^n$,$x \in \mathbb{R}$,满足 $b_n > 0$,$\lim_{n \to \infty} \frac{a_n}{b_n} = 1$,则 $\lim_{x \to +\infty} \frac{f(x)}{g(x)} = 1$。
\end{itemize}
\end{example}
\begin{example}
	给定 $\{a_n\}_{n=0}^{\infty} \subset \mathbb{R}$,设 $f(x) = \sum_{n=0}^{\infty} a_n x^n$,$x \in (-1, 1)$,若 $f(x) = \sum_{n=0}^{\infty} a_n x^n$,$x \in (-1, 1)$,证明:$\lim_{x \to 1^-} f(x) = +\infty$(或 $-\infty$),并指出 $\lim_{n \to \infty} \left| \sum_{k=0}^n a_k \right| = \infty$ 推不出 $\lim_{x \to 1^-} |f(x)| = \infty$。
	\end{example}
	
	\begin{example}
	(2024 复旦数学夏令营) 已知 $\lim_{n \to \infty} a_n = 1$,求 $\lim_{x \to 1^-} \frac{\sum_{n=0}^{\infty} a_n x^n}{\ln(1 - x)}$。
	\end{example}
\subsection{Tauber定理}
\begin{example}
	若 $\sum_{n=0}^{\infty} a_n x^n$ 在 $(-1, 1)$ 收敛且 $\lim_{x \to 1^-} \sum_{n=0}^{\infty} a_n x^n = A$,若 $a_n \geq 0$,则 $\sum_{n=0}^{\infty} a_n = A$。
	\end{example}
	
	\begin{example}
	(Tauber 定理) 设级数 $\sum_{n=0}^{\infty} a_n x^n$ 收敛半径为 1,左极限 $\lim_{x \to 1^-} \sum_{n=0}^{\infty} a_n x^n = A$ 存在,且 $\lim_{n \to \infty} n a_n = 0$,则 $\sum_{n=0}^{\infty} a_n = A$。
	\end{example}
\chapter{多元微积分}
\section{连续性和可微性}
\begin{example}
	设 $f(x, y) = \sqrt{|xy|}$,求证:
\begin{itemize}
\item $f(x, y)$ 在 $(0, 0)$ 点连续;
\item $\frac{\partial f}{\partial x}(0, 0)$ 和 $\frac{\partial f}{\partial y}(0, 0)$ 都存在;
\item $f(x, y)$ 在 $(0, 0)$ 点不可微。
\end{itemize}
\begin{example}
	(2024 同济夏令营改编) 设 $f(x, y) = |x - y| \phi(x, y)$,其中 $\phi(x, y)$ 在点 $(0, 0)$ 的一个邻域上有定义,要求给出函数 $\phi(x, y)$ 加上适当的条件,使得:
\begin{itemize}
\item $f(x, y)$ 在点 $(0, 0)$ 连续;
\item $f(x, y)$ 在点 $(0, 0)$ 存在偏导数;
\item $f(x, y)$ 在点 $(0, 0)$ 可微。
\end{itemize}
\end{example}
\end{example}
\begin{example}
	设 $f(x, y) = \begin{cases} \frac{\sqrt{|x - y|}}{x^2 + y^2} \sin(x^2 + y^2), & x^2 + y^2 \neq 0, \\ 0, & x^2 + y^2 = 0, \end{cases}$
讨论:
\begin{enumerate}
\item $f(x, y)$ 在点 $(0, 0)$ 是否连续?
\item $f(x, y)$ 在点 $(0, 0)$ 是否可微?
\end{enumerate}
\end{example}
\section{曲线积分}

\section{重积分计算}
\begin{example}
	(2021 复旦夏令营) 计算积分 $\iint_{0\le x + y \leq \pi} \ln|\sin(x - y)| \, dx \, dy$。
\end{example}
\begin{example}
	(2024 中科院夏令营) 计算积分:
\[
\iint_{D} (x + y) \, dx \, dy,
\]
其中 $D = \{ (x, y) \mid (x - 1)^2 + (y - 1)^2 \leq 2, y \geq x \}$。

\end{example}
\begin{example}
	(2023 中科院推免生) 计算四重积分:
\[
\int_{Q(x)\leq 1} e^{Q(x)} \, dx \, dy \, dz \, dt.
\]
\end{example}
\begin{example}
	(2024 人大预推免) 计算 $n$ 维超球体的体积。
\end{example}
\section{曲线积分}

\subsection{第一型曲线积分}
第一型曲线积分是对弧长的积分,与曲线的方向无关。即:
\[
I = \int_{AB} f(x) \, ds = \int_{BA} f(x) \, ds.
\]

若 $\Gamma$ 是 $\mathbb{R}^n$ 中简单可求长的曲线,且 $f(x)$ 在 $\Gamma$ 上连续,则 $f(x)$ 在 $\Gamma$ 上的第一型曲线积分存在。

若 $\Gamma$ 是 $\mathbb{R}^n$ 中逐段光滑的简单曲线,且有参数表示:
\[
\Gamma: x = x(t) \in \mathbb{R}^n, \quad a \leq t \leq b,
\]

则弧长微分:
\[
ds = |x'(t)| dt = \left( \sum_{i=1}^{n} (x_i'(t))^2 \right)^{1/2} dt,
\]
则:
\[
\int_{\Gamma} f(x) \, ds = \int_{a}^{b} f(x(t)) |x'(t)| dt.
\]


\begin{example}
	求 $I = \oint_{C} x^2 \, ds$,其中 $C$ 为:
\[
\begin{cases}
x^2 + y^2 + z^2 = R^2, \\
x + y + z = 0.
\end{cases}
\]
\end{example}

\begin{example}
	设椭圆 $\frac{x^2}{4} + \frac{y^2}{3} = 1$ 的周长为 $a > 0$,计算 $\oint_{\frac{x^2}{4} + \frac{y^2}{3} = 1} (2xy + 3x^2 + 4y^2) \, ds$。

\end{example}

\begin{example}
	设 $a > 0$,$\Gamma: \left\{ \begin{array}{l} x^2 + y^2 + z^2 = a^2, \\ x + y + z = 0 \end{array} \right.$,试计算 $\int_{\Gamma} (1 + x)^2 \, ds$。
\end{example}






\subsection{第二型曲线积分}
第二型曲线积分的形式为:
\[
I = \int_{\Gamma} P(x, y, z) \, dx + Q(x, y, z) \, dy + R(x, y, z) \, dz
\]

若逐段光滑的有向曲线 $\Gamma$ 有参数表示:
\begin{align*}
x &= x(t), \\
y &= y(t), \\
z &= z(t), \quad a \leq t \leq b,
\end{align*}

则:
\[
\int_{\Gamma} P \, dx + Q \, dy + R \, dz = \int_{a}^{b} \left[ P(x(t), y(t), z(t)) x'(t) + Q(x(t), y(t), z(t)) y'(t) + R(x(t), y(t), z(t)) z'(t) \right] dt.
\]

\begin{example}
	计算积分 $I = \oint_{C} (x^2 + 2xy) \, dy$,其中 $C$ 表示逆时针方向的上半椭圆 $\frac{x^2}{a^2} + \frac{y^2}{b^2} = 1$。

\end{example}

\begin{example}
	设 $C$ 为抛物线 $2x = \pi y^2$ 自 $(0, 0)$ 到 $\left( \frac{\pi}{2}, 1 \right)$ 的弧段,求积分:
\[
I = \int_{C} \left( 2xy^3 - y^2 \cos x \right) dx + \left( 1 - 2y \sin x + 3x^2 y^2 \right) dy.
\]
\end{example}
\begin{example}
	(2024 北师夏令营改编) 设 $f(x) \in C^1(-\infty, +\infty)$,$L$ 是上半平面 $(y > 0)$ 内的有向分段光滑曲线,起点为 $(a, b)$,终点为 $(c, d)$。计算:
\[
I = \int_{L} \frac{1}{y} [1 + y^2 f(xy)] dx + \frac{x}{y^2} [y^2 f(xy) - 1] dy.
\]
\end{example}

\begin{example}
	计算 $I = \oint_{x^2 + y^2 = 1} \frac{x \, dy - y \, dx}{x^2 + y^2}$,方向取逆时针。
\end{example}

\begin{example}
	求简单正定向光滑闭曲线 $L$ 使得积分 $\oint_{L} (y^3 - y) \, dx - 2x^3 \, dy$ 值最大,并计算最大值。
\end{example}
\begin{example}
	设 $D$ 是平面内的单连通区域,$\omega = P dx + Q dy$,其中 $P, Q$ 在 $D$ 上有连续的偏导数,则以下结论等价:
\begin{itemize}
\item 对 $D$ 内的任意一条闭曲线 $C$,有 $\oint_{C} \omega = 0$。
\item 对 $D$ 内的任一条路径 $C$,积分 $\int_{C} \omega$ 仅与 $C$ 的起点和终点有关,而与所沿的路径无关。
\item 在 $D$ 内处处成立 $\frac{\partial P}{\partial y} = \frac{\partial Q}{\partial x}$。
\item 存在函数 $\varphi(x, y)$,使得在 $D$ 内成立 $d\varphi(x, y) = P(x, y) dx + Q(x, y) dy$。
\end{itemize}
\end{example}

\begin{example}
	计算积分 $I = \oint_{C} \frac{e^y}{x^2 + y^2} \left[ (x \sin x + y \cos x) dx + (y \sin x - x \cos x) dy \right]$,其中 $C: x^2 + y^2 = 1$,取逆时针方向。
\end{example}




\end{document}