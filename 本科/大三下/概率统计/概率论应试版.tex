\documentclass[lang=cn,10pt,thmcnt=section]{elegantbook}
\usepackage{graphicx}
\usepackage{float}
\usepackage{esint}
\usepackage{mathtools}
\usepackage{tikz}
\usetikzlibrary{arrows.meta, positioning}
\usetikzlibrary{automata, positioning, arrows}
\title{概率统计应试版}



\author{Huang}
\date{\today}




\setcounter{tocdepth}{3}


\cover{cover.jpg}

% 本文档命令
\usepackage{array}
\newcommand{\ccr}[1]{\makecell{{\color{#1}\rule{1cm}{1cm}}}}

% 修改标题页的橙色带
% \definecolor{customcolor}{RGB}{32,178,170}
% \colorlet{coverlinecolor}{customcolor}

\begin{document}
	
	\maketitle
	\frontmatter
	
	\tableofcontents
	
	\mainmatter
	\chapter{看前须知}
	本书内容主要为老师课上内容的整理,旨在帮助同学期末的复习,其中作业的答案由claude-3.7-sonnet生成,如果觉得本书内容还行,请给作者点个star
	\href{https://github.com/Hilbert-beinghappy/-}{GitHub - Hilbert-beinghappy/-: 本科阶段上过课课程的笔记}。

	\chapter{概率论入门}
	定义事件 $A = \{x \mid x > 1\}$ 可以看出,事件的本质是集合,故事件之间的关系与集合之间的关系相似。

定义事件 $B = \{x \mid x > 0\}$


显然 $A \subseteq B$,从集合的角度看,$A$ 包含于 $B$,从概率论事件的角度看,若 $A$ 发生,则一定导致 $B$ 发生。


定义:$A$ 与 $B$ 相等,$A = B \Leftrightarrow A \subseteq B \text{ 且 } B \subseteq A$

$A$ 发生一定导致 $B$ 发生,$B$ 发生一定导致 $A$ 发生,此即概率论意义的事件相等。

\[
A \cup B : \quad \text{$A$, $B$ 中只要有一个发生,则 } A \cup B \text{ 发生。}
\]

\[
\bigcup_{k=1}^{n} A_k : \quad \text{$A_1, \ldots, A_n$ 中只要有一个发生,则 } \bigcup_{k=1}^{n} A_k \text{ 发生。}
\]

\[
A \cap B : \quad \text{必须 $A$, $B$ 同时发生,$A \cap B$ 才发生。}
\]

\[
\bigcap_{k=1}^{n} A_k : \quad \text{$A_1, \ldots, A_n$ 同时发生,$\bigcap_{k=1}^{n} A_k$ 发生。}
\]

\[
A - B : \quad \text{$A$ 发生,$B$ 不发生 时,$(A - B)$ 发生。}
\]
以后一般记作 $A \overline{B}$

定义:$A \cap B = \emptyset$,即 $A$, $B$ 不能同时发生,称 $A$, $B$ 互斥。

\[
AB = \emptyset
\]

定义全空间 $\Omega$,表示全部元素的集合。


若 $A \cup B = \Omega$,$A \cap B = \emptyset$,即 $A$, $B$ 不能同时发生,且 $A$, $B$ 至少有一个得发生,称 $A$, $B$ 为对立事件,记 $B = \overline{A}$。

背:
\begin{enumerate}
    \item $A \cup B = B \cup A$, $A \cap B = B \cap A$
    \item $A \cup (B \cup C) = (A \cup B) \cup C$,$A \cap (B \cap C) = (A \cap B) \cap C$
    \item $A \cup (B \cap C) = (A \cup B) \cap (A \cup C)$
    \item $A \cap (B \cup C) = (A \cap B) \cup (A \cap C)$
    \item $\overline{A \cup B} = \overline{A} \cap \overline{B}$,$\overline{A \cap B} = \overline{A} \cup \overline{B}$
\end{enumerate}


\end{document}


