\documentclass[lang=cn,10pt,thmcnt=section]{elegantbook}
\usepackage{graphicx}
\usepackage{float}
\usepackage{esint}
\usepackage{mathtools}
\usepackage{tikz}
\title{随机过程}



\author{Huang}
\date{\today}




\setcounter{tocdepth}{3}


\cover{cover.jpg}

% 本文档命令
\usepackage{array}
\newcommand{\ccr}[1]{\makecell{{\color{#1}\rule{1cm}{1cm}}}}

% 修改标题页的橙色带
% \definecolor{customcolor}{RGB}{32,178,170}
% \colorlet{coverlinecolor}{customcolor}

\begin{document}
	
	\maketitle
	\frontmatter
	
	\tableofcontents
	
	\mainmatter
	\chapter{离散时间马尔可夫链}
	\section{马氏链及其转移概率}
	\subsection{马氏链及其转移概率}
	有一类随机过程,它具备所谓的“无后效性”(Markov 性),
即要确定过程将来的状态,知道它此刻的情况就足够了,并不需要
对它以往状况的认识,这类过程称为Markov过程.我们将介绍离散
时间的Markov链(简称马氏链).

本章假定:$T = \{0, 1, \cdots\}$,$S = \{0, 1, 2, \cdots, N\}$(或者 $S := \mathbb{N}$),所有r.v.均定义在同一个概率空间上。用$i, j$表示$S$中元素。
\begin{definition}[离散时间马尔可夫链]
	随机过程$\{X_n, n = 0, 1, 2, \cdots\}$称为\textit{Markov链},若它只取有限或可列个值(若不另外说明,以非负整数集$\{0, 1, 2, \cdots\}$来表示),并且对任意的$n \geq 0$,及任意状态$i, j, i_0, i_1, \cdots, i_{n-1}$,有
\begin{equation}
P\{X_{n+1} = j | X_0 = i_0, X_1 = i_1, \cdots, X_{n-1} = i_{n-1}, X_n = i\} = P\{X_{n+1} = j | X_n = i\}
\end{equation}
其中$X_n = i$表示过程在时刻$n$处于状态$i$,称为$S$。式(1.1)刻画了\textit{Markov链}的特性,称为\textit{Markov性},或\textit{马氏性},或\textit{无记忆性}。
\end{definition}

\begin{definition}[转移概率]
	设 $\{X_n, n = 0, 1, \ldots\}$ 为马氏链,称
\[ P\{X_{n+1} = j | X_n = i\} =: p_{ij}(n) \]
为 $n$ 时刻的一步转移概率。若它与 $n$ 无关,则记作 $p_{ij}$,并称相应的马氏链为齐次的或时齐的。令 $P = (p_{ij})$,称 $P$ 为齐次马氏链的转移概率矩阵,简称为转移矩阵,$p_{ij}$ 为一步转移概率。我们只考虑齐次马氏链。

\end{definition}

\[ P\{X_{n+1} = j | X_0 = i_0, X_1 = i_1, \cdots, X_{n-1} = i_{n-1}, X_n = i\} \]
\[ = P\{X_{n+1} = j | X_n = i\} \text{马尔可夫性} \]
\[ = P\{X_1 = j | X_0 = i\} \text{齐次} \]

设 $\{X_n, n = 0, 1, \ldots\}$ 是齐次马氏链,具有转移矩阵 $P = (p_{ij})$,则有
\[ p_{ij} \geq 0 \quad \forall i, j \in S \text{且} \]
\[ \sum_{j \in S} p_{ij} = \sum_{j \in S} P(X_1 = j | X_0 = i) = P(X_1 \in S | X_0 = i) = 1 \quad \forall i \in S. \]

\begin{definition}[随机矩阵]
	称矩阵 $A = (a_{ij})_{S \times S}$ 为随机矩阵,若 $a_{ij} \geq 0 (\forall i, j \in S)$,且 $\sum_{j \in S} a_{ij} = 1 (\forall i \in S)$。
\end{definition}

由该定义知转移矩阵是随机矩阵。

\begin{example}[赌徒破产问题]
	系统的状态是 \(0 \sim n\),反映赌博者在赌博期间拥有的钱数,当他输光或拥有钱数为 \(n\) 时,赌博停止,否则他将持续赌博。每次以概率 \(p\) 赢得1,以概率 \(q = 1 - p\) 输掉1。则每个时刻,该赌徒拥有的钱数服从马尔可夫性吗?能否写出对应的转移概率矩阵?
\end{example}
\begin{proof}
	这个系统的转移矩阵为

\[
P = 
\begin{array}{c|ccccccccc}
 & 0 & 1 & 2 & 3 & \cdots & n-2 & n-1 & n \\
\hline
0 & 1 & 0 & 0 & 0 & \cdots & 0 & 0 & 0 \\
1 & q & 0 & p & 0 & \cdots & 0 & 0 & 0 \\
2 & 0 & q & 0 & p & \cdots & 0 & 0 & 0 \\
3 & 0 & 0 & q & 0 & \cdots & 0 & 0 & 0 \\
\vdots & \vdots & \vdots & \vdots & \vdots & \vdots & \vdots & \vdots & \vdots \\
n-2 & 0 & 0 & 0 & 0 & \cdots & 0 & p & 0 \\
n-1 & 0 & 0 & 0 & 0 & \cdots & q & 0 & p \\
n & 0 & 0 & 0 & 0 & \cdots & 0 & 0 & 1 \\
\end{array}
\]
\end{proof}
\begin{example}[简单随机游动]
	质点在直线的整数点上作简单随机游动:质点到达某个状态后,下次向右移动一步的概率是$p$,向左移动一步的概率是$q$,在原地不动的概率为$r$,且$p + q + r = 1$。$X_0$表示初始状态,$X_n$表示质点在时间$n$的状态。假设初始状态与每次移动相互独立。则$\{X_n\}$是马氏链,
\end{example}
\begin{proof}
	\[
\left\{
\begin{aligned}
p_{i, i-1} &= P(X_{n+1} = i - 1 | X_n = i) = q \\
p_{i, i+1} &= P(X_{n+1} = i + 1 | X_n = i) = p \\
p_{i, i} &= P(X_{n+1} = i | X_n = i) = r
\end{aligned}
\right.
\]
\end{proof}
\begin{example}
	设有一蚂蚁在下图爬行,当两个结点相临时,蚂蚁将爬向它临近的一点,并且爬向任何一个邻居的概率是相同的。

\begin{figure}[h]
    \centering
    \begin{tikzpicture}
        \node (1) at (0,0) {1};
        \node (2) at (2,2) {2};
        \node (3) at (2,0) {3};
        \node (4) at (2,-2) {4};
        \node (5) at (4,1) {5};
        \node (6) at (6,0) {6};
        
        \draw (1) -- (2);
        \draw (1) -- (3);
        \draw (2) -- (3);
        \draw (3) -- (4);
        \draw (3) -- (5);
        \draw (5) -- (6);
    \end{tikzpicture}
\end{figure}
\end{example}
\begin{proof}
	此Markov链的转移矩阵为

\[
\mathbf{P} = \begin{pmatrix}
0 & \frac{1}{2} & \frac{1}{2} & 0 & 0 & 0 \\
\frac{1}{2} & 0 & \frac{1}{2} & 0 & 0 & 0 \\
\frac{1}{4} & \frac{1}{4} & 0 & \frac{1}{4} & \frac{1}{4} & 0 \\
0 & 0 & 1 & 0 & 0 & 0 \\
0 & 0 & \frac{1}{2} & 0 & 0 & \frac{1}{2} \\
0 & 0 & 0 & 0 & 1 & 0
\end{pmatrix}
\]
\end{proof}

\begin{theorem}
	设 $A, B, C$ 为三个随机事件,则 $P(BC|A) = P(B|A)P(C|AB)$.
\end{theorem}
\begin{proof}
	\[
P(BC|A) = \frac{P(ABC)}{P(A)} = \frac{P(AB)P(ABC)}{P(A)P(AB)} = P(B|A)P(C|AB).
\]
\end{proof}
\begin{remark}
	令 $P(\cdot|A) := P_A$, 则应用乘法公式 $P(BC|A) = P_A(BC) = P_A(C|B) \cdot P_A(B) = P(C|BA)P(B|A)$.

\end{remark}
\begin{theorem}
	对于事件 $A, B, C$,当 $P(AB) > 0$,条件

\[
P(C|BA) = P(C|B),
\]

和条件

\[
P(AC|B) = P(A|B)P(C|B)
\]

等价。
\end{theorem}
\begin{proof}
	\[
\frac{P(ACB)}{P(B)} = \frac{P(AB)}{P(B)} \frac{P(BC)}{P(AB)} \text{ 可知 } \frac{P(ACB)}{P(AB)} = \frac{P(BC)}{P(B)}. \text{ 即 } P(C|BA) = P(C|B).
\]
\end{proof}
\begin{theorem}
	对于事件 $A, B, C$,当 $P(AB) > 0$,条件
\[ P(C|BA) = P(C|B), \]
和条件
\[ P(AC|B) = P(A|B)P(C|B) \]
等价。


\end{theorem}
马氏性的解释:

过去:$A = (X_0 = i_0, \ldots, X_{n-1} = i_{n-1})$,

现在:$B = (X_n = i_n)$,

将来:$C = (X_{n+1} = i_{n+1})$。

马氏性代表在已知现在的情况下,将来与过去无关。
\begin{theorem}
	设 $S$ 是马氏链 $\{X_n\}$ 的状态空间,则有
\begin{enumerate}
    \item 对任意的 $n, m \geq 1$ 有
    \[
    \begin{aligned}
    & P(X_{n+1} = i_{n+1}, \ldots, X_{n+m} = i_{n+m} | X_0 = i_0, \ldots, X_n = i) \\
    & = P(X_{n+1} = i_{n+1}, X_{n+2} = i_{n+2}, \ldots, X_{n+m} = i_{n+m} | X_n = i)
    \end{aligned}
    \]
    \item 对任意的 $n, m \geq 1$,以及 $C \subset S^m, A \subset S^n$ 有
    \[
    \begin{aligned}
    & P((X_{n+1}, X_{n+2}, \ldots, X_{n+m}) \in C | (X_0 \ldots, X_{n-1}) \in A, X_n = i) \\
    & = P((X_{n+1}, X_{n+2}, \ldots, X_{n+m}) \in C | X_n = i)
    \end{aligned}
    \]
    \item 对任意的 $k, m \geq 1$,以及 $t_0 < t_1 < \ldots < t_k < t_{k+1} < \ldots < t_{k+m}, i \in S, C \subset S^m, A \subset S^k$ 有
    \[
    \begin{aligned}
    & P((X_{t_{k+1}}, X_{t_{k+2}}, \ldots, X_{t_{k+m}}) \in C | (X_{t_0} \ldots, X_{t_{k-1}}) \in A, X_{t_k} = i) \\
    & = P((X_{t_{k+1}}, X_{t_{k+2}}, \ldots, X_{t_{k+m}}) \in C | X_{t_k} = i)
    \end{aligned}
    \]
\end{enumerate}
\end{theorem}
\begin{proof}
	(1)对 \(m\) 进行归纳证明。当 \(m = 1\) 时有马氏性即得。现在设对 \(m = k\) 成立,即已知
	\[
	\begin{aligned}
	& P(X_{n+1} = i_{n+1}, \ldots, X_{n+k} = i_{n+k} | X_0 = i_0, \ldots, X_n = i) \\
	& = P(X_{n+1} = i_{n+1}, X_{n+2} = i_{n+2}, \ldots, X_{n+k} = i_{n+k} | X_n = i)
	\end{aligned}
	\]
	
	\[
	\begin{aligned}
	& P(X_{n+1} = i_{n+1}, \ldots, X_{n+k+1} = i_{n+k+1} | X_0 = i_0, \ldots, X_n = i) \\
	& = P(X_{n+2} = i_{n+2}, \ldots, X_{n+k+1} = i_{n+k+1} | X_0 = i_0, \ldots, X_{n+1} = i_{n+1}) \\
	& \quad \cdot P(X_{n+1} = i_{n+1} | X_0 = i_0, \ldots, X_n = i) \\
	& = P(X_{n+2} = i_{n+2}, \ldots, X_{n+k+1} = i_{n+k+1} | X_{n+1} = i_{n+1}) \\
	& \quad \cdot P(X_{n+1} = i_{n+1} | X_n = i) \\
	& = P(X_{n+1} = i_{n+1}, X_{n+2} = i_{n+2}, \ldots, X_{n+k+1} = i_{n+k+1} | X_n = i)
	\end{aligned}
	\]
	
	\[
	\begin{aligned}
	& P(X_{n+2} = i_{n+2}, \ldots, X_{n+k+1} = i_{n+k+1} | X_n = i, X_{n+1} = i_{n+1}) \\
	& = \sum_{i_{n-1} \in S, \ldots, i_0 \in S} P(X_{n+k+1} = i_{n+k+1}, \ldots, X_{n+2} = i_{n+2}, X_{n+1} = i_{n+1}, \\
	& \quad X_n = i, X_{n-1} = i_{n-1}, \ldots, X_0 = i_0) / P(X_{n+1} = i_{n+1}, X_n = i) \\
	& = \sum_{i_{n-1} \in S, \ldots, i_0 \in S} P(X_{n+2} = i_{n+2}, \ldots, X_{n+k+1} = i_{n+k+1} | X_{n+1} = i_{n+1}, \\
	& \quad X_n = i, X_{n-1} = i_{n-1}, \ldots, X_0 = i_0) P(X_{n+1} = i_{n+1}, \\
	& \quad X_n = i, X_{n-1} = i_{n-1}, \ldots, X_0 = i_0) / P(X_{n+1} = i_{n+1}, X_n = i) \\
	& = \sum_{i_{n-1} \in S, \ldots, i_0 \in S} P(X_{n+2} = i_{n+2}, \ldots, X_{n+k+1} = i_{n+k+1} | X_{n+1} = i_{n+1}) \\
	& \quad P(X_{n+1} = i_{n+1}, X_n = i, X_{n-1} = i_{n-1}, \ldots, X_0 = i_0) \\
	& \quad / P(X_{n+1} = i_{n+1}, X_n = i) \\
	& = P(X_{n+2} = i_{n+2}, \ldots, X_{n+k+1} = i_{n+k+1} | X_{n+1} = i_{n+1})
	\end{aligned}
	\]

	(2)\[
\begin{aligned}
& P((X_{n+1}, \ldots, X_{n+m}) \in C | (X_0 \ldots, X_{n-1}) \in A, X_n = i) \\
& = \frac{P((X_{n+1}, \ldots, X_{n+m}) \in C, X_n = i, (X_{n-1} \ldots, X_0) \in A)}{P((X_0 \ldots, X_{n-1}) \in A, X_n = i)} \\
& = \sum_{(i_{n+1}, i_{n+2}, \ldots, i_{n+m}) \in C} \sum_{(i_{n-1}, \ldots, i_0) \in A} P(X_{n+m} = i_{n+m}, \ldots, X_{n+1} = i_{n+1}, \\
& \quad X_n = i, X_{n-1} = i_{n-1}, \ldots, X_0 = i_0) / P(X_n = i, (X_{n-1} \ldots, X_0) \in A) \\
& = \sum_{(i_{n+1}, i_{n+2}, \ldots, i_{n+m}) \in C} \sum_{(i_{n-1}, \ldots, i_0) \in A} P(X_{n+m} = i_{n+m}, \ldots, X_{n+1} = i_{n+1} | X_n = i, \\
& \quad X_{n-1} = i_{n-1}, \ldots, X_0 = i_0) \cdot P(X_n = i, X_{n-1} = i_{n-1}, \ldots, X_0 = i_0) \\
& \quad / P(X_n = i, (X_{n-1} \ldots, X_0) \in A) \\
& = \sum_{(i_{n+1}, i_{n+2}, \ldots, i_{n+m}) \in C} P(X_{n+m} = i_{n+m}, \ldots, X_{n+1} = i_{n+1} | X_n = i) \\
& \quad \cdot \sum_{(i_{n-1}, \ldots, i_0) \in A} P(X_n = i, X_{n-1} = i_{n-1}, \ldots, X_0 = i_0) \\
& \quad / P(X_n = i, (X_{n-1} \ldots, X_0) \in A) \\
& = P((X_{n+1}, X_{n+2}, \ldots, X_{n+m}) \in C | X_n = i)
\end{aligned}
\]

(3)用增补变量的方法:只举特殊情形证明,其余类似。证明 \(P((X_7, X_5) \in C | X_3 = i, X_1 \in A) = P((X_7, X_5) \in C | X_3 = i)\)

令 \(\tilde{C} := \{(i_7, i_6, i_5, i_4) | (i_7, i_5) \in C, i_6 \in S, i_4 \in S\}\),\(\tilde{A} := \{(i_2, i_1, i_0) | i_2 \in S, i_1 \in A, i_0 \in S\}\)

\[
\begin{aligned}
P((X_7, X_5) \in C | X_3 = i, X_1 \in A) \\
& = P((X_7, X_6, X_5, X_4) \in \tilde{C} | X_3 = i, (X_2, X_1, X_0) \in \tilde{A}) \\
& = P((X_7, X_6, X_5, X_4) \in \tilde{C} | X_3 = i) \\
& = P((X_7, X_5) \in C | X_3 = i)
\end{aligned}
\]
\end{proof}
\begin{remark}
	设 $S$ 是马氏链 $\{X_n\}$ 的状态空间,对任意的 $k, m \geq 1$,以及 $t_0 < t_1 < \ldots < t_k < t_{k+1} < \ldots < t_{k+m}$,$B \subset S$,$A \subset S^k$,$C \subset S^m$,有
\[
P((X_{t_{k+1}}, X_{t_{k+2}}, \ldots, X_{t_{k+m}}) \in C | X_{t_k} \in B, (X_{t_{k-1}}, \ldots, X_{t_0}) \in A) \neq P((X_{t_{k+1}}, X_{t_{k+2}}, \ldots, X_{t_{k+m}}) \in C | X_{t_k} \in B)
\]
\end{remark}
\begin{remark}
	看不懂也没关系,不影响后面的学习
\end{remark}
\begin{example}
	质点在直线的整数点上作简单随机游动:质点到达某个状态后,下次向右移动一步的概率是 $p$,向左移动一步的概率是 $q = 1 - p$。$X_0$ 表示初始状态,$X_n$ 表示质点在时间 $n$ 的状态。假设初始状态与每次移动相互独立。则 $\{X_n\}$ 是马氏链,
\[
\begin{cases}
p_{i, i-1} = P(X_{n+1} = i - 1 | X_n = i) = q \\
p_{i, i+1} = P(X_{n+1} = i + 1 | X_n = i) = p
\end{cases}
\]
设初始分布 $P(X_0 = 0) = P(X_0 = 2) = \frac{1}{2}$,$D := \{1, 3\}$。证明:当 $p \neq q$ 时,$P(X_2 = 2 | X_0 = 0, X_1 \in D) \neq P(X_2 = 2 | X_1 \in D)$.
\end{example}
\begin{proof}
	\[
\begin{aligned}
P(X_1 = 1) &= P(X_0 = 0)P(X_1 = 1 | X_0 = 0) + P(X_0 = 2)P(X_1 = 1 | X_0 = 2) = \frac{1}{2}p + \frac{1}{2}q = \frac{1}{2}, \\
P(X_1 = 3) &= P(X_1 = 3, X_0 = 2) + P(X_1 = 3, X_0 = 0) = P(X_0 = 2)P(X_1 = 3 | X_0 = 2) = \frac{1}{2}p.
\end{aligned}
\]

故
\[
\begin{aligned}
& P(X_2 = 2 | X_1 \in D) \\
& = \frac{P(X_2 = 2, X_1 \in D)}{P(X_1 \in D)} \\
& = \frac{P(X_2 = 2, X_1 = 1) + P(X_2 = 2, X_1 = 3)}{P(X_1 \in D)} \\
& = \frac{P(X_1 = 1)P(X_2 = 2 | X_1 = 1) + P(X_1 = 3)P(X_2 = 2 | X_1 = 3)}{P(X_1 = 1) + P(X_1 = 3)} \\
& = \frac{\frac{1}{2}p + \frac{1}{2}pq}{\frac{1}{2} + \frac{1}{2}p} = p\frac{1 + q}{1 + p}
\end{aligned}
\]

\[
\begin{aligned}
& P(X_2 = 2 | X_1 \in D, X_0 = 0) \\
& = \frac{P(X_2 = 2, X_1 \in D, X_0 = 0)}{P(X_1 \in D, X_0 = 0)} \\
& = \frac{P(X_2 = 2, X_1 = 1, X_0 = 0)}{P(X_1 \in D, X_0 = 0)} \\
& = \frac{P(X_2 = 2 | X_1 = 1, X_0 = 0)}{P(X_1 \in D, X_0 = 0)} = p
\end{aligned}
\]

$P(X_2 = 2 | X_0 = 0, X_1 \in D) = P(X_2 = 2 | X_0 = 0, X_1 = 1) = p$。当 $p \neq q$ 时,$P(X_2 = 2 | X_0 = 0, X_1 \in D) \neq P(X_2 = 2 | X_1 \in D)$.

\end{proof}
\begin{theorem}
	设随机过程 $\{X_n, n \geq 0\}$ 满足:
\begin{enumerate}
    \item $X_n = f(X_{n-1}, \xi_n) (n \geq 1)$,其中 $f: S \times S \rightarrow S$,且 $\xi_n$ 取值在 $S$ 上,
    \item $\{\xi_n, n \geq 1\}$ 为独立同分布随机变量,且 $X_0$ 与 $\{\xi_n, n \geq 1\}$ 也相互独立,
\end{enumerate}
则 $\{X_n, n \geq 0\}$ 是马尔可夫链,而且其一步转移概率为
\[
p_{ij} = P(f(i, \xi_1) = j).
\]
\end{theorem}
\begin{remark}
	这个定理讲的是如何生成一个马尔可夫链。
\end{remark}
\begin{example}
	质点在直线的整数点上作简单随机游动:质点到达某个状态后,下次向右移动一步的概率是 $p$,向左移动一步的概率是 $q = 1 - p$。$X_0$ 表示初始状态,$X_n$ 表示质点在时间 $n$ 的状态。假设初始状态与每次移动相互独立。证明 $\{X_n\}$ 是马氏链,且
\[
\begin{cases}
p_{i, i-1} = P(X_{n+1} = i - 1 | X_n = i) = q \\
p_{i, i+1} = P(X_{n+1} = i + 1 | X_n = i) = p
\end{cases}
\]

\end{example}
\begin{proof}
	令 $X_n$ 为质点在时刻 $n \geq 0$ 的位置,
\[
\xi_n := 
\begin{cases}
1 & \text{第 } n \text{ 次向右移动} \\
-1 & \text{第 } n \text{ 次向左移动}
\end{cases}
\]
$X_n = X_{n-1} + \xi_n$。

转移概率:
\[
p_{ij} = P(i + \xi_1 = j) = P(\xi_1 = j - i) = 
\begin{cases}
q & j = i - 1 \\
p & j = i + 1
\end{cases}
\]
\end{proof}
\begin{remark}
	这道例题就是讲述如何生成一个马尔可夫链。
\end{remark}

设 $\{X_n, n = 0, 1, \ldots\}$ 是齐次马氏链,具有转移矩阵 $P = (p_{ij})$,则有
\[ p_{ij} \geq 0 \quad \forall i, j \in S \text{且} \sum_{j \in S} p_{ij} = 1 \quad \forall i \in S. \]
\subsection{随机矩阵}
\begin{definition}[随机矩阵]
	称矩阵 $A = (a_{ij})_{S \times S}$ 为随机矩阵,若 $a_{ij} \geq 0 (\forall i, j \in S)$,且 $\sum_{j \in S} a_{ij} = 1 (\forall i \in S)$。
\end{definition}
\begin{remark}
	随机矩阵就是转移矩阵。
\end{remark}
特别地,记 $P^0 = I(S$ 上的单位矩阵),
\[
p_{ij}^{(0)} = \delta_{ij} = 
\begin{cases} 
1 & j = i \\
0 & j \neq i 
\end{cases}
\]
且
\[
p_{ij}^{(n)} := P(X_n = j | X_0 = i) = P(X_{n+m} = j | X_m = i) \quad \text{与 } m \text{ 无关! }
\]
表示从 $i$ 出发经 $n$ 步到达 $j$ 的概率。称 $P^{(n)} := (p_{ij}^{(n)})_{i,j \in S}$ 为 $\{X_n\}$ 的 $n$ 步转移概率矩阵。显然 $P^{(n)}$ 为随机矩阵。

\begin{theorem}[Chapman-Kolmogorov 方程]
	设 $\{X_n\}$ 是齐次马氏链,具有转移矩阵 $P$,则对任意的 $m, n \geq 0$,有
\[
p_{ij}^{(n+m)} = \sum_{k \in S} p_{ik}^{(n)} p_{kj}^{(m)} \quad \forall i, j \in S, m, n \geq 0.
\]

\[
P^{(n+m)} = P^{(m)} P^{(n)} = P^{n+m}, \text{其中 } P^{n+m} \text{表示矩阵 } P \text{的} n+m \text{次乘积}
\]
\end{theorem}
\begin{proof}
	\[
\begin{aligned}
p_{ij}^{(n+m)} &= P(X_{n+m} = j | X_0 = i) = P(X_{n+m} = j, X_n \in S | X_0 = i) \\
&= \sum_{k \in S} P(X_n = k, X_{n+m} = j | X_0 = i) \\
&= \sum_{k \in S} P(X_n = k | X_0 = i) P(X_{n+m} = j | X_0 = i, X_n = k) \\
&= \sum_{k \in S} p_{ik}^{(n)} p_{kj}^{(m)}.
\end{aligned}
\]

从状态 $i$ 出发经 $n+m$ 步到达 $j$ 的概率可以由转移矩阵 $P$ 及归纳法证明。
\end{proof}
\begin{theorem}
	\[
p_{ij}^{(n+m)} = \sum_{k \in S} p_{ik}^{(n)} p_{kj}^{(m)} \quad \forall i, j \in S, m, n \geq 0.
\]

\[
P^{(n+m)} = P^{(m)} P^{(n)} = P^{n+m}, \text{其中 } P^{n+m} \text{表示矩阵} P \text{的} n+m \text{次乘积}
\]
\end{theorem}
\begin{corollary}
	对任意的正整数 $n, m, k, n_1, n_2, \cdots, n_k$ 和状态 $i, j, l$, 有
\begin{enumerate}
    \item $p_{ij}^{(n+m)} \geq p_{il}^{(n)} p_{lj}^{(m)}$;
    \item $p_{ii}^{(n+m+k)} \geq p_{ij}^{(n)} p_{jl}^{(k)} p_{li}^{(m)}$;
    \item $p_{ii}^{(n_1+n_2+\cdots+n_k)} \geq p_{ii}^{(n_1)} p_{ii}^{(n_2)} \cdots p_{ii}^{(n_k)}$;
    \item $p_{ii}^{(nk)} \geq (p_{ii}^{(n)})^k$.
\end{enumerate}

\end{corollary}
\begin{proof}
	\[
P^{(n)} = P^{(n-1)}P = P^{(n-2)}P \cdot P = \cdots = P^n.
\]
\end{proof}
\begin{example}
	已知马氏链的一步转移概率矩阵为
\[
\mathbf{P} = \begin{pmatrix}
0.7 & 0.3 \\
0.4 & 0.6
\end{pmatrix}
\]
求 $P^{(2)}$, $P^{(4)}$。
\end{example}
\begin{proof}
	\[
\mathbf{P}^{(2)} = \mathbf{P} \cdot \mathbf{P} = \begin{pmatrix}
0.61 & 0.39 \\
0.52 & 0.48
\end{pmatrix}
\]
\[
\mathbf{P}^{(4)} = \mathbf{P}^{(2)} \cdot \mathbf{P}^{(2)} = \begin{pmatrix}
0.5749 & 0.4251 \\
0.5668 & 0.4332
\end{pmatrix}
\]
\end{proof}
\begin{example}
	系统的状态是 $0 \sim n$,反映赌博者在赌博期间拥有的钱数,当他输光或拥有钱数为 $n$ 时,赌博停止,否则他将持续赌博。每次以概率 $p$ 赢得 1,以概率 $q = 1 - p$ 输掉 1。这个系统的转移矩阵为
\[
\mathbf{P} = \begin{pmatrix}
1 & 0 & 0 & 0 & \cdots & 0 & 0 & 0 \\
q & 0 & p & 0 & \cdots & 0 & 0 & 0 \\
0 & q & 0 & p & \cdots & 0 & 0 & 0 \\
\vdots & \vdots & \vdots & \vdots & \ddots & \vdots & \vdots & \vdots \\
0 & 0 & 0 & 0 & \cdots & q & 0 & p \\
0 & 0 & 0 & 0 & \cdots & 0 & 0 & 1
\end{pmatrix}_{(n+1) \times (n+1)}
\]
$n = 3$, $p = q = \frac{1}{2}$。赌博者从 2 元赌金开始赌博,求解他经过 4 次赌博之后输光的概率。
\end{example}
\begin{proof}
	这个概率为 $p_{20}^{(4)} = P\{X_4 = 0 | X_0 = 2\}$,一步转移矩阵为
\[
\mathbf{P} = \begin{pmatrix}
1 & 0 & 0 & 0 \\
\frac{1}{2} & 0 & \frac{1}{2} & 0 \\
0 & \frac{1}{2} & 0 & \frac{1}{2} \\
0 & 0 & 0 & 1
\end{pmatrix}
\]
利用矩阵乘法得
\[
\mathbf{P}^{(4)} = \mathbf{P}^4 = \begin{pmatrix}
1 & 0 & 0 & 0 \\
\frac{5}{16} & \frac{1}{16} & 0 & \frac{5}{16} \\
\frac{5}{16} & 0 & \frac{1}{16} & \frac{5}{8} \\
0 & 0 & 0 & 1
\end{pmatrix}
\]
故 $p_{20}^{(4)} = \frac{5}{16}$ ($\mathbf{P}^{(4)}$ 中第 3 行第 1 列)。

\end{proof}
\begin{example}
	甲乙两人进行某种比赛,设每局甲胜的概率是 $p$,乙胜的概率是 $q$,和局的概率是 $r$,$p + q + r = 1$。设每局比赛后,胜者记 “+1” 分,负者记 “-1” 分,和局不记分,且当两人中有一人获得 2 分时结束比赛。以 $X_n$ 表示比赛至第 $n$ 局时甲获得的分数,则 $\{X_n, n = 0, 1,2,\cdots\}$ 为时齐 Markov 链,求在甲获得 1分的情况下,不超过两局可结束比赛的概率。


\end{example}
\begin{proof}
	$\{X_n, n = 0, 1, 2,\cdots\}$ 的一步转移概率矩阵为
\[
\mathbf{P} = 
\begin{pmatrix}
-2 & 1 & 0 & 0 & 0 & 0 & 0 \\
-1 & q & r & p & 0 & 0 & 0 \\
0 & 0 & q & r & p & 0 & 0 \\
1 & 0 & 0 & 0 & q & r & p \\
2 & 0 & 0 & 0 & 0 & 0 & 1
\end{pmatrix}
\]

两步转移概率矩阵为
\[
\mathbf{P}^{(2)} = \mathbf{P} \cdot \mathbf{P} = 
\begin{pmatrix}
1 & 0 & 0 & 0 & 0 & 0 \\
q + rq & r^2 + pq & 2pr & p^2 & 0 & 0 \\
q^2 & 2rq & r^2 + 2pq & 2pr & p^2 & 0 \\
0 & q^2 & 2qr & r^2 + pq & p + pr & 0 \\
0 & 0 & 0 & 0 & 0 & 1
\end{pmatrix}
\]

故在甲获得 1分的情况下,不超过两局可结束比赛的概率为
\[
p_{1,2}^{(2)} + p_{1,-2}^{(2)} = p + pr
\]
\end{proof}

\begin{example}
	质点在直线的整数点上作简单随机游动:质点到达某个状态后,下次向右移动一步的概率是 $p$,向左移动一步的概率是 $q = 1 - p$。$X_0$ 表示初始状态,$X_n$ 表示质点在时间 $n$ 的状态。假设初始状态与每次移动相互独立。则 $\{X_n\}$ 是马氏链,求 $P^{(n)}$。

\[
\begin{cases}
p_{i, i-1} = P(X_{n+1} = i - 1 | X_n = i) = q \\
p_{i, i+1} = P(X_{n+1} = i + 1 | X_n = i) = p
\end{cases}
\]


\end{example}
\begin{proof}
从 $i$ 经过 $n$ 步到 $j$,其中向左走了 $x$ 步,向右走了 $y$ 步。则有
\[
\begin{cases}
x + y = n \\
i - x + y = j
\end{cases}
\]
故 $x = \frac{n - (j - i)}{2}$ 且 $y = \frac{n + j - i}{2}$,其中 $n + j - i$ 必须是偶数 $(n + j - i) + [n - (j - i)] = 2n$。

\[
p_{ij}^{(n)} = 
\begin{cases} 
C_n^{(n+j-i)/2} p^{(n+j-i)/2} q^{(n-j+i)/2} & n + j - i \text{为偶数} \\
0 & n + j - i \text{为奇数}
\end{cases}
\]
\end{proof}

回顾 $p_{ij}^{(n)} = P(X_n = j | X_0 = i)$ 与 $P(X_n = j)$

\[
\pi_i(n) = P(X_n = i), i \in \mathcal{S},
\]
\[
\pi(n) = (\pi_i(n), i \in \mathcal{S}).
\]
即 $\pi(n)$ 表示 $n$ 时刻 $X_n$ 的概率分布,称 $\pi(0) := (\pi_i(0), i \in \mathcal{S})$ 为马氏链 $\{X_n, n = 0, 1, \ldots\}$ 的初始分布。

对任意的 $n \geq 0$,
\[
\sum_{i \in \mathcal{S}} \pi_i(n) = P(X_n \in \mathcal{S}) = 1.
\]

\end{document}