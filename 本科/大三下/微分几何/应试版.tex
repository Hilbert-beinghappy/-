\documentclass[lang=cn,10pt,thmcnt=section]{elegantbook}
\usepackage{graphicx}
\usepackage{float}
\usepackage{esint}
\usepackage{mathtools}
\usepackage{tikz}
\usetikzlibrary{arrows.meta, positioning}
\usetikzlibrary{automata, positioning, arrows}
\title{微分几何应试版教材}



\author{Huang}
\date{\today}




\setcounter{tocdepth}{3}


\cover{cover.jpg}

% 本文档命令
\usepackage{array}
\newcommand{\ccr}[1]{\makecell{{\color{#1}\rule{1cm}{1cm}}}}

% 修改标题页的橙色带
% \definecolor{customcolor}{RGB}{32,178,170}
% \colorlet{coverlinecolor}{customcolor}

\begin{document}
	
	\maketitle
	\frontmatter
	
	\tableofcontents
	
	\mainmatter
	\chapter{曲线论}
	\section{求弧长参数/弧长}
	求弧长参数,三步走
	\begin{enumerate}
		\item 求$\overrightarrow{r} '$
		\item 求$\left\lvert \overrightarrow{r} '\right\rvert $
		\item 求积分$\int \left\lvert \overrightarrow{r} '\right\rvert \,dt $
	\end{enumerate}

	\begin{example}
		将圆柱螺线 $\vec{r} = (a\cos t, a\sin t, bt)$ 化为弧长参数
	\end{example}
	\begin{proof}
		三步走
	\begin{enumerate}
		\item \textbf{求导数 $\overrightarrow{r} '$}:\\
		对参数 $t$ 求导得:
		\[
			\vec{r}\,'(t) = \left( -a\sin t,\ a\cos t,\ b \right)
		\]
		
		\item \textbf{求模长 $\left\lvert \overrightarrow{r} '\right\rvert $}:\\
		计算导数的模:
		\[
			\left\lvert \vec{r}\,'(t) \right\rvert = \sqrt{(-a\sin t)^2 + (a\cos t)^2 + b^2} = \sqrt{a^2(\sin^2 t + \cos^2 t) + b^2} = \sqrt{a^2 + b^2}
		\]
		
		\item \textbf{求弧长参数 $s$}:\\
		积分模长得到弧长:
		\[
			s(t) = \int_{0}^{t} \left\lvert \vec{r}\,'(\tau) \right\rvert \,d\tau = \int_{0}^{t} \sqrt{a^2 + b^2} \,d\tau = \sqrt{a^2 + b^2} \cdot t
		\]
		解得 $t = \dfrac{s}{\sqrt{a^2 + b^2}}$,代入原参数方程得:
		\[
			\vec{r}(s) = \left( a\cos\left(\dfrac{s}{\sqrt{a^2 + b^2}}\right),\ a\sin\left(\dfrac{s}{\sqrt{a^2 + b^2}}\right),\ \dfrac{bs}{\sqrt{a^2 + b^2}} \right)
		\]
	\end{enumerate}
	\end{proof}
	\begin{example}
		求双曲螺线 $\vec{r} = (a\cosh t, a\sinh t, at)$ 的弧长参数表示。
	\end{example}
	\begin{proof}
		三步走
		\begin{enumerate}
		\item \textbf{求导数 $\overrightarrow{r} '$}:\\
		对参数 $t$ 求导得:
		\[
			\vec{r}\,'(t) = \left( a\sinh t,\ a\cosh t,\ a \right)
		\]
		
		\item \textbf{求模长 $\left\lvert \overrightarrow{r} '\right\rvert $}:\\
		利用双曲恒等式 $\cosh^2 t - \sinh^2 t = 1$:
		\[
			\left\lvert \vec{r}\,'(t) \right\rvert = \sqrt{a^2\sinh^2 t + a^2\cosh^2 t + a^2} = a\sqrt{2\cosh^2 t} = a\sqrt{2} \cosh t
		\]
		
		\item \textbf{求弧长参数 $s$}:\\
		积分模长得到弧长:
		\[
			s(t) = \int_{0}^{t} a\sqrt{2} \cosh \tau \,d\tau = a\sqrt{2} \sinh t
		\]
		反解得 $t = \sinh^{-1}\left(\dfrac{s}{a\sqrt{2}}\right)$,代入原方程得弧长参数化表示:
		\[
			\vec{r}(s) = \left( \sqrt{a^2 + \dfrac{s^2}{2}},\ \dfrac{s}{\sqrt{2}},\ a \sinh^{-1}\left(\dfrac{s}{a\sqrt{2}}\right) \right)
		\]
	\end{enumerate}
	\end{proof}
	\begin{example}
		求旋轮线 $\vec{r} = (a(t - \sin t), a(1 - \cos t))$ 在 $0 \leq t \leq 2\pi$ 时的弧长。
	\end{example}
	\begin{proof}
	三步走:
	\begin{enumerate}
		\item \textbf{求导数 $\overrightarrow{r} '$}:\\
		对参数 $t$ 求导得:
		\[
			\vec{r}\,'(t) = \left( a(1 - \cos t),\ a\sin t \right)
		\]
		
		\item \textbf{求模长 $\left\lvert \overrightarrow{r} '\right\rvert $}:\\
		利用三角恒等式 $1 - \cos t = 2\sin^2(t/2)$:
		\[
			\left\lvert \vec{r}\,'(t) \right\rvert = a\sqrt{(1 - \cos t)^2 + \sin^2 t} = 2a \sin(t/2)
		\]
		
		\item \textbf{求总弧长}:\\
		积分模长得总弧长:
		\[
			L = \int_{0}^{2\pi} 2a \sin(t/2) \,dt = \left[ -4a \cos(t/2) \right]_{0}^{2\pi} = 8a
		\]
	\end{enumerate}
\end{proof}

\section{切平面和法平面}
切线方程的坐标表示:
\[
\frac{x - x(t_0)}{x'(t_0)} = \frac{y - y(t_0)}{y'(t_0)} = \frac{z - z(t_0)}{z'(t_0)}
\]

法平面方程的坐标表示:
\[
[x - x(t_0)] x'(t_0) + [y - y(t_0)] y'(t_0) + [z - z(t_0)] z'(t_0) = 0
\]


\begin{example}
	求圆柱螺线 $\vec{r}(t) = \{a\cos t, a\sin t, bt\}$ 在 $t = \frac{\pi}{3}$ 处的切线方程
\end{example}
\begin{proof}
	三步走:
	\begin{enumerate}
		\item \textbf{求导向量}:
		\[
			\vec{r}\,'(t) = \{-a\sin t,\ a\cos t,\ b\}
		\]
		
		\item \textbf{求 $t = \frac{\pi}{3}$ 处的点和切向量}:
		\[
			\vec{r}\left(\frac{\pi}{3}\right) = \left(\frac{a}{2},\ \frac{a\sqrt{3}}{2},\ \frac{b\pi}{3}\right), \quad 
			\vec{r}\,'\left(\frac{\pi}{3}\right) = \left(-\frac{a\sqrt{3}}{2},\ \frac{a}{2},\ b\right)
		\]
		
		\item \textbf{写切线方程}:
		\[
			\frac{x - \frac{a}{2}}{-\frac{a\sqrt{3}}{2}} = \frac{y - \frac{a\sqrt{3}}{2}}{\frac{a}{2}} = \frac{z - \frac{b\pi}{3}}{b}
		\]
	\end{enumerate}
\end{proof}
\begin{example}
	对于圆柱螺线 $x = \cos t, y = \sin t, z = t$,求它在 $(1, 0, 0)$ 时的切线与法平面。
\end{example}
\begin{proof}
	三步走:
	\begin{enumerate}
		\item \textbf{确定参数 $t$}:\\
		由 $x=1$ 得 $\cos t=1 \Rightarrow t=0$,对应点 $(1,0,0)$
		
		\item \textbf{求导向量}:
		\[
			\vec{r}\,'(t) = \{-\sin t,\ \cos t,\ 1\} \quad \Rightarrow \quad 
			\vec{r}\,'(0) = (0,1,1)
		\]
		
		\item \textbf{切线方程}:
		\[
			\frac{x-1}{0} = \frac{y}{1} = \frac{z}{1} \quad \text{或} \quad 
			\begin{cases} 
				x = 1 \\ 
				y = z 
			\end{cases}
		\]
		
		\item \textbf{法平面方程}:\\
		法向量为 $\vec{r}\,'(0) = (0,1,1)$,方程为:
		\[
			0(x-1) + 1(y-0) + 1(z-0) = 0 \quad \Rightarrow \quad y + z = 0
		\]
	\end{enumerate}
\end{proof}
\begin{example}
	求三次挠曲线 $\vec{r} = \{at, bt^2, ct^3\}$ 在 $t_0$ 时的切线和法平面。
\end{example}
\begin{proof}
	三步走:
	\begin{enumerate}
		\item \textbf{求导向量}:
		\[
			\vec{r}\,'(t) = \{a, 2bt, 3ct^2\}
		\]
		
		\item \textbf{求 $t_0$ 处的点和切向量}:
		\[
			\vec{r}(t_0) = \{at_0, bt_0^2, ct_0^3\}, \quad 
			\vec{r}\,'(t_0) = \{a, 2bt_0, 3ct_0^2\}
		\]
		
		\item \textbf{切线方程}:
		\[
			\frac{x - at_0}{a} = \frac{y - bt_0^2}{2bt_0} = \frac{z - ct_0^3}{3ct_0^2}
		\]
		
		\item \textbf{法平面方程}:
		\[
			a(x - at_0) + 2bt_0(y - bt_0^2) + 3ct_0^2(z - ct_0^3) = 0
		\]
	\end{enumerate}
\end{proof}
\section{主法线、副法线、密切平面、从切平面}
\begin{definition}[Frenet 标架]
    设 $r$ 为正则曲线,$s$ 为弧长参数。
    \begin{itemize}
        \item 记 $T(s) = r'(s)$(自动为单位向量)
        \item 注意到 $|T(s)|^2 = 1 \Rightarrow T(s)' \cdot T(s) = 0$
        \item 若 $r'(s) \neq 0$,则令 $N(s) = \frac{T'(s)}{|T'(s)|}$,最后令 $B(s) = T(s) \times N(s)$。
    \end{itemize}
    在 $r'(s) \neq \vec{0}$ 的地方总是可以定义以下坐标系 $\{r(s); T(s), N(s), B(s)\}$,称其为曲线 $r$ 的 Frenet 标架。
\end{definition}
\begin{definition}[主法线]
    设正则曲线 $r(s)$ 的 Frenet 标架为 $\{r(s); T(s), N(s), B(s)\}$,则:
    \begin{itemize}
        \item 向量 $N(s) = \frac{T'(s)}{|T'(s)|}$ 称为曲线在 $s$ 处的主法线向量。
        \item 主法线方向是曲线弯曲方向的正交单位化结果。
    \end{itemize}
\end{definition}

\begin{definition}[副法线]
    在 Frenet 标架中:
    \begin{itemize}
        \item 向量 $B(s) = T(s) \times N(s)$ 称为曲线在 $s$ 处的副法线向量。
        \item 副法线方向由右手法则确定,且 $B(s)$ 始终垂直于密切平面。
    \end{itemize}
\end{definition}

\begin{definition}[密切平面]
    在 Frenet 标架 $\{r(s); T(s), N(s), B(s)\}$ 中:
    \begin{itemize}
        \item 由切向量 $T(s)$ 和主法线向量 $N(s)$ 张成的平面称为密切平面。
        \item 密切平面方程为:$(X - r(s)) \cdot B(s) = 0$,即所有与副法线正交的点构成的平面。
    \end{itemize}
\end{definition}
\begin{definition}[从切平面]
    对于 Frenet 标架 $\{r(s); T(s), N(s), B(s)\}$:
    \begin{itemize}
        \item 由切向量 $T(s)$ 和副法线向量 $B(s)$ 张成的平面称为从切平面(亦称矫正平面)。
        \item 从切平面方程为:$(X - r(s)) \cdot N(s) = 0$,即所有与主法线正交的点构成的平面。
    \end{itemize}
\end{definition}
\begin{definition}[法平面]
    \begin{itemize}
        \item 由主法线 $N(s)$ 和副法线 $B(s)$ 张成的平面称为法平面。
        \item 法平面方程为:$(X - r(s)) \cdot T(s) = 0$,即所有与切线正交的点构成的平面。
    \end{itemize}
\end{definition}

\begin{example}
	求 $\vec{r}(t) = \{\cos t, \sin t, t\}$ 在 $(1, 0, 0)$ 处的法平面、副法线、密切平面、主法线及从切平面。
\end{example}
\begin{proof}
	
	
	\begin{enumerate}
		\item \textbf{确定参数 $t$ 值}  
		由 $\cos t = 1$ 且 $\sin t = 0$,得 $t = 0$,对应点 $(1, 0, 0)$。
		
		\item \textbf{计算 Frenet 标架}:
		\begin{enumerate}
			\item \textbf{切向量 $\vec{T}$}:
			\[
				\vec{r}\,'(t) = \{-\sin t, \cos t, 1\} \quad \Rightarrow \quad 
				\vec{r}\,'(0) = \{0, 1, 1\}
			\]
			单位化得:
			\[
				\vec{T} = \frac{\vec{r}\,'(0)}{\|\vec{r}\,'(0)\|} = \left\{0, \frac{1}{\sqrt{2}}, \frac{1}{\sqrt{2}}\right\}
			\]
			
			\item \textbf{主法线 $\vec{N}$}:
			计算二阶导数:
			\[
				\vec{r}\,''(t) = \{-\cos t, -\sin t, 0\} \quad \Rightarrow \quad 
				\vec{r}\,''(0) = \{-1, 0, 0\}
			\]
			单位化得主法线方向:
			\[
				\vec{N} = \frac{\vec{r}\,''(0)}{\|\vec{r}\,''(0)\|} = \{-1, 0, 0\}
			\]
			
			\item \textbf{副法线 $\vec{B}$}:
			\[
				\vec{B} = \vec{T} \times \vec{N} = \begin{vmatrix}
					\mathbf{i} & \mathbf{j} & \mathbf{k} \\
					0 & \frac{1}{\sqrt{2}} & \frac{1}{\sqrt{2}} \\
					-1 & 0 & 0
				\end{vmatrix} = \left\{0, -\frac{1}{\sqrt{2}}, \frac{1}{\sqrt{2}}\right\}
			\]
		\end{enumerate}
		
		\item \textbf{几何对象方程}:
		\begin{enumerate}
			\item \textbf{切线方程}:\\
			方向向量 $\vec{r}\,'(0) = \{0, 1, 1\}$,过点 $(1, 0, 0)$:
			\[
				\frac{x-1}{0} = \frac{y}{1} = \frac{z}{1} \quad \text{或} \quad 
				\begin{cases} 
					x = 1 \\ 
					y = z 
				\end{cases}
			\]
			
			\item \textbf{法平面方程}:\\
			法向量为 $\vec{r}\,'(0)$,方程为:
			\[
				0(x-1) + 1(y-0) + 1(z-0) = 0 \quad \Rightarrow \quad y + z = 0
			\]
			
			\item \textbf{主法线方程}:\\
			方向向量 $\vec{N} = \{-1, 0, 0\}$,过点 $(1, 0, 0)$:
			\[
				\frac{x-1}{-1} = \frac{y}{0} = \frac{z}{0} \quad \text{或} \quad 
				\begin{cases} 
					y = 0 \\ 
					z = 0 
				\end{cases}
			\]
			
			\item \textbf{副法线方程}:\\
			方向向量 $\vec{B} = \left\{0, -\frac{1}{\sqrt{2}}, \frac{1}{\sqrt{2}}\right\}$,过点 $(1, 0, 0)$:
			\[
				x = 1, \quad \frac{y}{-\frac{1}{\sqrt{2}}} = \frac{z}{\frac{1}{\sqrt{2}}}
			\]
			
			\item \textbf{密切平面方程}:\\
			法向量为 $\vec{B}$,方程为:
			\[
				0(x-1) -\frac{1}{\sqrt{2}}(y-0) + \frac{1}{\sqrt{2}}(z-0) = 0 \quad \Rightarrow \quad z = y
			\]
			
			\item \textbf{从切平面方程}:\\
			法向量为 $\vec{N}$,方程为:
			\[
				-1(x-1) + 0(y-0) + 0(z-0) = 0 \quad \Rightarrow \quad x = 1
			\]
		\end{enumerate}
	\end{enumerate}
\end{proof}
\section{曲率挠率}
\begin{definition}[曲率]
    我们称 $k(s) = |T'(s)| = |r''(s)|$ 为曲线的曲率。
\end{definition}
\begin{definition}[挠率]
    \[
    \tau(s) = -B'(s) \cdot N(s)
    \]
\end{definition}
对于一般参数$t$
\begin{align*}
    k(t) &= k(s) = T'(s) \cdot N(s) \\
     &= \frac{|r' \times r''|}{|r'|^3}.
\end{align*}
\begin{align*}
    \tau(t) &= \tau(s) = -B'(s) \cdot N(s)\\
    &= \frac{[r', r'', r''']}{|r' \times r''|^2}.
\end{align*} 

\begin{example}
	求 $\vec{r} = \{a(3t - t^3), 3at^2, a(3t + t^3)\}$ 的曲率和挠率。
\end{example}
\begin{proof}
	\begin{enumerate}
		\item \textbf{计算导数}:
		\begin{align*}
			\vec{r}\,'(t) &= \{a(3 - 3t^2),\ 6at,\ a(3 + 3t^2)\} \\
			\vec{r}\,''(t) &= \{-6at,\ 6a,\ 6at\} \\
			\vec{r}\,'''(t) &= \{-6a,\ 0,\ 6a\}
		\end{align*}
		
		\item \textbf{计算叉乘 $\vec{r}\,' \times \vec{r}\,''$}:
		\[
			\vec{r}\,' \times \vec{r}\,'' = \begin{vmatrix}
				\mathbf{i} & \mathbf{j} & \mathbf{k} \\
				a(3-3t^2) & 6at & a(3+3t^2) \\
				-6at & 6a & 6at
			\end{vmatrix} = \left(18a^2(t^2-1),\ -36a^2t,\ 18a^2(1+t^2)\right)
		\]
		模长为:
		\[
			|\vec{r}\,' \times \vec{r}\,''| = 18a^2\sqrt{2}(t^2 + 1)
		\]
		
		\item \textbf{计算曲率}:
		\[
			k(t) = \frac{|\vec{r}\,' \times \vec{r}\,''|}{|\vec{r}\,'|^3} = \frac{18a^2\sqrt{2}(t^2+1)}{\left(3a\sqrt{2}(t^2+1)\right)^3} = \frac{1}{3a(t^2+1)^2}
		\]
		
		\item \textbf{计算三重标量积}:
		\[
			[\vec{r}\,', \vec{r}\,'', \vec{r}\,'''] = \begin{vmatrix}
				a(3-3t^2) & 6at & a(3+3t^2) \\
				-6at & 6a & 6at \\
				-6a & 0 & 6a
			\end{vmatrix} = 216a^3
		\]
		
		\item \textbf{计算挠率}:
		\[
			\tau(t) = \frac{[\vec{r}\,', \vec{r}\,'', \vec{r}\,''']}{|\vec{r}\,' \times \vec{r}\,''|^2} = \frac{216a^3}{\left(18a^2\sqrt{2}(t^2+1)\right)^2} = \frac{1}{3a(t^2+1)^2}
		\]
	\end{enumerate}
		

\end{proof}

\chapter{曲面论}

\section{切平面和法线}

  \textbf{u-曲线:} 令 $v = v_0$,
    $$
    r(u, v_0) = \big(x(u, v_0), y(u, v_0), z(u, v_0)\big).
    $$
    只有 $u$ 作为参数,是曲面上的曲线。
    其对应的切向量为
    $$
    \left. r'(u) \right|_{u_0} = r_u(u_0, v_0).
    $$
    
    \textbf{v-曲线:} 令 $u = u_0$,
    $$
    r(u_0, v) = \big(x(u_0, v), \ldots \big)
    $$
    其对应的切向量为
    $$
    \left. r'(v) \right|_{v_0} = r_v(u_0, v_0).
    $$
    \begin{definition}[切平面]
        设 $p \in S$ 为曲面上一点。
        
        定义 $T_p S$ 为所有经过 $p$ 且在 $S$ 上的曲线在 $p$ 点处的切向量构成的集合。
        
        称为曲面 $S$ 在 $p$ 点处的切平面(切空间)。
        
        $$
        T_p S = \left\{ \text{所有过 } p \text{ 且在曲面 } S \text{ 上的曲线在 } p \text{ 点处的切向量} \right\}
        $$
\end{definition}
\begin{proposition}
    $T_p S$ 是由 $\operatorname{Span}\{ r_u|_{(u_0,v_0)}, r_v|_{(u_0,v_0)} \}$ 张成的线性空间。  
因此,$T_p S$ 是一个线性空间。  
\[ 0 \iff (u_0, v_0) \iff P. \]
\end{proposition}
\begin{definition}[法向量与正则曲面片]
    我们称单位法向量
\[
\mathbf{n} = \frac{r_u \times r_v}{\lVert r_u \times r_v \rVert}
\]
为曲面片在参数域上的\textbf{定向法向量}。

对于正则曲面片,若其参数化满足 $r_u \times r_v \neq \mathbf{0}$,则可在每一点定义法向量。  
严格来说,$\mathbf{n}$ 和 $-\mathbf{n}$ 均可作为法向量方向,对应两种不同的定向方式(称为分法 (2))。
\end{definition}

\begin{example}
	设曲面 $\vec{r} = \vec{r}(u,v) = \{u+v, u-v, uv\}$,求点 $(1,2)$ 处的单位法向量、切平面方程、法线方程 
\end{example}
\begin{proof}
	\begin{enumerate}
		\item \textbf{计算偏导数}:
		\[
			\vec{r}_u = \left.\frac{\partial \vec{r}}{\partial u}\right|_{(1,2)} = \{1, 1, 2\}, \quad 
			\vec{r}_v = \left.\frac{\partial \vec{r}}{\partial v}\right|_{(1,2)} = \{1, -1, 1\}
		\]
		
		\item \textbf{求法向量}:
		\[
			\vec{n} = \vec{r}_u \times \vec{r}_v = \begin{vmatrix}
				\mathbf{i} & \mathbf{j} & \mathbf{k} \\
				1 & 1 & 2 \\
				1 & -1 & 1
			\end{vmatrix} = \{3, 1, -2\}
		\]
		模长 $|\vec{n}| = \sqrt{3^2 + 1^2 + (-2)^2} = \sqrt{14}$,单位法向量:
		\[
			\vec{n}_0 = \left\{\frac{3}{\sqrt{14}}, \frac{1}{\sqrt{14}}, -\frac{2}{\sqrt{14}}\right\}
		\]
		
		\item \textbf{切平面方程}:\\
		点 $\vec{r}(1,2) = (3, -1, 2)$,法向量 $\vec{n}$:
		\[
			3(x-3) + 1(y+1) - 2(z-2) = 0 \quad \Rightarrow \quad 3x + y - 2z = 4
		\]
		
		\item \textbf{法线方程}:
		\[
			\frac{x-3}{3} = \frac{y+1}{1} = \frac{z-2}{-2}
		\]
	\end{enumerate}
\end{proof}
\begin{example}
	求球面 $\vec{r} = ( \theta, \varphi ) = \{a\cos\theta\cos\varphi, a\cos\theta\sin\varphi, a\sin\theta \}$ 上任意点的切平面和法线方程。
\end{example}
\begin{proof}
	\begin{enumerate}
		\item \textbf{计算偏导数}:
		\[
			\vec{r}_\theta = \{-a\sin\theta\cos\varphi, -a\sin\theta\sin\varphi, a\cos\theta\}
		\]
		\[
			\vec{r}_\varphi = \{-a\cos\theta\sin\varphi, a\cos\theta\cos\varphi, 0\}
		\]
		
		\item \textbf{求法向量}:
		\[
			\vec{n} = \vec{r}_\theta \times \vec{r}_\varphi = -a^2\cos\theta \cdot \{a\cos\theta\cos\varphi, a\cos\theta\sin\varphi, a\sin\theta\}
		\]
		单位法向量:
		\[
			\vec{n}_0 = -\frac{\vec{r}}{a} \quad \text{(指向球心)}
		\]
		
		\item \textbf{切平面方程}:
		\[
			\cos\theta\cos\varphi \cdot x + \cos\theta\sin\varphi \cdot y + \sin\theta \cdot z = a
		\]
		
		\item \textbf{法线方程}:
		\[
			\frac{x - a\cos\theta\cos\varphi}{\cos\theta\cos\varphi} = \frac{y - a\cos\theta\sin\varphi}{\cos\theta\sin\varphi} = \frac{z - a\sin\theta}{\sin\theta}
		\]
	\end{enumerate}
\end{proof}
\section{曲面的第一基本形式}
\begin{definition}[第一基本形式]
    设正则曲面片 \( r: D \to \mathbb{E}^3 \),对任意点 \( p \in S \),其切空间 \( T_pS \) 由基向量 \(\{r_u, r_v\}\) 张成。定义第一基本形式 \( I \) 为:
\[
I = E \, du \otimes du + 2F \, du \otimes dv + G \, dv \otimes dv,
\]
其中系数由基向量的内积确定:
\[
E = r_u \cdot r_u, \quad F = r_u \cdot r_v, \quad G = r_v \cdot r_v.
\]
对应的度量张量矩阵为:
\[
(g_{ij}) = \begin{pmatrix}
E & F \\
F & G
\end{pmatrix}.
\]
\end{definition}
\subsection*{第一基本形式的公式}
设切向量 \(\alpha = a_1 r_u + b_1 r_v\) 和 \(\beta = a_2 r_u + b_2 r_v\),则:
\[
I(\alpha, \beta) = (a_1 \quad b_1) \begin{pmatrix}
E & F \\
F & G
\end{pmatrix} \begin{pmatrix}
a_2 \\
b_2
\end{pmatrix} = E a_1 a_2 + F(a_1 b_2 + b_1 a_2) + G b_1 b_2.
\]

\subsection*{第一基本形式的几何意义}

(一)曲面上的曲线长度

设曲面上的曲线为 $r(t) = r(u(t), v(t))$,其中 $t \in [a, b]$。

在点 $p$ 处,参数 $t$ 对应于 $t_0$。

曲线在点 $p$ 处的切向量为:

$$
\left. \frac{d}{dt} \right|_{t_0} r(t) = u_t \cdot r_u + v_t \cdot r_v \in T_p S
$$

曲线在点 $p$ 处的长度元素为:

$$
\left| \left. \frac{d}{dt} \right|_{t_0} r(t) \right| = \sqrt{E (u_t)^2 + 2 F (u_t)(v_t) + G (v_t)^2} = \sqrt{I \left( \frac{d}{dt} \gamma(t), \frac{d}{dt} \gamma(t) \right)}
$$
\begin{definition}[曲线长度]
    设曲面上的曲线为 $r(t) = r(u(t), v(t))$。

    曲线的长度 $l$ 定义为:
    
    $$
    l = \int_{a}^{b} |r'(t)| dt = \int_{a}^{b} \sqrt{I\left(\frac{dr}{dt}, \frac{dr}{dt}\right)} dt
    $$
    
    其中,$I$ 表示第一基本形式。
    
\end{definition}
\begin{example}
	计算 $S = \vec{r}(u,v) = \{a\cos u, a\sin u, v\}$ ($a > 0$) 上曲线 $u = t, v = t, t \in [0,1]$ 的长度。
\end{example}
\begin{proof}
	\begin{enumerate}
		\item \textbf{确定曲线参数方程}:\\
		将 $u = t$ 和 $v = t$ 代入曲面方程,得曲线:
		\[
			\vec{r}(t) = \{a\cos t,\ a\sin t,\ t\}, \quad t \in [0,1]
		\]
		
		\item \textbf{计算第一基本形式}:
		\begin{itemize}
			\item 曲面偏导数:
			\[
				\vec{r}_u = \{-a\sin u,\ a\cos u,\ 0\}, \quad 
				\vec{r}_v = \{0,\ 0,\ 1\}
			\]
			\item 第一基本量:
			\[
				E = \vec{r}_u \cdot \vec{r}_u = a^2, \quad 
				F = \vec{r}_u \cdot \vec{r}_v = 0, \quad 
				G = \vec{r}_v \cdot \vec{r}_v = 1
			\]
			\[
				\text{第一基本形式:} \quad ds^2 = a^2 du^2 + dv^2
			\]
		\end{itemize}
		
		\item \textbf{计算弧长微分}:\\
		曲线参数变化率 $\frac{du}{dt} = 1$, $\frac{dv}{dt} = 1$,代入第一基本形式:
		\[
			ds = \sqrt{E\left(\frac{du}{dt}\right)^2 + 2F\frac{du}{dt}\frac{dv}{dt} + G\left(\frac{dv}{dt}\right)^2} \, dt = \sqrt{a^2 \cdot 1 + 1 \cdot 1} \, dt = \sqrt{a^2 + 1} \, dt
		\]
		
		\item \textbf{积分求总长}:
		\[
			l = \int_{0}^{1} \sqrt{a^2 + 1} \, dt = \sqrt{a^2 + 1} \cdot \int_{0}^{1} dt = \boxed{\sqrt{a^2 + 1}}
		\]
	\end{enumerate}
\end{proof}
(二)曲面上区域的面积

曲面的面积元素 $d\sigma$ 定义为:

$$
d\sigma = |r_u \times r_v| du dv
$$

展开为:

$$
d\sigma = |r_u \times r_v| (\Delta u) (\Delta v)
$$

进一步化简为:

$$
d\sigma = \sqrt{|r_u|^2 |r_v|^2 - (r_u \cdot r_v)^2} (\Delta u) (\Delta v)
$$

即:

$$
d\sigma = \sqrt{EG - F^2} du dv
$$

其中,$E = |r_u|^2$,$F = r_u \cdot r_v$,$G = |r_v|^2$。

\begin{definition}[曲面面积]
    曲面的面积 $A(r(D))$ 定义为:

    $$
    A(r(D)) = \iint\limits_{D} \sqrt{EG - F^2} du dv
    $$

    其中,$D \subset \mathbb{R}^2$。
\end{definition}
\begin{example}
	求球面 $\vec{r}(\varphi, \theta) = a \{\cos\varphi \cos\theta, \cos\varphi \sin\theta, \sin\varphi\}$ 的面积。

\end{example}
\begin{proof}
	\begin{enumerate}
		\item \textbf{计算偏导数}:
		\[
			\vec{r}_\varphi = \{-a\sin\varphi \cos\theta,\ -a\sin\varphi \sin\theta,\ a\cos\varphi\}
		\]
		\[
			\vec{r}_\theta = \{-a\cos\varphi \sin\theta,\ a\cos\varphi \cos\theta,\ 0\}
		\]
		
		\item \textbf{计算第一基本量}:
		\[
			E = \vec{r}_\varphi \cdot \vec{r}_\varphi = a^2, \quad 
			F = \vec{r}_\varphi \cdot \vec{r}_\theta = 0, \quad 
			G = \vec{r}_\theta \cdot \vec{r}_\theta = a^2\cos^2\varphi
		\]
		
		\item \textbf{计算面积元}:
		\[
			\sqrt{EG - F^2} = a^2\cos\varphi
		\]
		
		\item \textbf{确定参数范围}:\\
		标准球面参数域 $\varphi \in [-\frac{\pi}{2}, \frac{\pi}{2}],\ \theta \in [0, 2\pi]$
		
		\item \textbf{积分求面积}:
		\[
			A = \int_{0}^{2\pi} \int_{-\frac{\pi}{2}}^{\frac{\pi}{2}} a^2\cos\varphi \, d\varphi d\theta = 2\pi a^2 \cdot 2 = \boxed{4\pi a^2}
		\]
	\end{enumerate}
\end{proof}
\begin{example}
	计算圆柱面 $S: \vec{r}(u,v) = \{a\cos u, a\sin u, v\}$ ($a > 0$) 上由曲线 $u=1, v=0, u-v=0$ 所围成曲面的面积。

\end{example}
\begin{proof}
	\begin{enumerate}
		\item \textbf{确定参数区域 $D$}:\\
		曲线 $u=1$, $v=0$, $u-v=0$ 在参数平面内围成三角形区域:
		\[
			D = \{(u,v) \mid 0 \leq u \leq 1,\ 0 \leq v \leq u\}
		\]
		
		\item \textbf{计算第一基本量}:
		\[
			\vec{r}_u = \{-a\sin u,\ a\cos u,\ 0\}, \quad 
			\vec{r}_v = \{0,\ 0,\ 1\}
		\]
		\[
			E = a^2, \quad F = 0, \quad G = 1 \quad \Rightarrow \quad \sqrt{EG - F^2} = a
		\]
		
		\item \textbf{积分求面积}:
		\[
			A = \iint\limits_{D} a \, du dv = a \int_{0}^{1} \int_{0}^{u} dv du = a \int_{0}^{1} u \, du = \boxed{\frac{a}{2}}
		\]
	\end{enumerate}
\end{proof}
\section{第二基本形式}
\begin{definition}[ 第二基本形式]
    $$
II = L \, du \otimes du + M \, du \otimes dv + M \, dv \otimes du + N \, dv \otimes dv
$$

其中,

$$
II \sim \begin{pmatrix} L & M \\ M & N \end{pmatrix}
$$

也是 $T_p S$ 上的一个张量积,线性、对称,但不一定正定。

其中:

\begin{align*}
L &= r_{uu} \cdot n = (r_u \cdot n)_u - r_u \cdot n_u \\
M &= r_{uv} \cdot n = -r_u \cdot n_v = -r_v \cdot n_u \\
N &= r_{vv} \cdot n = -r_v \cdot n_v
\end{align*}
\end{definition}
\begin{example}
	计算悬链面 $\vec{r} = \{\cosh u \cos v, \cosh u \sin v, u\}$ 的第一、第二类基本量。
\end{example}
\begin{proof}

	
	\begin{enumerate}
		\item \textbf{第一类基本量}:
		\begin{enumerate}
			\item \textbf{计算偏导数}:
			\[
				\vec{r}_u = \{\sinh u \cos v,\ \sinh u \sin v,\ 1\}, \quad 
				\vec{r}_v = \{-\cosh u \sin v,\ \cosh u \cos v,\ 0\}
			\]
			
			\item \textbf{计算第一基本量}:
			\[
				E = \vec{r}_u \cdot \vec{r}_u = \cosh^2 u, \quad 
				F = \vec{r}_u \cdot \vec{r}_v = 0, \quad 
				G = \vec{r}_v \cdot \vec{r}_v = \cosh^2 u
			\]
		\end{enumerate}
		
		\item \textbf{第二类基本量}:
		\begin{enumerate}
			\item \textbf{计算二阶偏导数}:
			\[
				\vec{r}_{uu} = \{\cosh u \cos v,\ \cosh u \sin v,\ 0\}
			\]
			\[
				\vec{r}_{uv} = \{-\sinh u \sin v,\ \sinh u \cos v,\ 0\}
			\]
			\[
				\vec{r}_{vv} = \{-\cosh u \cos v,\ -\cosh u \sin v,\ 0\}
			\]
			
			\item \textbf{计算单位法向量}:
			\[
				\vec{n} = \frac{\vec{r}_u \times \vec{r}_v}{|\vec{r}_u \times \vec{r}_v|} = \left\{ -\frac{\cos v}{\cosh u},\ -\frac{\sin v}{\cosh u},\ \tanh u \right\}
			\]
			
			\item \textbf{计算第二基本量}:
			\[
				L = \vec{r}_{uu} \cdot \vec{n} = -1, \quad 
				M = \vec{r}_{uv} \cdot \vec{n} = 0, \quad 
				N = \vec{r}_{vv} \cdot \vec{n} = 1
			\]
		\end{enumerate}
	\end{enumerate}
	
\end{proof}
\section{法曲率的计算}
\begin{proposition}[法曲率的一般公式]


	若 $v$ 非单位向量,则 $\frac{v}{|v|}$ 为单位向量。
	
	定义
	\[
	k_n(v) \equiv k_n\left(\frac{v}{|v|}\right) = II\left(\frac{v}{|v|}, \frac{v}{|v|}\right)
	\]
	\[
	= \frac{1}{|v|^2} II(v,v) = \frac{II(v,v)}{I(v,v)}
	\]
	上述称为法曲率的一般公式。
\end{proposition}
\begin{example}
	计算椭圆抛物面 $z = \frac{1}{2}(ax^2 + by^2)$ ($a, b > 0$) 在点 $(0,0,0)$ 处沿任一方向 $\frac{dx}{dy}$ 的法曲率
\end{example}
\begin{proof}
	
	\begin{enumerate}
		\item \textbf{参数化曲面}:\\
		将椭圆抛物面表示为 $\vec{r}(x,y) = \left(x,\ y,\ \frac{1}{2}(ax^2 + by^2)\right)$
		
		\item \textbf{计算一阶偏导数}:
		\[
			\vec{r}_x = \left(1,\ 0,\ ax\right), \quad 
			\vec{r}_y = \left(0,\ 1,\ by\right)
		\]
		在点 $(0,0,0)$ 处:
		\[
			\vec{r}_x = (1,0,0), \quad 
			\vec{r}_y = (0,1,0)
		\]
		
		\item \textbf{第一类基本量}:
		\[
			E = \vec{r}_x \cdot \vec{r}_x = 1, \quad 
			F = \vec{r}_x \cdot \vec{r}_y = 0, \quad 
			G = \vec{r}_y \cdot \vec{r}_y = 1
		\]
		
		\item \textbf{计算单位法向量}:
		\[
			\vec{n} = \frac{\vec{r}_x \times \vec{r}_y}{|\vec{r}_x \times \vec{r}_y|} = (0,0,1)
		\]
		
		\item \textbf{计算二阶偏导数}:
		\[
			\vec{r}_{xx} = (0,0,a), \quad 
			\vec{r}_{xy} = (0,0,0), \quad 
			\vec{r}_{yy} = (0,0,b)
		\]
		
		\item \textbf{第二类基本量}:
		\[
			L = \vec{r}_{xx} \cdot \vec{n} = a, \quad 
			M = \vec{r}_{xy} \cdot \vec{n} = 0, \quad 
			N = \vec{r}_{yy} \cdot \vec{n} = b
		\]
		
		\item \textbf{法曲率公式}:\\
		设方向比为 $k = \frac{dx}{dy}$,则方向向量为 $(k, 1)$,法曲率为:
		\[
			k_n = \frac{Lk^2 + N}{k^2 + 1} = \frac{a k^2 + b}{k^2 + 1}
		\]
	\end{enumerate}
	
\end{proof}
\section{曲面的渐进方向}
\begin{itemize}
    \item 我们称 \( V \in T_pS \) 是一个渐近方向,若 \( k_n(v) = \frac{II(v, v)}{I(v,v)} = 0 \)。

    若 \( r(s) = r(u(s), v(s)) \) 的切方向 \( r'(s) \) 是渐近方向,则 称\( r(s) \) 为曲面上的一条渐近线。此时 \( k_n(r'(s)) \equiv 0 \)。  
\end{itemize}

渐近线方程

\[ k_n(r'(s)) \equiv 0 \iff II(r'(s), r'(s)) \equiv 0. \]

\[
\Leftrightarrow L\left( \frac{du}{ds} \right)^2 + 2M\left( \frac{du}{ds} \right) \left( \frac{dv}{ds} \right) + N\left( \frac{dv}{ds} \right)^2 = 0
\]

\[
\Leftrightarrow L\left( \frac{du}{dv} \right)^2 + 2M\left( \frac{du}{dv} \right) + N = 0.
\]

将 \( u \) 视为 \( v \) 的函数:
\[
\frac{du}{dv} = \frac{du/ds}{dv/ds}
\]

方程有解的必要条件是判别式 \( \Delta \geq 0 \):
\[
\Delta \geq 0 \iff \det II\leq 0 \iff K \leq 0.
\]

解得 \( u = f(v) + C \),于是可以讨论曲线 \( u = f(v) + C \):
\[
F(u,v) = C
\]

\begin{example}
	求双曲抛物面 $\vec{r}(u,v) = \{2(u+v), u-v, 2uv\}$ 在点 $(0,0,0)$ 的渐近方向。
\end{example}
\begin{proof}
	\begin{enumerate}
		\item \textbf{二阶偏导数计算}:
		\[
			\vec{r}_{uu} = (0,0,0), \quad 
			\vec{r}_{uv} = (0,0,2), \quad 
			\vec{r}_{vv} = (0,0,0)
		\]
		
		\item \textbf{单位法向量}:
		\[
			\vec{n} = \frac{\vec{r}_u \times \vec{r}_v}{|\vec{r}_u \times \vec{r}_v|} = (0,0,-1)
		\]
		
		\item \textbf{第二基本量}:
		\[
			L = \vec{r}_{uu} \cdot \vec{n} = 0, \quad 
			M = \vec{r}_{uv} \cdot \vec{n} = -2, \quad 
			N = \vec{r}_{vv} \cdot \vec{n} = 0
		\]
		
		\item \textbf{渐近方向方程}:
		\[
			0 \cdot (du)^2 + 2(-2) du dv + 0 \cdot (dv)^2 = 0 \quad \Rightarrow \quad du \cdot dv = 0
		\]
		解得渐近方向为 $\boxed{du=0}$ 或 $\boxed{dv=0}$,对应直母线方向。
	\end{enumerate}
\end{proof}

\begin{example}
	试求圆柱面 $S: \vec{r}(u,v) = \{a\cos u, a\sin u, v\}$ ($a > 0$) 上的渐近线。
\end{example}
\begin{proof}
	\begin{enumerate}
		\item \textbf{二阶偏导数计算}:
		\[
			\vec{r}_{uu} = (-a\cos u, -a\sin u, 0), \quad 
			\vec{r}_{uv} = (0,0,0), \quad 
			\vec{r}_{vv} = (0,0,0)
		\]
		
		\item \textbf{单位法向量}:
		\[
			\vec{n} = (\cos u, \sin u, 0)
		\]
		
		\item \textbf{第二基本量}:
		\[
			L = \vec{r}_{uu} \cdot \vec{n} = -a, \quad 
			M = 0, \quad 
			N = 0
		\]
		
		\item \textbf{渐近线方程}:
		\[
			-a (du)^2 = 0 \quad \Rightarrow \quad du = 0
		\]
		解得渐近线为 $\boxed{u = \text{常数}}$,即圆柱面的直母线。
	\end{enumerate}
\end{proof}
\begin{example}
	曲面 $z = xy^2$ 的渐近线。
\end{example}
\begin{proof}
	\begin{enumerate}
		\item \textbf{参数化与二阶偏导数}:
		\[
			\vec{r}(x,y) = (x, y, xy^2), \quad 
			\vec{r}_{xx} = (0,0,0), \quad 
			\vec{r}_{xy} = (0,0,2y), \quad 
			\vec{r}_{yy} = (0,0,2x)
		\]
		
		\item \textbf{单位法向量}:
		\[
			\vec{n} = \frac{(-y^2, -2xy, 1)}{\sqrt{y^4 + 4x^2y^2 + 1}}
		\]
		
		\item \textbf{第二基本量}:
		\[
			L = 0, \quad 
			M = \frac{2y}{\sqrt{D}}, \quad 
			N = \frac{2x}{\sqrt{D}} \quad (D = y^4 + 4x^2y^2 + 1)
		\]
		
		\item \textbf{渐近线方程}:
		\[
			4y dx dy + 2x (dy)^2 = 0 \quad \Rightarrow \quad 2 dy (2y dx + x dy) = 0
		\]
		解得渐近线为:
		\[
			\boxed{y = C} \quad \text{或} \quad \boxed{x = Cy^{-1/2}} \quad (C \text{为常数})
		\]
	\end{enumerate}
\end{proof}
\section{主方向}
\begin{definition}
	设 $(d) du:dv$ 为曲面 $S: \vec{r} = \vec{r}(u,v)$ 在 $P(u,v)$ 点的一个方向,则 $(d)$ 为主方向当且仅当
	\[
	\begin{vmatrix}
	du^2 & -du \, dv & dv^2 \\
	E & F & G \\
	L & M & N
	\end{vmatrix}
	= 0 
	\]
\end{definition}
\begin{example}
	计算双曲抛物面 $\vec{r} = \{a(u+v), b(u-v), 2uv\}$ 在点 $(0,0,0)$ 处的主方向。
\end{example}
\begin{proof}
    \begin{enumerate}
        \item \textbf{计算一阶偏导:}
        \[
            \vec{r}_u = (a, b, 2v), \quad \vec{r}_v = (a, -b, 2u)
        \]
        在 $(u,v) = (0,0)$ 处:
        \[
            \vec{r}_u = (a, b, 0), \quad \vec{r}_v = (a, -b, 0)
        \]
        
        \item \textbf{计算第一基本量:}
        \[
            E = \vec{r}_u \cdot \vec{r}_u = a^2 + b^2
        \]
        \[
            F = \vec{r}_u \cdot \vec{r}_v = a^2 - b^2
        \]
        \[
            G = \vec{r}_v \cdot \vec{r}_v = a^2 + b^2
        \]
        
        \item \textbf{计算单位法向量:}
        \[
            \vec{n} = \frac{\vec{r}_u \times \vec{r}_v}{|\vec{r}_u \times \vec{r}_v|} = \frac{(0, 0, -2ab)}{2|ab|} = (0, 0, -1)
        \]
        
        \item \textbf{计算二阶偏导:}
        \[
            \vec{r}_{uu} = (0, 0, 0), \quad \vec{r}_{uv} = (0, 0, 2), \quad \vec{r}_{vv} = (0, 0, 0)
        \]
        
        \item \textbf{计算第二基本量:}
        \[
            L = \vec{r}_{uu} \cdot \vec{n} = 0
        \]
        \[
            M = \vec{r}_{uv} \cdot \vec{n} = 2 \cdot (-1) = -2
        \]
        \[
            N = \vec{r}_{vv} \cdot \vec{n} = 0
        \]
        
        \item \textbf{主方向判别式:}
        设主方向为 $du:dv = k:1$,代入主方向判别式:
        \[
        \begin{vmatrix}
        k^2 & -k & 1 \\
        E & F & G \\
        L & M & N
        \end{vmatrix}
        = 0
        \]
        即
        \[
        \begin{vmatrix}
        k^2 & -k & 1 \\
        a^2+b^2 & a^2-b^2 & a^2+b^2 \\
        0 & -2 & 0
        \end{vmatrix}
        = 0
        \]
        展开行列式(按第三行):
        \[
        0 \cdot \begin{vmatrix} -k & 1 \\ a^2-b^2 & a^2+b^2 \end{vmatrix}
        - (-2) \cdot \begin{vmatrix} k^2 & 1 \\ a^2+b^2 & a^2+b^2 \end{vmatrix}
        + 0 \cdot \begin{vmatrix} k^2 & -k \\ a^2+b^2 & a^2-b^2 \end{vmatrix}
        \]
        只剩中间项:
        \[
        2 \cdot \left[ k^2(a^2+b^2) - 1(a^2+b^2) \right] = 2(a^2+b^2)(k^2 - 1) = 0
        \]
        所以 $k^2 = 1$,即 $k = \pm 1$。
        
        \item 
        主方向为 $du:dv = 1:1$ 和 $du:dv = -1:1$,即沿 $u = v$ 和 $u = -v$ 方向。
        \[
        \boxed{du:dv = 1:1 \quad \text{和} \quad du:dv = -1:1}
        \]
    \end{enumerate}
\end{proof}


\section{主曲率、高斯曲率、平均曲率}
\begin{definition}[主曲率]
    我们称 $W$ 的曲线在两个特征值(可以相等)$k_1, k_2$ 处的方向为曲面在该点的主曲率。
\end{definition}
\begin{theorem}
	曲面 $S: \vec{r} = \vec{r}(u,v)$ 在点 $P(u,v)$ 处的主曲率 $k_n$ 满足
\[
\begin{vmatrix}
L - k_n E & M - k_n F \\
M + k_n F & N - k_n G
\end{vmatrix}
= 0
\]
($EG - F^2$)$k_n^2$ + $(LG - 2MF + NE)k_n$ + $(LN - M^2) = 0$ (一元二次方程)
\end{theorem}
\begin{definition}[中曲率和高斯曲率]
    \begin{itemize}
        \item \textbf{高斯曲率}($K$):
        \[
        K = k_1 \cdot k_2 = \frac{LN - M^2}{EG - F^2}
        \]
        其中$k_1$和$k_2$是主曲率,$E,F,G$是第一基本形式的系数,$L,M,N$是第二基本形式的系数。
    
        \item \textbf{中曲率}($H$):
        \[
        H = \frac{1}{2}(k_1 + k_2) = \frac{LG + NE - 2MF}{2(EG - F^2)}
        \]
    \end{itemize}

    两者完全决定了曲面在一点处的弯曲
\end{definition}
\begin{example}
	计算曲面 $\vec{r}(u,v) = \{u+v, u^2-v^2, u-v\}$ 在点 $(0,0)$ 处的曲率。
\end{example}
\begin{proof}
    \begin{enumerate}
        \item \textbf{计算一阶偏导:}
        \[
        \vec{r}_u = (1, 2u, 1), \quad \vec{r}_v = (1, -2v, -1)
        \]
        在 $(u,v) = (0,0)$ 处:
        \[
        \vec{r}_u = (1, 0, 1), \quad \vec{r}_v = (1, 0, -1)
        \]

        \item \textbf{计算第一基本形式系数:}
        \[
        E = \vec{r}_u \cdot \vec{r}_u = 1^2 + 0^2 + 1^2 = 2
        \]
        \[
        F = \vec{r}_u \cdot \vec{r}_v = 1 \times 1 + 0 \times 0 + 1 \times (-1) = 0
        \]
        \[
        G = \vec{r}_v \cdot \vec{r}_v = 1^2 + 0^2 + (-1)^2 = 2
        \]

        \item \textbf{计算单位法向量:}
        \[
        \vec{n} = \frac{\vec{r}_u \times \vec{r}_v}{|\vec{r}_u \times \vec{r}_v|} = \frac{(0, 2, 0)}{2} = (0, 1, 0)
        \]

        \item \textbf{计算二阶偏导:}
        \[
        \vec{r}_{uu} = (0, 2, 0), \quad \vec{r}_{uv} = (0, 0, 0), \quad \vec{r}_{vv} = (0, -2, 0)
        \]

        \item \textbf{计算第二基本形式系数:}
        \[
        L = \vec{r}_{uu} \cdot \vec{n} = (0, 2, 0) \cdot (0, 1, 0) = 2
        \]
        \[
        M = \vec{r}_{uv} \cdot \vec{n} = (0, 0, 0) \cdot (0, 1, 0) = 0
        \]
        \[
        N = \vec{r}_{vv} \cdot \vec{n} = (0, -2, 0) \cdot (0, 1, 0) = -2
        \]

        \item \textbf{代入主曲率判别式:}
        \[
        (EG - F^2)k_n^2 + (LG - 2MF + NE)k_n + (LN - M^2) = 0
        \]
        代入 $E=2, F=0, G=2, L=2, M=0, N=-2$ 得
        \[
        (2 \times 2 - 0^2)k_n^2 + (2 \times 2 - 2 \times 0 \times 0 + (-2) \times 2)k_n + (2 \times -2 - 0^2) = 0
        \]
        \[
        4k_n^2 + (4 - 4)k_n - 4 = 0
        \]
        \[
        4k_n^2 - 4 = 0
        \]
        \[
        k_n^2 = 1 \implies k_n = \pm 1
        \]

        \item \textbf{高斯曲率与平均曲率:}
        \[
        K = \frac{LN - M^2}{EG - F^2} = \frac{2 \times (-2) - 0^2}{2 \times 2 - 0^2} = \frac{-4}{4} = -1
        \]
        \[
        H = \frac{LG + NE - 2MF}{2(EG - F^2)} = \frac{2 \times 2 + (-2) \times 2 - 0}{2 \times 4} = \frac{4 - 4}{8} = 0
        \]

        \item 
        \[
        \boxed{
            \begin{aligned}
                &\text{主曲率:} \quad k_1 = 1,\quad k_2 = -1 \\\\
                &\text{高斯曲率:} \quad K = -1 \\\\
                &\text{平均曲率:} \quad H = 0
            \end{aligned}
        }
        \]
    \end{enumerate}
\end{proof}
\begin{example}
	计算正螺面 $S: \vec{r}(u,v) = \{u\cos v, u\sin v, v\}$ 的高斯曲率和平均曲率。
\end{example}
\begin{proof}
    \begin{enumerate}
        \item \textbf{计算一阶偏导:}
        \[
        \vec{r}_u = (\cos v,\, \sin v,\, 0), \quad \vec{r}_v = (-u\sin v,\, u\cos v,\, 1)
        \]

        \item \textbf{计算第一基本形式系数:}
        \[
        E = \vec{r}_u \cdot \vec{r}_u = \cos^2 v + \sin^2 v = 1
        \]
        \[
        F = \vec{r}_u \cdot \vec{r}_v = \cos v \cdot (-u\sin v) + \sin v \cdot (u\cos v) + 0 \cdot 1 = 0
        \]
        \[
        G = \vec{r}_v \cdot \vec{r}_v = (-u\sin v)^2 + (u\cos v)^2 + 1^2 = u^2(\sin^2 v + \cos^2 v) + 1 = u^2 + 1
        \]

        \item \textbf{计算单位法向量:}
        \[
        \vec{n} = \frac{\vec{r}_u \times \vec{r}_v}{|\vec{r}_u \times \vec{r}_v|}
        \]
        先计算叉积:
        \[
        \vec{r}_u \times \vec{r}_v =
        \begin{vmatrix}
        \mathbf{i} & \mathbf{j} & \mathbf{k} \\
        \cos v & \sin v & 0 \\
        -u\sin v & u\cos v & 1
        \end{vmatrix}
        = \left( \sin v,\, -\cos v,\, u \right)
        \]
        其模长为
        \[
        |\vec{r}_u \times \vec{r}_v| = \sqrt{\sin^2 v + \cos^2 v + u^2} = \sqrt{1 + u^2}
        \]
        所以
        \[
        \vec{n} = \left( \frac{\sin v}{\sqrt{1+u^2}},\, -\frac{\cos v}{\sqrt{1+u^2}},\, \frac{u}{\sqrt{1+u^2}} \right)
        \]

        \item \textbf{计算二阶偏导:}
        \[
        \vec{r}_{uu} = (0,\, 0,\, 0)
        \]
        \[
        \vec{r}_{uv} = (-\sin v,\, \cos v,\, 0)
        \]
        \[
        \vec{r}_{vv} = (-u\cos v,\, -u\sin v,\, 0)
        \]

        \item \textbf{计算第二基本形式系数:}
        \[
        L = \vec{r}_{uu} \cdot \vec{n} = 0
        \]
        \[
        M = \vec{r}_{uv} \cdot \vec{n} = (-\sin v,\, \cos v,\, 0) \cdot \left( \frac{\sin v}{\sqrt{1+u^2}},\, -\frac{\cos v}{\sqrt{1+u^2}},\, \frac{u}{\sqrt{1+u^2}} \right)
        \]
        \[
        = -\frac{\sin^2 v}{\sqrt{1+u^2}} - \frac{\cos^2 v}{\sqrt{1+u^2}} = -\frac{1}{\sqrt{1+u^2}}
        \]
        \[
        N = \vec{r}_{vv} \cdot \vec{n} = (-u\cos v,\, -u\sin v,\, 0) \cdot \left( \frac{\sin v}{\sqrt{1+u^2}},\, -\frac{\cos v}{\sqrt{1+u^2}},\, \frac{u}{\sqrt{1+u^2}} \right)
        \]
        \[
        = -u\cos v \cdot \frac{\sin v}{\sqrt{1+u^2}} + (-u\sin v) \cdot \left(-\frac{\cos v}{\sqrt{1+u^2}}\right)
        \]
        \[
        = -\frac{u\sin v\cos v}{\sqrt{1+u^2}} + \frac{u\sin v\cos v}{\sqrt{1+u^2}} = 0
        \]

        \item \textbf{高斯曲率与平均曲率公式:}
        \[
        K = \frac{LN - M^2}{EG - F^2}
        \]
        \[
        H = \frac{LG + NE - 2MF}{2(EG - F^2)}
        \]
        代入 $E=1, F=0, G=u^2+1, L=0, M=-\frac{1}{\sqrt{1+u^2}}, N=0$ 得
        \[
        EG - F^2 = 1 \cdot (u^2+1) - 0 = u^2+1
        \]
        \[
        K = \frac{0 \cdot 0 - \left(-\frac{1}{\sqrt{1+u^2}}\right)^2}{u^2+1} = -\frac{1}{(u^2+1)^2}
        \]
        \[
        H = \frac{0 \cdot (u^2+1) + 0 \cdot 1 - 0}{2(u^2+1)} = 0
        \]

        \item \textbf{结论:}
        \[
        \boxed{
            \begin{aligned}
                &\text{高斯曲率:} \quad K = -\frac{1}{(u^2+1)^2} \\\\
                &\text{平均曲率:} \quad H = 0
            \end{aligned}
        }
        \]
    \end{enumerate}
\end{proof}
\section{可展曲面}
\begin{definition}[直纹面]
    称由单参数直线族给出的曲面为直纹面,换而言之,直纹面一定有以下参数表达。
\[
r(u,v) = a(u) + v b(u).
\]
\end{definition}
\begin{definition}[可展曲面]
    $K=0$ 的直纹面称为可展曲面。
\end{definition}
\begin{proposition}
    直纹面可展 $\Leftrightarrow [a'(u), b(u), b'(u)] \equiv 0$.
\end{proposition}

\begin{example}
	证明 $\vec{r} = \{u^2 + \frac{1}{3}v, 2u^3 + uv, u^4 + \frac{2}{3}u^2v\}$ 是可展曲面。
\end{example}
\begin{proof}

	\textbf{方法一:高斯曲率法}
	
	\begin{enumerate}
		\item \textbf{写为直纹面形式}:
		\[
		a(u) = \left(u^2,\, 2u^3,\, u^4\right),\quad b(u) = \left(\frac{1}{3},\, u,\, \frac{2}{3}u^2\right)
		\]
		\item \textbf{计算一阶偏导}:
		\[
		\vec{r}_u = (2u,\, 6u^2 + v,\, 4u^3 + \frac{4}{3}uv)
		\]
		\[
		\vec{r}_v = \left(\frac{1}{3},\, u,\, \frac{2}{3}u^2\right) = b(u)
		\]
		\item \textbf{计算二阶偏导}:
		\[
		\vec{r}_{uu} = (2,\, 12u,\, 12u^2 + \frac{4}{3}v)
		\]
		\[
		\vec{r}_{uv} = (0,\, 1,\, \frac{4}{3}u)
		\]
		\[
		\vec{r}_{vv} = (0,\, 0,\, 0)
		\]
		\item \textbf{计算法向量}:
		\[
		\vec{n} = \vec{r}_u \times \vec{r}_v
		\]
		\item \textbf{计算 $E, F, G, L, M, N$,代入高斯曲率公式},可验证 $K \equiv 0$。
	\end{enumerate}
	
	\textbf{方法二:混合积法}
	
	\begin{enumerate}
		\item $a(u) = (u^2,\, 2u^3,\, u^4)$,$b(u) = \left(\frac{1}{3},\, u,\, \frac{2}{3}u^2\right)$
		\item $a'(u) = (2u,\, 6u^2,\, 4u^3)$,$b'(u) = (0,\, 1,\, \frac{4}{3}u)$
		\item 计算混合积
		\[
		[a'(u), b(u), b'(u)] = \det
		\begin{pmatrix}
		2u & \frac{1}{3} & 0 \\
		6u^2 & u & 1 \\
		4u^3 & \frac{2}{3}u^2 & \frac{4}{3}u
		\end{pmatrix}
		\]
		展开可得 $[a'(u), b(u), b'(u)] \equiv 0$,所以是可展曲面。
	\end{enumerate}
	\end{proof}
\begin{example}
	证明 $\vec{r} = \{\cos v - (u+v)\sin v, \sin v + (u+v)\cos v, u + 2v\}$ 是可展曲面。
\end{example}
\begin{proof}
	\textbf{方法一:高斯曲率法}
	
	\begin{enumerate}
		\item \textbf{写为直纹面形式}:
		\[
		a(u) = \{-u\sin v,\, u\cos v,\, u\},\quad b(u) = \{\cos v - v\sin v,\, \sin v + v\cos v,\, 2v\}
		\]
		但更直接地,视为 $a(u) = (0,0,u)$,$b(u) = (\cos v - v\sin v,\, \sin v + v\cos v,\, 2v)$,参数交换也可。
		\item \textbf{计算 $K$,可验证 $K \equiv 0$}。
	\end{enumerate}
	
	\textbf{方法二:混合积法}
	
	\begin{enumerate}
		\item $a(u) = (0,0,u)$,$b(u) = (\cos v - v\sin v,\, \sin v + v\cos v,\, 2v)$
		\item $a'(u) = (0,0,1)$,$b'(u) = (-\sin v - v\cos v,\, \cos v - v\sin v,\, 2)$
		\item 计算混合积
		\[
		[a'(u), b(u), b'(u)] = \det
		\begin{pmatrix}
		0 & \cos v - v\sin v & -\sin v - v\cos v \\
		0 & \sin v + v\cos v & \cos v - v\sin v \\
		1 & 2v & 2
		\end{pmatrix}
		\]
		展开可得 $[a'(u), b(u), b'(u)] \equiv 0$,所以是可展曲面。
	\end{enumerate}
	\end{proof}
\begin{example}
	证明正螺面 $\vec{r} = \{v\cos u, v\sin u, au + b\}$ 不是可展曲面。

\end{example}
\begin{proof}
	\textbf{方法一:高斯曲率法}
	
	\begin{enumerate}
		\item \textbf{写为直纹面形式}:
		\[
		a(u) = (0, 0, au + b),\quad b(u) = (\cos u, \sin u, 0)
		\]
		\item \textbf{计算 $K$,可得 $K \not\equiv 0$,不是可展曲面。}
	\end{enumerate}
	
	\textbf{方法二:混合积法}
	
	\begin{enumerate}
		\item $a(u) = (0, 0, au + b)$,$b(u) = (\cos u, \sin u, 0)$
		\item $a'(u) = (0, 0, a)$,$b'(u) = (-\sin u, \cos u, 0)$
		\item 计算混合积
		\[
		[a'(u), b(u), b'(u)] = \det
		\begin{pmatrix}
		0 & \cos u & -\sin u \\
		0 & \sin u & \cos u \\
		a & 0 & 0
		\end{pmatrix}
		= a(\cos^2 u + \sin^2 u) = a \neq 0
		\]
		所以不是可展曲面。
	\end{enumerate}
	\end{proof}

\section{曲面论基本定理、一些符号}
\begin{align*}
    &\text{高斯符号:} \quad u \quad v \quad du \quad dv \quad \vec{r_0} \quad \vec{r_1} \quad \vec{r_2} \quad \vec{r_{uv}} \quad \vec{r_{uv}}' \\
    &\text{张量符号:} \quad u^1 \quad u^2 \quad du^1 \quad du^2 \quad \vec{r}^1 \quad \vec{r}^2 \quad (\vec{r}^1)' \quad (\vec{r}^2)' \quad \vec{r_{12}} \quad \vec{r_{22}}' \\
\end{align*}

\begin{align*}
    &\text{(高斯符号)} \quad E = \vec{r}_u \cdot \vec{r}_u \quad F = \vec{r}_u \cdot \vec{r}_v \quad G = \vec{r}_v \cdot \vec{r}_v \\
    &\quad \quad \quad \quad \quad L = \vec{r}_{uu} \cdot \vec{n} \quad M = \vec{r}_{uv} \cdot \vec{n} \quad N = \vec{r}_{vv} \cdot \vec{n} \\
    &\text{(张量符号)} \quad g_{11} = \vec{r}_1 \cdot \vec{r}_1 \quad g_{12} = \vec{r}_1 \cdot \vec{r}_2 \quad g_{22} = \vec{r}_2 \cdot \vec{r}_2 \\
    &\quad \quad \quad \quad \quad L_{11} = \vec{r}_{11} \cdot \vec{n} \quad L_{12} = \vec{r}_{12} \cdot \vec{n} \quad L_{22} = \vec{r}_{22} \cdot \vec{n} \\
\end{align*}

\begin{align*}
    &\text{(高斯符号)} 
    \begin{vmatrix}
        E & F \\
        F & G
    \end{vmatrix} = EG - F^2 \quad
    \begin{vmatrix}
        E & F \\
        F & G
    \end{vmatrix}^{-1} \quad
     k = \frac{LN - M^2}{EG - F^2} \\
    &\text{(张量符号)} 
    g = \begin{vmatrix}
        g_{11} & g_{12} \\
        g_{21} & g_{22}
    \end{vmatrix} \quad
    (g^{ij}) = \frac{1}{g} \begin{vmatrix}
        g_{22} & -g_{12} \\
        -g_{21} & g_{11}
    \end{vmatrix} \quad
    k = \frac{L_{11}L_{22} - L_{12}L_{21}}{g}
\end{align*}
\begin{align*}
    &\text{高斯符号} 
    I = E \, du^2 + 2F \, du \, dv + G \, dv^2 \quad
    II = L \, du^2 + 2M \, du \, dv + N \, dv^2 \\
    &\text{张量符号} 
    I = \sum_{i,j} g_{ij} \, du^i \, dv^j \quad (i,j = 1,2) \quad
    II = \sum_{i,j} L_{ij} \, du^i \, dv^j \quad (i,j = 1,2)
\end{align*}
\[
\Gamma_{ij}^k = \sum_{l} g^{kl} [ij,l]= \sum_{l} \frac{1}{2} g^{kl} \left( \frac{\partial g_{il}}{\partial u^j} + \frac{\partial g_{jl}}{\partial u^i} - \frac{\partial g_{ij}}{\partial u^l} \right), \quad i,j,k=1,2.
\]

\[
\Gamma_{ij}^k \text{ 用 } E, F, G \text{ 表示:}
\]

\[
\Gamma_{11}^1 = \frac{GE_u - F(2F_u - E_v)}{2(EG - F^2)}, \quad \Gamma_{11}^2 = \frac{E(2F_u - E_v) - FE_u}{2(EG - F^2)}
\]

\[
\Gamma_{12}^1 = \frac{E_v - F G_u}{2(EG - F^2)}, \quad \Gamma_{12}^2 = \frac{E G_u - F E_v}{2(EG + F^2)}
\]

\[
\Gamma_{22}^1 = \frac{G(2F_v - G_u) - F G_v}{2(EG - F^2)}, \quad \Gamma_{22}^2 = \frac{E G_v + F(2F_v - G_u)}{2(EG - F^2)}
\]
\begin{example}
	$\Gamma_{11}^1 $
\end{example}
\begin{proof}
	由 Christoffel 符号的定义:
	\[
	\Gamma_{ij}^k = \sum_{l=1}^2 \frac{1}{2} g^{kl} \left( \frac{\partial g_{il}}{\partial u^j} + \frac{\partial g_{jl}}{\partial u^i} - \frac{\partial g_{ij}}{\partial u^l} \right)
	\]
	对于二维曲面,第一基本形式为
	\[
	(g_{ij}) = \begin{pmatrix}
	E & F \\
	F & G
	\end{pmatrix}
	\]
	其逆矩阵为
	\[
	(g^{ij}) = \frac{1}{EG - F^2}
	\begin{pmatrix}
	G & -F \\
	-F & E
	\end{pmatrix}
	\]
	我们要求 $\Gamma_{11}^1$,即 $i=1, j=1, k=1$,代入得
	\[
	\Gamma_{11}^1 = \sum_{l=1}^2 \frac{1}{2} g^{1l} \left( \frac{\partial g_{1l}}{\partial u^1} + \frac{\partial g_{1l}}{\partial u^1} - \frac{\partial g_{11}}{\partial u^l} \right)
	\]
	\[
	= \frac{1}{2} \left[
	g^{11} \left( 2\frac{\partial g_{11}}{\partial u^1} - \frac{\partial g_{11}}{\partial u^1} \right)
	+ g^{12} \left( 2\frac{\partial g_{12}}{\partial u^1} - \frac{\partial g_{11}}{\partial u^2} \right)
	\right]
	\]
	注意 $g_{11}=E,\,g_{12}=F,\,u^1=u,\,u^2=v$,所以
	\[
	\Gamma_{11}^1 = \frac{1}{2} \left[
	g^{11} E_u
	+ g^{12} (2F_u - E_v)
	\right]
	\]
	其中 $E_u = \frac{\partial E}{\partial u}$,$F_u = \frac{\partial F}{\partial u}$,$E_v = \frac{\partial E}{\partial v}$。
	
	再代入 $g^{11} = \frac{G}{EG-F^2}$,$g^{12} = -\frac{F}{EG-F^2}$,得
	\[
	\Gamma_{11}^1 = \frac{1}{2(EG-F^2)} \left[
	G E_u - F (2F_u - E_v)
	\right]
	\]
	即
	\[
	\boxed{
	\Gamma_{11}^1 = \frac{G E_u - F (2F_u - E_v)}{2(EG-F^2)}
	}
	\]
\end{proof}
\begin{example}
	平面上取极坐标时,第一基本形式为 $ds^2 = d\rho^2 + \rho^2 d\theta^2$,试计算第二类克里斯托费尔符号 $\Gamma_{ij}^k$
\end{example}
\begin{proof}
	极坐标下,第一基本形式为
	\[
	ds^2 = d\rho^2 + \rho^2 d\theta^2
	\]
	即
	\[
	E = 1, \quad F = 0, \quad G = \rho^2
	\]
	记 $u^1 = \rho,\, u^2 = \theta$。
	
	计算各个分量的偏导数:
	\[
	E_\rho = 0,\quad E_\theta = 0,\quad F_\rho = 0,\quad F_\theta = 0,\quad G_\rho = 2\rho,\quad G_\theta = 0
	\]
	
	第一基本形式的逆矩阵为
	\[
	(g^{ij}) = \frac{1}{EG - F^2}
	\begin{pmatrix}
	G & -F \\
	-F & E
	\end{pmatrix}
	= \frac{1}{\rho^2}
	\begin{pmatrix}
	\rho^2 & 0 \\
	0 & 1
	\end{pmatrix}
	= \begin{pmatrix}
	1 & 0 \\
	0 & \frac{1}{\rho^2}
	\end{pmatrix}
	\]
	
	下面分别计算各个 $\Gamma_{ij}^k$:
	
	\textbf{1.} $\Gamma_{11}^1$:
	\[
	\Gamma_{11}^1 = \frac{G E_\rho - F (2F_\rho - E_\theta)}{2(EG - F^2)} = \frac{\rho^2 \cdot 0 - 0 \cdot (0 - 0)}{2\rho^2} = 0
	\]
	
	\textbf{2.} $\Gamma_{11}^2$:
	\[
	\Gamma_{11}^2 = \frac{E(2F_\rho - E_\theta) - F E_\rho}{2(EG - F^2)} = \frac{1 \cdot (0 - 0) - 0 \cdot 0}{2\rho^2} = 0
	\]
	
	\textbf{3.} $\Gamma_{12}^1$:
	\[
	\Gamma_{12}^1 = \frac{E_\theta - F G_\rho}{2(EG - F^2)} = \frac{0 - 0 \cdot 2\rho}{2\rho^2} = 0
	\]
	
	\textbf{4.} $\Gamma_{12}^2$:
	\[
	\Gamma_{12}^2 = \frac{E G_\rho - F E_\theta}{2(EG - F^2)} = \frac{1 \cdot 2\rho - 0 \cdot 0}{2\rho^2} = \frac{2\rho}{2\rho^2} = \frac{1}{\rho}
	\]
	
	\textbf{5.} $\Gamma_{22}^1$:
	\[
	\Gamma_{22}^1 = \frac{G(2F_\theta - G_\rho) - F G_\theta}{2(EG - F^2)} = \frac{\rho^2 (0 - 2\rho) - 0 \cdot 0}{2\rho^2} = \frac{-2\rho^3}{2\rho^2} = -\rho
	\]
	
	\textbf{6.} $\Gamma_{22}^2$:
	\[
	\Gamma_{22}^2 = \frac{E G_\theta + F(2F_\theta - G_\rho)}{2(EG - F^2)} = \frac{1 \cdot 0 + 0 \cdot (0 - 2\rho)}{2\rho^2} = 0
	\]
	
	\textbf{综上,极坐标下非零的克里斯托费尔符号为:}
	\[
	\boxed{
	\Gamma_{12}^2 = \Gamma_{21}^2 = \frac{1}{\rho}, \qquad
	\Gamma_{22}^1 = -\rho
	}
	\]
	其余分量均为零。
\end{proof}
\end{document}

