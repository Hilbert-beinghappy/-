\chapter{一致有界性原理和闭图像定理}

\section{贝尔纲定理}

下面的经典结果在第二章的证明中起着至关重要的作用。

\begin{theorem}[贝尔]\label{theorem2.1}
设$X$是一个完备度量空间,$(X_n)_{n \geq 1}$是$X$中的一列闭子集。假设对于每个$n \geq 1$,
\[
\text{Int } X_n = \emptyset.
\]
那么
\[
\text{Int } \left(\bigcup_{n=1}^\infty X_n\right) = \emptyset.
\]
\end{theorem}

\begin{remark}
贝尔纲定理通常以下面的形式使用。设$X$是一个非空完备度量空间。设$(X_n)_{n \geq 1}$是一列闭子集,使得
\[
\bigcup_{n=1}^\infty X_n = X.
\]
那么存在某个$n_0$使得$\text{Int } X_{n_0} \neq \emptyset$。
\end{remark}

\begin{proof}
令$O_n = X_n^c$,那么对于每个$n \geq 1$,$O_n$在$X$中是开集且稠密。我们的目标是证明$G = \bigcap_{n=1}^\infty O_n$在$X$中是稠密的。设$\omega$是$X$中的一个非空开集;我们将证明$\omega \cap G \neq \emptyset$。
通常,设
\[
B(x, r) = \{y \in X; d(y, x) < r\}.
\]
选择任意$x_0 \in \omega$和$r_0 > 0$使得
\[
\overline{B(x_0, r_0)} \subset \omega.
\]
然后,选择$x_1 \in B(x_0, r_0) \cap O_1$和$r_1 > 0$使得
\[
\begin{cases}
\overline{B(x_1, r_1)} \subset B(x_0, r_0) \cap O_1, \\
0 < r_1 < \frac{r_0}{2},
\end{cases}
\]
这是可能的,因为$O_1$是开集且稠密的。通过归纳,我们构造序列$(x_n)$和$(r_n)$使得
\[
\begin{cases}
\overline{B(x_{n+1}, r_{n+1})} \subset B(x_n, r_n) \cap O_{n+1}, \quad \forall n \geq 0, \\
0 < r_{n+1} < \frac{r_n}{2}.
\end{cases}
\]
由此可知$(x_n)$是一个柯西序列;设$x_n \to \ell$。由于对于每一个$p \geq 0$,$x_{n+p} \in B(x_n, r_n)$,我们在极限(当$p \to \infty$时)得到
\[
\ell \in \overline{B(x_n, r_n)}, \quad \forall n \geq 0.
\]
特别地,$\ell \in \omega \cap G$。
\end{proof}

\section{一致有界性原理}


设$E$和$F$是两个赋范向量空间(n.v.s.)。我们用$\mathcal{L}(E, F)$表示从$E$到$F$的连续(=有界)线性算子的空间,其范数为
\[
\|T\|_{\mathcal{L}(E, F)} = \sup_{\substack{x \in E \\ \|x\| \leq 1}} \|Tx\|.
\]
通常,我们写$\mathcal{L}(E)$而不是$\mathcal{L}(E, E)$。

\begin{theorem}[Banach-Steinhaus, 一致有界性原理]\label{theorem2.2}
设$E$和$F$是两个Banach空间,设$(T_i)_{i \in I}$是从$E$到$F$的一族(不必可数)连续线性算子。假设
\begin{equation}
\sup_{i \in I} \|T_i x\| < \infty \quad \forall x \in E. \label{eq:2.1}
\end{equation}
那么
\begin{equation}
\sup_{i \in I} \|T_i\|_{\mathcal{L}(E, F)} < \infty. \label{eq:2.2}
\end{equation}
换句话说,存在一个常数$c$使得
\[
\|T_i x\| \leq c\|x\| \quad \forall x \in E, \quad \forall i \in I.
\]
\end{theorem}

\begin{remark}
定理\ref{theorem2.2}的结论是相当出人意料和令人惊讶的。从逐点估计可以推导出一个全局(一致)估计。
\end{remark}

\begin{proof}
对于每个$n \geq 1$,令
\[
X_n = \{x \in E; \quad \forall i \in I, \|T_i x\| \leq n\},
\]
所以$X_n$是闭集,并且由\eqref{eq:2.1}我们有
\[
\bigcup_{n=1}^\infty X_n = E.
\]
由贝尔纲定理可知,存在某个$n_0 \geq 1$使得$\text{Int}(X_{n_0}) \neq \emptyset$。选择$x_0 \in E$和$r > 0$使得$B(x_0, r) \subset X_{n_0}$。我们有
\[
\|T_i(x_0+rz)\| \leq n_0 \quad \forall i \in I, \quad \forall z \in B(0, 1).
\]
这导致
\[
r\|T_i\|_{\mathcal{L}(E,F)} \leq n_0 + \|T_i x_0\|,
\]
这意味着\eqref{eq:2.2}成立。
\end{proof}

\begin{remark}
回想一下,连续映射的\textbf{逐点极限}不一定是连续的。线性假设在定理\ref{theorem2.2}中起着至关重要的作用。注意,在定理\ref{theorem2.2}的证明中,我们没有用到$F$是完备的。然而,在下面的设定中,它确实起作用了。
\end{remark}

下面是一致有界性原理的一些直接推论。

\begin{corollary}\label{corollary2.3}
设$E$和$F$是两个Banach空间。设$(T_n)$是从$E$到$F$的一列连续线性算子,使得对于每个$x \in E$,$T_n x$收敛(当$n \to \infty$时)到一个极限,记为$Tx$。那么我们有
\begin{enumerate}[(a)]
\item $\sup_n \|T_n\|_{\mathcal{L}(E,F)} < \infty$,
\item $T \in \mathcal{L}(E,F)$,
\item $\|T\|_{\mathcal{L}(E,F)} \leq \liminf_{n \to \infty} \|T_n\|_{\mathcal{L}(E,F)}$.
\end{enumerate}
\end{corollary}

\begin{proof}
(a) 直接由定理\ref{theorem2.2}得出,因此存在一个常数$c$使得
\[
\|T_n x\| \leq c\|x\| \quad \forall n, \quad \forall x \in E.
\]
在极限处我们发现
\[
\|Tx\| \leq c\|x\| \quad \forall x \in E.
\]
由于$T$显然是线性的,我们得到(b)。
最后,我们有
\[
\|T_n x\| \leq \|T_n\|_{\mathcal{L}(E,F)} \|x\| \quad \forall x \in E,
\]
(c)也直接得出。
\end{proof}

\begin{corollary}\label{corollary2.4}
设$G$是一个Banach空间,设$B$是$G$的一个子集。假设
\begin{equation}
\text{对于每个 } f \in G^*, \text{ 集合 } f(B) = \{\langle f, x \rangle; x \in B\} \text{ 在 } \mathbb{R} \text{ 中是有界的}. \label{eq:2.3}
\end{equation}
那么
\begin{equation}
B \text{ 是有界的}. \label{eq:2.4}
\end{equation}
\end{corollary}

\begin{proof}
我们将使用定理\ref{theorem2.2},其中$E=G^*$, $F=\mathbb{R}$,$I=B$。对于每个$b \in B$,设
\[
T_b(f) = \langle f, b \rangle, \quad f \in E = G^*,
\]
这样由\eqref{eq:2.3},
\[
\sup_{b \in B} |T_b(f)| < \infty \quad \forall f \in E.
\]
由定理\ref{theorem2.2}可知,存在一个常数$c$使得
\[
|\langle f, b \rangle| \leq c\|f\| \quad \forall f \in G^* \quad \forall b \in B.
\]
因此我们发现(使用推论\ref{corollary1.4})
\[
\|b\| \leq c \quad \forall b \in B.
\]
\end{proof}

\begin{remark}
推论\ref{corollary2.4}表明,为了证明一个集合$B$是有界的,只需通过有界线性泛函来“观察”$B$。这是一种在有限维空间中熟悉的过程,其中线性泛函是与某个基相关的分量。在某种意义上,推论\ref{corollary2.4}取代了在无限维空间中的分量使用。有时,人们通过说“弱有界”$\iff$“强有界”来表达推论\ref{corollary2.4}的结论(见第三章)。
\end{remark}

接下来我们有推论\ref{corollary2.4}的一个对偶陈述:

\begin{corollary}\label{corollary2.5}
设$G$是一个Banach空间,设$B^*$是$G^*$的一个子集。假设
\begin{equation}
\text{对于每个 } x \in G, \text{ 集合 } \langle B^*, x \rangle = \{\langle f, x \rangle; f \in B^*\} \text{ 在 } \mathbb{R} \text{ 中是有界的}. \label{eq:2.5}
\end{equation}
那么
\begin{equation}
B^* \text{ 是有界的}. \label{eq:2.6}
\end{equation}
\end{corollary}

\begin{proof}
使用定理\ref{theorem2.2},其中$E=G$, $F=\mathbb{R}$,$I=B^*$。对于每个$b \in B^*$设
\[
T_b(x) = \langle b, x \rangle \quad (x \in G=E).
\]
我们发现存在一个常数$c$使得
\[
|\langle b, x \rangle| \leq c\|x\| \quad \forall b \in B^*, \quad \forall x \in G.
\]
我们得出(从对偶范数的定义)
\[
\|b\| \leq c \quad \forall b \in B^*.
\]
\end{proof}

\section{开映射定理和闭图像定理}

这里是两个由Banach提出的基本结果。

\begin{theorem}[开映射定理]\label{theorem2.6}
设$E$和$F$是两个Banach空间,设$T$是从$E$到$F$的连续线性算子,且$T$是满射的(=映上的)。那么存在一个常数$c>0$使得
\begin{equation}
T(B_E(0,1)) \supset B_F(0,c). \label{eq:2.7}
\end{equation}
\end{theorem}

\begin{remark}
性质\eqref{eq:2.7}意味着$E$中任何开集的像在$F$中是开集(这证明了这个定理的名称!)。确实,让我们假设$U$是$E$中的一个开集,并让我们证明$T(U)$是开的。在$F$中任取一点$y_0 \in T(U)$,设$y_0=Tx_0$对于某个$x_0 \in U$。设$r>0$使得$B(x_0, r) \subset U$,即$x_0+B(0,r) \subset U$。由此可知
\[
y_0 + T(B(0,r)) \subset T(U).
\]
使用\eqref{eq:2.7}我们得到$T(B(0,r)) \supset B(0, rc)$,因此
\[
B(y_0, rc) \subset T(U).
\]
\end{remark}

定理\ref{theorem2.6}的一些重要推论如下。

\begin{corollary}\label{corollary2.7}
设$E$和$F$是两个Banach空间,设$T$是从$E$到$F$的连续线性算子,且$T$是双射的,即单射(=一对一)和满射。那么$T^{-1}$是连续的(从$F$到$E$)。
\end{corollary}

\begin{proof}
推论\ref{corollary2.7}的证明和定理\ref{theorem2.6}的假设立即意味着如果$x \in E$被选择使得$\|Tx\|<c$,那么$\|x\|<1$。通过齐次性,我们有
\[
\|x\| \leq \frac{1}{c} \|Tx\| \quad \forall x \in E
\]
因此$T^{-1}$是连续的。
\end{proof}

\begin{corollary}\label{corollary2.8}
设$E$是一个向量空间,赋有两个范数$\|\cdot\|_1$和$\|\cdot\|_2$。假设$E$对于两个范数都是Banach空间,并且存在一个常数$C \geq 0$使得
\[
\|x\|_2 \leq C\|x\|_1 \quad \forall x \in E.
\]
那么这两个范数是\textbf{等价的},即存在一个常数$c>0$使得
\[
\|x\|_1 \leq c\|x\|_2 \quad \forall x \in E.
\]
\end{corollary}

\begin{proof}
用$E=(E, \|\cdot\|_1)$, $F=(E, \|\cdot\|_2)$和$T=I$来应用推论\ref{corollary2.7}。
\end{proof}

\begin{proof}
我们将证明分为两个步骤。

\textbf{步骤1.} 假设$T$是从$E$到$F$的线性满射算子。那么存在一个常数$c>0$使得
\begin{equation}
\overline{T(B(0,1))} \supset B(0, 2c). \label{eq:2.8}
\end{equation}
\begin{proof}
设$X_n = n\overline{T(B(0,1))}$。由于$T$是满射的,我们有$\bigcup_{n=1}^\infty X_n = F$,并且根据贝尔纲定理,存在某个$n_0$使得$\text{Int}(X_{n_0}) \neq \emptyset$。由此可知
\[
\text{Int}[\overline{T(B(0,1))}] \neq \emptyset.
\]
选择$c>0$和$y_0 \in F$使得
\begin{equation}
B(y_0, 4c) \subset \overline{T(B(0,1))}. \label{eq:2.9}
\end{equation}
特别地,$y_0 \in \overline{T(B(0,1))}$,并且通过对称性,
\begin{equation}
-y_0 \in \overline{T(B(0,1))}. \label{eq:2.10}
\end{equation}
将\eqref{eq:2.9}和\eqref{eq:2.10}相加得到
\[
B(0, 4c) \subset \overline{T(B(0,1))} + \overline{T(B(0,1))}.
\]
另一方面,由于$T(B(0,1))$是凸的,我们有
\[
\overline{T(B(0,1))} + \overline{T(B(0,1))} = 2\overline{T(B(0,1))},
\]
\eqref{eq:2.8}也随之成立。
\end{proof}

\textbf{步骤2.} 假设$T$是从$E$到$F$的连续线性算子,满足\eqref{eq:2.8}。那么我们有
\begin{equation}
T(B(0,1)) \supset B(0,c). \label{eq:2.11}
\end{equation}
\begin{proof}
选择任意$y \in F$使得$\|y\|<c$。我们的目标是找到一个$x \in E$使得
\[
\|x\| < 1 \quad \text{和} \quad Tx=y.
\]
由\eqref{eq:2.8}我们知道
\begin{equation}
\forall \varepsilon > 0 \quad \exists z \in E \text{ with } \|z\| < \frac{1}{2} \text{ and } \|y - Tz\| < \varepsilon. \label{eq:2.12}
\end{equation}
选择$\varepsilon = c/2$,我们找到一些$z_1 \in E$使得
\[
\|z_1\| < \frac{1}{2} \quad \text{和} \quad \|y-Tz_1\| < \frac{c}{2}.
\]
通过对$y-Tz_1$应用与$c/2$(而不是$c$)相同的构造,我们找到一些$z_2 \in E$使得
\[
\|z_2\| < \frac{1}{4} \quad \text{和} \quad \|(y-Tz_1) - Tz_2\| < \frac{c}{4}.
\]
类似地进行,通过归纳我们得到一个序列$(z_n)$使得
\[
\|z_n\| < \frac{1}{2^n} \quad \text{和} \quad \|y-T(z_1+z_2+\dots+z_n)\| < \frac{c}{2^n} \quad \forall n.
\]
由此可知序列$x_n = z_1+z_2+\dots+z_n$是一个柯西序列。设$x_n \to x$,显然,$\|x\| < 1$且$y=Tx$(因为$T$是连续的)。
\end{proof}
\end{proof}

\begin{theorem}[闭图像定理]\label{theorem2.9}
设$E$和$F$是两个Banach空间,设$T$是从$E$到$F$的线性算子。假设$T$的图像$G(T)$在$E \times F$中是闭集。那么$T$是连续的。
\end{theorem}

\begin{remark}
反之亦然,因为任何连续映射的图像都是闭的。
\end{remark}

\begin{proof}
考虑$E$上的两个范数
\[
\|x\|_1 = \|x\|_E + \|Tx\|_F \quad \text{和} \quad \|x\|_2 = \|x\|_E
\]
(范数$\|\cdot\|_1$被称为\textbf{图像范数})。
很容易检查,由于$G(T)$是闭的,$(E, \|\cdot\|_1)$是一个Banach空间。另一方面,$E$也是范数$\|\cdot\|_2$的Banach空间。此外,我们显然有$\|x\|_2 \leq \|x\|_1$。由推论\ref{corollary2.8}可知,这两个范数是等价的,因此存在一个常数$c>0$使得$\|x\|_1 \leq c\|x\|_2$。我们得出$\|Tx\|_F \leq c\|x\|_E$。
\end{proof}

\section{补子空间,线性算子的左右可逆性}

我们从Banach空间中闭子空间的一些几何性质开始,这些性质源于开映射定理。

\begin{theorem}\label{theorem2.10}
设$E$是一个Banach空间。假设$G$和$L$是两个闭线性子空间,使得$G+L$是闭的。那么存在一个常数$C \geq 0$使得
\begin{equation}
\begin{cases}
\text{每个 } z \in G+L \text{ 承认一个形式为} \\
z = x+y \text{ 的分解,其中 } x \in G, y \in L, \text{ 并且 } \|x\| \leq C\|z\| \text{ 和 } \|y\| \leq C\|z\|.
\end{cases} \label{eq:2.13}
\end{equation}
\end{theorem}

\begin{proof}
考虑乘积空间$G \times L$及其范数
\[
\|[x,y]\| = \|x\| + \|y\|
\]
和空间$G+L$及其范数。映射$T: G \times L \to G+L$定义为$T[x,y] = x+y$是连续、线性和满射的。通过开映射定理,存在一个常数$c>0$使得$G+L$中的每个$z$都可以写成$z=x+y$的形式,其中$x \in G, y \in L$且$\|[x,y]\| < c$可以写成$z=x+y$的形式,其中$x \in G, y \in L$且$\|[x,y]\| < c$。通过齐次性,$G+L$中的每个$z$都可以写成
\[
z = x+y \quad \text{其中 } x \in G, y \in L, \text{ 且 } \|x\| + \|y\| \leq (1/c)\|z\|.
\]
\end{proof}

\begin{corollary}\label{corollary2.11}
在与定理\ref{theorem2.10}相同的假设下,存在一个常数$C$使得
\begin{equation}
\text{dist}(x, G \cap L) \leq C\{\text{dist}(x,G) + \text{dist}(x,L)\} \quad \forall x \in E. \label{eq:2.14}
\end{equation}
\end{corollary}

\begin{proof}
给定$x \in E$和$\varepsilon > 0$,存在$a \in G$和$b \in L$使得
\[
\|x-a\| \leq \text{dist}(x,G) + \varepsilon, \quad \|x-b\| \leq \text{dist}(x,L) + \varepsilon.
\]
性质\eqref{eq:2.13}应用于$z=a-b$说明存在$a' \in G$和$b' \in L$使得
\[
a-b = a' + b', \quad \|a'\| \leq C\|a-b\|, \quad \|b'\| \leq C\|a-b\|.
\]
由此可知$a-a' = b+b' \in G \cap L$且
\begin{align*}
\text{dist}(x, G \cap L) &\leq \|x-(a-a')\| \leq \|x-a\| + \|a'\| \\
&\leq \|x-a\| + C\|a-b\| \leq \|x-a\| + C(\|x-a\| + \|x-b\|) \\
&\leq (1+C)\text{dist}(x,G) + \text{dist}(x,L) + (1+2C)\varepsilon.
\end{align*}
最后,我们通过令$\varepsilon \to 0$得到\eqref{eq:2.14}。
\end{proof}

\begin{remark}
推论\ref{corollary2.11}的反命题也成立:如果$G$和$L$是两个闭线性子空间使得\eqref{eq:2.14}成立,那么$G+L$是闭的(见练习2.16)。
\end{remark}

\begin{definition}
设$G \subset E$是Banach空间$E$的一个闭子空间。一个子空间$L \subset E$被称为$G$的\textbf{拓扑补}或简称$G$的\textbf{补},如果
\begin{enumerate}[(i)]
    \item $L$是闭的,
    \item $G \cap L = \{0\}$且$G+L=E$。
\end{enumerate}
我们也说$G$和$L$是$E$的\textbf{互补子空间}。如果成立,那么每个$z \in E$都可以唯一地写成$z=x+y$的形式,其中$x \in G$和$y \in L$。由定理\ref{theorem2.10}可知,投影算子$z \mapsto x$和$z \mapsto y$是连续线性算子。(这个性质也可以作为互补子空间的定义。)
\end{definition}

\begin{example}
\begin{enumerate}
    \item 每个有限维子空间$G$都承认一个补。实际上,设$e_1, e_2, \dots, e_n$是$G$的一个基。$G$中的每个$x$都可以写成$x=\sum_{i=1}^n x_i e_i$。设$\varphi_i(x)=x_i$。使用Hahn-Banach(解析形式)——或者更准确地说是推论\ref{corollary1.2}——每个$\varphi_i$都可以延拓成一个连续线性泛函$\bar{\varphi}_i$定义在$E$上。很容易检查$L = \bigcap_{i=1}^n (\bar{\varphi}_i)^{-1}(0)$是$G$的补。
    \item 每个有限余维的闭子空间$G$都承认一个补。要选择任何与$G \cap L = \{0\}$和$G+L=E$的有限维空间$L$就足够了($L$是闭的,因为它是有限维的)。
    这里是这种情况的一个典型例子。设$N \subset E^*$是$p$维的一个子空间。那么
    \[ G = \{x \in E; \langle f, x \rangle = 0 \quad \forall f \in N\} = N^\perp \]
    是闭的且余维为$p$。实际上,设$f_1, f_2, \dots, f_p$是$N$的一个基。那么存在$e_1, e_2, \dots, e_p \in E$使得
    \[ \langle f_i, e_j \rangle = \delta_{ij} \quad \forall i, j=1, 2, \dots, p. \]
    [考虑映射$\Phi: E \to \mathbb{R}^p$定义为
    \[ \Phi(x) = (\langle f_1, x \rangle, \langle f_2, x \rangle, \dots, \langle f_p, x \rangle) \]
    并注意$\Phi$是满射的;否则,会存在——由Hahn-Banach(第二几何形式)——某个$\alpha=(\alpha_1, \alpha_2, \dots, \alpha_p) \neq 0$使得
    \[ \alpha \cdot \Phi(x) = \sum_{i=1}^n \alpha_i \langle f_i, x \rangle = 0 \quad \forall x \in E, \]
    这是荒谬的]。
    很容易检查由向量$(e_i)_{1 \leq i \leq p}$生成的空间是$G$的线性补。$N^\perp$的余维为$p$的另一个证明是命题\ref{proposition1.11}的事实。
    \item 在希尔伯特空间中,每个闭子空间都承认一个补(见第5.2节)。
\end{enumerate}
\end{example}

\begin{remark}
重要的是要知道一些闭子空间(即使在自反Banach空间中)\textbf{没有补}。事实上,J. Lindenstrauss和L. Tzafriri [1]的一个著名结果断言,在一个Banach空间中,每个闭子空间都承认一个补,当且仅当这个空间与一个希尔伯特空间同构。
\end{remark}

\begin{definition}
设$T \in \mathcal{L}(E,F)$。算子$S \in \mathcal{L}(F,E)$是$T$的\textbf{右逆},如果$T \circ S = I_F$。算子$S \in \mathcal{L}(F,E)$是$T$的\textbf{左逆},如果$S \circ T = I_E$。
\end{definition}

我们的下一个结果为这些逆的存在提供了必要和充分的条件。

\begin{theorem}\label{theorem2.12}
设$T \in \mathcal{L}(E,F)$是\textbf{满射}的。以下性质是等价的:
\begin{enumerate}[(i)]
    \item $T$承认一个右逆。
    \item $N(T) = T^{-1}(0)$在$E$中承认一个补。
\end{enumerate}
\end{theorem}
\begin{proof}
(i) $\implies$ (ii). 设$S$是$T$的一个右逆。很容易看出(请检查)$R(S)=S(F)$是$N(T)$的一个补。

(ii) $\implies$ (i). 设$L$是$N(T)$的一个补。设$P$是从$E$到$L$的(连续)投影算子。给定$f \in F$,我们用$x$表示方程$Tx=f$的任意解。设$S f = Px$并注意$S$独立于$x$的选择。很容易检查$S \in \mathcal{L}(F,E)$且$T \circ S = I_F$。
\end{proof}

\begin{remark}
鉴于注记8和定理\ref{theorem2.12},很容易构造出没有右逆的满射算子$T$。确实,设$G$是一个没有补的闭子空间,并设$F=E/G$,并设$T$是从$E$到$F$的典范投影(关于$E/G$的定义和性质,见第11.2节)。
\end{remark}

\begin{theorem}\label{theorem2.13}
假设$T \in \mathcal{L}(E,F)$是\textbf{单射}的。以下性质是等价的:
\begin{enumerate}[(i)]
    \item $T$承认一个左逆。
    \item $R(T)=T(E)$是闭的并且在$F$中承认一个补。
\end{enumerate}
\end{theorem}
\begin{proof}
(i) $\implies$ (ii). 很容易检查$R(T)$是闭的并且$N(S)$是$R(T)$的补[写$f=TSf+(f-TSf)$]。

(ii) $\implies$ (i). 设$P$是从$F$到$R(T)$的连续投影算子。设$f \in F$;由于$Pf \in R(T)$,存在唯一的$x \in E$使得$Tx=Pf$。设$Sf=x$。很明显$S \circ T = I_E$;此外,根据推论\ref{corollary2.7},$S$是连续的。
\end{proof}


\section{再论正交性}

这里有一些给出和或交的正交表示的简单公式。

\begin{proposition}\label{proposition2.14}
设$G$和$L$是$E$中的两个闭子空间。那么
\begin{gather}
(G \cap L)^\perp = \overline{G^\perp + L^\perp}, \label{eq:2.16} \\
(G+L)^\perp = G^\perp \cap L^\perp. \label{eq:2.17}
\end{gather}
\end{proposition}
\begin{proof}
很明显$G^\perp + L^\perp \subset (G \cap L)^\perp$;因此,如果$x \in (G^\perp+L^\perp)^\perp$并且$f \in G^\perp+L^\perp$那么$\langle f, x \rangle=0$。反之,我们有$G^\perp \subset (G \cap L)^\perp$和$L^\perp \subset (G \cap L)^\perp$。同样地,$(G^\perp+L^\perp)^\perp \subset G \cap L$。因此$(G \cap L)^\perp \supset \overline{G^\perp+L^\perp}$。
\end{proof}

\begin{corollary}\label{corollary2.15}
设$G$和$L$是$E$中的两个闭子空间。那么
\begin{gather}
(G \cap L)^\perp \supset G^\perp+L^\perp, \label{eq:2.18} \\
(G^\perp \cap L^\perp)^\perp = \overline{G+L}. \label{eq:2.19}
\end{gather}
\end{corollary}
\begin{proof}
使用命题\ref{proposition1.9}和\ref{proposition2.14}。
\end{proof}

这里有一个更深的结果。
\begin{theorem}\label{theorem2.16}
设$G$和$L$是Banach空间$E$中的两个闭子空间。以下性质是等价的:
\begin{enumerate}[(a)]
    \item $G+L$在$E$中是闭的,
    \item $G^\perp+L^\perp$在$E^*$中是闭的,
    \item $G+L = (G^\perp \cap L^\perp)^\perp$,
    \item $G^\perp+L^\perp = (G \cap L)^\perp$.
\end{enumerate}
\end{theorem}
\begin{proof}
(a) $\iff$ (c)由\eqref{eq:2.19}得出。(d) $\iff$ (b)是明显的。
我们剩下(a) $\implies$ (d)和(b) $\implies$ (a)的推导。
(a) $\implies$ (d). 鉴于\eqref{eq:2.18}足以证明$(G \cap L)^\perp \subset G^\perp+L^\perp$。给定$f \in (G \cap L)^\perp$,考虑泛函$\varphi: G+L \to \mathbb{R}$定义如下。对于每个$x \in G+L$写$x=a+b$其中$a \in G$和$b \in L$。设
\[ \varphi(x) = \langle f, a \rangle. \]
显然,$\varphi$与$x$的分解无关,且$\varphi$是线性的。另一方面,根据定理\ref{theorem2.10}我们可以选择一种分解$x$的方式使得$\|a\| \leq C\|x\|$,因此
\[ |\varphi(x)| \leq C\|f\| \|x\| \quad \forall x \in G+L. \]
将$\varphi$通过连续线性泛函$\bar{\varphi}$延拓到$E$的所有部分(见推论\ref{corollary1.2})。因此我们有
\[ f = (f-\bar{\varphi}) + \bar{\varphi} \quad \text{其中 } f-\bar{\varphi} \in G^\perp \text{ 且 } \bar{\varphi} \in L^\perp. \]
(b) $\implies$ (a). 我们从推论\ref{corollary2.11}知道存在一个常数$C$使得
\begin{equation}
\text{dist}(f, G^\perp \cap L^\perp) \leq C\{\text{dist}(f, G^\perp) + \text{dist}(f, L^\perp)\} \quad \forall f \in E^*. \label{eq:2.20}
\end{equation}
另一方面,我们有
\begin{equation}
\text{dist}(f, G^\perp) = \sup_{\substack{x \in G \\ \|x\| \leq 1}} \langle f, x \rangle \quad \forall f \in E^*. \label{eq:2.21}
\end{equation}
[使用定理\ref{theorem1.12}其中$\varphi(x)=I_{B_E}(x)-\langle f,x \rangle$和$\psi(x)=I_{G}(x)$,其中
\[ B_E = \{x \in E; \|x\| \leq 1\}.] \]
同样地,我们有
\begin{equation}
\text{dist}(f, L^\perp) = \sup_{\substack{x \in L \\ \|x\| \leq 1}} \langle f, x \rangle \quad \forall f \in E^*. \label{eq:2.22}
\end{equation}
并且由\eqref{eq:2.17})
\begin{equation}
\text{dist}(f, G^\perp \cap L^\perp) = \text{dist}(f, (G+L)^\perp) = \sup_{\substack{x \in G+L \\ \|x\| \leq 1}} \langle f, x \rangle \quad \forall f \in E^*. \label{eq:2.23}
\end{equation}
结合\eqref{eq:2.20}, \eqref{eq:2.21}和\eqref{eq:2.23}我们得到
\begin{equation}
\sup_{\substack{x \in G+L \\ \|x\| \leq 1}} \langle f, x \rangle \leq C \left\{ \sup_{\substack{x \in G \\ \|x\| \leq 1}} \langle f, x \rangle + \sup_{\substack{x \in L \\ \|x\| \leq 1}} \langle f, x \rangle \right\} \quad \forall f \in E^*. \label{eq:2.24}
\end{equation}
由此可知
\begin{equation}
\overline{B_{G+L}} \supset \frac{1}{C} B_{\overline{G+L}}. \label{eq:2.25}
\end{equation}
事实上,通过矛盾来证明,假设存在$x_0 \in \overline{G+L}$其中$\|x_0\| \leq 1/C$且$x_0 \notin \overline{B_G+B_L}$。那么存在一个严格分离$\{x_0\}$和$\overline{B_G+B_L}$的闭超平面。因此存在某个$f_0 \in E^*$和$\alpha \in \mathbb{R}$使得
\[ \langle f_0, x \rangle < \alpha < \langle f_0, x_0 \rangle \quad \forall x \in \overline{B_G+B_L}. \]
因此,我们会有
\[ \sup_{x \in G, \|x\| \leq 1} \langle f_0, x \rangle + \sup_{x \in L, \|x\| \leq 1} \langle f_0, x \rangle \leq \alpha < \langle f_0, x_0 \rangle, \]
这与\eqref{eq:2.24}和\eqref{eq:2.25}的证明相矛盾。
最后,考虑空间$X=G \times L$及其范数
\[ \|[x,y]\| = \max\{\|x\|, \|y\|\} \]
和空间$Y=\overline{G+L}$及其范数。映射$T:X \to Y$定义为$T([x,y])=x+y$是线性和连续的。从\eqref{eq:2.25}我们有
\[ \overline{T(B_X)} \supset \frac{1}{2C} B_Y. \]
使用定理\ref{theorem2.6}(开映射定理)的证明的步骤2我们得出
\[ T(B_X) \supset \frac{1}{2C} B_Y. \]
由此可知$T$从$X$到$Y$是满射的,即$G+L=\overline{G+L}$。
\end{proof}

\section{无界线性算子简介,伴随的定义}

\begin{definition}
设$E$和$F$是两个Banach空间。从$E$到$F$的\textbf{无界线性算子}是一个线性映射$A:D(A) \subset E \to F$定义在一个线性子空间$D(A) \subset E$上,取值在$F$中。$D(A)$被称为$A$的\textbf{定义域}。
如果$D(A)=E$并且存在一个常数$c \geq 0$使得
\[ \|Au\| \leq c\|u\| \quad \forall u \in E. \]
就称$A$是\textbf{有界的}(或\textbf{连续的})。
有界算子的范数定义为
\[ \|A\|_{\mathcal{L}(E,F)} = \sup_{u \neq 0} \frac{\|Au\|}{\|u\|}. \]
\end{definition}

\begin{remark}
当然可能会发生一个无界线性算子结果是有界的。这种术语稍微不一致,但通常被使用并且不会导致混淆。
\end{remark}

这里有一些重要的定义和进一步的记号:
\begin{gather*}
\text{A的图像} = G(A) = \{[u, Au]; u \in D(A)\} \subset E \times F, \\
\text{A的值域} = R(A) = \{Au; u \in D(A)\} \subset F, \\
\text{A的核} = N(A) = \{u \in D(A); Au=0\} \subset E.
\end{gather*}
如果$G(A)$在$E \times F$中是闭的,则称映射$A$是\textbf{闭的}。

\begin{remark}
为了证明一个算子$A$是闭的,一般会进行如下操作。取一个序列$(u_n)$在$D(A)$中使得$u_n \to u$在$E$中且$Au_n \to f$在$F$中。然后检查两个事实:
\begin{enumerate}[(a)]
    \item $u \in D(A)$,
    \item $f=Au$.
\end{enumerate}
注意,仅假设序列$(u_n)$使得$u_n \to 0$在$E$中且$Au_n \to f$在$F$中(并证明$f=0$)是不够的。
\end{remark}

\begin{remark}
如果$A$是闭的,那么$N(A)$是闭的;然而,$R(A)$不一定是闭的。
\end{remark}

\begin{remark}
在实践中,大多数无界算子是\textbf{闭的}并且是\textbf{稠密定义的},即$D(A)$在$E$中是稠密的。
\end{remark}

\begin{definition}
设$A:D(A) \subset E \to F$是一个\textbf{稠密定义}的无界线性算子。我们将引入一个无界线性算子$A^*: D(A^*) \subset F^* \to E^*$。首先,我们定义其定义域:
\[ D(A^*) = \{v \in F^*; \exists c \geq 0 \text{ s.t. } |\langle v, Au \rangle| \leq c\|u\| \quad \forall u \in D(A)\}. \]
\end{definition}

很明显$D(A^*)$是$F^*$的一个线性子空间。我们现在将定义$A^*v$。给定$v \in D(A^*)$,考虑映射$g:D(A) \to \mathbb{R}$定义为
\[ g(u) = \langle v, Au \rangle \quad \forall u \in D(A). \]
我们有
\[ |g(u)| \leq c\|u\| \quad \forall u \in D(A). \]
根据Hahn-Banach(解析形式;见定理\ref{theorem1.1})存在一个线性映射$f:E \to \mathbb{R}$延拓$g$并且使得
\[ |f(u)| \leq c\|u\| \quad \forall u \in E. \]
由此可知$f \in E^*$。注意$g$的延拓是唯一的,因为$D(A)$在$E$中是稠密的。
设
\[ A^*v = f. \]
无界线性算子$A^*:D(A^*) \subset F^* \to E^*$被称为$A$的\textbf{伴随}。简而言之,$A$和$A^*$之间的基本关系由
\[ \langle v, Au \rangle_{F^*,F} = \langle A^*v, u \rangle_{E^*,E} \quad \forall u \in D(A), v \in D(A^*). \]
给出。

\begin{remark}
调用Hahn-Banach来延拓$g$不是必要的。只需使用经典延拓定理,因为$g$在$D(A)$上是一致连续的,并且$E$是完备的(见,例如,H. L. Royden [1](第7章的命题11)或J. Dugundji [1](第XIV章的定理5.2))。
\end{remark}

\begin{remark}
可能会发生$D(A^*)$在$F^*$中不是稠密的(即使$A$是闭的);但这是一个相当病理的情况(见练习2.22)。如果$A$是闭的,则$D(A^*)$在$F^*$的弱$*$拓扑$\sigma(F^*,F)$中总是稠密的(见第三章,问题9)。如果$F$是自反的,那么$D(A^*)$在$F^*$的通常(范数)拓扑中是稠密的(见定理3.24)。
\end{remark}

\begin{remark}
如果$A$是一个有界算子,那么$A^*$也是一个有界算子(从$F^*$到$E^*$)并且此外,
\[ \|A^*\|_{\mathcal{L}(F^*,E^*)} = \|A\|_{\mathcal{L}(E,F)}. \]
确实,很明显$D(A^*)=F^*$。从基本关系,我们有
\[ |\langle A^*v, u \rangle| \leq \|A\| \|v\| \|u\| \quad \forall u \in E, \quad \forall v \in F^*, \]
这意味着$\|A^*v\| \leq \|A\| \|v\|$因此$\|A^*\| \leq \|A\|$。
我们也有
\[ |\langle v, Au \rangle| \leq \|A^*\| \|v\| \|u\| \quad \forall u \in E, \quad \forall v \in F^*, \]
根据推论\ref{corollary1.4},这意味着$\|Au\| \leq \|A^*\|\|u\|$因此$\|A\| \leq \|A^*\|$。
\end{remark}

\begin{proposition}\label{proposition2.17}
设$A:D(A) \subset E \to F$是一个稠密定义的无界线性算子。那么$A^*$是闭的,即$G(A^*)$在$F^* \times E^*$中是闭的。
\end{proposition}
\begin{proof}
设$v_n \in D(A^*)$使得$v_n \to v$在$F^*$中,$A^*v_n \to f$在$E^*$中。我们需要检查(a) $v \in D(A^*)$和(b) $A^*v=f$。
我们有
\[ \langle v_n, Au \rangle = \langle A^*v_n, u \rangle \quad \forall u \in D(A). \]
在极限我们得到
\[ \langle v, Au \rangle = \langle f, u \rangle \quad \forall u \in D(A). \]
因此$v \in D(A^*)$(因为$|\langle v, Au \rangle| \leq \|f\|\|u\| \quad \forall u \in D(A)$)并且$A^*v=f$。
\end{proof}
$A$和$A^*$的图像通过一个非常简单的正交关系相关联:
考虑同构$I:F^* \times E^* \to E^* \times F^*$定义为
\[ I([v,f]) = [-f, v]. \]
设$A:D(A) \subset E \to F$是一个稠密定义的无界线性算子。那么
\[ I[G(A^*)] = G(A)^\perp. \]
确实,设$[v,f] \in F^* \times E^*$,那么
\begin{align*}
[v,f] \in G(A^*) &\iff \langle f,u \rangle = \langle v,Au \rangle \quad \forall u \in D(A) \\
&\iff -\langle f,u \rangle + \langle v,Au \rangle = 0 \quad \forall u \in D(A) \\
&\iff [-f,v] \in G(A)^\perp.
\end{align*}

这里有一些关于值域和核的标准正交关系。
\begin{corollary}\label{corollary2.18}
设$A:D(A) \subset E \to F$是一个稠密定义且闭的无界线性算子。那么
\begin{enumerate}[(i)]
    \item $N(A^*) = R(A)^\perp$,
    \item $N(A) = R(A^*)_\perp$,
    \item $N(A^*)_\perp = \overline{R(A)}$,
    \item $N(A)^\perp = \overline{R(A^*)}$.
\end{enumerate}
\end{corollary}
\begin{proof}
注意(iii)和(iv)直接由(i)和(ii)结合命题\ref{proposition1.9}得出。有一个简单的(i)和(ii)的直接证明(见练习2.18)。然而,通过以下设备将这些事实与命题\ref{proposition2.14}联系起来是有益的。考虑空间$X=E \times F$,那么$X^* = E^* \times F^*$,以及$X$的子空间
\[ G=G(A) \quad \text{和} \quad L=E \times \{0\}. \]
很容易检查
\begin{gather}
N(A) \times \{0\} = G \cap L, \label{eq:2.26} \\
E \times R(A) = G+L, \label{eq:2.27} \\
\{0\} \times N(A^*) = G^\perp \cap L^\perp, \label{eq:2.28} \\
R(A^*) \times F^* = G^\perp + L^\perp. \label{eq:2.29}
\end{gather}
(i)的证明。由\eqref{eq:2.29}我们有
\[ R(A^*) \times \{0\} = (G^\perp+L^\perp) \cap G \cap L \quad \text{(由\eqref{eq:2.16})} \]
\[ = N(A) \times \{0\} \quad \text{(由\eqref{eq:2.26})}. \]
(ii)的证明。由\eqref{eq:2.27}我们有
\[ \{0\} \times R(A)^\perp = (G+L)^\perp = G^\perp \cap L^\perp \quad \text{(由\eqref{eq:2.17})} \]
\[ = \{0\} \times N(A^*) \quad \text{(由\eqref{eq:2.28})}. \]
\end{proof}

\begin{remark}
即使$A$是有界线性算子,也可能发生$N(A) \neq \overline{R(A^*)}$(见练习2.23)。然而,$N(A)^\perp$总是$R(A^*)$在弱$*$拓扑$\sigma(E^*,E)$下的闭包(见问题9)。特别地,如果$E$是自反的,那么$N(A)^\perp = \overline{R(A^*)}$。
\end{remark}

\section{算子的闭值域刻画}
主要结果关注具有闭值域的算子。

\begin{theorem}\label{theorem2.19}
设$A:D(A) \subset E \to F$是一个稠密定义且闭的无界线性算子。以下性质是等价的:
\begin{enumerate}[(i)]
    \item $R(A)$是闭的,
    \item $R(A^*)$是闭的,
    \item $R(A) = N(A^*)_\perp$,
    \item $R(A^*) = N(A)^\perp$.
\end{enumerate}
\end{theorem}
\begin{proof}
与推论\ref{corollary2.18}的证明中相同的记号,我们有
\begin{enumerate}[(i)]
    \item $G+L$在$X$中是闭的(见\eqref{eq:2.27}),
    \item $G^\perp+L^\perp$在$X^*$中是闭的(见\eqref{eq:2.29}),
    \item $G+L = (G^\perp \cap L^\perp)^\perp$(见\eqref{eq:2.27}和\eqref{eq:2.28}),
    \item $(G \cap L)^\perp = G^\perp+L^\perp$(见\eqref{eq:2.26}和\eqref{eq:2.29}).
\end{enumerate}
结论则由定理\ref{theorem2.16}得出。
\end{proof}

\begin{remark}
设$A:D(A) \subset E \to F$是一个闭无界线性算子。那么$R(A)$是闭的当且仅当存在一个常数$C$使得
\[ \text{dist}(u, N(A)) \leq C\|Au\| \quad \forall u \in D(A); \]
见练习2.14。
\end{remark}

下一个结果提供了满射算子的一个有用刻画。
\begin{theorem}\label{theorem2.20_again}
设$A:D(A) \subset E \to F$是一个稠密定义的无界线性算子。以下性质是等价的:
\begin{enumerate}[(a)]
    \item $A$是满射的,即$R(A)=F$,
    \item 存在一个常数$C$使得
    \[ \|v\| \leq C\|A^*v\| \quad \forall v \in D(A^*), \]
    (c) $N(A^*) = \{0\}$且$R(A^*)$是闭的。
\end{enumerate}
\end{theorem}
\begin{remark}
(b) $\implies$ (a)的推导有时在建立算子$A$是满射时很有用。一个过程如下。假设$v$满足$A^*v=f$。一个人试图证明$\|v\| \leq C\|f\|$(其中$C$与$f$无关)。这就是所谓的\textbf{先验估计}方法。一个人不关心方程$A^*v=f$是否有解;一个人假设$v$是一个先验解,并试图估计其范数。
\end{remark}
\begin{proof}
(a) $\iff$ (b). 设
\[ B^* = \{v \in D(A^*); \|A^*v\| \leq 1\}. \]
通过齐次性,足以证明$B^*$是有界的。为此——鉴于推论\ref{corollary2.5}(一致有界性原理)——我们只需要证明对于任何$f_0 \in F$集合$\langle B^*, f_0 \rangle$在$\mathbb{R}$中有界。由于$A$是满射的,存在某个$u_0 \in D(A)$使得$Au_0=f_0$。对于每个$v \in B^*$我们有
\[ \langle v, f_0 \rangle = \langle v, Au_0 \rangle = \langle A^*v, u_0 \rangle \]
因此$|\langle v, f_0 \rangle| \leq \|u_0\|$。
(b) $\implies$ (c). 假设$f_n = A^*v_n \to f$。使用(b)我们看到$(v_n)$是柯西序列,所以$v_n \to v$。由于$A^*$是闭的(见命题\ref{proposition2.17}),我们得出$A^*v=f$。
(c) $\implies$ (a). 由于$R(A^*)$是闭的,我们从定理\ref{theorem2.19}推断出$R(A)=N(A^*)_\perp = F$。
\end{proof}

这里有一个“对偶”陈述。
\begin{theorem}\label{theorem2.21_again}
设$A:D(A) \subset E \to F$是一个稠密定义且闭的无界线性算子。以下性质是等价的:
\begin{enumerate}[(a)]
    \item $A^*$是满射的,即$R(A^*)=E^*$,
    \item 存在一个常数$C$使得
    \[ \|u\| \leq C\|Au\| \quad \forall u \in D(A), \]
    (c) $N(A)=\{0\}$且$R(A)$是闭的。
\end{enumerate}
\end{theorem}
\begin{proof}
证明与定理\ref{theorem2.20}的证明类似,我们将其作为一个练习。
\end{proof}
\begin{remark}
如果一个人假设$\dim E < \infty$或$\dim F < \infty$,那么以下是等价的:
\begin{gather*}
A \text{ 满射} \iff A^* \text{ 单射}, \\
A^* \text{ 满射} \iff A \text{ 单射},
\end{gather*}
这确实是有限维空间中线性算子的经典结果。这些等价性成立的原因是$R(A)$和$R(A^*)$是有限维的(因此是闭的)。
在\textbf{一般情况}下,只有以下推论
\begin{gather*}
A \text{ 满射} \implies A^* \text{ 单射}, \\
A^* \text{ 满射} \implies A \text{ 单射}.
\end{gather*}
反例可以从下面的简单例子中看出。设$E=F=\ell^2$;对于每个$x \in \ell^2$写$x=(x_n)_{n \geq 1}$并设$Ax = (\frac{1}{n}x_n)_{n \geq 1}$。很容易看出$A$是一个有界算子且$A^*=A$;$A$(相应地$A^*$)是单射的但$A$(相应地$A^*$)不是满射的;$R(A)$(相应地$R(A^*)$)是稠密的而不是闭的。
\end{remark}


