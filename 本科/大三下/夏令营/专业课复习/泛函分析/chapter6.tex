\chapter{紧算子。自伴紧算子的谱分解}

\section{定义、基本性质、伴随算子}

在本章中,除非另有说明,\(E\) 和 \(F\) 表示两个巴拿赫空间。

\begin{definition}
一个有界算子 \(T \in \mathcal{L}(E, F)\) 被称为是\textbf{紧的},如果 \(T(B_E)\) 在 \(F\) 中有紧闭包(在强拓扑下)。
\end{definition}

所有从 \(E\) 到 \(F\) 的紧算子的集合用 \(K(E, F)\) 表示。为简单起见,我们记 \(K(E) = K(E, E)\)。

\begin{theorem}\label{theorem:6.1}
集合 \(K(E, F)\) 是 \(\mathcal{L}(E, F)\) 的一个闭线性子空间(在与范数 \(\|\cdot\|_{\mathcal{L}(E, F)}\) 相关的拓扑下)。
\end{theorem}

\begin{proof}
两个紧算子的和是紧算子是显然的。假设 \((T_n)\) 是 \(K(E, F)\) 中的一个算子序列,并且 \(T\) 是一个有界算子,使得 \(\|T_n - T\|_{\mathcal{L}(E, F)} \to 0\)。我们断言 \(T\) 是一个紧算子。由于 \(F\) 是完备的,为了证明 \(T(B_E)\) 是预紧的,我们只需证明对于每个 \(\varepsilon > 0\),存在一个用半径为 \(\varepsilon\) 的球对 \(T(B_E)\) 的有限覆盖(参见,例如,J. R. Munkres [1],第 7.3 节)。固定一个整数 \(n\),使得 \(\|T_n - T\|_{\mathcal{L}(E, F)} < \varepsilon/2\)。由于 \(T_n(B_E)\) 有紧闭包,存在一个半径为 \(\varepsilon/2\) 的球对 \(T_n(B_E)\) 的有限覆盖,记为 \(T_n(B_E) \subset \bigcup_{i \in I} B(f_i, \varepsilon/2)\)。由此可知 \(T(B_E) \subset \bigcup_{i \in I} B(f_i, \varepsilon)\)。
\end{proof}

\begin{definition}
一个算子 \(T \in \mathcal{L}(E, F)\) 被称为是\textbf{有限秩}的,如果它的值域 \(R(T)\) 是有限维的。
\end{definition}

显然,任何有限秩算子都是紧的,因此我们有以下推论。

\begin{corollary}\label{corollary:6.2}
设 \((T_n)\) 是一个有限秩算子序列,并设 \(T \in \mathcal{L}(E, F)\) 使得 \(\|T_n - T\|_{\mathcal{L}(E, F)} \to 0\)。那么 \(T \in K(E, F)\)。
\end{corollary}

\begin{remark}\label{remark:6.1}
著名的“逼近问题”(由巴拿赫、格罗滕迪克提出)询问推论 \ref{corollary:6.2} 的逆命题是否成立:给定一个紧算子 \(T\),是否总存在一个有限秩算子序列 \((T_n)\) 使得 \(\|T_n - T\|_{\mathcal{L}(E, F)} \to 0\)? 这个问题在很长一段时间内都是开放的,直到 1972 年 P. Enflo [1] 发现了一个反例。最初的构造非常复杂,后来又发现了更简单的例子,例如,如果 \(F\) 是 \(l^p\) 的某个闭子空间(对于任何 \(1 < p < \infty, p \ne 2\))。感兴趣的读者可以在 J. Lindenstrauss-L. Tzafriri [2] 的详细讨论中找到逼近问题的介绍。请注意,在一些特殊情况下,这个问题的答案是肯定的——例如,如果 \(F\) 是一个\textbf{希尔伯特空间}。事实上,设 \(K = \overline{T(B_E)}\)。给定 \(\varepsilon > 0\),存在一个用半径为 \(\varepsilon\) 的球对 \(K\) 的有限覆盖,即 \(K \subset \bigcup_{i \in I} B(f_i, \varepsilon)\)。设 \(G\) 表示由 \(f_i\) 张成的向量空间,并设 \(P_G\) 是到 \(G\) 上的投影,因此 \(T_\varepsilon = P_G T\) 是有限秩的。我们断言 \(\|T_\varepsilon - T\|_{\mathcal{L}(E, F)} < 2\varepsilon\)。对于每个 \(x \in B_E\),存在某个 \(i_0 \in I\) 使得
\begin{equation}\label{eq:6.1}
\|Tx - f_{i_0}\| < \varepsilon.
\end{equation}
因此,
\[
\|P_G Tx - P_G f_{i_0}\| < \varepsilon,
\]
也就是说,
\begin{equation}\label{eq:6.2}
\|P_G Tx - f_{i_0}\| < \varepsilon.
\end{equation}
结合 \eqref{eq:6.1} 和 \eqref{eq:6.2},我们得到
\[
\|P_G Tx - Tx\| < 2\varepsilon \quad \forall x \in B_E,
\]
也就是说,
\[
\|T_\varepsilon - T\|_{\mathcal{L}(E, F)} < 2\varepsilon.
\]
[更一般地,我们可以看到,如果 \(F\) 有一个 Schauder 基,那么对于每个在 \(E\) 和 \(F\) 之间的紧算子,逼近问题的答案是肯定的。]

在与逼近问题相关的方面,让我们提一个在非线性分析中非常有用的技术,用有限秩映射来逼近一个连续映射(线性的或非线性的)。设 \(X\) 是一个拓扑空间,设 \(F\) 是一个巴拿赫空间,并设 \(T: X \to F\) 是一个连续映射,使得 \(\overline{T(X)}\) 在 \(F\) 中是紧的。我们断言对于每个 \(\varepsilon > 0\),存在一个有限秩的连续映射 \(T_\varepsilon: X \to F\) 使得
\begin{equation}\label{eq:6.3}
\|T_\varepsilon(x) - T(x)\| < \varepsilon \quad \forall x \in X.
\end{equation}
事实上,由于 \(K = \overline{T(X)}\) 是紧的,存在一个对 \(K\) 的有限覆盖,即 \(K \subset \bigcup_{i \in I} B(f_i, \varepsilon/2)\)。设
\[
T_\varepsilon(x) = \frac{\sum_{i \in I} q_i(x) f_i}{\sum_{i \in I} q_i(x)} \quad \text{其中 } q_i(x) = \max\{\varepsilon - \|Tx - f_i\|, 0\};
\]
显然 \(T_\varepsilon\) 满足 \eqref{eq:6.3}。

这个近似的性质非常有用,例如,可以用来从 Brouwer 不动点定理(见,例如,K. Deimling [1], A. Granas-J. Dugundji [1], J. Franklin [1] 和练习 6.26)推导出 Schauder 不动点定理。一个类似但更精巧的构造被 Lomonosov 在一个令人惊讶的证明中用来证明非平凡不变子空间的存在性,该证明适用于一大类线性算子(见,例如,C. Pearcy [1], N. Akhiezer-I. Glazman [1], A. Granas-J. Dugundji [1] 和问题 42)。另一个基于 Schauder 不动点定理的简单证明得出的重要结果是 Krein-Rutman 定理(见定理 \ref{theorem:6.13} 和问题 41)。
\end{remark}

\begin{proposition}\label{proposition:6.3}
设 \(E, F, G\) 是三个巴拿赫空间。设 \(T \in \mathcal{L}(E, F)\) 且 \(S \in \mathcal{L}(F, G)\)。如果 \(T \in K(E, F)\) [或 \(S \in K(F, G)\)],那么 \(S \circ T \in K(E, G)\)。
\end{proposition}
\begin{proof}
证明是显然的。
\end{proof}

\begin{theorem}[Schauder]\label{theorem:6.4}
如果 \(T \in K(E, F)\),那么 \(T^* \in K(F^*, E^*)\)。反之亦然。
\end{theorem}
\begin{proof}
我们需要证明 \(T^*(B_{F^*})\) 在 \(E^*\) 中有紧闭包。设 \((v_n)\) 是 \(B_{F^*}\) 中的一个序列。设 \(K = \overline{T(B_E)}\);这是一个紧度量空间。考虑由
\[
\mathcal{H} = \{\varphi_n: x \in K \mapsto \langle v_n, x \rangle; n=1, 2, \ldots\}
\]
定义的集合 \(\mathcal{H} \subset C(K)\)。
Arzelà-Ascoli 定理(定理 4.25)的假设是满足的。因此,存在一个子序列,记为 \(\varphi_{n_k}\),它在 \(K\) 上一致收敛到一个连续函数 \(\varphi \in C(K)\)。特别地,我们有
\[
\sup_{u \in B_E} |\langle v_{n_k}, Tu \rangle - \varphi(Tu)| \to 0 \quad \text{当 } k \to \infty.
\]
因此
\[
\sup_{u \in B_E} |\langle v_{n_k} - v_{n_l}, Tu \rangle| \to 0 \quad \text{当 } k,l \to \infty,
\]
即 \(\|T^*v_{n_k} - T^*v_{n_l}\|_{E^*} \to 0\)。因此 \(T^*v_{n_k}\) 在 \(E^*\) 中收敛。
反过来,假设 \(T^* \in K(F^*, E^*)\)。我们已经从第一部分知道 \(T^{**} \in K(E^{**}, F^{**})\)。特别地,\(T^{**}(B_{E^{**}})\) 在 \(F^{**}\) 中有紧闭包。但 \(T(B_E) = T^{**}(B_E)\) 并且 \(F\) 在 \(F^{**}\) 中是闭的。因此 \(T(B_E)\) 在 \(F\) 中有紧闭包。
\end{proof}

\begin{remark}\label{remark:6.2}
设 \(E\) 和 \(F\) 是两个巴拿赫空间,并设 \(T \in K(E, F)\)。如果 \((u_n)\) 在 \(E\) 中弱收敛于 \(u\),则 \((Tu_n)\) 在 \(F\) 中强收敛于 \(Tu\)。如果 \(E\) 是自反的,则反之亦然(见练习 6.7)。
\end{remark}

\section{Riesz-Fredholm 理论}

我们从一些有用的预备结果开始。

\begin{lemma}[Riesz 引理]\label{lemma:6.1}
设 \(E\) 是一个赋范线性空间,设 \(M \subset E\) 是一个闭线性子空间,使得 \(M \ne E\)。那么
\[
\forall \varepsilon > 0 \quad \exists u \in E \text{ 使得 } \|u\| = 1 \text{ 且 } \mathrm{dist}(u, M) \ge 1 - \varepsilon.
\]
\end{lemma}
\begin{proof}
取 \(v \in E\) 且 \(v \notin M\)。由于 \(M\) 是闭的,则
\[
d = \mathrm{dist}(v, M) > 0.
\]
选择任意 \(m_0 \in M\) 使得
\[
d \le \|v - m_0\| \le d / (1 - \varepsilon).
\]
那么
\[
u = \frac{v - m_0}{\|v - m_0\|}
\]
满足所需性质。事实上,对于任意 \(m \in M\),我们有
\[
\|u - m\| = \left\| \frac{v - m_0}{\|v - m_0\|} - m \right\| = \frac{\|v - m_0 - \|v - m_0\|m\|}{\|v - m_0\|} \ge \frac{d}{\|v - m_0\|} \ge 1 - \varepsilon,
\]
因为 \(m_0 + \|v - m_0\|m \in M\)。
\end{proof}

\begin{remark}\label{remark:6.3}
如果 \(M\) 是有限维的(或者更一般地,如果 \(M\) 是自反的),我们甚至可以选择 \(\varepsilon = 0\)(见练习 1.17)。但对于一般情况这是不成立的。
\end{remark}

\begin{theorem}[Riesz]\label{theorem:6.5}
设 \(E\) 是一个赋范线性空间,其单位球 \(B_E\) 是紧的。那么 \(E\) 是有限维的。
\end{theorem}
\begin{proof}
通过反证法,假设 \(E\) 是无限维的。那么我们可以构造一个无限维子空间序列 \((E_n)\) 使得 \(E_{n-1} \subsetneq E_n\) 且 \(\dim E_n = n\)。根据 Riesz 引理(引理 \ref{lemma:6.1}),存在一个序列 \((u_n)\) 使得 \(u_n \in E_n\),\(\|u_n\| = 1\) 且 \(\mathrm{dist}(u_n, E_{n-1}) \ge 1/2\)。特别地,对于 \(m < n\),\(\|u_n - u_m\| \ge 1/2\)。因此 \((u_n)\) 没有收敛子序列,这与 \(B_E\) 是紧的这一事实相矛盾。
\end{proof}

\begin{theorem}[Fredholm 二择一]\label{theorem:6.6}
设 \(T \in K(E)\)。那么
\begin{itemize}
    \item[(a)] \(N(I-T)\) 是有限维的。
    \item[(b)] \(R(I-T)\) 是闭的,并且更精确地 \(R(I-T) = N(I-T^*)^\perp\)。
    \item[(c)] \(N(I-T) = \{0\} \iff R(I-T) = E\)。
    \item[(d)] \(\dim N(I-T) = \dim N(I-T^*)\)。
\end{itemize}
\end{theorem}

\begin{remark}\label{remark:6.4}
Fredholm 二择一处理方程 \(u-Tu=f\) 的可解性。它表明
\begin{itemize}
    \item 要么对于每个 \(f \in E\),方程 \(u-Tu=f\) 有唯一解,
    \item 要么齐次方程 \(u-Tu=0\) 有 \(n\) 个线性无关的解,在这种情况下,非齐次方程 \(u-Tu=f\) 可解当且仅当 \(f\) 满足 \(n\) 个正交性条件,即
\end{itemize}
\[
f \in N(I-T^*)^\perp.
\]
\end{remark}

\begin{remark}\label{remark:6.5}
性质 (c) 在有限维空间中是熟悉的。如果 \(\dim E < \infty\),一个从 \(E\) 到其自身的线性算子是单射的(一对一)当且仅当它是满射的(映上)。然而,在无限维空间中,一个有界算子可能是单射的但非满射,或者满射的但非单射。例如 \(l^2\) 上的右移算子(见注记 6)。因此,对于 \(T \in K(E)\) 的算子 \(I-T\) 的这一性质是显著的。
\end{remark}

\begin{proof}
(a) 设 \(E_1 = N(I-T)\)。那么 \(B_{E_1} \subset T(B_E)\) 并且因此 \(B_{E_1}\) 是紧的。根据定理 \ref{theorem:6.5},\(E_1\) 必须是有限维的。

(b) 设 \(f_n = u_n - Tu_n \to f\)。我们需要证明 \(f \in R(I-T)\)。设 \(d_n = \mathrm{dist}(u_n, N(I-T))\)。由于 \(N(I-T)\) 是有限维的,存在 \(v_n \in N(I-T)\) 使得 \(d_n = \|u_n - v_n\|\)。我们有
\begin{equation}\label{eq:6.4}
f_n = (u_n - v_n) - T(u_n - v_n).
\end{equation}
我们断言 \(\|u_n - v_n\|\) 保持有界。如果不是,则存在一个子序列使得 \(\|u_n - v_n\| \to \infty\)。设 \(w_n = (u_n - v_n) / \|u_n - v_n\|\)。从 \eqref{eq:6.4} 我们看到 \(w_n - Tw_n \to 0\)。为简单起见,选择一个进一步的子序列,我们不妨假设 \(Tw_n \to z\)。因此 \(w_n \to z\) 且 \(z \in N(I-T)\),所以 \(\mathrm{dist}(w_n, N(I-T)) \to 0\)。另一方面,
\[
\mathrm{dist}(w_n, N(I-T)) = \frac{\mathrm{dist}(u_n, N(I-T))}{\|u_n - v_n\|} = 1
\]
(因为 \(v_n \in N(I-T)\));一个矛盾。
因此 \(\|u_n-v_n\|\) 是有界的,并且由于 \(T\) 是一个紧算子,\((u_n-v_n)\) 的一个子序列收敛到一个极限 \(\ell\)。从 \eqref{eq:6.4} 我们推断 \(u_{n_k}-v_{n_k} \to f + \ell\)。设 \(g=f+\ell\),我们有 \(g-Tg=f\),即 \(f \in R(I-T)\)。这就完成了算子 \(I-T\) 的值域是闭的这一部分的证明。我们现在应用定理 2.19 并推断
\[
R(I-T) = N(I-T^*)^\perp, \quad R(I-T^*) = N(I-T)^\perp.
\]

(c) 我们首先证明 \(\implies\) 的部分。通过反证法,假设
\[
E_1 = R(I-T) \ne E.
\]
那么 \(E_1\) 是一个巴拿赫空间并且 \(T(E_1) \subset E_1\)。因此 \(T|_{E_1} \in K(E_1)\) 并且 \(E_2 = (I-T)(E_1)\) 是 \(E_1\) 的一个闭子空间。此外,\(E_2 \ne E_1\) (因为 \(I-T\) 在 \(E\) 上是单射的)。设 \(E_n = (I-T)^n(E)\)。我们得到一个严格递减的闭子空间序列 \((u_n)\) 使得 \(u_n \in E_n\), \(\|u_n\|=1\) 且 \(\mathrm{dist}(u_n, E_{n+1}) \ge 1/2\)。我们有
\[
Tu_n - Tu_m = -(u_n - Tu_n) + (u_m - Tu_m) + (u_n-u_m).
\]
注意如果 \(n>m\),那么 \(E_n \subset E_{n+1} \subset E_m\),因此
\[
-(u_n-Tu_n) + (u_m-Tu_m) \in E_{m+1}.
\]
由此可知 \(\|Tu_n - Tu_m\| \ge \mathrm{dist}(u_n, E_{m+1}) \ge 1/2\)。这是不可能的,因为 \(T\) 是紧的。
反过来,假设 \(R(I-T)=E\)。根据推论 \ref{corollary2.18} 我们知道 \(N(I-T^*) = R(I-T)^\perp = \{0\}\)。由于 \(K(E,E^*)\) 是紧的,我们可以应用前面的步骤来推断 \(R(I-T^*) = E^*\)。使用推论 \ref{corollary2.18},我们再次得出结论 \(N(I-T)=R(I-T^*)^\perp=\{0\}\)。

(d) 设 \(d=\dim N(I-T)\) 和 \(d^*=\dim N(I-T^*)\)。我们将首先证明 \(d \le d^*\)。假设不是,即 \(d>d^*\)。由于 \(N(I-T)\) 是有限维的,它在 \(E\) 中有一个补(见第 2.4 节,例 2.4)。另一方面,\(R(I-T)=N(I-T^*)^\perp\) 有余维数 \(d^*\)(见第 2.4 节,例 2),因此它有一个维数为 \(d^*\) 的补,记为 \(F\)。由于 \(d<d^*\),存在一个线性映射 \(A:N(I-T) \to F\) 它是单射的但非满射的。设 \(S=T+\Lambda \circ P\)。那么 \(S \in K(E)\),因为 \(\Lambda \circ P\) 是有限秩的。我们断言 \(N(I-S) = \{0\}\)。事实上,如果
\[
0 = u - Su = (u-Tu) - (\Lambda \circ Pu),
\]
那么
\[
u-Tu=0 \quad \text{且} \quad \Lambda \circ Pu = 0,
\]
即 \(u \in N(I-T)\) 且 \(\Lambda u = 0\)。因此 \(u=0\)。应用 (c) 到算子 \(S\),我们得到 \(R(I-S)=E\)。这是荒谬的,因为存在一些 \(f \in F\) 使得 \(f \notin R(\Lambda)\),并且方程 \(u-Su=f\) 没有解。
因此我们已经证明了 \(d \le d^*\)。应用这个结果于 \(T^*\),我们得到
\[
\dim N(I-T^{**}) \le \dim N(I-T^*) \le \dim N(I-T).
\]
但 \(N(I-T^{**}) \supset N(I-T)\),因此 \(d=d^*\)。
\end{proof}

\section{紧算子的谱}
这里有一些重要的定义。

\begin{definition}
设 \(T \in \mathcal{L}(E)\)。
\textbf{预解集},记为 \(\rho(T)\),定义为
\[
\rho(T) = \{\lambda \in \mathbb{R}; (T - \lambda I) \text{ 是从 } E \text{ 到 } E \text{ 的双射}\}.
\]
\textbf{谱},记为 \(\sigma(T)\),是预解集的补集,即 \(\sigma(T) = \mathbb{R} \setminus \rho(T)\)。
一个实数 \(\lambda\) 被称为 \(T\) 的\textbf{特征值},如果
\[
N(T - \lambda I) \ne \{0\};
\]
对应的非零向量空间被称为\textbf{特征空间}。所有特征值的集合用 \(EV(T)\) 表示。
\end{definition}
1. 记住如果 \(\lambda \in \rho(T)\),那么 \((T - \lambda I)^{-1} \in \mathcal{L}(E)\) 是有用的(见推论 \ref{corollary2.7})。
2. 很明显 \(EV(T) \subset \sigma(T)\)。一般来说,这个包含关系可以是严格的:可能存在一些 \(\lambda\) 使得
\[
N(T - \lambda I) = \{0\} \quad \text{但} \quad R(T - \lambda I) \ne E
\]
(这样的 \(\lambda\) 属于谱但不属于特征值)。例如,在 \(E = l^2\) 上考虑右移算子 \(T=(u_1, u_2, \ldots)\) 定义为 \(Tu=(0, u_1, u_2, \ldots)\)。那么 \(0 \in \sigma(T)\) 但 \(0 \notin EV(T)\)。事实上,在这种情况下 \(EV(T) = \emptyset\),而 \(\sigma(T) = [-1, +1]\)(见练习 6.18)。这当然可能发生,在有限维或无限维空间中,\(\sigma(T)=\emptyset\)。例如,\(l^2\) 中绕 \(\pi/2\) 的旋转,或者更一般地 \(l^2\) 上的算子 \(T = (-u_2, u_1, -u_4, u_3, \ldots)\)。如果我们研究复数域上的向量空间,情况就完全不同了;众所周知的有限维复数域上的特征值和谱的研究结果(它们是特征多项式的根)以及无限维复数域上的一个非平凡结果断言 \(\sigma(T)\) 总是非空的(见第 11.4 节)。然而,\(EV(T)=\emptyset\) 仍然可能发生(例如 \(E=l^2\) 的情况)。

\begin{proposition}\label{proposition:6.7}
有界算子 \(T\) 的谱 \(\sigma(T)\) 是紧的,且
\[
\sigma(T) \subset [- \|T\|, + \|T\|].
\]
\end{proposition}

\begin{proof}
设 \(\lambda \in \mathbb{R}\) 使得 \(|\lambda| > \|T\|\)。我们将证明 \(T-\lambda I\) 是双射的,这意味着 \(\sigma(T) \subset [-\|T\|, +\|T\|]\)。给定 \(f \in E\),方程 \(Tu-\lambda u = f\) 有唯一解 \(u = \lambda^{-1}(Tu-f)\),因为它可以通过收缩映射原理(定理 5.7)求解。
我们现在证明 \(\rho(T)\) 是开集。设 \(\lambda_0 \in \rho(T)\)。给定 \(\lambda \in \mathbb{R}\) (靠近 \(\lambda_0\)) 并且 \(f \in E\),我们尝试求解
\begin{equation}\label{eq:6.5}
Tu - \lambda u = f,
\end{equation}
即
\[
Tu - \lambda_0 u = f + (\lambda - \lambda_0)u,
\]
也就是说,
\begin{equation}\label{eq:6.6}
u = (T - \lambda_0 I)^{-1}[f + (\lambda - \lambda_0)u].
\end{equation}
再一次应用收缩映射原理,我们看到只要
\[
|\lambda - \lambda_0| \|(T - \lambda_0 I)^{-1}\| < 1,
\]
方程 \eqref{eq:6.6} 就有解。
\end{proof}

\begin{theorem}\label{theorem:6.8}
设 \(T \in K(E)\) 且 \(\dim E = \infty\),则我们有:
\begin{itemize}
    \item[(a)] \(0 \in \sigma(T)\)。
    \item[(b)] \(\sigma(T) \setminus \{0\} = EV(T) \setminus \{0\}\)。
    \item[(c)] 以下情况之一成立:
    \begin{itemize}
        \item \(\sigma(T) = \{0\}\)。
        \item \(\sigma(T)\) 是一个有限集。
        \item \(\sigma(T)\) 是一个收敛到 0 的序列。
    \end{itemize}
\end{itemize}
\end{theorem}

\begin{proof}
(a) 假设不是,即 \(0 \notin \sigma(T)\)。那么 \(T\) 是双射的,并且 \(I = T \circ T^{-1}\) 是紧的。因此 \(B_E\) 是紧的,并且 \(\dim E < \infty\) (根据定理 \ref{theorem:6.5});一个矛盾。

(b) 设 \(\lambda \in \sigma(T)\) 且 \(\lambda \ne 0\)。我们需要证明 \(\lambda\) 是一个特征值。假设不是,那么 \(N(T - \lambda I) = \{0\}\)。因此根据定理 \ref{theorem:6.6}(c),我们知道 \(R(T - \lambda I) = E\) 并且 \(\lambda \in \rho(T)\);一个矛盾。

为了证明的其余部分,我们将使用以下引理。
\end{proof}

\begin{lemma}\label{lemma:6.2}
设 \(T \in K(E)\) 并且设 \((\lambda_n)_{n \ge 1}\) 是一个不同的实数序列,使得
\[
\lambda_n \to \lambda
\]
且
\[
\lambda_n \in \sigma(T) \setminus \{0\} \quad \forall n.
\]
那么 \(\lambda = 0\)。
换句话说,\(\sigma(T)\setminus\{0\}\) 的所有点都是孤立点。
\end{lemma}

\begin{proof}
我们知道 \(\lambda_n \in EV(T)\);设 \(e_n \ne 0\) 使得 \((T - \lambda_n I)e_n = 0\)。令 \(E_n\) 是由 \(\{e_1, e_2, \ldots, e_n\}\) 张成的空间。我们断言,对于所有 \(n\),向量 \(e_1, e_2, \ldots, e_n\) 是线性无关的。证明通过归纳法进行。假设它对 \(n\) 成立,并假设 \(e_{n+1} = \sum_{i=1}^n \alpha_i e_i\)。那么
\[
Te_{n+1} = \sum_{i=1}^n \alpha_i \lambda_i e_i = \sum_{i=1}^n \alpha_i \lambda_{n+1} e_i.
\]
由此可知 \(\alpha_i(\lambda_i - \lambda_{n+1}) = 0\) 对于 \(i=1, 2, \ldots, n\),因此 \(\alpha_i=0\) 对于 \(i=1, 2, \ldots, n\);一个矛盾。因此我们已经证明了 \(E_n \subsetneq E_{n+1}\) 对于所有 \(n\)。
应用 Riesz 引理(引理 \ref{lemma:6.1}),我们可以构造一个序列 \((u_n)\) 使得 \(u_n \in E_n\),\(\|u_n\|=1\) 且 \(\mathrm{dist}(u_n, E_{n-1}) \ge 1/2\) 对于所有 \(n \ge 2\)。对于 \(2 \le m < n\),我们有
\[
\frac{Tu_n}{\lambda_n} - \frac{Tu_m}{\lambda_m} = u_n - \left( \frac{Tu_m}{\lambda_m} - u_n + u_m \right).
\]
我们有
\[
E_{m-1} \subset E_{n-1} \subset E_n.
\]
另一方面,很明显 \((T - \lambda_n I)E_n \subset E_{n-1}\)。因此我们有
\[
\left\| \frac{Tu_n}{\lambda_n} - \frac{Tu_m}{\lambda_m} \right\| = \left\| u_n - \frac{1}{\lambda_n}(Tu_n - \lambda_n u_n) - \frac{Tu_m}{\lambda_m} \right\| \ge \mathrm{dist}(u_n, E_{n-1}) \ge 1/2.
\]
如果 \(\lambda_n \to \lambda\) 且 \(\lambda \ne 0\),我们有一个矛盾,因为 \((Tu_n)\) 有一个收敛的子序列。
\end{proof}

\begin{proof}[定理 \ref{theorem:6.8} 的证明结束]
对于每个整数 \(n \ge 1\),集合
\[
\sigma(T) \cap \{\lambda \in \mathbb{R}; |\lambda| \ge 1/n\}
\]
是有限的或空的(如果它有无限多个不同的点,那么会有一个子序列收敛到一个 \(\lambda\) 且 \(|\lambda| \ge 1/n\)——因为 \(\sigma(T)\) 是紧的——这与引理 \ref{lemma:6.2} 相矛盾)。因此 \(\sigma(T) \setminus \{0\}\) 有无限多个点,我们可以按递减顺序排列它们。
\end{proof}

\begin{remark}\label{remark:6.7}
给定任何收敛到 0 的序列 \((\alpha_n)\),存在一个紧算子 \(T\) 使得 \(\sigma(T) = (\alpha_n) \cup \{0\}\)。在 \(l^2\) 中,定义乘法算子 \(T\) 为 \((Tu)_n = \alpha_n u_n\),其中 \(u = (u_1, u_2, \ldots, u_n, \ldots)\)。那么 \(T\) 是紧的,因为它是一系列有限秩算子 \(T_n\) 的极限,其中 \((T_n u)_i = \alpha_i u_i\) 当 \(i \le n\) 时,\(0\) 当 \(i > n\) 时。更准确地说,\(\|T_n-T\| \to 0\)。那么我们看到 \(\alpha_i\) 可能属于也可能不属于 \(EV(T)\)。如果 \(\alpha_i\) 在序列中出现有限次,则 \(\alpha_i \in EV(T)\),相应的特征空间 \(N(T)\) 是有限维或无限维的。
\end{remark}

\section{自伴紧算子的谱分解}

在接下来的内容中,我们假设 \(E=H\) 是一个希尔伯特空间,并且 \(T \in \mathcal{L}(H)\)。通过将 \(H^*\) 和 \(H\) 等同起来,我们可以将 \(T^*\) 视为从 \(H\) 到 \(H\) 的有界算子。

\begin{definition}
一个有界算子 \(T \in \mathcal{L}(H)\) 被称为是\textbf{自伴的},如果 \(T^*=T\),即
\[
(Tu, v) = (u, Tv) \quad \forall u, v \in H.
\]
\end{definition}

\begin{proposition}\label{proposition:6.9}
设 \(T \in \mathcal{L}(H)\) 是一个自伴算子。设
\[
m = \inf_{u \in H, \|u\|=1} (Tu, u) \quad \text{和} \quad M = \sup_{u \in H, \|u\|=1} (Tu, u).
\]
那么 \(\sigma(T) \subset [m, M]\),并且 \(m \in \sigma(T)\) 且 \(M \in \sigma(T)\)。此外,\(\|T\| = \max\{|m|, |M|\}\)。
\end{proposition}

\begin{proof}
设 \(\lambda > M\);我们将证明 \(\lambda \in \rho(T)\)。我们有
\[
(Tu, u) \le M \|u\|^2 \quad \forall u \in H,
\]
因此
\[
(\lambda u - Tu, u) \ge (\lambda - M)\|u\|^2 = \alpha \|u\|^2 \quad \forall u \in H, \text{其中 } \alpha > 0.
\]
应用 Lax-Milgram 定理(推论 \ref{cor:5.8}),我们推断 \(\lambda I - T\) 是双射的,因此 \(\lambda \in \rho(T)\)。类似地,任何 \(\lambda < m\) 都属于 \(\rho(T)\)。因此 \(\sigma(T) \subset [m, M]\)。
我们现在证明 \(M \in \sigma(T)\) ( \(m \in \sigma(T)\) 的证明是类似的)。定义双线性形式 \(a(u,v) = (Mu - Tu, v)\) 是对称的并且满足
\[
a(v,v) \ge 0 \quad \forall v \in H.
\]
因此它满足 Cauchy-Schwarz 不等式
\[
|a(u,v)| \le a(u,u)^{1/2} a(v,v)^{1/2} \quad \forall u, v \in H,
\]
即
\[
|(Mu - Tu, v)| \le (Mu - Tu, u)^{1/2} (Mv - Tv, v)^{1/2} \quad \forall u, v \in H.
\]
由此可知
\begin{equation}\label{eq:6.7}
|Mu - Tu| \le C (Mu - Tu, u)^{1/2} \quad \forall u \in H.
\end{equation}
根据 \(M\) 的定义,存在一个序列 \((u_n)\) 使得 \(\|u_n\|=1\) 并且 \((Tu_n, u_n) \to M\)。从 \eqref{eq:6.7} 可知 \(\|Mu_n - Tu_n\| \to 0\),因此 \(M \in \sigma(T)\) (因为如果 \(M \in \rho(T)\),那么 \(\| (MI-T)^{-1}(Mu_n - Tu_n) \to 0\),这是不可能的)。最后,我们证明 \(\|T\| = \mu\),其中 \(\mu = \max\{|m|, |M|\}\)。写出 \(u, v \in H\),
\[
(T(u+v), u+v) = (Tu, u) + (Tv, v) + 2(Tu, v),
\]
\[
(T(u-v), u-v) = (Tu, u) + (Tv, v) - 2(Tu, v),
\]
因此
\[
4(Tu, v) = (T(u+v), u+v) - (T(u-v), u-v)
\]
\[
\le M\|u+v\|^2 - m\|u-v\|^2,
\]
因此
\[
4(Tu, v) \le \mu(\|u+v\|^2 + \|u-v\|^2) = 2\mu(\|u\|^2 + \|v\|^2).
\]
用 \(\alpha v\) 替换 \(v\) 得到
\[
4|\alpha (Tu, v)| \le 2\mu \left( \|u\|^2 + |\alpha|^2 \|v\|^2 \right).
\]
接下来我们最小化右侧,即选择 \(\alpha = \|u\|/\|v\|\),然后我们得到
\[
|(Tu, v)| \le \mu \|u\| \|v\| \quad \forall u, v \in H, \text{ 所以 } \|T\| \le \mu.
\]
另一方面,很明显 \(|(Tu, u)| \le \|T\| \|u\|^2\),所以 \(|m| \le \|T\|\) 和 \(|M| \le \|T\|\),因此 \(\mu \le \|T\|\)。
\end{proof}

\begin{corollary}\label{corollary:6.10}
设 \(T \in \mathcal{L}(H)\) 是一个自伴算子,使得 \(\sigma(T)=\{0\}\)。那么 \(T=0\)。
\end{corollary}

我们最后的陈述是一个基本结果。它断言每个紧自伴算子都可以在某个合适的基上对角化。

\begin{theorem}\label{theorem:6.11}
设 \(H\) 是一个可分的希尔伯特空间,设 \(T \in \mathcal{L}(H)\) 是一个紧自伴算子。那么存在一个由 \(T\) 的特征向量组成的希尔伯特基。
\end{theorem}

\begin{proof}
设 \((\lambda_n)_{n \ge 1}\) 是 \(T\) 的(不同的)非零特征值的序列。设
\[
\lambda_0 = 0, \quad E_0 = N(T), \quad \text{且} \quad E_n = N(T - \lambda_n I).
\]
回顾
\[
0 \le \dim E_0 \le \infty \quad \text{和} \quad 0 < \dim E_n < \infty.
\]
我们断言 \(H\) 是 \(E_n\), \(n=0, 1, 2, \ldots\) 的希尔伯特和(在第 5.4 节的意义上)。
(i) 空间 \((E_n)_{n \ge 0}\) 是相互正交的。事实上,如果 \(u \in E_m\) 且 \(v \in E_n\) 且 \(m \ne n\),则
\[
Tu = \lambda_m u \quad \text{和} \quad Tv = \lambda_n v,
\]
所以
\[
(Tu, v) = \lambda_m(u, v) = (u, Tv) = \lambda_n(u,v).
\]
因此
\[
(u,v) = 0.
\]
(ii) 设 \(F\) 是由 \((E_n)_{n \ge 0}\) 张成的向量空间。我们将证明 \(F\) 在 \(H\) 中是稠密的。
很明显 \(T(F) \subset F\)。由此可知 \(T(F^\perp) \subset F^\perp\),我们有
\[
(Tu, v) = (u, Tv) = 0 \quad \forall u \in F^\perp.
\]
设 \(T_0\) 是 \(T\) 到 \(F^\perp\) 的限制。这个算子 \(T_0\) 是一个在 \(F^\perp\) 上的自伴紧算子。我们断言 \(\sigma(T_0)=\{0\}\)。假设不是;那么 \(0 \ne \lambda\) 属于 \(\sigma(T_0)\)。因此 \(\lambda \in EV(T_0)\),也就是说,存在某个 \(u \in F^\perp\),\(u \ne 0\) 使得 \(T_0 u = \lambda u\)。因此,\(\lambda\) 是 \(T\) 的特征值之一,比如 \(\lambda=\lambda_n\) 对于某个 \(n \ge 1\)。因此 \(u \in E_n\)。但 \(u \in F^\perp \cap E_n\),这意味着 \(u=0\),一个矛盾。应用推论 \ref{corollary:6.10},我们推断 \(T_0\) 在 \(F^\perp\) 上为零。这意味着 \(F^\perp \subset N(T)\) 并且因此 \(F^\perp \subset F\)。这意味着 \(F^\perp = \{0\}\),因此 \(F\) 在 \(H\) 中是稠密的。

最后,我们在每个子空间 \((E_n)_{n \ge 0}\) 中选择一个希尔伯特基(对于 \(E_0\) 存在这样的基是根据定理 5.11;对于其他的 \(E_n\), \(n \ge 1\),这是显然的,因为它们是有限维的)。这些基的并集显然是 \(H\) 的一个由 \(T\) 的特征向量组成的希尔伯特基。
\end{proof}

\begin{remark}\label{remark:6.8}
设 \(T\) 是一个紧自伴算子。根据前面的分析,我们可以将 \(H\) 中的任何元素 \(u\) 写成
\[
u = \sum_{n=0}^\infty u_n \quad \text{其中 } u_n \in E_n.
\]
那么 \(Tu = \sum_{n=1}^\infty \lambda_n u_n\)。给定一个整数 \(k \ge 1\),设
\[
T_k u = \sum_{n=1}^k \lambda_n u_n.
\]
显然,\(T_k\) 是一个有限秩算子并且
\[
\|T_k - T\| \le \sup_{n \ge k+1} |\lambda_n| \to 0 \quad \text{当 } k \to \infty.
\]
回想一下,事实上,在希尔伯特空间中,每个紧算子——不一定是自伴的——都是有限秩算子序列的极限(见注记 \ref{remark:6.1})。
\end{remark}
