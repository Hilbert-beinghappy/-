\chapter{$L^p$ 空间}

设 $(\Omega, \mathcal{M}, \mu)$ 表示一个测度空间,即 $\Omega$ 是一个集合,并且

(i) $\mathcal{M}$ 是 $\Omega$ 中的一个 $\sigma$-代数,即 $\mathcal{M}$ 是 $\Omega$ 的子集的一个集合,满足:
\begin{itemize}
    \item[ (a)] $\emptyset \in \mathcal{M}$,
    \item[ (b)] $A \in \mathcal{M} \implies A^c \in \mathcal{M}$,
    \item[ (c)] 每当 $A_n \in \mathcal{M} \quad \forall n$ 时,$\cup_{n=1}^\infty A_n \in \mathcal{M}$,
\end{itemize}

(ii) $\mu$ 是一个测度,即 $\mu: \mathcal{M} \to [0, \infty]$ 满足
\begin{itemize}
    \item[ (a)] $\mu(\emptyset) = 0$,
    \item[ (b)] 每当 $(A_n)$ 是 $\mathcal{M}$ 中可数个不交成员的族时,$\mu\left(\bigcup_{n=1}^\infty A_n\right) = \sum_{n=1}^\infty \mu(A_n)$。
\end{itemize}
$\mathcal{M}$ 的成员被称为\textit{可测集}。有时我们用 $|A|$ 代替 $\mu(A)$。我们还假设——尽管这不是必需的——

(iii) $\Omega$ 是\textit{$\sigma$-有限的},即存在 $\mathcal{M}$ 中的一个可数族 $(\Omega_n)$,使得 $\Omega = \cup_{n=1}^\infty \Omega_n$ 且 $\mu(\Omega_n) < \infty \quad \forall n$。

具有性质 $\mu(E)=0$ 的集合 $E \in \mathcal{M}$ 被称为\textit{零测集}。如果一个性质在 $\Omega$ 上除一个零测集外处处成立,我们说该性质\textit{几乎处处}(a.e.)成立(或对\textit{几乎所有} $x \in \Omega$ 成立)。

我们假设读者熟悉\textbf{可测函数}和\textbf{可积函数} $f: \Omega \to \mathbb{R}$ 的概念;参见,例如,H. L. Royden [1], G. B. Folland [2], A. Knapp [1], D. L. Cohn [1], A. Friedman [3], W. Rudin [2], P. Halmos [1], E. Hewitt–K. Stromberg [1], R. Wheeden–A. Zygmund [1], J. Neveu [1], P. Malliavin [1], A. J. Weir [1], A. Kolmogorov–S. Fomin [1], I. Fonseca–G. Leoni [1]。我们用 $L^1(\Omega, \mu)$,或简称 $L^1(\Omega)$(或仅 $L^1$),表示从 $\Omega$ 到 $\mathbb{R}$ 的可积函数空间。

我们通常用 $\int f$ 代替 $\int_\Omega f \,d\mu$,并且我们也使用记号
\[ \|f\|_{L^1} = \|f\|_1 = \int_\Omega |f| \,d\mu = \int |f|. \]
像往常一样,我们等同几乎处处相等的两个函数。我们回顾以下基本事实。

\section{一些每个人都必须知道的积分结果}

\begin{theorem}[单调收敛定理, Beppo Levi]\label{theorem4.1}
设 $(f_n)$ 是 $L^1$ 中的一个函数序列,满足
\begin{itemize}
    \item[(a)] 在 $\Omega$ 上几乎处处有 $f_1 \le f_2 \le \cdots \le f_n \le f_{n+1} \le \cdots$,
    \item[(b)] $\sup_n \int f_n < \infty$。
\end{itemize}
那么 $f_n(x)$ 在 $\Omega$ 上几乎处处收敛到一个有限极限,我们记为 $f(x)$;函数 $f$ 属于 $L^1$ 且 $\|f_n - f\|_1 \to 0$。
\end{theorem}

\begin{theorem}[控制收敛定理, Lebesgue]\label{theorem4.2}
设 $(f_n)$ 是 $L^1$ 中的一个函数序列,满足
\begin{itemize}
    \item[(a)] 在 $\Omega$ 上几乎处处有 $f_n(x) \to f(x)$,
    \item[(b)] 存在一个函数 $g \in L^1$,使得对所有 $n$,在 $\Omega$ 上几乎处处有 $|f_n(x)| \le g(x)$。
\end{itemize}
那么 $f \in L^1$ 且 $\|f_n - f\|_1 \to 0$。
\end{theorem}

\begin{lemma}[Fatou 引理]\label{lemma4.1}
设 $(f_n)$ 是 $L^1$ 中的一个函数序列,满足
\begin{itemize}
    \item[(a)] 对所有 $n$,几乎处处有 $f_n \ge 0$。
    \item[(b)] $\sup_n \int f_n < \infty$。
\end{itemize}
对几乎所有 $x \in \Omega$,我们设 $f(x) = \liminf_{n\to\infty} f_n(x) \le +\infty$。那么 $f \in L^1$ 且
\[ \int f \le \liminf_{n\to\infty} \int f_n. \]
\end{lemma}

一个基本的例子是 $\Omega = \mathbb{R}^N$,$\mathcal{M}$ 由勒贝格可测集构成,$\mu$ 是 $\mathbb{R}^N$ 上的勒贝格测度。

\textbf{注记.} 我们用 $C_c(\mathbb{R}^N)$ 表示 $\mathbb{R}^N$ 上所有具有紧支集的连续函数空间,即
\[ C_c(\mathbb{R}^N) = \{f \in C(\mathbb{R}^N); f(x)=0 \quad \forall x \in \mathbb{R}^N \setminus K, \text{ 其中 } K \text{ 是紧集}\}. \]

\begin{theorem}[稠密性]\label{theorem4.3}
空间 $C_c(\mathbb{R}^N)$ 在 $L^1(\mathbb{R}^N)$ 中是稠密的;即
\[ \forall f \in L^1(\mathbb{R}^N) \quad \forall \varepsilon > 0 \quad \exists \varphi \in C_c(\mathbb{R}^N) \text{ 使得 } \|f - \varphi\|_1 \le \varepsilon. \]
\end{theorem}

设 $(\Omega_1, \mathcal{M}_1, \mu_1)$ 和 $(\Omega_2, \mathcal{M}_2, \mu_2)$ 是两个 $\sigma$-有限的测度空间。可以在笛卡尔积 $\Omega = \Omega_1 \times \Omega_2$ 上以标准方式定义测度空间 $(\Omega, \mathcal{M}, \mu)$ 的结构。

\begin{theorem}[Tonelli]\label{theorem4.4}
设 $F(x, y): \Omega_1 \times \Omega_2 \to \mathbb{R}$ 是一个可测函数,满足
\begin{itemize}
    \item[(a)] 对几乎处处的 $x \in \Omega_1$,$\int_{\Omega_2} |F(x, y)| d\mu_2 < \infty$,
\end{itemize}
且
\begin{itemize}
    \item[(b)] $\int_{\Omega_1} d\mu_1 \int_{\Omega_2} |F(x,y)| d\mu_2 < \infty$。
\end{itemize}
则 $F \in L^1(\Omega_1 \times \Omega_2)$。
\end{theorem}

\begin{theorem}[Fubini]\label{theorem4.5}
假设 $F \in L^1(\Omega_1 \times \Omega_2)$。那么对几乎处处的 $x \in \Omega_1$,$F(x, y) \in L^1_y(\Omega_2)$ 且 $\int_{\Omega_2} F(x, y) d\mu_2 \in L^1_x(\Omega_1)$。类似地,对几乎处处的 $y \in \Omega_2$,$F(x, y) \in L^1_x(\Omega_1)$ 且 $\int_{\Omega_1} F(x, y) d\mu_1 \in L^1_y(\Omega_2)$。此外,我们有
\[ \int_{\Omega_1} d\mu_1 \int_{\Omega_2} F(x,y) d\mu_2 = \int_{\Omega_2} d\mu_2 \int_{\Omega_1} F(x, y) d\mu_1 = \iint_{\Omega_1 \times \Omega_2} F(x,y) d\mu_1 d\mu_2. \]
\end{theorem}

\section{$L^p$ 空间的定义和基本性质}

\begin{definition}
设 $p \in \mathbb{R}$ 且 $1 < p < \infty$;我们定义
\[ L^p(\Omega) = \{f: \Omega \to \mathbb{R}; f \text{ 是可测的且 } |f|^p \in L^1(\Omega)\} \]
其范数为
\[ \|f\|_{L^p} = \|f\|_p = \left[ \int_\Omega |f(x)|^p d\mu \right]^{1/p}. \]
我们稍后将验证 $\| \cdot \|_p$ 是一个范数。
\end{definition}

\begin{definition}
我们定义
\[ L^\infty(\Omega) = \{f: \Omega \to \mathbb{R}; f \text{ 是可测的且存在一个常数 } C \text{ 使得在 } \Omega \text{上几乎处处 } |f(x)| \le C \} \]
其范数为
\[ \|f\|_{L^\infty} = \|f\|_\infty = \inf\{C; \text{在 } \Omega \text{ 上几乎处处 } |f(x)| \le C \}. \]
\end{definition}
下面的注记意味着 $\| \cdot \|_\infty$ 是一个范数。
\begin{remark}
如果 $f \in L^\infty$ 那么我们有
\[ |f(x)| \le \|f\|_\infty \quad \text{在 } \Omega \text{ 上几乎处处}. \]
事实上,存在一个序列 $C_n$ 使得 $C_n \to \|f\|_\infty$ 并且对每个 $n$,在 $\Omega$ 上几乎处处有 $|f(x)| \le C_n$。因此 $|f(x)| \le C_n$ 对所有 $n$ 和所有 $x \in \Omega \setminus E_n$ 成立,其中 $|E_n|=0$。令 $E = \cup_{n=1}^\infty E_n$,那么 $|E|=0$ 且
\[ |f(x)| \le C_n \quad \forall n, \quad \forall x \in \Omega \setminus E; \]
由此得出 $|f(x)| \le \|f\|_\infty \quad \forall x \in \Omega \setminus E$。
\end{remark}

\textbf{注记.} 设 $1 \le p \le \infty$; 我们用 $p'$ 表示共轭指数,
\[ \frac{1}{p} + \frac{1}{p'} = 1. \]

\begin{theorem}[Hölder 不等式]\label{theorem4.6}
假设 $f \in L^p$ 且 $g \in L^{p'}$,其中 $1 \le p \le \infty$。那么 $fg \in L^1$ 并且
\begin{equation}\label{eq:holder}
\int |fg| \le \|f\|_p \|g\|_{p'}.
\end{equation}
\end{theorem}

\begin{proof}
当 $p=1$ 或 $p=\infty$ 时结论是显然的;因此我们假设 $1 < p < \infty$。我们回顾杨氏不等式:\footnote{有时使用形式 $ab \le \varepsilon a^p + C_\varepsilon b^{p'}$,其中 $C_\varepsilon = \varepsilon^{-1/(p-1)}$,也很方便。}
\begin{equation}\label{eq:young}
ab \le \frac{a^p}{p} + \frac{b^{p'}}{p'} \quad \forall a \ge 0, \forall b \ge 0.
\end{equation}
不等式 (\ref{eq:young}) 是函数 $t \mapsto \log t$ 在 $(0, \infty)$ 上凹性的直接推论:
\[ \log\left(\frac{1}{p}a^p + \frac{1}{p'}b^{p'}\right) \ge \frac{1}{p}\log a^p + \frac{1}{p'}\log b^{p'} = \log ab. \]
我们有
\[ |f(x)g(x)| \le \frac{1}{p}|f(x)|^p + \frac{1}{p'}|g(x)|^{p'} \quad \text{a.e. } x \in \Omega. \]
由此得出 $fg \in L^1$ 且
\begin{equation}\label{eq:holder_proof_1}
\int |fg| \le \frac{1}{p}\|f\|_p^p + \frac{1}{p'}\|g\|_{p'}^{p'}.
\end{equation}
在 (\ref{eq:holder_proof_1}) 中用 $\lambda f$ ($\lambda > 0$) 替换 $f$ 得到
\begin{equation}\label{eq:holder_proof_2}
\int |fg| \le \frac{\lambda^{p-1}}{p}\|f\|_p^p + \frac{1}{\lambda p'}\|g\|_{p'}^{p'}.
\end{equation}
选择 $\lambda = \|f\|_p^{-1} \|g\|_{p'}^{p'/p}$ 以最小化 (\ref{eq:holder_proof_2}) 的右边,我们得到 (\ref{eq:holder})。
\end{proof}

\begin{remark}
记住 Hölder 不等式的以下扩展很有用。假设 $f_1, f_2, \dots, f_k$ 是函数,使得
\[ f_i \in L^{p_i}, \quad 1 \le i \le k \text{ 且 } \frac{1}{p_1} + \frac{1}{p_2} + \dots + \frac{1}{p_k} \le 1. \]
那么乘积 $f = f_1 f_2 \cdots f_k$ 属于 $L^p$ 且
\[ \|f\|_p \le \|f_1\|_{p_1} \|f_2\|_{p_2} \cdots \|f_k\|_{p_k}. \]
特别地,如果 $f \in L^p \cap L^q$ 且 $1 \le p \le q \le \infty$,那么 $f \in L^r$ 对所有 $p \le r \le q$ 成立,并且下面的“插值不等式”成立:
\[ \|f\|_r \le \|f\|_p^\alpha \|f\|_q^{1-\alpha}, \quad \text{其中 } \frac{1}{r} = \frac{\alpha}{p} + \frac{1-\alpha}{q}, \quad 0 \le \alpha \le 1; \]
见练习 4.4。
\end{remark}

\begin{theorem}\label{theorem4.7}
$L^p$ 是一个向量空间,且 $\| \cdot \|_p$ 对任意 $1 \le p \le \infty$ 都是一个范数。
\end{theorem}
\begin{proof}
$p=1$ 和 $p=\infty$ 的情况是清楚的。因此我们假设 $1 < p < \infty$。设 $f, g \in L^p$。我们有
\[ |f(x)+g(x)|^p \le (|f(x)|+|g(x)|)^p \le 2^p(|f(x)|^p + |g(x)|^p). \]
因此,$f+g \in L^p$。另一方面,
\[ \|f+g\|_p^p = \int |f+g|^p = \int |f+g|^{p-1}|f+g| \le \int |f+g|^{p-1}|f| + \int |f+g|^{p-1}|g|. \]
但 $|f+g|^{p-1} \in L^{p'}$,通过 Hölder 不等式我们得到
\[ \|f+g\|_p^p \le \|f+g\|_p^{p/p'} (\|f\|_p + \|g\|_p), \]
即 $\|f+g\|_p \le \|f\|_p + \|g\|_p$。
\end{proof}

\begin{theorem}[Fischer-Riesz]\label{theorem4.8}
$L^p$ 对任意 $1 \le p \le \infty$ 都是一个 Banach 空间。
\end{theorem}
\begin{proof}
我们区分 $p=\infty$ 和 $1 \le p < \infty$ 的情况。

\textbf{情况 1: $p=\infty$。} 设 $(f_n)$ 是 $L^\infty$ 中的一个柯西序列。给定任意整数 $k \ge 1$,存在一个整数 $N_k$ 使得对 $m, n \ge N_k$,$\|f_m - f_n\|_\infty \le \frac{1}{k}$。因此存在一个零测集 $E_k$ 使得
\begin{equation}\label{eq:fischer_riesz_p_inf}
|f_m(x) - f_n(x)| \le \frac{1}{k} \quad \forall x \in \Omega \setminus E_k, \quad \forall m, n \ge N_k.
\end{equation}
然后我们令 $E = \cup_k E_k$——这是一个零测集——我们看到对所有 $x \in \Omega \setminus E$,序列 $(f_n(x))$ 是柯西序列(在 $\mathbb{R}$ 中)。因此 $f_n(x) \to f(x)$ 对所有 $x \in \Omega \setminus E$。在 (\ref{eq:fischer_riesz_p_inf}) 中令 $m \to \infty$ 我们得到
\[ |f(x) - f_n(x)| \le \frac{1}{k} \quad \text{对所有 } x \in \Omega \setminus E, \quad \forall n \ge N_k. \]
我们得出 $f \in L^\infty$ 并且 $\|f - f_n\|_\infty \le \frac{1}{k} \quad \forall n \ge N_k$;因此 $f_n \to f$ 在 $L^\infty$ 中。

\textbf{情况 2: $1 \le p < \infty$。} 设 $(f_n)$ 是 $L^p$ 中的一个柯西序列。为了得出序列在 $L^p$ 中收敛,提取一个子序列 $(f_{n_k})$ 使得
\[ \|f_{n_{k+1}} - f_{n_k}\|_p \le \frac{1}{2^k} \quad \forall k \ge 1. \]
[一个人可以这样进行:选择 $n_1$ 使得 $\|f_m - f_n\|_p \le \frac{1}{2}$ 对所有 $m, n \ge n_1$;然后选择 $n_2 \ge n_1$ 使得 $\|f_m - f_n\|_p \le \frac{1}{2^2}$ 对所有 $m, n \ge n_2$ 等等。] 我们断言 $f_{n_k}$ 在 $L^p$ 中收敛。为了简化符号,我们写 $f_k$ 代替 $f_{n_k}$,所以我们有
\begin{equation}\label{eq:fischer_riesz_p_finite}
\|f_{k+1} - f_k\|_p \le \frac{1}{2^k} \quad \forall k \ge 1.
\end{equation}
令
\[ g_n(x) = \sum_{k=1}^n |f_{k+1}(x) - f_k(x)|, \]
那么
\[ \|g_n\|_p \le 1. \]
作为单调收敛定理的一个推论,$g_n(x)$ 在 $\Omega$ 上几乎处处收敛到一个有限极限,且 $g \in L^p$。另一方面,对 $m \ge n \ge 2$ 我们有
\[ |f_m(x) - f_n(x)| \le |f_m(x) - f_{m-1}(x)| + \dots + |f_{n+1}(x) - f_n(x)| \le g(x) - g_{n-1}(x). \]
由此得出,在 $\Omega$ 上几乎处处,$f_n(x)$ 是柯西序列并收敛到一个有限极限,记为 $f(x)$。
我们在 $\Omega$ 上几乎处处有,
\begin{equation}\label{eq:fischer_riesz_p_finite_2}
|f(x) - f_n(x)| \le g(x) \quad \text{对 } n \ge 2,
\end{equation}
特别地 $f \in L^p$。最后,通过控制收敛我们得到 $\|f_n - f\|_p \to 0$,因为 $|f_n(x) - f(x)|^p \to 0$ 几乎处处且 $|f_n - f|^p \le g^p \in L^1$。
\end{proof}

\begin{theorem}\label{theorem4.9}
设 $(f_n)$ 是 $L^p$ 中的一个序列,且 $f \in L^p$ 使得 $\|f_n - f\|_p \to 0$。
那么存在一个子序列 $(f_{n_k})$ 和一个函数 $h \in L^p$ 使得
\begin{itemize}
    \item[(a)] 在 $\Omega$ 上几乎处处有 $f_{n_k}(x) \to f(x)$,
    \item[(b)] 在 $\Omega$ 上几乎处处有 $|f_{n_k}(x)| \le h(x) \quad \forall k$。
\end{itemize}
\end{theorem}
\begin{proof}
当 $p=\infty$ 时结论是显然的。因此我们假设 $1 \le p < \infty$。由于 $(f_n)$ 是一个柯西序列,我们可以回到定理 \ref{theorem4.8} 的证明并考虑一个子序列 $(f_{n_k})$——记为 $(g_k)$——满足 $\|g_{k+1} - g_k\|_p \le 1/2^k$。我们已经看到 $g_k(x)$ 几乎处处收敛到一个极限,该极限必须与 $f(x)$ 几乎处处相等。此外,我们有 $|g_k(x) - f(x)| \le G(x)$,其中 $G(x) = \sum_{j=k}^\infty |g_{j+1}(x) - g_j(x)|$。因此 $|g_k(x)| \le |f(x)| + G(x) = h(x)$。
\end{proof}

\section{自反性, $L^p$ 的对偶}

我们将分别考虑以下三种情况:
\begin{itemize}
    \item[(A)] $1 < p < \infty$,
    \item[(B)] $p=1$,
    \item[(C)] $p=\infty$.
\end{itemize}

\textbf{A. 研究 $1 < p < \infty$ 时的 $L^p(\Omega)$}

这是最“有利”的情况;$L^p$ 是自反的、可分的,并且 $L^p$ 的对偶是 $L^{p'}$。

\begin{theorem}\label{theorem4.10}
$L^p$ 对任意 $1 < p < \infty$ 都是自反的。
\end{theorem}

证明包括三个步骤:
\textbf{步骤 1 (Clarkson 不等式的第一部分).} 设 $2 \le p < \infty$。我们断言
\begin{equation}\label{eq:clarkson1}
\left\| \frac{f+g}{2} \right\|_p^p + \left\| \frac{f-g}{2} \right\|_p^p \le \frac{1}{2}(\|f\|_p^p + \|g\|_p^p) \quad \forall f, g \in L^p.
\end{equation}
(\ref{eq:clarkson1}) 的证明。显然,只需对 $\mathbb{R}$ 中的点证明不等式:
\[ \left| \frac{a+b}{2} \right|^p + \left| \frac{a-b}{2} \right|^p \le \frac{1}{2}(|a|^p + |b|^p) \quad \forall a, b \in \mathbb{R}. \]
我们首先注意到
\[ \alpha^p + \beta^p \le (\alpha^2 + \beta^2)^{p/2} \quad \forall \alpha, \beta \ge 0 \]
(通过齐次性,假设 $\beta=1$ 并观察函数 $(x^2+1)^{p/2} - x^p - 1$ 在 $[0, \infty)$ 上是增加的)。选择 $\alpha = |\frac{a+b}{2}|$ 和 $\beta=|\frac{a-b}{2}|$,我们得到
\[ \left| \frac{a+b}{2} \right|^p + \left| \frac{a-b}{2} \right|^p \le \left( \left| \frac{a+b}{2} \right|^2 + \left| \frac{a-b}{2} \right|^2 \right)^{p/2} = \left( \frac{a^2+b^2}{2} \right)^{p/2} \le \frac{1}{2}(|a|^p + |b|^p) \]
(最后一个不等式源于函数 $x \mapsto |x|^{p/2}$ 的凸性,因为 $p \ge 2$)。

\textbf{步骤 2:} $L^p$ 对 $2 \le p < \infty$ 是一致凸的。事实上,设 $\varepsilon > 0$ 并设 $f, g \in L^p$ 满足 $\|f\|_p \le 1$, $\|g\|_p \le 1$ 且 $\|f-g\|_p > \varepsilon$。我们从 (\ref{eq:clarkson1}) 中推断出
\[ \left\| \frac{f+g}{2} \right\|_p^p < 1 - \left(\frac{\varepsilon}{2}\right)^p \]
因此 $\|f+g/2\|_p < 1-\delta$ 其中 $\delta = 1 - [1 - (\varepsilon/2)^p]^{1/p} > 0$。因此,$L^p$ 是一致凸的,并根据定理 \ref{theorem3.31} 是自反的。

\textbf{步骤 3:} $L^p$ 对 $1 < p \le 2$ 是自反的。
\begin{proof}
设 $1 < p < \infty$。考虑算子 $T: L^p \to (L^{p'})^*$ 定义如下:设 $u \in L^p$ 是固定的;映射 $f \mapsto \int uf$ 是 $L^{p'}$ 上的连续线性泛函,因此定义了 $(L^{p'})^*$ 中的一个元素,记为 $Tu$,使得
\[ \langle Tu, f \rangle = \int uf \quad \forall f \in L^{p'}. \]
我们断言
\begin{equation}\label{eq:lp_iso}
\|Tu\|_{(L^{p'})^*} = \|u\|_p \quad \forall u \in L^p.
\end{equation}
确实,由 Hölder 不等式,我们有
\[ |\langle Tu, f \rangle| \le \|u\|_p \|f\|_{p'} \quad \forall f \in L^{p'} \]
因此 $\|Tu\|_{(L^{p'})^*} \le \|u\|_p$。另一方面,设
\[ f_0(x) = |u(x)|^{p-2}u(x) \quad (f_0(x)=0 \text{ if } u(x)=0). \]
显然我们有
\[ f_0 \in L^{p'}, \quad \|f_0\|_{p'}^{p'} = \|u\|_p^p \text{ 且 } \langle Tu, f_0 \rangle = \|u\|_p^p; \]
因此
\begin{equation}\label{eq:lp_iso_2}
\|Tu\|_{(L^{p'})^*} \ge \frac{\langle Tu, f_0 \rangle}{\|f_0\|_{p'}} = \|u\|_p.
\end{equation}
因此,我们已经证明了 $T$ 是从 $L^p$ 到 $(L^{p'})^*$ 的一个等距,这意味着 $T(L^p)$ 是 $(L^{p'})^*$ 的一个闭合子空间(因为 $L^p$ 是一个 Banach 空间)。
假设现在 $1 < p \le 2$。由于 $L^{p'}$ 是自反的(根据步骤 2),可以得出 $(L^{p'})^*$ 也是自反的(推论 \ref{corollary3.21})。因此,通过命题 \ref{prop3.20},$T(L^p)$ 是自反的,结果是 $L^p$ 也是自反的。
\end{proof}
\begin{remark}
事实上,$L^p$ 对 $1 < p \le 2$ 也是一致凸的。这是 Clarkson 第二不等式的一个推论,它对 $1 < p \le 2$ 成立:
\[ \left\| \frac{f+g}{2} \right\|_{p'}^{p'} + \left\| \frac{f-g}{2} \right\|_{p'}^{p'} \le \left(\frac{1}{2}\|f\|_p^p + \frac{1}{2}\|g\|_p^p\right)^{1/(p-1)} \quad \forall f, g \in L^p. \]
这个不等式比 Clarkson 的第一个不等式更难证明(见,例如,问题 20 或 E. Hewitt–K. Stromberg [1])。显然,它意味着 $L^p$ 对 $1 < p \le 2$ 是一致凸的;对于另一种方法,见 C. Morawetz [1] (练习 4.12) 或 J. Diestel [1]。
\end{remark}

\begin{theorem}[Riesz 表示定理]\label{theorem4.11}
设 $1 < p < \infty$ 并设 $\phi \in (L^p)^*$。那么存在一个唯一的函数 $u \in L^{p'}$ 使得
\[ \langle \phi, f \rangle = \int uf \quad \forall f \in L^p. \]
此外,
\[ \|u\|_{p'} = \|\phi\|_{(L^p)^*}. \]
\end{theorem}
\begin{remark}
定理 \ref{theorem4.11} 非常重要。它说每个在 $L^p$($1 < p < \infty$)上的连续线性泛函都可以“具体地”表示为一个积分。映射 $\phi \mapsto u$ 是一个线性满射等距,允许我们将“抽象”空间 $(L^p)^*$ 与 $L^{p'}$ 等同起来。
接下来,我们将系统地进行这种等同
\[ (L^p)^* = L^{p'}. \]
\end{remark}

\begin{proof}
我们考虑算子 $T: L^{p'} \to (L^p)^*$ 定义为 $\langle T(u, f) \rangle = \int uf \quad \forall u \in L^{p'}, \forall f \in L^p$。在定理 \ref{theorem4.10}(步骤 3)的证明中,论证表明
\[ \|Tu\|_{(L^p)^*} = \|u\|_{p'} \quad \forall u \in L^{p'}. \]
我们断言 $T$ 是满射的。确实,设 $E=T(L^{p'})$。由于 $E$ 是一个闭子空间,足以证明 $E$ 在 $(L^p)^*$ 中是稠密的。设 $h \in (L^p)^{**}$ 满足 $\langle h, Tu \rangle = 0 \quad \forall u \in L^{p'}$。由于 $L^p$ 是自反的,$h \in L^p$ 并且满足 $\int uh = 0 \quad \forall u \in L^{p'}$。选择 $u = |h|^{p-2}h$,我们看到 $h=0$。
\end{proof}

\begin{theorem}\label{theorem4.12}
空间 $C_c(\mathbb{R}^N)$ 在 $L^p(\mathbb{R}^N)$ 中是稠密的,对任意 $1 \le p < \infty$。
\end{theorem}

\textbf{注记.} 截断算子 $T_n: \mathbb{R} \to \mathbb{R}$ 定义为
\[ T_n r = \begin{cases} r & \text{如果 } |r| \le n, \\ \frac{nr}{|r|} & \text{如果 } |r| > n. \end{cases} \]
给定一个集合 $E \subset \Omega$,我们定义其特征函数\footnote{不要与第一章介绍的示性函数 $I_E$ 混淆。} $\chi_E$ 为
\[ \chi_E(x) = \begin{cases} 1 & \text{如果 } x \in E, \\ 0 & \text{如果 } x \in \Omega \setminus E. \end{cases} \]

\begin{definition}
如果由可数个成员 $(E_n)$ 生成的 $\sigma$-代数(即,包含 $(E_n)$ 的最小 $\sigma$-代数)与 $\mathcal{M}$ 重合,则称测度空间 $\Omega$ 是可分的。
\end{definition}

\begin{theorem}\label{theorem4.13}
假设 $\Omega$ 是一个可分的测度空间。那么 $L^p(\Omega)$ 对任意 $1 \le p < \infty$ 是可分的。
\end{theorem}

\textbf{B. $L^1(\Omega)$ 的研究}
我们从 $L^1(\Omega)$ 的对偶空间的描述开始。

\begin{theorem}[Riesz 表示定理]\label{theorem4.14}
设 $\phi \in (L^1)^*$。那么存在一个唯一的函数 $u \in L^\infty$ 使得
\[ \langle \phi, f \rangle = \int uf \quad \forall f \in L^1. \]
此外,
\[ \|u\|_\infty = \|\phi\|_{(L^1)^*}. \]
\end{theorem}
\begin{remark}
定理 \ref{theorem4.14} 断言 $L^1$ 上的每个连续线性泛函都可以“具体地”表示为一个积分。映射 $\phi \mapsto u$ 是一个线性满射等距,允许我们将“抽象”空间 $(L^1)^*$ 与 $L^\infty$ 等同起来。接下来,我们将系统地进行这种等同
\[ (L^1)^* = L^\infty. \]
\end{remark}

\textbf{C. $L^\infty(\Omega)$ 的研究}
我们已经知道(定理 \ref{theorem4.14})$L^\infty = (L^1)^*$。作为一个对偶空间,$L^\infty$ 具有一些不错的性质。我们有以下内容:
\begin{itemize}
    \item[(i)] 闭单位球 $B_{L^\infty}$ 在弱*拓扑 $\sigma(L^\infty, L^1)$ 中是紧的(根据定理 \ref{theorem3.16})。
    \item[(ii)] 如果 $\Omega$ 是 $\mathbb{R}^N$ 的一个可测子集,且 $(f_n)$ 是 $L^\infty(\Omega)$ 中的一个有界序列,那么存在一个子序列 $(f_{n_k})$ 和一个函数 $f \in L^\infty(\Omega)$ 使得 $f_{n_k} \to f$ 在弱*拓扑 $\sigma(L^\infty, L^1)$ 中(这是推论 \ref{corollary3.30} 和定理 \ref{theorem4.13} 的一个推论)。
\end{itemize}
然而,$L^\infty(\Omega)$ 不是自反的,除非在 $\Omega$ 由有限个原子组成的平凡情况下;否则 $L^1(\Omega)$ 将是自反的(推论 \ref{corollary3.21}),而我们知道事实并非如此(注记 6)。结果是,对偶空间 $(L^\infty)^*$ 包含 $L^1$(因为 $L^\infty = (L^1)^*$) 并且 $(L^\infty)^*$ 严格大于 $L^1$。换句话说,存在不能表示为
\[ \langle \phi, f \rangle = \int uf \quad \forall f \in L^\infty \text{ 和某个 } u \in L^1 \]
的 $L^\infty$ 上的连续线性泛函。

下表总结了当 $\Omega$ 是 $\mathbb{R}^N$ 的可测子集时空间 $L^p(\Omega)$ 的主要性质:
\begin{center}
\begin{tabular}{|c|c|c|c|}
\hline
 & \textbf{自反} & \textbf{可分} & \textbf{对偶空间} \\
\hline
$L^p$ with $1<p<\infty$ & 是 & 是 & $L^{p'}$ \\
\hline
$L^1$ & 否 & 是 & $L^\infty$ \\
\hline
$L^\infty$ & 否 & 否 & 严格大于 $L^1$ \\
\hline
\end{tabular}
\end{center}

\section{卷积与正则化}

我们首先定义函数 $f \in L^1(\mathbb{R}^N)$ 与函数 $g \in L^p(\mathbb{R}^N)$ 的卷积。

\begin{theorem}[Young]\label{theorem4.15}
设 $f \in L^1(\mathbb{R}^N)$ 且设 $g \in L^p(\mathbb{R}^N)$,其中 $1 \le p \le \infty$。
那么对几乎所有的 $x \in \mathbb{R}^N$,函数 $y \mapsto f(x-y)g(y)$ 在 $\mathbb{R}^N$ 上是可积的,并且我们定义
\[ (f \star g)(x) = \int_{\mathbb{R}^N} f(x-y)g(y)dy. \]
此外,$f \star g \in L^p(\mathbb{R}^N)$ 且
\[ \|f \star g\|_p \le \|f\|_1 \|g\|_p. \]
\end{theorem}

\begin{proof}
当 $p=\infty$ 时,结论是显然的。我们考虑两种情况:
(i) $p=1$,
(ii) $1 < p < \infty$.

\textbf{情况 (i): $p=1$。} 设 $F(x, y) = f(x-y)g(y)$。
对几乎所有的 $y \in \mathbb{R}^N$ 我们有
\[ \int |F(x,y)|dx = |g(y)| \int |f(x-y)|dx = |g(y)| \|f\|_1 < \infty \]
并且,
\[ \int dy \int |F(x,y)|dx = \|g\|_1 \|f\|_1 < \infty. \]
我们从 Tonelli 定理(定理 \ref{theorem4.4})推断出 $F \in L^1(\mathbb{R}^N \times \mathbb{R}^N)$。应用 Fubini 定理(定理 \ref{theorem4.5}),我们看到
\[ \int |F(x,y)|dy < \infty \quad \text{对几乎所有 } x \in \mathbb{R}^N \]
并且,
\[ \int dx \int |F(x,y)|dy = \int dy \int |F(x,y)|dx = \|f\|_1 \|g\|_1. \]
这恰好是定理 \ref{theorem4.15} 在 $p=1$ 时的结论。

\textbf{情况 (ii): $1 < p < \infty$。} 通过情况 (i) 我们知道对几乎所有固定的 $x \in \mathbb{R}^N$,函数 $y \mapsto |f(x-y)| |g(y)|^p$ 是可积的,即,
\[ |f(x-y)|^{1/p} |g(y)| \in L^p_y(\mathbb{R}^N). \]
由于 $|f(x-y)|^{1/p'} \in L^{p'}_y(\mathbb{R}^N)$,我们从 Hölder 不等式推断出
\[ \int |f(x-y)g(y)|dy \le \|f(x-\cdot)\|_1^{1/p'} \left( \int |f(x-y)||g(y)|^p dy \right)^{1/p}, \]
也就是说,
\[ |(f \star g)(x)| \le \|f\|_1^{1/p'} (|f| \star |g|^p)(x)^{1/p}. \]
我们通过情况 (i) 得出,如果 $g \in L^p(\mathbb{R}^N)$ 那么 $f \star g \in L^p(\mathbb{R}^N)$ 且
\[ \|f \star g\|_p^p \le \|f\|_1^{p/p'} \| |f| \star |g|^p \|_1 = \|f\|_1^{p/p'} \|f\|_1 \|g\|_p^p, \]
即,
\[ \|f \star g\|_p \le \|f\|_1 \|g\|_p. \]
\end{proof}
\textbf{注记.} 给定一个函数 $f$,我们记 $\check{f}(x) = f(-x)$。

\begin{proposition}\label{proposition4.16}
设 $f \in L^1(\mathbb{R}^N)$, $g \in L^p(\mathbb{R}^N)$ 且 $h \in L^{p'}(\mathbb{R}^N)$。那么我们有
\[ \int (f \star g)h = \int g(\check{f} \star h). \]
\end{proposition}

\begin{proof}
函数 $F(x,y) = f(x-y)g(y)h(x)$ 属于 $L^1(\mathbb{R}^N \times \mathbb{R}^N)$,因为
\[ \iint |F(x,y)|dx dy = \int |g(y)|dy \int |f(x-y)||h(x)|dx < \infty \]
通过定理 \ref{theorem4.15} 和 Hölder 不等式。因此我们有
\[ \int (f \star g)(x) h(x) dx = \iint F(x,y) dy dx = \int dy \int F(x,y) dx = \int g(y) (\check{f} \star h)(y) dy. \]
\end{proof}

\textbf{支集与卷积.} 函数的支集(support)的概念是标准的:supp $f$ 是使 $f$ 不为零的最大开集的补集;换句话说,supp $f$ 是集合 $\{x; f(x) \ne 0\}$ 的闭包。当处理等价类时,这个概念是不够的,就像在空间 $L^p$ 中一样。当 $f_1=f_2$ a.e. 时,我们需要一个定义,它能确保 supp $f_1$ = supp $f_2$(或者在一个零测集上不同)。读者会轻易地确信,通常的定义没有意义;例如,如果 $f = \chi_Q$ 在 $\mathbb{R}$ 上,我们有 $f=0$ a.e. 但 supp $f = \bar{Q}$。为此,我们引入适当的概念。

\begin{proposition}[以及支集的定义]\label{proposition4.17}
设 $f: \mathbb{R}^N \to \mathbb{R}$ 是任意函数。考虑 $\mathbb{R}^N$ 中所有开集 $\omega$ 的族 $(\omega_i)_{i \in I}$,使得对每个 $i \in I$, $f=0$ 在 $\omega_i$ 上几乎处处成立。设 $\omega = \cup_{i \in I} \omega_i$。那么 $f=0$ 在 $\omega$ 上几乎处处成立。
\end{proposition}

\textbf{定义.} \textit{supp $f$} 是 $\mathbb{R}^N$ 中 $\omega$ 的补集。

\begin{remark}
(a) 假设 $f_1 = f_2$ 在 $\mathbb{R}^N$ 上几乎处处成立;显然我们有 supp $f_1$ = supp $f_2$。因此我们可以在不说它在等价类中定义的情况下,谈论 $L^p$ 中函数的支集。
(b) 如果 $f$ 是 $\mathbb{R}^N$ 上的连续函数,则很容易检验支集的新定义与通常的定义一致。
\end{remark}

\begin{proof}[命题 \ref{proposition4.17} 的证明]
由于我们可能要恢复可数情况,集合 $I$ 不一定是可数的这一事实并不清楚。在 $\omega$ 上有一个可数的开集族 $(O_n)$,使得每个开集 $O_n$ 是某个 $\omega_i$ 的子集,并且 $\omega = \cup_n O_n$。写 $\omega_i = \cup_{n \in A_i} O_n$ 并且 $\omega = \cup_{i \in I} \omega_i = \cup_{n \in B} O_n$,其中 $B = \cup_{i \in I} A_i$。由于 $f=0$ 在每个 $O_n$ 上几乎处处成立,其中 $n \in B$,我们得出 $f=0$ 在 $\omega$ 上几乎处处成立。
\end{proof}

\begin{proposition}\label{proposition4.18}
设 $f \in L^1(\mathbb{R}^N)$ 且 $g \in L^p(\mathbb{R}^N)$,其中 $1 \le p \le \infty$。那么
\[ \mathrm{supp}(f \star g) \subset \overline{\mathrm{supp}\,f + \mathrm{supp}\,g}. \]
\end{proposition}

\begin{proof}
固定 $x \in \mathbb{R}^N$,使得函数 $y \mapsto f(x-y)g(y)$ 是可积的(见定理 \ref{theorem4.15})。我们有
\[ (f \star g)(x) = \int f(x-y)g(y)dy = \int_{(x-\mathrm{supp}\,f) \cap \mathrm{supp}\,g} f(x-y)g(y)dy. \]
如果 $x \notin \overline{\mathrm{supp}\,f + \mathrm{supp}\,g}$,则 $(x-\mathrm{supp}\,f) \cap \mathrm{supp}\,g = \emptyset$ 并且因此 $(f \star g)(x) = 0$。因此 $(f \star g)(x) = 0$ 在 $(\overline{\mathrm{supp}\,f + \mathrm{supp}\,g})^c$ 上几乎处处成立。特别地,
\[ (f \star g)(x) = 0 \quad \text{a.e. on } \mathrm{Int}[(\overline{\mathrm{supp}\,f + \mathrm{supp}\,g})^c] \]
因此
\[ \mathrm{supp}(f \star g) \subset \overline{\mathrm{supp}\,f + \mathrm{supp}\,g}. \]
\end{proof}

\begin{remark}
10. 如果 $f$ 和 $g$ 都有紧支集,那么 $f \star g$ 也有紧支集。然而,如果它们中只有一个有紧支集,$f \star g$ 不一定有紧支集。
\end{remark}

\begin{definition}
设 $\Omega \subset \mathbb{R}^N$ 是一个开集,设 $1 \le p \le \infty$。我们说函数 $f: \Omega \to \mathbb{R}$ 属于 $L^p_{loc}(\Omega)$,如果对每个紧集 $K \subset \Omega$,$f\chi_K \in L^p(\Omega)$。
注意如果 $f \in L^p(\Omega)$,那么 $f \in L^p_{loc}(\Omega)$。
\end{definition}

\begin{proposition}\label{proposition4.19}
设 $f \in C_c(\mathbb{R}^N)$ 且 $g \in L^1_{loc}(\mathbb{R}^N)$。那么 $(f \star g)(x)$ 对每个 $x \in \mathbb{R}^N$ 都有定义,并且 $(f \star g) \in C(\mathbb{R}^N)$。
\end{proposition}

\begin{proof}
注意到对每个 $x \in \mathbb{R}^N$,函数 $y \mapsto f(x-y)g(y)$ 是可积的,并且因此 $(f \star g)(x)$ 对每个 $x \in \mathbb{R}^N$ 都有定义。
设 $x_n \to x$ 且设 $K$ 是 $\mathbb{R}^N$ 中的一个紧集,使得 $(x_n-\mathrm{supp}\,f) \subset K$ $\forall n$。我们有 $f(x_n - y) = 0$ $\forall n, \forall y \notin K$。我们从 $f$ 的一致连续性推断出
\[ |f(x_n - y) - f(x-y)| \le \varepsilon_n \chi_K(y) \quad \forall n, \quad \forall y \in \mathbb{R}^N \]
其中 $\varepsilon_n \to 0$。我们得出结论
\[ |(f \star g)(x_n) - (f \star g)(x)| \le \varepsilon_n \int_K |g(y)|dy \to 0. \]
\end{proof}

\textbf{注记.} 设 $\Omega \subset \mathbb{R}^N$ 是一个开集。
$C(\Omega)$ 是 $\Omega$ 上连续函数的空间。
$C^k(\Omega)$ 是 $\Omega$ 上 $k$ 次连续可微函数的空间($k \ge 1$ 是一个整数)。
$C^\infty(\Omega) = \cap_k C^k(\Omega)$。
$C_c(\Omega)$ 是 $\Omega$ 上具有紧支集的连续函数的空间,即在某个紧集 $K \subset \Omega$ 之外为零。
$C_c^k(\Omega) = C^k(\Omega) \cap C_c(\Omega)$。
$C_c^\infty(\Omega) = C^\infty(\Omega) \cap C_c(\Omega)$。
(一些作者用 $\mathcal{D}(\Omega)$ 或 $C_0^\infty(\Omega)$ 代替 $C_c^\infty(\Omega)$)。
如果 $f \in C^1(\Omega)$,其梯度定义为
\[ \nabla f = \left( \frac{\partial f}{\partial x_1}, \frac{\partial f}{\partial x_2}, \dots, \frac{\partial f}{\partial x_N} \right). \]
如果 $f \in C^k(\Omega)$ 且 $\alpha = (\alpha_1, \alpha_2, \dots, \alpha_N)$ 是一个长度为 $|\alpha| = \alpha_1 + \dots + \alpha_N$ 的多重指标,小于 $k$,我们写
\[ D^\alpha f = \frac{\partial^{|\alpha|} f}{\partial x_1^{\alpha_1} \partial x_2^{\alpha_2} \cdots \partial x_N^{\alpha_N}}. \]

\begin{proposition}\label{proposition4.20}
设 $f \in C_c^k(\mathbb{R}^N)$ ($k \ge 1$) 且设 $g \in L^1_{loc}(\mathbb{R}^N)$。那么 $f \star g \in C^k(\mathbb{R}^N)$ 且
\[ D^\alpha(f \star g) = (D^\alpha f) \star g \quad \forall \alpha \text{ with } |\alpha| \le k. \]
特别地,如果 $f \in C_c^\infty(\mathbb{R}^N)$ 且 $g \in L^1_{loc}(\mathbb{R}^N)$,那么 $f \star g \in C^\infty(\mathbb{R}^N)$。
\end{proposition}

\begin{proof}
通过归纳法,只需考虑 $k=1$ 的情况。给定 $x \in \mathbb{R}^N$,我们声称 $f \star g$ 在 $x$ 处可微,并且
\[ \nabla(f \star g)(x) = (\nabla f) \star g(x). \]
设 $h \in \mathbb{R}^N$ 且 $|h| < 1$。我们有,对所有 $y \in \mathbb{R}^N$,
\[ f(x+h-y) - f(x-y) - h \cdot \nabla f(x-y) = \int_0^1 [h \cdot \nabla f(x+sh-y) - h \cdot \nabla f(x-y)] ds \]
\[ = \int_0^1 [h \cdot \nabla f(x+sh-y) - h \cdot \nabla f(x-y)] ds \le |h| \varepsilon(|h|) \]
其中 $\varepsilon(|h|) \to 0$ 当 $|h| \to 0$ 时(因为 $\nabla f$ 在 $\mathbb{R}^N$ 上是一致连续的)。
设 $K$ 是 $\mathbb{R}^N$ 中的一个紧集,使得对所有 $x$ 和 $|h|<1$, supp $f(x+h-\cdot) \subset K$。我们有
\[ f(x+h-y) - f(x-y) - h \cdot \nabla f(x-y) = 0 \quad \forall y \notin K, \quad \forall h \in B(0,1) \]
因此
\[ |f(x+h-y) - f(x-y) - h \cdot \nabla f(x-y)| \le |h|\varepsilon(|h|)\chi_K(y) \quad \forall y \in \mathbb{R}^N, \forall h \in B(0,1). \]
我们得出结论,对 $h \in B(0,1)$,
\[ |(f \star g)(x+h) - (f \star g)(x) - h \cdot (\nabla f \star g)(x)| \le |h|\varepsilon(|h|) \int_K |g(y)| dy. \]
由此可见,$f \star g$ 在 $x$ 处可微且 $\nabla(f \star g)(x) = (\nabla f) \star g(x)$。
\end{proof}

\subsection*{柔化子}

\begin{definition}
\textbf{柔化子序列}是指 $\mathbb{R}^N$ 上的函数序列 $(\rho_n)_{n \ge 1}$,满足
\[ \rho_n \in C_c^\infty(\mathbb{R}^N), \quad \mathrm{supp}\,\rho_n \subset B(0, 1/n), \quad \int \rho_n = 1, \quad \rho_n \ge 0 \text{ on } \mathbb{R}^N. \]
\end{definition}

在下文中,我们将系统地使用符号 $(\rho_n)$ 来表示一个柔化子序列。
很容易从一个函数 $\rho \in C_c^\infty(\mathbb{R}^N)$,$\rho \ge 0$ 且 $\rho$ 不恒等于零来构造一个柔化子序列——例如函数
\[ \rho(x) = \begin{cases} e^{1/(|x|^2-1)} & \text{如果 } |x| < 1, \\ 0 & \text{如果 } |x| \ge 1. \end{cases} \]
我们通过令 $\rho_n(x) = C n^N \rho(nx)$ 得到一个柔化子序列,其中 $C=1/\int \rho$。

\begin{proposition}\label{proposition4.21}
假设 $f \in C(\mathbb{R}^N)$。那么 $(\rho_n \star f) \to f$ 在 $\mathbb{R}^N$ 的紧集上一致收敛。
\end{proposition}

\begin{proof}
设 $K \subset \mathbb{R}^N$ 是一个固定的紧集。给定 $\varepsilon > 0$,存在 $\delta > 0$(依赖于 $K$ 和 $\varepsilon$),使得
\[ |f(x-y)-f(x)| < \varepsilon \quad \forall x \in K, \quad \forall y \in B(0, \delta). \]
我们有,对 $x \in K$,
\[ (\rho_n \star f)(x) - f(x) = \int [f(x-y)-f(x)] \rho_n(y) dy = \int_{B(0,1/n)} [f(x-y)-f(x)] \rho_n(y) dy. \]
对 $n > 1/\delta$ 和 $x \in K$,我们得到
\[ |(\rho_n \star f)(x) - f(x)| \le \varepsilon \int \rho_n = \varepsilon. \]
\end{proof}

\begin{theorem}\label{theorem4.22}
假设 $f \in L^p(\mathbb{R}^N)$,其中 $1 \le p < \infty$。那么 $(\rho_n \star f) \to f$ 在 $L^p(\mathbb{R}^N)$ 中。
\end{theorem}
\begin{proof}
给定 $\varepsilon > 0$,找一个函数 $f_1 \in C_c(\mathbb{R}^N)$ 使得 $\|f - f_1\|_p < \varepsilon$ (见定理 \ref{theorem4.12})。根据命题 \ref{proposition4.21} 我们知道 $(\rho_n \star f_1) \to f_1$ 在 $\mathbb{R}^N$ 的每个紧集上一致收敛。另一方面,我们有 (通过命题 \ref{proposition4.18})
\[ \mathrm{supp}(\rho_n \star f_1) \subset \overline{B(0, 1/n) + \mathrm{supp}\,f_1}, \]
这是一个紧集。由此得出
\[ \|\rho_n \star f_1 - f_1\|_p \to 0 \quad \text{当 } n \to \infty. \]
最后,我们写
\[ (\rho_n \star f) - f = [\rho_n \star (f - f_1)] + [(\rho_n \star f_1) - f_1] + [f_1 - f] \]
因此
\[ \|\rho_n \star f - f\|_p \le 2\|f-f_1\|_p + \|\rho_n \star f_1 - f_1\|_p \]
(通过定理 \ref{theorem4.15})。
我们得出
\[ \limsup_{n\to\infty} \|\rho_n \star f - f\|_p \le 2\varepsilon \quad \forall \varepsilon > 0 \]
因此 $\lim_{n\to\infty} \|\rho_n \star f - f\|_p = 0$。
\end{proof}

\begin{corollary}\label{corollary4.23}
设 $\Omega \subset \mathbb{R}^N$ 是一个开集。那么 $C_c^\infty(\Omega)$ 在 $L^p(\Omega)$ 中是稠密的,对任意 $1 \le p < \infty$。
\end{corollary}
\footnote{通过卷积进行正则化的技术最初由 Leray 和 Friedrichs 引入。}

\begin{proof}
给定 $f \in L^p(\Omega)$ 我们设
\[ \tilde{f}(x) = \begin{cases} f(x) & \text{如果 } x \in \Omega, \\ 0 & \text{如果 } x \in \mathbb{R}^N \setminus \Omega, \end{cases} \]
所以 $\tilde{f} \in L^p(\mathbb{R}^N)$。
设 $(K_n)$ 是 $\mathbb{R}^N$ 中的一个紧集序列,使得
\[ \bigcup_{n=1}^\infty K_n = \Omega \quad \text{且} \quad \mathrm{dist}(K_n, \Omega^c) \ge 2/n \quad \forall n. \]
[我们可以选择,例如,$K_n = \{x \in \mathbb{R}^N; |x| \le n \text{ and } \mathrm{dist}(x, \Omega^c) \ge 2/n\}$。]
设 $g_n = \chi_{K_n} \tilde{f}$ 且 $f_n = \rho_n \star g_n$,所以
\[ \mathrm{supp}\,f_n \subset \overline{B(0,1/n) + K_n} \subset \Omega. \]
由此得出 $f_n \in C_c^\infty(\Omega)$。另一方面,我们有
\[ \|f_n - f\|_{L^p(\Omega)} \le \|f_n - \tilde{f}\|_{L^p(\mathbb{R}^N)} \le \|(\rho_n \star g_n) - g_n\|_{L^p(\mathbb{R}^N)} + \|g_n - \tilde{f}\|_{L^p(\mathbb{R}^N)}. \]
最后,注意 $\|g_n - \tilde{f}\|_{L^p(\mathbb{R}^N)} \to 0$ 通过控制收敛,并且 $\|(\rho_n \star g_n) - g_n\|_{L^p(\mathbb{R}^N)} \to 0$ 通过定理 \ref{theorem4.22}。我们得出结论 $\|f_n - f\|_{L^p(\Omega)} \to 0$。
\end{proof}

\begin{corollary}\label{corollary4.24}
设 $\Omega \subset \mathbb{R}^N$ 是一个开集,设 $u \in L^1_{loc}(\Omega)$ 使得
\[ \int u \varphi = 0 \quad \forall \varphi \in C_c^\infty(\Omega). \]
那么 $u=0$ 在 $\Omega$ 上几乎处处成立。
\end{corollary}
\begin{proof}
设 $g \in L^\infty(\mathbb{R}^N)$ 是一个支集为紧集且包含在 $\Omega$ 中的函数。
设 $g_n = \rho_n \star g$,使得对足够大的 $n$,$\mathrm{supp}\,g_n$ 是紧集且包含在 $\Omega$ 中。因此我们有
\begin{equation}\label{eq:4.24_1_temp}
\int u g_n = 0 \quad \forall n. \tag{19}
\end{equation}
由于 $g_n \to g$ 在 $L^1(\mathbb{R}^N)$ 中(通过定理 \ref{theorem4.22}),存在一个子序列——仍然记为 $g_n$——使得 $g_n \to g$ 在 $\mathbb{R}^N$ 上几乎处处成立(见定理 \ref{theorem4.9})。此外,我们有 $\|g_n\|_{L^\infty(\mathbb{R}^N)} \le \|g\|_{L^\infty(\mathbb{R}^N)}$。在 (19) 式中取极限(通过控制收敛),我们得到
\begin{equation}\label{eq:4.24_2_temp}
\int u g = 0. \tag{20}
\end{equation}
设 $K$ 是包含在 $\Omega$ 中的一个紧集。我们选择函数 $g$ 如下
\[ g = \begin{cases} \mathrm{sign}\,u & \text{在 } K \text{上}, \\ 0 & \text{在 } \mathbb{R}^N \setminus K \text{上}. \end{cases} \]
我们从 (20) 式推断出 $\int_K |u| = 0$,因此 $u=0$ 在 $K$ 上几乎处处成立。由于这对任何紧集 $K \subset \Omega$ 都成立,我们得出 $u=0$ 在 $\Omega$ 上几乎处处成立。
\end{proof}

\section{$L^p$ 中强紧性的判据}

能够判断 $L^p(\Omega)$ 中的一个函数族是否在 $L^p(\Omega)$ 的强拓扑下有紧闭包是很重要的。我们回顾 Ascoli-Arzelà 定理,它回答了在 $C(K)$ 空间中的同样问题,其中 $K$ 是紧度量空间。

\begin{theorem}[Ascoli-Arzelà]\label{theorem4.25}
设 $K$ 是一个紧度量空间,设 $\mathcal{H}$ 是 $C(K)$ 的一个有界子集。假设 $\mathcal{H}$ 是等度连续的,即
\begin{equation}\label{eq:ascoli_arzelà}
\forall \varepsilon > 0 \quad \exists \delta > 0 \text{ 使得 } d(x_1, x_2) < \delta \implies |f(x_1) - f(x_2)| < \varepsilon \quad \forall f \in \mathcal{H}. \tag{21}
\end{equation}
那么 $\mathcal{H}$ 在 $C(K)$ 中有紧闭包。
\end{theorem}
[关于 Ascoli-Arzelà 定理的证明,见,例如,W. Rudin [1], [2], A. Knapp [1], J. R. Munkres [1], A. Friedman [3], L. C. Royden [1], J. Dixmier [1], G. B. Folland [2], K. Yosida [1]]。

\textbf{注记 (平移算子).} 我们设 $(\tau_h f)(x) = f(x+h)$,其中 $x \in \mathbb{R}^N, h \in \mathbb{R}^N$。

下面的定理及其推论是 Ascoli-Arzelà 定理的 "$L^p$ 版本"。

\begin{theorem}[Kolmogorov-M. Riesz-Fréchet]\label{theorem4.26}
设 $\mathcal{F}$ 是 $L^p(\mathbb{R}^N)$ 中的一个有界集,其中 $1 \le p < \infty$。假设
\begin{equation}\label{eq:kolmogorov_riesz_frechet}
\lim_{|h| \to 0} \|\tau_h f - f\|_p = 0 \quad \text{在 } f \in \mathcal{F} \text{ 中一致}. \tag{22}
\end{equation}
i.e., $\forall \varepsilon > 0 \quad \exists \delta > 0$ 使得 $\forall f \in \mathcal{F}, \forall h \in \mathbb{R}^N$ with $|h| < \delta, \|\tau_h f - f\|_p < \varepsilon$.
那么,$\mathcal{F}|_\Omega$ 在 $L^p(\Omega)$ 中对任何有限测度的可测集 $\Omega \subset \mathbb{R}^N$ 都有紧闭包。
[这里 $\mathcal{F}|_\Omega$ 表示 $\mathcal{F}$ 中函数在 $\Omega$ 上的限制。]
\footnote{假设 (22) 应与 (21) 进行比较。它是一个"积分"等度连续性假设。}
\end{theorem}
\begin{proof}
证明包括四个步骤:
\textbf{步骤 1:} 我们断言
\begin{equation}\label{eq:step1_proof_4.26}
\| \rho_n \star f - f \|_p \le \varepsilon \quad \forall f \in \mathcal{F}, \quad \forall n > 1/\delta. \tag{23}
\end{equation}
确实,我们有
\[ (\rho_n \star f)(x) - f(x) = \int [f(x-y)-f(x)] \rho_n(y)dy = \int [(\tau_{-y}f)(x) - f(x)]\rho_n(y)dy. \]
由 Hölder 不等式。因此我们得到
\[ \int |(\rho_n \star f)(x) - f(x)|^p dx \le \int \left[ \int |f(x-y)-f(x)|^p \rho_n(y)dy \right]^{1/p} dx \le \int_{B(0,1/n)} \rho_n(y) \left[ \int |f(x-y)-f(x)|^p dx \right] dy \le \varepsilon^p, \]
只要 $1/n < \delta$。

\textbf{步骤 2:} 我们断言
\begin{equation}\label{eq:step2_proof_4.26_1}
\|\rho_n \star f\|_{L^\infty(\mathbb{R}^N)} \le C_n \|f\|_{L^p(\mathbb{R}^N)} \quad \forall f \in \mathcal{F}, \tag{24}
\end{equation}
以及
\begin{equation}\label{eq:step2_proof_4.26_2}
|(\rho_n \star f)(x_1) - (\rho_n \star f)(x_2)| \le C_n \|f\|_p |x_1 - x_2| \quad \forall f \in \mathcal{F}, \forall x_1, x_2 \in \mathbb{R}^N, \tag{25}
\end{equation}
其中 $C_n$ 只依赖于 $n$。
不等式 (\ref{eq:step2_proof_4.26_1}) 从 Hölder 不等式得出,其中 $C_n = \|\rho_n\|_{p'}$。另一方面,我们有 $\nabla(\rho_n \star f) = (\nabla \rho_n) \star f$ 并且因此
\[ \|\nabla(\rho_n \star f)\|_{L^\infty(\mathbb{R}^N)} \le \|\nabla \rho_n\|_{p'} \|f\|_p. \]
因此我们得到 (\ref{eq:step2_proof_4.26_2}),其中 $C_n = \|\nabla \rho_n\|_{p'}$。

\textbf{步骤 3:} 给定 $\varepsilon > 0$ 和一个有限测度的 $\Omega \subset \mathbb{R}^N$,存在一个有界可测集 $\omega \subset \Omega$,使得
\begin{equation}\label{eq:step3_proof_4.26}
\|f\|_{L^p(\Omega \setminus \omega)} < \varepsilon \quad \forall f \in \mathcal{F}. \tag{26}
\end{equation}
的确,我们写
\[ \|f\|_{L^p(\Omega \setminus \omega)} \le \|f - \rho_n \star f\|_{L^p(\mathbb{R}^N)} + \|\rho_n \star f\|_{L^p(\Omega \setminus \omega)}. \]
鉴于 (24) 选择 $\omega$ 使得 $|\Omega \setminus \omega|$ 足够小。

\textbf{步骤 4: 结论。} 由于 $L^p(\Omega)$ 是完备的,只需证明(见,例如,A. Knapp [1] 或 J. R. Munkres [1], 定理 7.3)$\mathcal{F}|_\Omega$ 是\textbf{完全有界}的,即,给定任何 $\varepsilon>0$,存在 $\mathcal{F}|_\Omega$ 的一个有限覆盖,由半径为 $\varepsilon$ 的球构成。给定 $\varepsilon>0$ 我们固定 $n > 1/\delta$。
一个有界可测集 $\omega \subset \Omega$ 使得 (26) 成立。族 $\mathcal{H} = \{(\rho_n \star f)|_\omega : f \in \mathcal{F}\}$ 满足 Ascoli-Arzelà 定理的所有假设(通过步骤 2)。因此 $\mathcal{H}$ 在 $C(\bar{\omega})$ 中有紧闭包;因此 $\mathcal{H}$ 在 $L^p(\omega)$ 中也有紧闭包。因此我们可以用有限个半径为 $\varepsilon$ 的球覆盖 $\mathcal{H}$,比如
\[ \mathcal{H} \subset \cup_i B(g_i, \varepsilon) \text{ with } g_i \in L^p(\omega). \]
考虑函数 $\tilde{g}_i: \Omega \to \mathbb{R}$ 定义为
\[ \tilde{g}_i = \begin{cases} g_i & \text{在 } \omega \text{上}, \\ 0 & \text{在 } \Omega \setminus \omega \text{上}. \end{cases} \]
以及 $L^p(\Omega)$ 中的球 $B(\tilde{g}_i, 3\varepsilon)$。
我们声称它们覆盖了 $\mathcal{F}|_\Omega$。确实,给定 $f \in \mathcal{F}$,存在某个 $i$ 使得
\[ \|\rho_n \star f - g_i\|_{L^p(\omega)} < \varepsilon. \]
由于
\[ \|f - \tilde{g}_i\|_{L^p(\Omega)}^p = \int_\omega |f-g_i|^p + \int_{\Omega \setminus \omega} |f|^p, \]
我们有,通过 (26),
\[ \|f - \tilde{g}_i\|_{L^p(\Omega)} \le \|f-g_i\|_{L^p(\omega)} + \|f\|_{L^p(\Omega \setminus \omega)} \le \|f - \rho_n \star f\|_{L^p(\omega)} + \|\rho_n \star f - g_i\|_{L^p(\omega)} + \|f\|_{L^p(\Omega \setminus \omega)} < 3\varepsilon. \]
我们得出结论,$\mathcal{F}|_\Omega$ 在 $L^p(\Omega)$ 中有紧闭包。
\end{proof}

\begin{remark}
11. 当试图证明 $L^p(\Omega)$ 中的一个族 $\mathcal{F}$ 在 $L^p(\Omega)$ 中有紧闭包时,其中 $\Omega$ 有界,通常方便将函数全部扩展到 $\mathbb{R}^N$,然后在 $\mathbb{R}^N$ 上应用定理 \ref{theorem4.26} 并考虑对 $\Omega$ 的限制。
\end{remark}

\begin{remark}
12. 在定理 \ref{theorem4.26} 的假设下,我们不能得出 $\mathcal{F}$ 本身在 $L^p(\mathbb{R}^N)$ 中有紧闭包(构造一个例子,或见练习 4.33)。需要一个额外的假设;我们现在描述它。
\end{remark}

\begin{corollary}\label{corollary4.27}
设 $\mathcal{F}$ 是 $L^p(\mathbb{R}^N)$ 中的一个有界集,其中 $1 \le p < \infty$。假设 (22) 成立并且
\begin{equation}\label{eq:corollary_4.27}
\forall \varepsilon > 0 \quad \exists \Omega \subset \mathbb{R}^N, \text{有界, 可测使得} \quad \|f\|_{L^p(\mathbb{R}^N \setminus \Omega)} < \varepsilon \quad \forall f \in \mathcal{F}. \tag{27}
\end{equation}
那么 $\mathcal{F}$ 在 $L^p(\mathbb{R}^N)$ 中有紧闭包。
\end{corollary}

\begin{proof}
给定 $\varepsilon > 0$ 我们固定一个有界可测集 $\Omega$ 使得 (27) 成立。通过定理 \ref{theorem4.26} 我们知道 $\mathcal{F}|_\Omega$ 在 $L^p(\Omega)$ 中有紧闭包。因此我们可以用有限个半径为 $\varepsilon$ 的球覆盖 $\mathcal{F}|_\Omega$,比如
\[ \mathcal{F}|_\Omega \subset \cup_i B(g_i, \varepsilon) \text{ with } g_i \in L^p(\Omega). \]
设
\[ \tilde{g}_i(x) = \begin{cases} g_i(x) & \text{在 } \Omega \text{中}, \\ 0 & \text{在 } \mathbb{R}^N \setminus \Omega \text{上}. \end{cases} \]
很明显 $\mathcal{F}$ 被 $L^p(\mathbb{R}^N)$ 中的球 $B(\tilde{g}_i, 2\varepsilon)$ 覆盖。
\end{proof}

\begin{remark}
13. 推论 \ref{corollary4.27} 的逆命题也成立(见练习 4.34)。因此我们对 $L^p(\mathbb{R}^N)$ 中紧集的完整刻画。
我们以定理 \ref{theorem4.26} 的一个有用应用来结束。
\end{remark}

\begin{corollary}\label{corollary4.28}
设 $G$ 是 $L^1(\mathbb{R}^N)$ 中的一个固定函数,设
\[ \mathcal{F} = \{G \star B \mid B \text{ 是 } L^p(\mathbb{R}^N) \text{ 中的有界集}\}, \]
其中 $1 \le p < \infty$。那么 $\mathcal{F}|_\Omega$ 在 $L^p(\Omega)$ 中对任何有限测度的可测集 $\Omega$ 都有紧闭包。
\end{corollary}

\begin{proof}
显然 $\mathcal{F}$ 在 $L^p(\mathbb{R}^N)$ 中是有界的。另一方面,如果我们写 $f=G\star u$ 其中 $u \in B$ 我们有
\[ \|\tau_h f - f\|_p = \|(\tau_h G - G) \star u\|_p \le \|\tau_h G - G\|_1 \|u\|_p \le C \|\tau_h G - G\|_1, \]
并且当我们用以下引理帮助时,我们得出结论。
\end{proof}

\begin{lemma}\label{lemma4.3}
设 $G \in L^q(\mathbb{R}^N)$,其中 $1 \le q < \infty$。那么
\[ \lim_{h \to 0} \|\tau_h G - G\|_q = 0. \]
\end{lemma}
\begin{proof}
给定 $\varepsilon > 0$,存在(通过定理 \ref{theorem4.12})一个函数 $G_1 \in C_c(\mathbb{R}^N)$ 使得 $\|G-G_1\|_q < \varepsilon$。
我们写
\[ \|\tau_h G - G\|_q \le \|\tau_h G - \tau_h G_1\|_q + \|\tau_h G_1 - G_1\|_q + \|G_1 - G\|_q \le 2\varepsilon + \|\tau_h G_1 - G_1\|_q. \]
由于 $\lim_{h \to 0} \|\tau_h G_1 - G_1\|_q = 0$ 我们看到
\[ \limsup_{h\to 0} \|\tau_h G - G\|_q \le 2\varepsilon \quad \forall \varepsilon > 0. \]
\end{proof}
