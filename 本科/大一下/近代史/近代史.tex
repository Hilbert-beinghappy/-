\documentclass[lang=cn,10pt]{elegantbook}
\usepackage{graphicx}
\usepackage{float}

\title{近代史纲要}



\author{ Huang}
\date{\today}


\extrainfo{咱就是说,水课还是要救一下的}

\setcounter{tocdepth}{3}


\cover{cover.jpg}

% 本文档命令
\usepackage{array}
\newcommand{\ccr}[1]{\makecell{{\color{#1}\rule{1cm}{1cm}}}}

% 修改标题页的橙色带
% \definecolor{customcolor}{RGB}{32,178,170}
% \colorlet{coverlinecolor}{customcolor}

\begin{document}
	
	\maketitle
	\frontmatter
	
	\tableofcontents
	
	\mainmatter
	\chapter{导言}
	\begin{example}
		中国近代史的主流和本质是什么?
	\end{example}
	\begin{solution}
		是中国人民为救亡图存和实现中华民族伟大复兴而英勇奋斗、艰辛探索并不断取得伟大成就的历史。尤其是全国各族人民在中国共产党领导下,进行艰苦卓绝的斗争,经过新民主主义革命,赢得民族独立、人民解放,建立中华人民共和国的历史;经过社会主义革命、建设、改革,把极度贫穷落后的中国逐步改变成持续走向繁荣富强,充满生机活力的社会主义中国的历史
	\end{solution}
	\begin{example}
		学习近代史的目的和要求是什么?
	\end{example}
	\begin{solution}
		学习历史的主要目的是以史鉴今、资政育人
	\end{solution}
	\begin{example}
		四个选择是什么?
	\end{example}
	\begin{solution}
		人民是怎样选择了马克思主义、选择了中国共产党、选择了社会主义道路,选择了改革开放
	\end{solution}
	\begin{example}
		三个为什么是什么?
	\end{example}
	\begin{solution}
		中国共产党为什么能、马克思主义为什么行、中国特色社会主义为什么好
	\end{solution}
	\chapter{进入近代后中华民族的磨难与抗争}
	\begin{example}
		中国为什么会落后?
	\end{example}
	\begin{solution}
		1.封闭的经济政策:清朝时期,中国实行封闭的经济政策,对外商和技术的交流受到了限制,同时政府只顾着收税而忽略了促进产业发展和技术升级。这使得中国在工业、农业、交通运输等领域的发展慢于欧洲和日本等国家。
		
		2.内部政治动荡:清朝时期各种政治和军事混乱不断,外敌入侵、战事不断,国内政治和经济发展都受到严重影响。同时,清朝的统治也存在腐败、不作为和官僚主义等问题,政府的无能也导致了国家的衰落。
		
		3.科技和知识落后:在19世纪中叶以前,中国的科学、技术和哲学都比较保守,没有像欧洲那样的科学革命和文艺复兴。同时,国内的儒家知识和文化传统也限制了中国思想的创新和发展。
		
		4.西方列强的侵略和剥削:19世纪中叶以后,西方列强侵略中国,迫使中国签订了不平等条约,开放了它的港口和贸易,带走了大量的资源和财富。这使得中国不仅失去了经济上的优势,还剥夺了它自主发展的机会。
	\end{solution}
	\begin{example}
		资本主义帝国对中国是怎样侵略的?
	\end{example}
	\begin{solution}
		1.它们发动侵列战争,屠杀中国人民,侵占中国领土,划分势力范围,勒索赔款,抢掠财富;
		
		2.它们控制中国的内政、外交、镇压中国人民的反抗,扶植、收买代理人;
		
		3.它们控制中国的通商口岸,剥夺中国的关税自主权,实行商品倾销和资本输出,操纵中国的经济命脉;
		
		4.它们披着宗教外衣,进行侵略活动,为侵略中国制造舆论.
	\end{solution}
	\begin{example}
		人民的反抗为什么会失败?
	\end{example}
	\begin{solution}
		
		反侵略斗争失败的原因首先是中国半殖民地半封建社会的腐败社会制度决定的.
		
		其次,是国家经济特别是经济技术和作战能力的落后.
	\end{solution}
	\begin{example}
		如何理解半殖民地半封建社会?
	\end{example}
	\begin{solution}
		
		半殖民地:即在政治上是指丧失部分主权而不是全部的独立主权,外国侵略者逐渐控制了中国内政外交;经济上是指中国逐渐被卷入资本主义世界市场,沦为资本主义国家的原料产地、商品市场和资本输出场所,中国社会经济丧失独立地位,成为资本主义的经济附庸;在文化上则表现为“西学东渐”。
		
		半封建:即保存了封建主义又发展了资本主义;在政治上是指地主阶级仍居于统治地位,但产生了新的阶级,并提出了对政权的要求;同时,中国的封建势力与外国侵略势力相勾结,共同维持在中国的反动统治。在经济上是指自然经济仍占主导地位,但产生了新的资本主义经济,并冲击瓦解着旧的自然经济。在思想文化上近代西方民主思想意识传入并有所发展,但传统儒家伦理思想仍占统治地位。
	\end{solution}
	\chapter{不同社会力量对国家出路的早期探索}
	\begin{example}
		如何认识太平天国农民战争的意义和失败的原因、教训?
	\end{example}
	\begin{solution}
		意义:	
		
		(1)太平天国起义沉重打击了封建统治阶级动摇了清王朝封建统治的基础加速了清王朝的衰败过程;	
		
		(2)太平天国起义还有力地打击了外国侵略势力给侵略者应有的教训;	
		
		(3)太平天国起义是中国旧式农民战争的最高峰具有不同于以往农民战争的新的历史特点;	
		
		(4)太平天国起义冲击了孔子和儒家经典的正统权威在一定程度上削弱了封建统治的精神支柱;	
		
		(5)在19世纪中叶的亚洲民族解放运动中太平天国起义和其他亚洲国家的民族解放运动汇合在一起冲击了西方殖民主义者在亚洲的统治。	
		
		失败原因:
		
		从主观方面看	
		
		一是农民阶级的局限性。他们无法从根本上提出完整的、正确的政治纲领和社会改革方案未能制止和克服领导集团自身腐败现象的滋长也未能长期保持领导集团的团结从而大大削弱了太平天国的向心力和战斗力。	
		
		二是战略上的失误。太平军偏师北伐孤军深入分散了兵力。	
		
		三是在太平天国后期拜上帝教的思想理论绐太平天国起义带来了危害。
		
		从客观方面看中外反动势力勾结起来联合镇压太平天国。	
		
		教训:太平天国起义及其失败表明在半殖民地半封建的中国由于受阶级和时代的局限农民阶级不能担负起领导反帝反封建斗争取得胜利的重任。单纯的农民战争也不可能完成民族独立和人民解放的历史任务。
	\end{solution}
	\begin{example}
		如何认识洋务运动的作用和失败的原因、教训?
	\end{example}
	\begin{solution}
		作用:
		
		1、洋务运动推动了近代中国生产力的发展,促使中国民族资本主义的产生。
				
		2、洋务运动打开了封建教育制度的缺口,是中国近代教育的开始。
		
		3、洋务运动促进传统社会风气和价值观念的改变
		
		失败原因:
		
		(1)洋务运动具有封建性。洋务运动的指导思想是“中学为体西学为用”洋务派企图在不改变中国固有的制度与道德的前提下以吸取西方近代生产技术为手段来达到维护和巩固中国封建统治的目的这就严重限制了洋务运动的发展。
		
		(2)洋务运动对西方列强具有依赖性。西方列强依据种种特权。从政治、经济等各方面加紧对中国的侵略和控制它们并不希望中国真正富强起来而洋务派却处处仰赖外国企图以此来达到自强求富的目的这无异于与虎谋皮。
		
		(3)洋务企业管理具有腐朽性。洋务企业虽然具有一定的资本主义性质但其管理却是封建式的企业内部充斥着营私舞弊、贪污中饱、挥霍浪费等腐败现象。
		
		教训:
		
		洋务运动的失败说明在不触动封建专制统治、没有摆脱外国资本—帝国主义的侵略与控制的前提下试图通过局部的枝节改革发展本国资本主义达到自强求富的目的是不可能的。
	\end{solution}
	\begin{example}
		如何认戊戌维新运动的意义和失败的原因、教训?
	\end{example}
	\begin{solution}
		(1)戊戌维新运动的意义
		
		第一,戊戌维新运动是一次爱国救亡运动。
		
		第二,戊戌维新运动是一场资产阶级性质的政治改革运动。
		
		第三,戊戌维新运动更是一场思想启蒙运动。
		
		第四,戊戌维新运动不仅在思想启蒙和文化教育方面开创了新的局面,而且在社会风习方面也提出了许多新的主张。
		
		(2)戊戌维新运动失败的原因
		
		戊戌维新运动的失败,主要是由于维新派自身的局限和以慈禧太后为首的强大的守旧势力的反对。维新派本身的局限性突出表现在:
		
		首先,不敢否定封建主义。
		
		其次,对帝国主义报有幻想。
		
		再次,惧怕人民群众。
		
		(3)戊戌维新运动失败的教训
		
		戊戌维新运动的失败不仅暴露了中国民族资产阶级的软弱性,
		
		同时,也说明在半殖民地半封建的旧中国,企图通过统治着自上而下的改良道路,是根本行不通的。
		
		要想争取国家的独立、民主、富强,必须用革命的手段,推翻帝国主义、封建主义联合统治的半殖民地半封建的社会制度。
	\end{solution}
	\begin{example}
		如何评价《天朝田亩制度》和《资政新篇》
	\end{example}
	\begin{solution}
		
		《天朝田亩制度》是一个以解决土地问题为中心的比较完整的社会改革方案,其并没有超出农民小生产者的狭隘眼界,表明农民起义难以建立起足以代替腐朽制度的新的社会制度
		
		《资政新篇》是中国近代史上第一个比较系统的发展资本主义的方案,但通篇未涉及农民问题和土地问题
	\end{solution}
	\chapter{辛亥革命与君主专制制度的终结}
	\begin{example}
		辛亥革命的历史意义和失败的原因
	\end{example}
	\begin{solution}
		
		历史意义
		
		第一,辛亥革命推翻了封建势力的政治代表、帝国主义在中国的代理人——清王朝的统治,沉重的打击了中外反动势力,使中国反动统治者在政治上乱了阵脚。
		
		第二,辛亥革命结束了统治中国两千多年的封建君主专制制度,建立了中国历史上第一个资产阶级共和政府。
		
		第三,辛亥革命给人们带来一次思想上的解放。
		
		第四,辛亥革命促使社会经济、思想习惯和社会风俗等方面发生了新的积极变化。
		
		第五,辛亥革命不仅在一定程度上打击了帝国主义的侵略势力,而且推动了亚洲各国民族解放运动的高涨。
		
		
		失败原因:
		
		首先,从根本上说,是因为在帝国主义时代,在半殖民地半封建的中国,资本主义的建国方案是行不通的。
		
		其次,从主观方面来说,在于它的领导者资产阶级革命派本身存在着许多弱点和错误。
		
		第一,没有提出彻底的反帝反封建的革命纲领。
		
		第二,不能充分发动和依靠人民群众。
		
		第三,不能建立坚强的革命政党,作为团结一切革命力量的强有力的核心。
	\end{solution}
	\begin{example}
		
		三民主义的内容是什么,有什么局限性?
	\end{example}
	\begin{solution}
		民族主义、民权主义、民生主义
		
		局限性
		
		民族:没有明确的反帝主张,为没有明确的把汉族军阀、官僚、地主作为革命对象
		
		民权:没有明确广大劳动人民在国家中的地位
		
		民生:没有正面触及封建土地所有制,不能满足农民的土地需求,难以成为发动群众的理论武器
	\end{solution}
	\begin{example}
		资产阶级革命派的历史局限性?
	\end{example}
	\begin{solution}
		
		1、革命派未能认清帝国主义的本质,不仅不敢旗帜鲜明地提出反帝口号,反而希望争取帝国主义的支持。
		
		2、革命派停留在对民主制度形式的理解上,缺乏对民主建政的深入认识。
		
		3、革命派未能把土地制度改革和反对封建主义联系起来,从而无法真正解决农民土地问题
	\end{solution}
	\chapter{中国共产党的成立和中国革命新局面}
	\begin{example}
		五四运动的历史意义是什么?
	\end{example}
	\begin{solution}
		
		第一,五四运动是中国旧民主主义革命走向新民主主义革命的转折点,在近代以来中华民族追求民族独立和发展进步的历史进程中具有里程碑意义。 五四运动成为中国新民主主义革命的开端。
		
		第二,五四运动以全民族的力量高举起爱国主义的伟大旗帜,孕育了以爱国、进步、民主、科学为主要内容的伟大五四精神,其核心是爱国主义。 
		
		第三,五四运动以全民族的行动激发了追求真理、追求进步的伟大觉醒。它改变了以往只有觉悟的革命者而缺少觉醒的人民大众的斗争状况,实现了中国人民和中华民族自鸦片战争以来的第一次全面觉醒。 
		
		第四,五四运动以全民族的搏击培育了永久奋斗的伟大传统。
	\end{solution}
	
	\begin{example}
		新文化运动的口号,局限,历史意义是什么?
	\end{example}
	\begin{solution}
		
		口号:提倡民主与科学
		
		局限性:
		
		1.资产阶级共和国方案在中国行不通,提倡资产阶级民主主义,并不能为人民提供一种有效的思想武器去对中国社会进行改造
		
		2.把改造国民性置于优先地位,但是离开改造产生封建思想的社会环境的革命实践,没有深入到基层群众中去,只是针对少数知识分子。
		
		3.在思想方法上存在绝对肯定或绝对否定的形式主义偏向。
		
		历史意义:动摇了封建思想的统治地位,也使民主和科学思想得到弘扬,为五四运动的爆发作了思想准备,而后期传播的社会主义思想,更是启发了中国一批先进的知识分子,同时新文化运动也促进了文化的普及和繁荣。
	\end{solution}
	\begin{example}
		中国先进分子为什么选择马克思主义?
	\end{example}
	\begin{solution}
		
		1、马克思主义指导的十月革命发生在其国情与中国相同的俄国,因而中国先进分子具有特殊吸引力。

		2、十月革命诞生的社会主义俄国号召反对帝国主义,并以新的平等态度对待中国,有力推动了社会主义思想在中国的传播。

		3、十月革命取得的胜利的事实,给予中国先进知识分子新的革命方法的启示,推动他们去研究这个革命所遵循的主义。

		4、李大钊率先在中国举起马克思主义旗帜,比较系统地介绍马克思主义理论,吸引先进知识分子的注意。
	\end{solution}
	\begin{example}
		中国共产党成立的意义
	\end{example}
	\begin{solution}
		
		1.使中国革命有了坚强的领导核心,中国人民有了可依赖的组织者和领导者,保证了中国革命胜利的发展。
		
		2.中国人民由此踏上了争取民族独立、人民解放的光明道路,开启了实现国家富强、人民幸福的历史征程
		
		3.形成了坚持真理、坚守理想,践行初心、担当使命,不怕牺牲、英勇斗争,对党忠诚、不负人民的伟大建党精神,这是中国共产党的精神之源
		
		4.中国共产党的成立,深刻改变了近代以后中华民族发展的方向和进程,深刻改变了中国人民和中华民族的前途和命运,深刻改变了世界发展的趋势和格局
	
	\end{solution}
	\begin{example}
		中共一大和二大分别得出什么样的纲领?
	\end{example}
	\begin{solution}
		
		一大:革命军队必须与无产阶级一起推翻资本家阶级的政权;承认无产阶级专政,直到阶级斗争结束;消灭资本家私有制;联合第三国际
		
		二大:
		
		最高纲领:
		实现社会主义、共产主义
		
		最低纲领:
		打倒军阀,推翻国际帝国主义的压迫,统一中国成为真正民主共和国
	\end{solution}
	\begin{example}
		中国共产党在国民大革命中的作用?
	\end{example}
	\begin{solution}
		
		1.大革命是近代中国历史上空前而广泛的群众运动,而中国是人民群众的主要发动者和组织者
		
		2.共产党人不仅帮助和推动了国民革命军的建立,而且在军队中进行了卓有成效的政治工作,积极提高国民革命军的素质,增强它的凝聚力和战斗力
		
		3.共产党员在战斗中更是身先士卒,起着先锋作用和表率作用
		
		4.共产党人还建立了一定数量的工农武装 配合正规军作战
	\end{solution}
	\begin{example}
		大革命失败的原因?
	\end{example}
	\begin{solution}
		
		1.从客观方面讲,是由于反革命力量强大,革命统一战线内部出现剧烈分化。
		
		2.从主观方面说,是由于这时的中国共产党还处在幼年时期,缺乏应对复杂环境的政治经验,缺乏对中国社会和中国革命基本问题的深刻认识。还不善于将马克思列宁主义基本原理同中国革命的具体实际结合起来
		
		3.以陈独秀为代表的中共中央领导机关在大革命后期犯了右倾机会主义错误
	\end{solution}
	\chapter{中国革命的新道路}
	\begin{example}
		八七会议的内容和意义?
	\end{example}
	\begin{solution}
		内容:
		
		1.确定了土地革命和武装斗争的方针
		
	 	2.毛泽东提出了“政权是从枪杆子中取得的”
	 	
	 	意义:
	 	
	 	1.给中国共产党指明了出路
	 	
	 	2.是由大革命失败到土地革命战争兴起的历史转折点
	\end{solution}
	\begin{example}
		古田会议的内容和意义
	\end{example}
	\begin{solution}
		
		内容:
		
		(1)确立了思想建党、政治建军原则
		
		(2)规定红军是一个执行革命的政治任务的武装集团,必须绝对服从共产党的领导,必须全心全意为党的纲领、路线和政策而奋斗,必须加强思想和政治路线的教育,纠正党内的错误思想
		
		(3)提出红军必须担负打仗、筹款和做群众工作的任务,必须加强政治工作
		
		意义:
		
		(1)古田会议决议是中国共产党和红军建设的纲领性文献,是党和人民军队建设史上的重要里程碑
		
		(2)创造性地解决了在农村环境中、在党组织和军队以农民为主要成分的条件下,如何保持党的无产阶级先锋队性质和建设党领导的新型人民军队的重大问题
	\end{solution}
	\begin{example}
		遵义会议的内容和意义
	\end{example}
	\begin{solution}
		
		内容:
		
		1、批评了博古、李德在第五次"反围剿"中的错误,增选了毛泽东为中央政治局常委
		
		2、会后决定由张闻天代替博古负总的责任
		
		3、会后成立了由毛泽东、周恩来、王稼祥组成的新的三人团,全权负责红军的军事行动
		
		意义:
		
		1、开始确立了以毛泽东为代表的马克思主义正确路线在党中央的领导地位
		
		2、在极其危机的情况下挽救了中国共产党、挽救了中国工农红军、挽救了中国革台
		
		3、成为中国共产党历史上一个生死攸关的转折点
		
		4、为党和革命事业转危为安、不断打开新局面提供了最重要的保证
	\end{solution}
	\begin{example}
		土地革命时,采取了什么样的土地政策?
	\end{example}
	\begin{solution}
		1、农民已经分得的田归农民个人私有,可以自主租借买卖,别人不得侵犯。

		2、坚定地依靠贫农,固农联合中农,限制富农。

		3、保护中小工商业者,消灭地主阶级。

		4、以乡为单位,按人口平分土地,在原耕地的基础上实行抽多补少,抽肥补瘦。
开展土地革命的作用。
	\end{solution}
	\begin{example}
		土地革命的作用
	\end{example}
	\begin{solution}
		1.推翻了封建地主阶级的统治,废除了封建土地所有制,为农民的土地所有制奠定了基础。
		
		2.动员和组织了广大农民群众参与革命,提高了农民的政治、经济和文化地位,壮大了革命力量。
		
		3.打击了国民党反动派,使其元气大伤,为中国革命的发展创造了有利的条件。
		
		4.促进了中国革命的全面发展,推动了中国新民主主义革命的胜利。
		
		5.为新中国的成立奠定了基础,土地革命后,中国共产党领导下的人民政权得到了广大农民的支持,为新中国的成立奠定了基础。
	\end{solution}
	\begin{example}
		中国革命道路新在哪里?
	\end{example}
	\begin{solution}
		
		中国革命新道路之“新”体现在:
		
		1、创新:农村包围城市,武装夺取政权,建立农村革命根据地。(突破了苏联模式)中国特色革命道路的实践和理论意义 。
		
		2、土地革命:土地革命,是党在革命根据地开展打土豪、分田地、废除封建剥削和债务,满足农民土地要求的革命。
		
		3、新与旧是相对的,当时中国革命的道路被称为新,是要站在当时的历史舞台思考问题,相比以前的旧传统体制,对照革命以来走的弯路和错误,所以说新提出的革命道路是新的。
	\end{solution}
	\begin{example}
		长征的意义?
	\end{example}
	\begin{solution}
		
		1.长征为中国革命保存了有生力量,为党培养了一大批优秀的干部,中国共产党正是依靠这支队伍作基干,使革命力量逐步恢复、发展、壮大,直到取得全国的胜利。
		
		2.在长征的过程中,中国共产党通过总结成功的经验和挫折、失败的教训,一方面反对右倾机会主义,又一方面反对"左"倾机会主义,使自己从两条战线斗争中巩固和壮大起来,把党领导的革命事业坚持下来并向前推进。
		
		3.红军的长征宣告了国民党反动派消灭中国共产党和红军的图谋彻底失败,宣告了中国共产党和红军肩负着民族希望胜利实现了北上抗日的战略转移,实现了中国共产党和中国革命事业从挫折走向胜利的伟大转折
	\end{solution}
	\chapter{中华民族的抗日战争}
	\begin{example}
		抗日民族统一战线如何形成的?
	\end{example}
	\begin{solution}
		
		首先,华北事变后中日民族矛盾上升为中国社会的主要矛盾。日本的侵略激发了中国人民的愤慨,唤起了全民族的抗日救亡运动。中国共产党代表最广大人民的利益,提出了抗日民族统一战线的主张,得到了以国民党为首的各个爱国党派的响应。
		
		接着,中共瓦窑堡会议和八一宣言提出建立抗日民族统一战线的方针。瓦窑堡会议解决了党的中心任务、政治基础、军队建设和党的组织建设等一系列问题;八一宣言公开发表中共关于逼蒋抗日的政治主张,这都为抗日民族统一战线的正式建立准备了必要的前提。
		
		然后,西安事变和平解决标志着抗日民族统一战线初步形成。西安事变的和平解决,成为时局转换的枢纽。中国共产党对国内和平的积极贡献,促进了全国抗日高潮的到来。
		
		最后,七七事变后,中国共产党发表抗日通电,号召全国人民团结起来,实行全面抗战。七七事变揭开了全民族抗战的序幕,标志着抗日民族统一战线正式形成。
	\end{solution}
	\begin{example}
		为什么说中国共产党是中国人民抗日战争的中流砥柱?
	\end{example}
	\begin{solution}
		
		(1)中国共产党积极倡导、促成、维护抗日统一战线,最大限度动员全国军
		民共同抗战,成为凝聚全民族抗战力量的杰出组织者和鼓舞者。
		
		(2)以毛泽东为领导的中国共产党人,把马列主义基本原理同中国具体实践相结合创立和发展了毛泽东思想,制定、实施了一套完整的抗战策略和方针提出了持久战的战略思想,对抗战胜利发挥了重要作用
		
		
		(3)中国共产党通过游击战开辟了敌后战场,建立抗日根据地,牵制和消灭了日本大量有生力量,减轻了正面战场的压力,也为抗日战争的战略反攻准备了条件。
		
		(4)中国共产党人以自己最富于献身的爱国主义、不怕流血牺牲的模范行动,支撑起全民救亡图存的希望,成为夺取抗战胜利的民族先锋。
	\end{solution}
	
	\begin{example}
		抗日战争胜利的原因,意义?
	\end{example}
	\begin{solution}
		
		原因:
		
		第一,以爱国主义为核心的民族精神是中国人民抗日战争胜利的决定因素
		
		第二、中国共产党的中流砥柱作用是中国人民抗日战争胜利的关键
		
		第三,全民族抗战是中国人民抗日战争胜利的重要法宝。
		
		第四,中国人民抗日战争的胜利,同世界所有爱好和平和正义的国家和人民、国际组织以及各种反法西斯力量的同情和支持也是分不开 。
		
		意义:
		
		1.中国人民抗日战争的胜利,彻底粉碎了日本军国主义殖民奴役中国的图谋,有力捍卫了国家主权和领土完整,彻底洗刷了近代以来抗击外来侵略屡战屡败的民族耻辱。
		
		2.中国人民抗日战争的胜利,促进了中华民族的大团结,形成了伟大的抗战精神。 
		
		3.中国人民抗日战争的胜利,对世界各国夺取反法西斯战争的胜利,维护世界和平产生了巨大影响。中国人民为最终战胜法西斯势力作出了历史性贡献,国际地位显著提高
		
		4.中国人民抗日战争的胜利,坚定了中国人民追求民族独立、自由、解放的意志,开启了古老中国凤凰涅槃、浴火重生的历史新征程,为中国共产党团结带领全国人民继续奋斗,赢得新民主主义革命胜利,奠定了重基础。
	\end{solution}
	\chapter{为建立新中国而奋斗}
	
	\begin{example}
		抗战胜利后,国民党陷入全民的包围并迅速走向崩溃的原因?
	\end{example}
	\begin{solution}
		
		(1)国民党政府实行专制独裁统治,官员们贪污腐化,在抗战后期就严重丧失民心。抗战胜利之际,国民党政府所派官员"劫收"沦陷区,进一步丧失人心。
		
		(2)国民党政府在抗战胜利后违背全国人民迫切要求体养生息、和平建国的意愿。执行反人民的内战政策。为了筹措内战经费,向人民征收各种捐税,滥发纸币,导致恶性通货膨胀,工农业生产严重萎缩,人民生活困苦。
		
		
		(3)代表大地主、大资产阶级利益的国民党政府在抗战胜利后,拒绝全国人民要和平、要民主、要自由的愿望,仍然继续并加强独裁统治。因此,国民党政府置自身于人民的对立面,丧失了人心,激起了全国人民的反抗,从而陷入了全国的包围之中,并迅速走向崩溃。
		
		(4)共产党代表中国最广大劳动人民的利益,坚决进行反对封建主义、帝国主义和官僚资本主义三座大山的斗争。
	\end{solution}
	\begin{example}
		新民主主义革命的胜利的意义和胜利的基本经验是什么?
	\end{example}
	\begin{solution}
		
		意义:
		
		1.新民主主义革命的胜利,结束了帝国主义、封建主义和官僚资本主义在中国的统治,建立了人民民主专政的新中国。
		
		2.新民主主义革命的胜利,使中国开始了由新民主主义向社会主义过渡的历史。
		
		3.新民主主义革命的胜利,改变了世界政治格局,巩固了世界和平。
		
		经验:
		
		1.必须有一个领导中心,有了坚强的领导中心,才能团结起全国人民,进行艰苦卓绝的斗争,并最终取得革命的胜利。
		
		2.必须有一支基本的武装力量中国共产党领导的人民军队,是彻底地为人民服务的,完全地保护人民的利益,是新民主主义革命胜利的可靠保证
		
		3.必须采取正确的革命策略土地革命是新民主主义革命的基本内容和主要方法;武装反抗国民党反动派是新民主主义革命的主要形式和手段。只有采取这样的革命策略,才能使中国革命少走弯路,较快地取得胜利。
		
		4.必须坚持共产党的领导和广泛的统一战线。中国共产党是新民主主义革命的领导核心,只有坚持党的领导,才能保证革命的方向和最后的成功;只有坚持统一战线,才能有效地团结一切可能团结的力量,共同进行斗争,并取得革命的最后胜利。
		
		5.必须有一个恰当的革命纲领。新民主主义革命的纲领,是中国共产党经过长期斗争考验得出的正确结论,是革命胜利的重要保证。只有制定正确的革命纲领,并将其贯彻到革命的实践中去,才能推动中国革命走向胜利
	\end{solution}
	\chapter{中华人民共和国的成立和中国社会主义建设道路的探索}
	\begin{example}
		新中国成立之后,面临什么样的考验?采取了什么措施?
	\end{example}
	\begin{solution}
		
		国际上是新中国同帝国主义的矛盾,国内是工人阶级和资产阶级的矛盾
		
		措施:
		
		1、没收官僚资本,确立社会主义性质的国营经济的领导地位
		
		2、开始将资本主义工商业纳入国家资本主义轨道
		
		3、引导个体农民在土地改革后逐步走上互助合作的道路
	\end{solution}
	\begin{example}
		早期建设社会主义的曲折中,我们能得到哪些教训,取得了哪些成就?
	\end{example}
	\begin{solution}
		
		教训:
		1.必须从我国的实际出发,以经济建设为中心,大力发展社会生产力。
		
		2.必须正确认识和处理社会主义社会的矛盾,坚持适当的斗争策略,以促进生产力的发展和社会的进步。
		
		3.必须坚持马克思列宁主义、毛泽东思想,不断总结经验,探索建设社会主义的规律,不断开创有中国特色的社会主义建设道路。
		
		成就:
		
		1.新民主主义革命的胜利,实现了民族独立、人民解放。
		
		2.社会主义改造的基本完成,确立了社会主义制度,走上了社会主义建设的道路。
		
		3.社会主义建设的探索和实践,提出了一系列独创性的理论观点和政策措施,形成了中国特色社会主义理论体系。
		
		4.社会主义现代化建设取得了重大成就,我国国际地位日益提高,在国际事务中的作用不断扩大。
	\end{solution}
	\begin{example}
		改革开放前后两个历史时期有怎样的关系?
	\end{example}
	\begin{solution}
		 
		改革开放前后两个历史时期是相互联系、相互贯通的。改革开放前的社会主义实践探索为改革开放后的社会主义实践探索积累了条件,改革开放后的社会主义实践探索是对改革开放前的社会主义实践探索的继承、改革和发展。两个时期共同围绕着社会主义实践这一时代主轴,在实践探索中将社会主义与现代化建设、民族复兴融为一体。
	\end{solution}
	\begin{example}
		第十一届三中全会的内容和意义是什么?
	\end{example}
	\begin{solution}
		
		内容:
		
		1.全会冲破长期"左"的错误的严重束缚,彻底否定"两个凡是"的错误方针
		
		2.决定从1979年1月起,把全党的工作重心转移到社会主义现代化建设上来
		
		3.全会还特别强调要正确对待毛泽东的历史地位和毛泽东思想的科学体系
		
		意义
		
		1.标志着中国共产党重新确立了马克思主义的思想路线、政治路线、组织路线
		
		2.开启了我国改革开放和社会主义现代化建设新时期
		
		3.实现了新中国成立以来党的历史上具有深远意义的伟大转折
		
	\end{solution}
	\begin{example}
		
	\end{example}
	
\end{document}