\documentclass[lang=cn,10pt]{elegantbook}
\usepackage{graphicx}
\usepackage{float}

\title{思政}



\author{ Huang}
\date{\today}


\extrainfo{水课还是得救救的,别挂了}

\setcounter{tocdepth}{3}


\cover{yangren.jpeg}

% 本文档命令
\usepackage{array}
\newcommand{\ccr}[1]{\makecell{{\color{#1}\rule{1cm}{1cm}}}}

% 修改标题页的橙色带
% \definecolor{customcolor}{RGB}{32,178,170}
% \colorlet{coverlinecolor}{customcolor}

\begin{document}
	
	\maketitle
	\frontmatter
	
	\tableofcontents
	
	\mainmatter
	\chapter{领悟认识真谛,把握人生方向}
	\begin{example}
		个人与社会的辩证关系p15(简答题/辨析题)
	\end{example}
	\begin{solution}
		
		1.个人与社会是对立统一的关系。两者相互依存、相互制约,相互促
		进。社会是由一个个具 的人组成的,离开了人就没有社会。同时,人是
		社会的人,离开了社会,人也无法生活
		
		2.个人与社会的关系,最根本的是个人利益与社会利益的关系.在社会主义社会中,个人利益与社会利益在根本上是一致的。
		
		3.人的社会性决定了人只有在推动社会进步的过程中,才能实现自我的发展。大学生应把自己的人生追求同社会的发展进步紧密结合起来,在为社会作贡献的过程中成长进步,实现自己的人生价值。
	\end{solution}
	\begin{example}
		自我价值和社会价值的关系p19(好像没说出什么)
	\end{example}
	\begin{solution}
		
		人生价值内在地包含了人生的自我价值和社会价值两个方面。
		
		人生的自我价值,是个体的人生活动对自己的生存和发展所具有的价值。
		
		 人生的社会价值,是个体的实践活动对社会、他人所具有的价值 。
		 
		 一方面,人生的自我价值是个体生存和发展的必要条件,人生的自我价值的实现是个体为社会创造更大价值的前提。 
		 
		 另一方面,人生的社会价值是社会存在和发展的重要条件,人生社会价值的实现是个体自我完善、全面发展的保障。
	\end{solution}
	\chapter{不考}
	\chapter{继承优良传统,弘扬爱国精神}
	\begin{example}
		怎么做新时代的爱国者?p82(20分,看着办)
	\end{example}
	\begin{solution}
		
		1.大学生要坚持爱国爱党爱社会主义相统一。
		
		当代中国,爱国主义的本质就是坚持爱国和爱党、爱社会主义高度统一。。我们爱的“国”是中国共产党领导的社会主义中国。拥护国家的基本制度,遵守国家的宪法法律,维护国家安全和统一,捍卫国家的利益,为国家繁荣贡献自己的力量,是爱国主义的基本要求。爱国主义与爱社会主义的统一是中国历史发展的必然结果。没有共产党,就没有新中国。在现阶段,爱国主义主要表现为在中国共产党领导下,献身于实现中华民族伟大复兴的中国梦的实践。
		
		2.大学生要维护祖国统一和民族团结。
		
		(1)维护和推进祖国统一。保持香港、澳门长期繁荣稳定,解决台湾问题、实现祖国完全统一,是实现中华民族伟大复兴的必然要求,是不可阻挡的历史进程,也是全体中华儿女的共同心愿。
		
		(2)促进民族团结。处理好民族问题、促进民族团结,是关系祖国统一和边疆巩固的大事,新时代大学生要
		自觉做民族团结进步事业的建设者、维护者、促进者。
		
		3.大学生要尊重和传承中华民族历史文化。
		
		(1)历史文化是民族生生不息的丰厚滋养。中华优秀传统文化是中华民族的精神命脉,我们必须尊重和传承中华民族历史文化,以时代精神激活中华优秀传统文化的生命力,不断推进中华优秀传统文化创造性转化和创造性发展。
		
		(2)旗帜鲜明反对历史虚无主义。
		抛弃传统、丢掉根本,就等于割断了自己的精神命脉。一个有希望的民族,不能没有英雄,一个有前途的国家不能没有先锋。我们要对中华民族的英雄心怀崇敬,自觉传承好中华民族辉煌灿烂的历史文化。
		
		4.大学生应坚持立足中国又面向世界。
		
		(1)维护国家发展主体性。当今世界,国家仍然是民族存在的最高组织形式,是国际社会活动中的独立主体。 在新形势下,我们一定要保持清醒的认识,坚持独立自主、自力更生 坚决维护国家的主权和尊严、按照本国国情坚持、发展自己的政治制度和民族文化,把中国发展进步的命运始终牢牢掌握在自己手中。
		
		(2)自觉维护国家安全。在国家安全形势越来越复杂的今天,大学生要增强国家安全意识,切实履行维护国家安全的义务。同时我们要增强国防意识,履行国家安全的义务。
		
		(3)推动构建人类命运共同体。当今世界,没有哪个国家能够独自应对人类面临的各种挑战,也没有哪个国家能够退回到自我封闭的孤岛。共同建设一个持久和平、普遍安全、共同繁荣、开放包容、清洁美丽的世界,是全人类的共同利益和共同价值追求。要把弘扬爱国主义精神与扩大对外开放结合起来,尊重各国历史特点、文化传统,尊重各国人民选择的发展道路,善于从不同文明中寻求智慧、汲取营养,促进人类和平与发展的崇高事业,共同推动人类文明发展进步。
	\end{solution}
	\begin{example}
		经济全球化背景下还需要爱国主义吗?p93(辨析or大)有点多,自行根据实际情况删减。
	\end{example}
	\begin{solution}
		
		1.虽然经济全球化对爱国主义造成了较大冲击,但是爱国主义在今天仍然有其存在的理由。经济全球化是社会生产力发展的客观要求和科技进步的必然结果。 各个国家之间的利益济全球化不等于政治全球化,更不意味着政治一体化,只要国家存在,爱冲突和竞争强度没有减弱,一定程度上还强化了人们的爱国主义情感。经国主义就有坚实的基础和丰富的意义。
		
		2.大学生应维护国家发展主体性。当今世界,国家仍然是民族存在的最高组织形式,是国际社会活动中的独立主体。 在新形势下,我们一定要保持清醒的认识,坚持独立自主、自力更生 坚决维护国家的主权和尊严、按照本国国情坚持、发展自己的政治制度和民族文化,把中国发展进步的命运始终牢牢掌握在自己手中。
		
		3.大学生应自觉维护国家安全。在国家安全形势越来越复杂的今天,大学生要增强国家安全意识,切实履行维护国家安全的义务。同时我们要增强国防意识,履行国家安全的义务。
		
	\end{solution}
	\begin{example}
		如何做改革创新主力军?p101(不知道考啥,背就完事)
	\end{example}
	\begin{solution}
		1.树立改革创新的自我意识。
		
		(1)大学生要增强改革创新的责任感。改革创新充满艰辛、奉献甚至牺牲,没有强烈的责任感,很难克服和战胜改革创新过程中的艰难曲折。大学生要以时不我待、只争朝夕的紧迫感投身改革创新的实践
		
		(2)大学生要树立敢于突破陈规的意识。要有强烈的创新意识,凡事要有打破砂锅问到底的劲头,敢于质疑现有定论,勇于开拓新的方向,攻坚克难,追求卓越。敢于大胆突破陈规甚至常规,敢于大胆探索尝试。
		
		(3)树立大胆探索未知领域的信心。创新就是要走前人没有走过的路。
		要创新,就要有强烈的创新自信。	
		
		2.增强改革创新的能力本领。
		
		(1)大学生应夯实创新基础,应从扎实系统的专业知识学习起步和入手,不能好高骛远,空谈改革创新。
		
		(2)大学生应培养创新思维。创新思维善于发现问题,灵活而开放,发散而多维。大学生应在专业学习与社会实践中应自觉培养创新思维,勤于思考,善于发现,勇于创新。
		
		(3)大学生应投身创新实践。积极参加创新创业。
	\end{solution}
	\chapter{明确价值要求,践行价值准则}
	\begin{example}
		如果题目问社会主义核心价值观的意义,作用,为什么,答这个
		
		当代中国发展的精神指引p118
	\end{example}
	\begin{solution}
		1.是坚持和发展中国特色社会主义的价值遵循。社会主义核心价值观,集中体现了马克思主义所倡导的价值理念,是中国特色社会主义的根本价值导向。在全社会大力弘扬社会主义核心价值观,明确中国特色社会主义事业到底追求什么、反对什么,要朝着什么方向走、不能朝什么方向走, 保证中国特色社会主义事业始终沿着正确方向前进,是中国特色社会主义的铸魂工程。
		
		2.是提高国家文化软实力的迫切要求。一个国家的文化软实力,从根本上说,取决于其核心价值观的生命力、凝聚力、感召力。当今世界,文化越来越成为综合国力竞争的重要因素,成为经济社会发展的重要支撑, 文化软实力的竞争,本质上是不同文化所代表的核心价值观的竞争。 
		
		培育和践行社会主义核心价值观,
		
		有利于增进国际社会对中国的理解,扩大中华文化的影响力,展示社会主义中国的良好形象;
		
		有利于增强社会主义意识形态的竞争力,掌握话语权,赢得主动权,逐步打破西方的话语垄断、舆论垄断
		
		3。是推进社会团结奋进的“最大公约数”。社会主义核心价值观 是 凝聚人心、汇聚民力的强大力量。
		历史和现实一再表明,只有建立共同的价值目标,一个国家和民族才会有赖以维系的精神纽带, 我国是一个有着14亿多人口、56个民族的大国,确立反映全国各族人民共同认同的价值观"最大公约数",使全体人民同心同德、团结奋进,关乎国家前途命运,关乎人民幸福安康。
	\end{solution}
	\begin{example}
		全人类共同价值与所谓的“普世价值存在根本不同”:
		
		认清西方“普世价值”的实质p129(辨析)
	\end{example}
	\begin{solution}
		
		1.“普世价值”理论上的虚伪性。“普世价值”概括起来即普遍适用、永恒存在的价值。西方国家所谓的"普世价值"并非指人类道德评价、审美评价的普遍性或共性,而是特指资本主义价值观。 其从抽象的"人性论"出发,将人看作无差别的价值符号。事实上根本不存在抽象的人性,也没有放之四海而皆准的价值观及其相应的制度。
		
		2.“普世价值”在实践上的虚伪性。种族歧视、劳资对立、贫富分化在西方长期存在。
		
		3.反对西方所谓的“普世价值”,并不是说人类社会不存在共同价值。人类生活在同一个地球村里,越来越成为你中有我、我中有你的命运共同体,客观存在共同利益,必然要求共同价值。我们所主张的共同价值,是要倡导求同存异、和而不同,充分尊重文明的多样性,尊重各国自主选择社会制度和发展道路的权利。这与唯我独尊、强施于人、旨在推行资本主义政治理念和制度模式的所谓"普世价值"根本不同。
		
	\end{solution}
	\chapter{遵守道德规范,锤炼道德品格}
	\begin{example}
		道德的本质是什么?p141(简答)
	\end{example}
	\begin{solution}
		
		1.道德是反映社会关系的特殊意识形态。道德的产生、发展和变化,归根结底源于社会经济关系。其一,道德的性质和基本原则、规范反映了与之相应的社会经济关系的性质和内容。
		其二,道德随着社会经济的关系变化而变化(书上四点,随便选两点)
		
		2.道德是社会利益关系的特殊调节方式。道德与法律规定、政治规范的不同之处在于,它是用善恶标准去评价,依靠社会舆论、传统习俗、内心信念来维持的,因此是一种非强制性的规范。
		
		3.道德是一种实践精神。作为实践精神,道德是一种旨在通过把握世界的善恶现象而规范人们的行为,并通过人们的实践活动体现出来的社会意识。在本质上是知行合一。道德立足现实而追求理想,并以理想来改造和提升现实。
	\end{solution}
	\begin{example}
		为什么为人民服务是道德的核心问题?p146
	\end{example}
	\begin{solution}
		为人民服务是社会主义道德的本质要求
		
		1.为人民服务是社会主义经济基础和人际关系的客观要求。在社会主义社会,每个劳动者和建设者都在为社会、为他人同时也是为自己而劳动和工作。 每个人都是服务对象,又都为他人服务,全体人民通过社会分工和相互服务来实现共同利益。团结互助、平等友爱、共为人民服务是社会主义市场经济健康发展的要求。
		
		
		2.为人民服务是社会主义市场经济健康发展的要求。在社会主义市场,
		经济条件下,市场主体必须通过向社会和他人提供一定数量和质量的产品,建立满足社会和他人需求的良好信誉。社会主义市场经济,不仅不排斥为社会和他人服务,而且需要通过服务甚至是优质服务,才能实现市场主体的利益。大学生践行为人民服务,就是要弘扬为人好事、多作贡献。
		民服务的精神,尊重人、理解人、关心人,为人民、为社会、为国家多做
		好事、多作贡献。
	\end{solution}
	\begin{example}
		坚持以集体主义为原则p149
	\end{example}
	\begin{solution}
		
		1.集体主义强调国家利益、社会集体利益和个人利益的辩证统一。
		
		2.集体主义强调国家利益、社会利益高于个人利益。
		
		3.集体主义强调在个人利益与国家利益、社会整体利益发生矛盾尤其是发生激烈冲突的时候,必须坚持国家利益、社会整体利益高于个人利益的原则
		
		4.集体主义要求个人为国家、社会作出牺牲并不是随意的,只有在不牺牲个人利益就不能保全国家利益、社会整体利益的情况下,才要求个人作出牺牲。
		
		5.社会主义集体主义之所以强调个人利益要服从国家利益、社会整体利益,归根到底,既是为了维护国家、社会的共同利益,最终也是为了维护个人的根本利益和长远利益。
	\end{solution}
	\begin{example}
		网络中的道德要求p168(辨析)
	\end{example}
	\begin{solution}
		
		1.从本质上说,网络交往仍然是人与人的现实交往,网络生活也是人的真实生活。网络生活中的道德要求 是社会公德在网络空间的运用和扩展.
		
		2.正确使用网络工具。大学生要提高信息获取能力,加强信息辨识能力,增进信息应用能力,使网络成为开阔视野、提高能力的重要工具。
		
		3.加强网络文明自律。
		
		(1)首先,进行健康网络交往。大学生应通过网络开展健康有益的交往活动,重视个人信息安全,树立自我保护意识,避免给自己的人身和财产安全带来危害。
		
		(2)其次,自觉避免沉迷网络。大学生应当合理安排上网时间,约束上网行为。
		
		(3)最后,加强网络道德自律。大学生应当在网络生活中培养自律精神 ,做到自律而"不逾矩。
		
		4.营造良好的网络道德环境。大学生应当带头引导网络舆论,对模糊认识要即使廓清,对怨气怨言要及时化解,对错误看法要及时纠正,促进网络空间日益清朗。
	\end{solution}
	\begin{example}
		道德修养重在践行p182
	\end{example}
	\begin{solution}
		
		1.掌握道德修养的正确方法。做到学思并用、省察克治、慎独自律、知行合一、积善成德。
		
		2.向道德模范学习。(以XXX为榜样,学习他xx的品质,剩下自己掰扯点)
		
		3.参与志愿服务活动。积极的参加xxx志愿活动。(自己填空,不要太扯)
		
		4.积极引领社会风尚。我们要弘扬真善美,摒弃假恶丑。
	\end{solution}
	\chapter{学习法制思想,提升法制素养}
	\begin{example}
		为什么要走中国特色社会主义法治道路?(简答)
	\end{example}
	\begin{solution}
		
		1.走中国特色社会主义法制道路,是历史的必然结论。党的十八大以来,以习近平同志为核心的党中央把全面依法治国作为新时代坚持和发展中国特色社会主义‘四个全面’战略布局的重要组成部分部分。不断坚持和拓展了中国特色社会主义法制道路。
		
		2.走中国特色社会主义法治道路,是由我国社会主义国家性质决定的。中国特色社会主义坚持人民为主体地位,坚持法律面前人人平等,能够保证人民在党的领导下,按照法律规定管理国家事务。
		
		3.走中国特色社会主义法治道路,是立足我国基本国情的必然选择。就我们这个14亿多人口的社会主义大国而言,要在短时间内建成法治国家,必须走中国特色社会主义法治道路。
	\end{solution}
	\begin{example}
		坚持中国特色社会主义法治道路必须遵守的原则p204(简答)
	\end{example}
	\begin{solution}
		
		1.坚持共产党的领导。党的领导是中国特色社会主义最本质的特征,是社会主义法治最根本的特征。
		
		2.坚持人民主体地位。全面依法治国最广泛、最深厚的基础是人民,必须坚持为了人民,依靠人民
		
		3.坚持法律面前人人平等。平等是社会主义法律的基本属性,是社会主义法治的基本要求。
		
		4.坚持依法治国和以德治国相结合.法治和德治,是治国理政不可或缺的两种方式,法制与德治相得益彰,做到法安天下,德润人心。
		
		5.坚持从中国实际出发。建设法治中国,必须从我国实际出发,同完善和发展中国特色社会主义制度、推进国家治理体系和治理能力现代化相适应。
	\end{solution}
	\begin{example}
		
		党大还是法大?p205(辨析,简答,论述)
	\end{example}
	\begin{solution}
		
		1.党大还是法大是一个伪命题。党的领导和依法治国不是对立的,而是统一的。我国法律充分体现了党和人民意志,党依法办事,这个关系是相互同意的关系。两者不存在比谁大的问题,否则就会落入话语陷阱。我们说不存在党大还是法大的问题,是把党作为一个执政整体、就党的政治地位和领导地位而言的,具体到每一个政党组织、每个领导干部,就必须服从和遵守宪法法律。
		
		2.党和法的关系是政治和法治关系的集中反映。法治倒着念叫治法,就是党领导治国理政的一种方法。法治背后是政治,政治背后是政党,政党背后是人民
		
		3.法治并不是目的,政通人和,国富民强才是目的
		
		4.法治与政治的关系:政治为体。法治为用
		
		5.政治是内容,法制是形式式;政治是目的,法治是手段;政治是法制的源头,法治是政治的制度和规范化
	\end{solution}
	\begin{example}
		坚持依法治国和以德治国相结合
	\end{example}
	\begin{solution}
		
		1.强化道德对法治的支撑作用,重视发展道德的教化作用
		
		2.要把道德要求贯彻到法制建设中,以法治承载道德理念
		
		3.要运用法制手段解决道德领域突出问题,如老赖,让座等。
		
		4.德治以国家强制力为后盾,依靠预测、惩罚、威慑、预防对公民和社会组织行为进行约束,并对违法行为追究责任
		
		5.德治通过对人民的内心信念、传统习俗、社会舆论经行教化,对违反行为经行谴责。
	\end{solution}
\end{document}