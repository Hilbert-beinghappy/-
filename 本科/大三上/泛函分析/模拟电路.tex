\documentclass[lang=cn,10pt]{elegantbook}
\usepackage{graphicx}
\usepackage{float}

\title{泛函分析}



\author{ Huang}
\date{\today}




\setcounter{tocdepth}{3}


\cover{cover.jpg}

% 本文档命令
\usepackage{array}
\newcommand{\ccr}[1]{\makecell{{\color{#1}\rule{1cm}{1cm}}}}

% 修改标题页的橙色带
% \definecolor{customcolor}{RGB}{32,178,170}
% \colorlet{coverlinecolor}{customcolor}

\begin{document}
	
	\maketitle
	\frontmatter
	
	\tableofcontents
	
	\mainmatter
	\chapter{度量空间}
	\section{压缩映射原理}
	\begin{exercise}
		证明:完备空间的闭子集是一个完备的子空间,而任一
		度量空间中的完备子空间必是闭子集。
	\end{exercise}
	
	\begin{proof}
		先证:完备空间的闭子集是一个完备的子空间
		
		不妨记子集为$M$,由于$M$为闭集,其含有其所有的收敛点,即
		\begin{equation*}
			\forall x_n\in M,\lim_{n\rightarrow \infty} x_n=x\in M
		\end{equation*}
		
		于是,我们有
		\begin{equation*}
			\forall x_n,x_m\in M,\left\| x_n-x_m \right\| \rightarrow 0
		\end{equation*}
		
		再证:任一
		度量空间中的完备子空间必是闭子集,即
		\begin{equation*}
			\forall x_n\in M,\lim_{n\rightarrow \infty} x_n=x\in M
		\end{equation*}。
		
		由完备性,我们有
		\begin{equation*}
			\forall x_n,x_m\in M,\left\| x_n-x_m \right\| \rightarrow 0
		\end{equation*}
		
		此时显然有属于$M$的极限,否则就不完备 
	\end{proof}
	\begin{exercise}
		设$f$是定义在$[a,b]$上的二次连续可微的实值函数,$\widehat{x}\in(a,b)$使得$f(\widehat{x})=0,f^\prime(\widehat{x})\neq0$.求证:存在$\widehat{x}$的邻域$U(\widehat{x})$,使得$\forall x_0\in U(\widehat{x})$,迭代序列
		\begin{equation*}
			x_{n+1}=x_n-\frac{f(x_n)}{f^{\prime}(x_n)}\quad(n=0,1,2,\cdots)
		\end{equation*}
		是收敛的,并且
		\begin{equation*}
			\lim_{n\to\infty}x_n=\widehat{x}.
		\end{equation*}
	\end{exercise}
	\begin{proof}
		不妨记
		\begin{equation*}
			T(x)=x-\frac{f(x)}{f^{\prime}(x)}
		\end{equation*}
		
		此时
		\begin{equation*}
			T'(x)=\frac{f(x)f''\left( x \right)}{\left( f'(x) \right) ^2}
		\end{equation*}
		
		于是,$\exists\widehat{x}\in(a,b)$,s.t
		\begin{equation*}
			T'(\widehat{x})=0
		\end{equation*}
		
		故存在,$\widehat{x}$的一个领域,使得只要$x$取自该领域,就有
		\begin{equation*}
			|T'|=\alpha\le1
		\end{equation*}
		
		显然满足压缩映射条件,故收敛且收敛点唯一
	\end{proof}
	\begin{exercise}
		设$(\mathscr{X},\rho)$是度量空间,映射$T:\mathscr{X}\to\mathscr{X}$满足
		\begin{equation*}
				\rho(Tx,Ty)<\rho(x,y)\quad(\forall x\neq y),
		\end{equation*}
		并已知 $T$ 有不动点,求证:此不动点是唯一的.
	\end{exercise}
	\begin{proof}
		不妨设不动点为$x$,下证唯一性
		
		若不唯一,不妨设$y$为另一个不动点
		
		由不动点的性质,我们有
		\begin{equation*}
			\rho(Tx,Ty)=\rho(x,y)
		\end{equation*}
		
		与题设矛盾
	\end{proof}
	\begin{exercise}
	$	\text{设 }T\text{ 是度量空间上的压缩映射},\text{ 求证: }T\text{ 是连续的}.$
	\end{exercise}
	\begin{proof}
		
		由于其为压缩映射,故$\exists \alpha <1$,使得
		\begin{equation*}
			\rho(Tx_n,Tx)<\alpha\rho(x_n,x)\rightarrow 0
		\end{equation*}
		
		故连续
	\end{proof}
	\begin{exercise}
	$\text{设 }T\text{ 是压缩映射},\text{ 求证: }T^n(n\in\mathbb{N})\text{ 也是压缩映射},\text{ 并}\text{说明逆命题不一定成立}.$
	\end{exercise}
	\begin{proof}
		
		由于$T$其为压缩映射,故$\exists 0<\alpha <1$,使得
		\begin{equation*}
			\rho \left( T^nx,T^ny \right) <\alpha \rho \left( T^{n-1}x,T^{n-1}y \right) <\cdots <\alpha ^{n-1}\rho \left( Tx,Ty \right) <\alpha ^n\rho \left( x,y \right) 
		\end{equation*}
		
		显然成立
	\end{proof}
	\begin{exercise}
		设$M$是$(\mathbb{R}^n,\rho)$中的有界闭集,映射$T:M\to M$满足:$\rho(Tx,Ty)<\rho(x,y)(\forall x,y\in M,x\neq y)$.求证:$T$在$M$中存在
		唯一的不动点
	\end{exercise}
	\begin{proof}
		
		由$\rho(Tx,Ty)<\rho(x,y)(\forall x,y\in M,x\neq y)$,则存在$\alpha\in(0,1)$,使得
		\begin{equation*}
			\rho(Tx,Ty)<\alpha\rho(x,y)(\forall x,y\in M,x\neq y).
		\end{equation*}
		
		由压缩映射定理,得证
	\end{proof}
	\begin{exercise}
	 对于积分方程
		
		\begin{equation*}
			x(t)-\lambda\int_0^1\mathrm{e}^{t-s}x(s)\mathrm{d}s=y(t)
		\end{equation*}
		其中$y(t)\in C[0,1]$为一给定函数,$\lambda$为常数,$|\lambda|<1$,求证:存在
		唯一解$x(t)\in C[0,1].$
	\end{exercise}
	\begin{proof}
		
		变形可得
		\begin{equation*}
			e^{-t}x(t)=e^{-t}y(t)+\lambda \int_0^1{\mathrm{e}^{-s}}x(s)\mathrm{d}s
		\end{equation*}
		
		如下定义
		\begin{equation*}
			T:e^{-t}x(t)\rightarrow e^{-t}y(t)+\lambda \int_0^1{\mathrm{e}^{-s}}x(s)\mathrm{d}s
		\end{equation*}
		
		于是我们有
		\begin{equation*}
			\rho(Tu,Tv)=\max_{t\in[0,1]}\left|\lambda\int_0^1u(s)ds-\lambda\int_0^1v(s)ds\right|\le|\lambda|\max_{t\in[0,1]} \int_0^1|u(s) - v(s)|ds = |\lambda|\max_{t\in[0,1]}|u(t) - v(t)| = |\lambda|\rho(u,v)
		\end{equation*}
		
		由压缩映射原理得证
	\end{proof}
	\section{完备化}
	\begin{exercise}
		
		\text{令 }S\text{ 为一切实 (或复)数列}
		\begin{equation*}
			x=(\xi_1,\xi_2,\cdots,\xi_n,\cdots)
		\end{equation*}
		\text{组成的集合,在 }S\text{ 中定义距离为}
		
		\begin{equation*}
			\rho(x,y)=\sum_{k=1}^\infty\frac1{2^k}\cdot\frac{|\xi_k-\eta_k|}{1+|\xi_k-\eta_k|},
		\end{equation*}		
	$	\text{其中 }x=(\xi_1,\xi_2,\cdots,\xi_k,\cdots),y=(\eta_1,\eta_2,\cdots,\eta_k,\cdots).\text{ 求证:}S\text{ 为}\text{一个完备的度量空间}.$
	\end{exercise}
	\begin{proof}
		显然,满足非负性和对称性,下证满足三角不等式
		
		考虑函数
		\begin{equation*}
			f\left( x \right) =\frac{x}{1+x}
		\end{equation*}
		
		显然其单调递增,又有
		\begin{equation*}
			|x+y|<|x|+|y|
		\end{equation*}
		
		故
		\begin{equation*}
			\frac{|x+y|}{1+|x+y|}\le \frac{|x|}{1+|x|}+\frac{|y|}{1+|y|}
		\end{equation*}
		
		带入验证即可证明范数成立,下证完备性
		
		$\text{记}x_m=(\xi _{1m},\xi _{2m},\cdots ,\xi _{km},\cdots )$为$S$中的基本列
		
		只要
		\begin{equation*}
			\rho(x_{n+p},x_n)\rightarrow 0
		\end{equation*}
		即可
		
		由于是基本列,此时
		\begin{equation*}
			\rho (x_{n+p},x_n)=\sum_{k=1}^{\infty}{\frac{1}{2^k}}\cdot \frac{|\xi _{k\left( k+p \right)}-\xi _{kk}|}{1+|\xi _{k\left( k+p \right)}-\xi _{kk}|}<\frac{\varepsilon}{2^k}		
		\end{equation*}
		
		对于其中的一项,我们有
		\begin{equation*}
			\frac{1}{2^k}\frac{|\xi _{k\left( k+p \right)}-\xi _{kk}|}{1+|\xi _{k\left( k+p \right)}-\xi _{kk}|}<\frac{\varepsilon}{2^k}
		\end{equation*}
		
		此时我们有
		\begin{equation*}
			|\xi _{k\left( k+p \right)}-\xi _{kk}|<\frac{\varepsilon}{1-\varepsilon}
		\end{equation*}
		
		
	\end{proof}
	\begin{exercise}
		在一个度量空间 $(\mathscr{X},\rho)$ 上,求证:基本列是收敛列,当
		且仅当其中存在一串收敛子列.
	\end{exercise}
	\begin{proof}
		左推右,显然成立,因为基本列本身就是存在收敛子列的
		
		右推左,一句话证明,子列收敛有极限,子列收敛于数列本身,基本列收敛。剩下的是常规数分写法
	\end{proof}
	\begin{exercise}
		 设$F$是只有有限项不为0的实数列全体,在$F$上引进距离
		 
		 \begin{equation*}
		 	\rho(x,y)=\sup_{k\geqslant1}|\xi_k-\eta_k|,
		 \end{equation*}		 
		 $\text{其中 }x=\{\xi_k\}\in F,y=\{\eta_k\}\in F,\text{ 求证: }(F,\rho)\text{ 不完备,并指出它}\text{的完备化空间.}$
	\end{exercise}
	\begin{proof}
		$x^{(n)}=\left(\underbrace{1,\frac{1}{2},\frac{1}{3},\cdots,\frac{1}{n}}_{n},0,0,\cdots\right)\in F,x^{(m)}=\left(\underbrace{1,\frac{1}{2},\frac{1}{3},\cdots,\frac{1}{m}}_{m},0,0,\cdots\right)\in F$
		
		不妨设$n>m$
		
		\begin{equation*}
			\rho(x^{(n)},x^{(m)})=\frac{1}{m+1}\rightarrow0
		\end{equation*}
		
		故为柯西列,下证完备性,不妨设收敛于$x$
		
		事实上
		\begin{equation*}
			\rho(x^{(n)},x)=\frac{1}{n+1}\rightarrow0
		\end{equation*}
		但$x$$\notin F$,故不完备
	\end{proof}
	\begin{exercise}
		\text{求证:}[0,1]\text{ 上的多项式全体按距离}
		
		\begin{equation*}
			\rho(p,q)=\int_0^1\lvert p(x)-q(x)\rvert\mathrm{d}x\quad(p,q\text{ 是多项式})
		\end{equation*}
		\text{是不完备的},\text{并指出它的完备化空间}.
	\end{exercise}
	\begin{proof}
		不妨设$p(x)=x^{p},q(x)=x^q(p<q)$
		
		则
		\begin{equation*}
			\rho (p,q)=\int_0^1{\left| p(x)-q(x) \right|}\mathrm{d}x=\int_0^1{x^q-x^p}\mathrm{d}x=\frac{1}{q+1}-\frac{1}{p+1}\rightarrow 0
		\end{equation*}
		
		直观上,多项式会收敛到任意连续函数,但连续函数不属于多项式空间,故不完备,完备化空间应为连续函数构成的空间
	\end{proof}
	\begin{exercise}
		在完备的度量空间 $(\mathscr{C},\rho)$ 中给定点列 $\{x_n\}$,如果 $\forall\varepsilon>0$,
		存在基本列$\{y_n\}$,使得
		\begin{equation*}
			\rho(x_n,y_n)<\varepsilon\quad(n\in\mathbb{N}),
		\end{equation*}
		$\text{求证: }\{x_n\}\text{ 收敛}.$
	\end{exercise}
	\begin{proof}
		\begin{equation*}
		\rho \left( x_n,y \right) <\rho \left( x_n,y_n \right) +\rho \left( y_n,y \right) =2\varepsilon 
		\end{equation*}
	\end{proof}
	\section{列紧性}
	\begin{exercise}
		在完备的度量空间中求证:子集 $A$ 列紧的充要条件是
		对$\forall\varepsilon>0$,存在$A$的列紧的$\varepsilon$网
	\end{exercise}
	\begin{proof}
		
		左推右显然
		
		右推左
		
		不妨设$N$是$A$的列紧的$\varepsilon$网,于是于是其存在有限$\varepsilon$网,记其为$M$,于是取$x_n \in A,y_{n}\in N,z_n \in M$
		
		我们有
		\begin{equation*}
			\rho(x_{n},z_{n})<\rho(x_{n},y_{n})+\rho(y_{n},z_{n})<2\varepsilon
		\end{equation*}
		
		于是得证
	\end{proof}
	\begin{exercise}
		在度量空间中求证:紧集上的连续函数必是有界的,并
		且达到它的上、下确界。
	\end{exercise}
	\begin{exercise}
		在度量空间中求证:完全有界的集合是有界的,并通过
		考虑$l^2$的子集$E=\{e_k\}_{k=1}^{\infty}$,其中
		\begin{equation*}
			e_k=\{\underbrace{0,0,\cdots,0,1}_k,0,\cdots\},
		\end{equation*}
		
		来说明一个集合可以是有界但不完全有界的
	\end{exercise}
	\begin{exercise}
		设 $(\mathscr{X},\rho)$ 是度量空间,$F_1,F_2$ 是它的两个紧子集,求
		证:$\exists x_i\in F_i(i=1,2)$, 使得 $\rho(F_1,F_2)=\rho(x_1,x_2)$,其中
		\begin{equation*}
			\rho(F_1,F_2)\triangleq\inf\left\{\rho(x,y)|x\in F_1,y\in F_2\right\}.
		\end{equation*}
	\end{exercise}
	\begin{exercise}
		\text{设 }M\text{ 是 }C[a,b]\text{ 中的有界集, 求证: 集合}
		\begin{equation*}
			\left\{F(x)=\int_a^xf(t)\mathrm{d}t|f\in M\right\}
		\end{equation*}
		是列紧集
	\end{exercise}
	\begin{exercise}
		$\text{设 }E=\{\sin nt\}_{n=1}^\infty,\text{ 求证:}E\text{ 在 }C[0,\pi]\text{ 中不是列紧的}.$
	\end{exercise}
	\begin{exercise}
		求证:$S$空间 (定义见习题1.2.1)的子集$A$列紧的充要条件是:$\forall n\in\mathbb{N},\exists C_n>0$,使得对$\forall x=(\xi_1,\xi_2,\cdots,\xi_n,\cdots)\in A$, 有$|\xi_n|\leqslant C_n(n=1,2,\cdots).$
	\end{exercise}
	\begin{exercise}
		$\text{设 }(\mathscr{H},\rho)\text{ 是度量空间},M\text{ 是 }\mathscr{H}\text{ 中的列紧集},\text{ 映射}\\f:\mathscr{X}\to M\text{ 满足}$
		\begin{equation*}
			\rho(f(x_1),f(x_2))<\rho(x_1,x_2)\quad(\forall x_1,x_2\in\mathscr{X},x_1\neq x_2).
		\end{equation*}
		$\text{求证: }f\text{ 在 }\mathscr{X}\text{ 中存在唯一的不动点}.$
	\end{exercise}
	\begin{exercise}
		设$(M,\rho)$是一个紧度量空间,又$E\subset C(M),E$中的函
		数一致有界并满足下列 Hölder 条件:
		\begin{equation*}
			|x(t_1)-x(t_2)|\leqslant C\rho(t_1,t_2)^\alpha\quad(\forall x\in E,\forall t_1,t_2\in M),
		\end{equation*}
		$\text{其中 }0<\alpha\leqslant1,C>0.\text{ 求证: }E\text{ 在 }C(M)\text{ 中是列紧集}.$
	\end{exercise}
\end{document}