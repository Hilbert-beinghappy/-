\documentclass[lang=cn,10pt]{elegantbook}
\usepackage{graphicx}
\usepackage{float}
\usepackage{gensymb}
\usepackage{txfonts}
\setmainfont{TeX Gyre Termes}
\title{习概}



\author{ Huang}

\extrainfo{这可能是最后一次整理。心伤了}

\setcounter{tocdepth}{3}


\cover{cover.jpg}
\logo{xiao.png}
% 本文档命令
\usepackage{array}
\newcommand{\ccr}[1]{\makecell{{\color{#1}\rule{1cm}{1cm}}}}

% 修改标题页的橙色带
% \definecolor{customcolor}{RGB}{32,178,170}
% \colorlet{coverlinecolor}{customcolor}

\begin{document}
	
	\maketitle
	\frontmatter
	
	\tableofcontents
	
	\mainmatter
	\chapter{必背重点,其余随便}
	(1)试着论述你对价值规律的认识
	
	1、价值规律是商品经济的基本经济规律
	
	2、价值规律的基本内容和要求是:商品的价值量由生产商品必要劳动时间决定:商品交换要按照价值量相等的原则进行,交换。
	
	3、价值规律的作用:
	
	A、自发地调节资源配置和经济活动。
	
	B、刺激商品生产者改进技术、提高劳动生产率,推动社会的发展。
	
	C、经济活动中形成优胜劣汰机制,同时也产生公平与效率矛盾。
	
	同时,价值规律也会造成社会资源的浪费和社会贫富分化。
	
	4、价值规律的作用是通过市场机制(价格机制、竞争机制、供求机制的有机整体)的作用来实现的。
	
	(2)关于劳动价值论
	
	1、马克思劳动价值论是关于劳动创造价值和劳动决定价值的学说。
	
	2、马克思劳动价值论的理论和实践意义:
	
	第一,劳动二重性理论是"理解政治经济学的枢纽"。
	
	第二,马克思劳动价值论揭示了私有制条件下商品经济的基本矛盾,为从物与物的关系背后揭示人与人的关系提供了理论依据。
	
	第三,马克思劳动价值论揭示了商品经济的一般规律,对理解社会主义市场经济具有指导意义。
	
	(3)科学技术在社会发展中的作用
	
	1、科学技术是一个复合概念。
	
	科学是指对客观世界的认识,是反映客观事实和客观规律的知识体系及其相关的活动。
	
	技术是指生产技术,即人类改造自然、进行生产的方法与手段
	
	2、科技革命是推动经济和社会发展的强大杠杆:
	
	(科技革命,不同程度地引起了生产方式、生活方式和思维方式的深刻变化和社会的巨大进步。)
	
	A、对生产方式产生了深刻影响。
	
	其一,改变了社会生产力的构成要素。
	
	其二,改变了人们的劳动形式。
	
	其三,改变了社会经济结构,特别是导致产业结构发生变革。
	
	B、对生活方式产生了巨大影响。
	
	C、促进了思维方式的变革。
	
	(4)生产力与生产关系的辩证关系
	
	1、生产力是人类在生产实践中形成的改造和影响自然以使其适合社会需要的物质力量。
	
	2、生产关系是人们在物质生产过程中形成的不以人的意志为转移的经济关系。
	
	3、生产力与生产关系的相互关系是:生产力决定生产关系,而生产关系又反作用于生产力。

	A、生产力决定生产关系。

	其一,生产力状况决定生产关系的性质。
	
	其二,生产力的发展决定生产关系的变化。
	
	B、生产关系对生产力具有能动的反作用。生产力与生产关系矛盾运动规律的原理具有极为重要的理论意义和现实意义。
	
	(5)社会存在和社会意识的辩证关系
	
	1、社会存在:社会存在也称社会物质生活条件,是社会生活的物质方面,主要包括自然地理环境、人口因素和物质生产方式。
	
	2、社会意识:社会意识是社会生活的精神方面,是社会存在的反映。
	
	
	3、社会存在和社会意识的辩证关系
	
	社会存在和社会意识是辩证统一的。社会存在决定社会意识,社会意识是社会存在的反映,并反作用于社会存在。
	
	4、社会意识的相对独立性。
	
	社会意识的能动作用是通过指导人们的实践活动实现的。
	
	(6)简述认识与实践的辩证关系
	
	1、实践是人类能动地改造世界的社会性的物质活动。
	
	2、认识的本质是主体在实践基础上对客体的能动反映
	
	3、实践与认识的辩证关系:
	
	第一,实践是认识的来源。
	
	第二,实践是认识发展的动力。
	
	第三,实践是认识的目的。
	
	第四,实践是检验认识真理性的唯一标准。
	
	(7)对立统一规律是事物发展的根本规律
	
	(一)矛盾的同一性和斗争性及其在事物发展中的作用
	
	矛盾是反映事物内部和事物之间对立统一关系的哲学范畴。对立和统一体现了矛盾的两种基本属性。矛盾的对立属性又称斗争性,矛盾的统一属性又称同一性。
	
	矛盾的同一性是指矛盾双方相互依存、相互贯通的性质和趋势。
	
	矛盾的斗争性是矛盾着的对立面相互排斥、相互分矛盾的性质和趋势。
	
	矛盾的同一性和斗争性相互联结、相辅相成。
	
	
	(二)矛盾的普遍性和特殊性及其相互关系
	
	矛盾的普遍性是指矛盾存在于一切事物中,存在于一切事物发展过程的始终
	
	矛盾的特殊性是指各个具体事物的矛盾、每一个矛盾的各个方面在发展的不同阶段上各有其特点。
	
	矛盾的普遍性寓于特殊性之中。
	
	(三)
	
	主要矛盾是矛盾体系中处于支配地位、对事物发展起决定作用的矛盾
	
	次要矛盾是矛盾体系中处于从属地位、对事物的发展起次要作用的矛盾
	
	事物的性质是由主要矛盾的主要方面所规定的。
	
	(8)矛盾分析法
	
	1.矛盾是客观事物具有的既相对独立又相互同一的本性
	
	2.矛盾是事物发展的根本动力:要善于发现矛盾并解决矛盾中取得发展
	
	3.矛盾的同一性与斗争性原理:既要把握矛盾的主要方面,又要正确处理矛盾双方的和谐
	
	4.矛盾的普遍性和特殊性原理:
	
	A、立足从实际问题出发,善于学习与借鉴
	
	B、具体问题具体分析
	
	5.主要矛盾和次要矛盾的原理:抓住主要矛盾,兼顾次要矛盾
	
	(9)试述如何发挥意识主观能动性?
	
	1、意识对物质的反作用就是意识的主观能动性
	
	意识的主观能动性表现在:
	
	第一,意识活动具有目的性和计划性。
	
	第二,意识活动具有创造性。
	
	第三,意识具有指导实践改造客观世界的作用。
	
	第四,意识具有调控人的行为和生理活动的作用。
	
	2、要求发挥意识主观能动性和尊重客观规律性的统一
	
	一方面,尊重客观规律是正确发挥主观能动性的前提。
	
	另一方面,只有充分发挥主观能动性,才能正确认识和利用客观
	
	3、发挥意识主观能动性,要有以下三个方面的前提和条件
	
	第一,从实际出发是正确发挥人的主观能动性的前提。
	
	第二,实践是正确发挥人的主观能动性的基本途径。
	
	第三,正确发挥人的主观能动性,还需要依赖于一定的物质条件质手段
	
	(10)简述哲学基本问题
	
	近代哲学的重大的基本问题,是思维和存在的关系问题。"
	存在和思维的关系问题包括两个方面的内容:其一,存在和思维究竟谁是世界的本原,即物质和精神何者是第一性、何者是第二性的问题。对这一问题的不同回答,构成了划分唯物主义和唯心主义的标准。其二,存在和思维有没有同一性,即思维能否正确认识存在的问题。对这一问题的不同回答,构成了划分可知论和不可知论的标准。1对哲学基本问题的回答是解决其他一切哲学问题的前提和基础。
	
	(11)请用唯物史观的相关原理,阐述人民至上
	
	1、人民群众是社会历史实践的主体,在创造历史中起决定性的作用。
	
	2、人民群众是物质财富的创造者。人民群众是精神财富的创造者。人民群众是社会变革中的决定力量。
	
	3、唯物史观关于人民群众是历史创造者原理,要求我们坚持马克思主义众观点,贯彻党的群众路线。群众观点主要内容是:人民群众自己解放自己的观点,全心意为人民服务的观点,一切向人民群众负责的观点,虚心向人民群众学习的观点。群众路是群众观点的具体应用,即一切为了群众,一切依靠群众,从群众中来,到群众中去。群众路的实质,在于充分相信群众,坚决依靠群众,密切联系群众,全心全意为人民群众服务。
	
	4、"我们要始终坚持人民至上"坚持以人民为中心的发展思想,始终走好群众路线。
	
	(12)用认识论的相关原理,阐述真理的检验标准
	
	1、实践是检验真理的唯一标准,此外再也没有别的标准。
	
	2、实践之所以能够作为检验真理的唯一标准,是由真理的本性和实践的特点决定的。
	
	第一,从真理的本性来看,真理是人们对客观事物及其发展规律的正确反映,它的本性在于主观和客观相符合。检验真理的标准,既不能是主观认识本身,也不能是客观事物。只有那种能够把主观认识与客观事物联系和沟通起来,从而使人们能够把二者加以比较和对照的东西,才能充当检验真理的标准。具有这种特性的东西,只能是作为主客观联系的桥梁的社会实践。
	
	第二,从实践的特点来看,实践具有直接现实性,能够把主观的东西变为客观的东西,如果实践的结果与实践之前的认识和预想相符合,那么之前的认识就得到了证实,成为真理性认识。"
	
	3、坚持实践是检验真理的唯一标准,还必须正确地理解实践标准的确定性与不确定性,准确把握实践检验真理的辩证发展过程。
	
	
	(13)唯物辩证法的原理
	
	1、矛盾是反映事物内部或事物之间对立统一关系的哲学范畴。对立和统一分别体现了矛盾的两种基本属性。矛盾的对立属性又称斗争性;矛盾的统一属性又称同一性。
	
	2、矛盾的同一性是矛盾着的对立面相互依存、相互贯通的性质和趋势。矛盾的斗争性是矛盾着的对立面相互排斥、相互分离的性质和趋势。
	
	3、矛盾的同一性是有条件的、相对的,矛盾的斗争性是无条件的、绝对的。矛盾的同一性和斗争性相结合,构成了事物的矛盾运动,推动着事物的变化发展。
	
	(14)请简述物质与意识的辩证关系。
	
	1、物质决定意识。从意识的起源来看,一是,意识是自然界长期发展的产物。二是,意发展的产物。从意识的本质来看,意识是人脑这样一种特殊物质的机能和属性,是客观世界的主要映象
	
	2、意识对物质具有反作用,这种反作用就是意识的能动作用。主要表现在:一是,意识性和计划性。二是,意识活动具有创造性。三是,意识具有指导实践改造客观世界的作用。四,调控人的行为和生理活动的作用。
	
	3、正确认识和把握物质与意识的辩证关系,还需要处理好主观能动性和客观规律性的关重客观规律是正确发挥主观能动性的前提。二是,只有充分发挥主观能动性,才能正确认识和利用客观规律
	
\end{document}