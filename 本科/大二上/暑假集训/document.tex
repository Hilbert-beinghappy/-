\documentclass[lang=cn,10pt]{elegantbook}
\usepackage{graphicx}
\usepackage{float}

\title{课堂笔记}



\author{ Huang}



\setcounter{tocdepth}{3}


\cover{cover.jpg}

% 本文档命令
\usepackage{array}
\newcommand{\ccr}[1]{\makecell{{\color{#1}\rule{1cm}{1cm}}}}

% 修改标题页的橙色带
% \definecolor{customcolor}{RGB}{32,178,170}
% \colorlet{coverlinecolor}{customcolor}

\begin{document}
	
	\maketitle
	\frontmatter
	
	\tableofcontents
	
	\mainmatter
\chapter{多元函数的极限和连续}
\section{2023/8/31}
\begin{definition}
	$\text{设}F\subset \mathbb{R} ^n\text{,若}F^c\text{为开集,则称}F\text{为闭集}$
\end{definition}
\begin{theorem}
	在$\mathbb{R} ^n$中
	
	$1.\mathbb{R} ^n,\oslash$均为闭集(很特殊的一点,这两个同时为开集和闭集)
	
	$2.\text{若}{F_{\alpha}}$为$\mathbb{R} ^n$中的一个闭集族,其中指标$\alpha$来自一个指标集合$I$,则$\bigcap_{\alpha \in I}{F_{\alpha}}$也是闭集
	
	$3.$设$F_{1},F_{2},\cdots,F_{m}$为有限个闭集,则它们的并集$\bigcup_{i=1}^m{F_i}$也是闭集
\end{theorem}
令$\check{B}\left( a,r \right) =B\left( a,r \right) \backslash\{a\}\text{为以}a\text{为心,半径为}r\text{的空心球}$
\begin{definition}
	$\text{设}E\subset \mathbb{R} ^n$,若点$a\in \mathbb{R} ^n$满足:对任意$r>0$,在空心
	球 $\check{B}\left( a,r \right)$总含有$ E$ 中的点,则称 $a$ 为 $E$ 的聚点
\end{definition}
\begin{remark}
	聚点可以属于$ E $也可以不属于 $E$。若 $E $中的点不是
	聚点,则称其为孤立点。
\end{remark}
\begin{definition}
	$E\subset \mathbb{R} ^n$的凝聚点的全体称为 $E$ 的导集,记作
	$E′$,记$\bar{E}=E\cup E'$ ,称其为 $E$ 的闭包
\end{definition}
\begin{theorem}
	$E $为闭集的充要条件是 $E'\subset  E$,即$\bar{E}=E$
\end{theorem}
\begin{proof}
	
	$\Longrightarrow $我们要证明的是包含关系,于是先把$E'$中的元素设出来,不妨假设$x\in E'$,只需要证明$x\text{同时也}\in E$即可,但直接证明不是很好证明,考虑反证法,即$x\notin E\Rrightarrow x\in E^c$
	
	点在集合中,直接默写定义
	\begin{equation*}
		\exists r>0,\text{使得}  B(x,r)\subset E^c
	\end{equation*}
	
	于是乎,有
	\begin{equation}
		\hat{B}\left( x,r \right) \cap E=\oslash 
	\end{equation}
	
	根据聚点的定义,我们是对任意的$r>0$的去心球均有交集,但我们推出了如$(1.1)$的结论,表明其不为$E$的聚点,故
	\begin{equation*}
		x\in E'
	\end{equation*}
	与先前假设$x\in  E'$ 矛盾,于是假设错误
	
	$\Longleftarrow$现在我们要证明$E$为闭集通常像这类的证明,我们要将其转化为该集合的补集进行证明。故我们只需证明$E^{c}$为开集即可,但事实上,直接进行证明有些难度,我们采取反证法
	
	假设$E^{c}$非开,按定义直接写(把开集的定义反着来)
	\begin{equation*}
		\exists x \in E^{c},\forall r >0,B(x,r)\cap E\ne\oslash (\text{这里的结论是}B(x,r)\subset E^{c}\text{的反例,不属于这个集合,等价于和这个集合的补集有交集})
	\end{equation*}
	
	$x$显然不在$E$中,则$x$为$E$的一个聚点,于是有
	\begin{equation*}		
		x\in E' \subset E
	\end{equation*}
	与$x \in E^{c}$矛盾,假设错误
\end{proof}
\begin{corollary}
	$E $是闭集的充要条件是$ E $中的任何收敛点列的极限必
	在 $E$ 中
\end{corollary}
\begin{proof}
	$\Longrightarrow$设$\left\{ a_n \right\} \subset E$ 且$\underset{n\rightarrow \infty}{\lim}a_n=x$,直接证明不好证明,我们采取反证法,假设$x\in E^{c}$,又因为$E^{c}$为开集,按定义
	\begin{equation*}
		\exists r>0,B(x,r)\subset E^{c}
	\end{equation*}
	
	又因为$a_{n}$收敛于$x$,于是存在无穷多个$a_{n}$在$x$附近的一个开球$B(x,r)$中,故$x$为$E$的一个聚点于是有
	\begin{equation*}		
		x\in E' \subset E
	\end{equation*}
	
	与$x \in E^{c}$矛盾,假设错误
	$\Longleftarrow$现在我们要证明$E$为闭集通常像这类的证明,我们要将其转化为该集合的补集进行证明。故我们只需证明$E^{c}$为开集即可,但事实上,直接进行证明有些难度,我们采取反证法
\end{proof}
\begin{corollary}
	完备度量空间的闭子集是完备集。
\end{corollary}
\begin{proof}
	
\end{proof}
\begin{theorem}
	$E $的导集 $E′$ 和闭包 $\bar E$ 均为闭集
\end{theorem}
\begin{proof}
	
\end{proof}
\end{document}