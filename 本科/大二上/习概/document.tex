\documentclass[lang=cn,10pt]{elegantbook}
\usepackage{graphicx}
\usepackage{float}
\usepackage{gensymb}
\usepackage{txfonts}
\setmainfont{TeX Gyre Termes}
\title{习概}



\author{ Huang}



\setcounter{tocdepth}{3}


\cover{cover.jpg}
\logo{xiao.png}
% 本文档命令
\usepackage{array}
\newcommand{\ccr}[1]{\makecell{{\color{#1}\rule{1cm}{1cm}}}}

% 修改标题页的橙色带
% \definecolor{customcolor}{RGB}{32,178,170}
% \colorlet{coverlinecolor}{customcolor}

\begin{document}
	
	\maketitle
	\frontmatter
	
	\tableofcontents
	
	\mainmatter
	\chapter{{\color{red} \text{以中国式现代化全面推进中华民族伟大复兴}}}
	\section{中国式现代化是强国建设、民族复兴的唯一正确道路}
	
	1.中国式现代化是中国共产党领导人民长期探索和实践的重大成果
	
	(1)\textbf{新民主主义革命时期},我们党团结带领人民进行革命斗争,取得新民主主义革命的胜利,\textbf{为实现现代化创造了根本社会条件}。
	
	(2)\textbf{社会主义革命和建设时期},我们党带领人民完成了社会主义革命、推进了社会主义建设,\textbf{为现代化建设奠定了根本政治前提和制度基础,提供了宝贵经验、理论准备、物质基础。}
	
	(3)\textbf{改革开放和社会主义现代化建设新时期},我们党作出把党和国家工作中心转移到经济建设上来、实行改革开放的历史性决策,开启了中国式现代化的新长征,\textbf{为中国式现代化提供了充满新的活力的体制保证和快速发展的物质条件}。
	
	(4)\textbf{进入新时代},我们党围绕解决现代化建设中存在的突出矛盾和问题,不断实现理论和实践上的创新突破,\textbf{成功推进和拓展了中国式现代化}。
	
	2.中国式现代化的中国特色(5个方面)
	
	(1)\textbf{中国式现代化是人口规模巨大的现代化}。这是中国式现代化的\textbf{显著特征}。
	
	(2)\textbf{中国式现代化是全体人民共同富裕的现代化}。这是中国式现代化
	的\textbf{本质特征},也是区别于西方现代化的\textbf{显著标志}。
	
	(3)\textbf{中国式现代化是物质文明和精神文明相协调的现代化}。物质富足、精神富有是社会主义现代化的\textbf{根本要求}。既要物质富足、也要精神富有,是中国式现代化的\textbf{崇高追求}。
	
	(4)\textbf{中国式现代化是人与自然和谐共生的现代化}。尊重自然、顺应自然、保护自然,促进人与自然和谐共生,是中国式现代化的\textbf{鲜明特点}。
	
	(5)\textbf{中国式现代化是走和平发展道路的现代化}。坚持和平发展,在坚定维护世界和平与发展中谋求自身发展,又以自身发展更好维护世界和平与发展,推动构建人类命运共同体,是中国式现代化的\textbf{突出特征}。
	
	3.中国式现代化的本质要求
	
	(1)本质要求(9条):坚持中国共产党领导,坚持中国特色社会主义,实现高质量发展,发展全过程人民民主,丰富人民精神世界,实现全人民共同富裕,促进人与自然和谐共生,推动构建人类命运共同体,创造人类文明新形态。
	
	这一本质要求符合人类现代化的一般规律,阐明了中国式现代化的内在规定性,明确了中国式现代化的领导力量、发展道路和根本方向、总体布局和战略要求,以及对人类文明和世界发展的重大意义,是推进中国式,现代化的重要遵循。
	
	(2)中国式现代化是中国共产党领导的社会主义现代化,这是对中国式现代化的定性,是管总、管根本的。党的领导直接关系中国式现代化的\textbf{根本方向、前途命运、最终成败}。
	
	\textbf{第一,党的领导决定中国式现代化的根本性质}(是社会主义现代化)
	
	\textbf{第二,党的领导确保中国式现代化锚定奋斗目标行稳致远}(一代一代地接力推进)。
	
	\textbf{第三,党的领导激发建设中国式现代化的强劲动力}(改革开放)。
	。
	
	\textbf{第四,党的领导凝聚建设中国式现代化的磅礴力量}(汇集全体人民的
	智慧和力量)。
	
	4.中国式现代化创造了人类文明新形态
	
	中国式现代化,\textbf{深深植根于}中华优秀传统文化,\textbf{体现}科学社会主义的先进本质,\textbf{借鉴吸收}一切人类优秀文明成果,\textbf{代表}人类文明进步的发展方向,\textbf{是一种全新的人类文明形态}。
	
	(1)\textbf{中国式现代化提供了一种全新的现代化模式}。人类走向现代化并不是只有一条路,中国式现代化,打破了"现代化=西方化"的迷思,展现了不同于西方现代化模式的新图景。
	
	(2)\textbf{中国式现代化是对西方式现代化理论和实践的重大超越}。中国现代化蕴含的独特世界观、价值观、历史观、文明观、民主观、生态观及其伟大实践,是对世界现代化理论和实践的重大创新。
	
	(3)\textbf{中国式现代化为广大发展中国家}独立自主迈向现代化、探索现化道路的多样性提供了\textbf{全新选择}
	\chapter{坚持以人民为中心}
	\section{坚持人民至上}
	
	坚持人民至上,是我们党百年奋斗的宝贵历史经验,也是\textbf{新时代党治国理政的根本价值取向}。
	
	1.人民对美好生活的向往就是党的奋斗目标
	
	(1)\textbf{坚持人民至上,必须始终把人民放在心中最高的位置},想人民之所想,行人民之所嘱。
	
	(2)为人民谋幸福是党始终坚守的初心,让人民过上好日子是党\textbf{一贯的追求}。我们党的百年奋斗\textbf{归根到底就是让中国人民过上好日子,实现国家富强和民族振兴}。
	
	(3)人民对美好生活的向往就是党的奋斗目标,\textbf{这体现了我国社会主要矛盾转化对党和国家工作的新要求}。要在高质量发展中努力为人民创造更美好、更幸福的生活。
	
	2.依靠人民创造历史伟业
	
	(1)\textbf{人民是我们党的生命之根、执政之基、力量之源}。一路走来,党紧紧依靠人民赢得胜利、取得成功。面向未来,党永远要依靠人民创造新的历史伟业。
	
	(2)依靠人民创造历史伟业的要求
	
	①必须尊重人民主体地位。
	
	②必须尊重人民首创精神。
	
	3.人民是党的工作的最高裁决者和最终评判者
	
	(1)时代是出卷人,我们是答卷人,人民是阅卷人。人民是党的工作
	的最高裁决者和最终评判者。
	
	(2)让群众满意是我们党做好一切工作的\textbf{价值取向和根本标准}。
	
	(3)必须牢固树立和践行正确的政绩观。
	
	\chapter{推动高质量发展}
	\section{加快构建创新发展格局}
	
	1.把握未来发展主动权的战略部署
	
	(1)构建新发展格局是事关全局的系统性、深层次变革,是立足当前着眼长远的战略谋划,\textbf{是适应我国发展新阶段要求、塑造国际合作和竞争新优势的必然选择。}
	
	改革开放以来特别是加入世界贸易组织后,我国深度参与国际分工,融入国际大循环,\textbf{形成市场和资源"两头在外"的发展格局},对我们抓住经济全球化机遇快速提升经济实力、改善人民生活发挥了重要作用。但是,近几年,全球政治经济环境变化,逆全球化趋势加剧,市场和资源"两头在外"的国际大循环动能明显减弱,而国内大循环活力日益强劲。在这种情况下,\textbf{必须进一步把发展立足点放在国内,更多依靠国内市场实现经济发展}。我们党提出构建新发展格局,\textbf{是对我国客观经济规律和发展趋势的自觉把握}。
	
	②\textbf{构建新发展格局是把握未来发展主动权的战略性布局和先手棋,不是被迫之举和权宜之计}。在当前国际形势充满不稳定性不确定性的背景下,立足国内、依托国内大市场优势,充分挖掘内需潜力,\textbf{有利于化解外部冲击和外需下降带来的影响,也有利于在极端情况下保证我国经济基本正常运行和社会大局总体稳定}。
	
	(2)正确认识构建新发展格局
	
	构建新发展格局是\textbf{开放的国内国际双循环},不是封闭的国内单循环。
	
	构建新发展格局是\textbf{以全国统一大市场基础上的国内大循环为主体}不是各地都搞自我小循环。
	
	2.以国内大循环为主体、国内国际双循环相互促进
	
	(1)\textbf{构建新发展格局必须具备强大的国内经济循环体系和稳固的基本盘},保持国内经济持续健康发展,巩固和发展我国经济的强大竞争力。国内循环越顺畅,越能形成对全球资源要素的引力场。要加强国内大循环在双循环中的\textbf{主导作用},把稳定经济运行作为\textbf{重点}。
	
	(2)\textbf{构建新发展格局必须发挥比较优势,以国内大循环吸引全球资源要素}。
	
	(3)\textbf{构建新发展格局必须保证经济循环的畅通无阻,实现生产、分配、流通、消费各环节有机衔接}。
	
	3.大力推动构建新发展格局
	
	(1)\textbf{着力推动实施扩大内需战略同深化供给侧结构性改革有机结合}。供给和需求是市场经济内在关系\textbf{的两个基本方面,是既对立又统一的辩证关系。要以满足国内需求为基本立足点},加快培育完整内需体系,着力提升供给体系对国内需求的适配性,把实施扩大内需战略同深化供给侧结构性改革有机结合起来,形成需求牵引供给、供给创造需求的更高水平动态平衡。
	
	(2)\textbf{着力发展实体经济}。实体经济是一国经济的\textbf{立身之本、财富之源},是构筑未来发展战略优势的\textbf{重要支撑}。要坚持把发展经济的\textbf{着力点}放在实体经济上。
	
	(3)\textbf{着力加快科技自立自强}。构建新发展格局\textbf{最本质的特征}是实现高水平的自立自强。
	
	(4)\textbf{着力推动产业链供应链优化升级}。建立稳定的产业链供应链,是稳固国内大循环主体地位、增强在国际大循环中带动能力的迫切需要。
	\chapter{社会主义现代化建设的教育、科技、人才战略}
	\section{加快建设科技强国}
	1.科技强则国家强
	
	科技是国之利器。
	
	(1)\textbf{实现高水平科技自立自强是国家强盛和民族复兴的战略基石}。全面建成社会主义现代化强国关键看科技自立自强,这是促进发展大局的根本支撑。
	
	(2)\textbf{实现高水平科技自立自强是应对风险挑战和维护国家利益的必然选择}。科技自立自强是国家强盛之基、安全之要。
	
	(3)\textbf{实现高水平科技自立自强是构建新发展格局、推动高质量发展满足人民美好生活需要的内在要求}。
	
	2.打赢关键核心技术攻坚战
	
	(1)关键核心技术是国之重器,\textbf{关键核心技术必须掌握在自己手}中关键核心技术是要不来、买不来、讨不来的。\textbf{只有把最关键最核心的技术掌握在自己手中,才能从根本上保障国家经济安全、国防安全和其他安全。}建设世界科技强国,需要标志性科技成就。掌握关键核心技术,才能真正掌握国际竞争的主动权、掌控产业发展的主导权,打造经济社会发展新引擎、赢得未来发展新优势。要以关键共性技术、前沿引领技术、现代工程技术、颠覆性技术创新为突破口,集聚力量进行原创性引领性科技功关,敢于走前人没走过的路,努力实现关键核心技术自主可控。
	
	(2)打赢关键核心技术攻坚战的举措。
	
	①打赢关键核心技术攻坚战,必须深入推进科技体制改革。
	
	②发挥新型举国体制优势开展科技攻关。
	
	3.增强自主创新能力
	
	(1)自力更生是中华民族自立于世界民族之林的\textbf{奋斗基点},自主创新是我们攀登世界科技高峰的\textbf{必由之路}。实现高水平科技自立自强,必须\textbf{增强自主创新能力},坚定不移走中国特色自主创新道路,\textbf{加强原创性、引领性科技攻关},强化国家战略科技力量,推动科技创新与经济社会发展紧密结合。
	
	(2)自主创新,就是从增强国家创新能力出发,加强\textbf{原始创新、集成创新和开放创新},确保国家拥有自主可控的科技创新能力。
	
	(3)增强自主创新能力举措。
	
	\textbf{基础研究是科技创新的源头}。要强化基础研究前瞻性、战略性、系统性布局,有组织推进\textbf{战略导向的体系化基础研究、前沿导向的探索性基础研究、市场导向的应用性基础研究},把握科技发展大趋势,下好自主创新先手棋。
	
	\textbf{国家战略科技力量是世界科技强国竞争的着力点}。
	
	\chapter{全面依法治国}
	\section{建设中国特色社会主义法治体系}
	1.全面推进依法治国的总抓手——建设中国特色社会主义法治体系
	
	(1)建设中国特色社会主义法治体系、建设社会主义法治国家,是坚持和发展中国特色社会主义的内在要求。
	
	(2)中国特色社会主义法治体系,\textbf{本质}上是中国特色社会主义制度的法律表现形式。
	
	2.坚持依宪治国、依宪执政
	
	(1)为什么必须坚持依宪治国、依宪执政
	
		\textbf{宪法是国家的根本法,是治国安邦的总章程,是党和人民意志的集中体现,是国家各种制度和法律法规的总依据},具有最高的法律地位、法律权威、法律效力。
		
	(2)坚持依宪治国、依宪执政的要求
	
	①必须坚持党的领导地位和我国国体、政体不动摇。
	
	②必须全面贯彻实施宪法。加强宪法实施,必须坚持维护宪法权威
	和尊严。
	
	3.更好推进中国特色社会主义法治体系建设
	
	法治体系是国家治理体系的骨干工程。更好推进中国特色社会主义法治体系建设,必须加快形成完备的法律规范体系、高效的\textbf{法治实施体系}、严密的\textbf{法治监督体系}、有力的\textbf{法治保障体系},形成完善的\textbf{党内法规体系}。
	
	(1)法律规范体系是以宪法为核心,由部门齐全、结构严谨、内部协调、体例科学、调整有效的法律规范所构成的有机整体。
	
	(2)法治实施体系包括执法、司法、守法等各个环节的协调高效运转。
	
	(3)\textbf{法治监督体系以规范和约束公权力的运行为重点}。
	
	(4)法治保障体系包括政治、思想、组织、制度、队伍、科技等保障
	条件。
	
	(5)把党内法规体系纳入中国特色社会主义法治体系,是我国法治区别于其他国家法治的\textbf{鲜明特征}。
	\chapter{建设社会主义生态文明}
	\section{建设美丽中国}
	
	1.加快形成绿色生产方式和生活方式
	
	(1)\textbf{绿色发展是新发展理念的重要内容},是发展观的一场深刻革命。要加快推动发展方式绿色低碳转型,坚持把\textbf{绿色低碳发展}作为解决生态环境问题的\textbf{治本之策},加快形成绿色生产方式和生活方式,厚植高质量发展的绿色底色。
	
	(2)加快形成绿色生产方式和绿色生活方式的举措。
	
	①\textbf{加快推动产业结构、能源结构、交通运输结构等调整优化}。调整优化经济结构是从源头推动发展方式绿色转型的重要任务,要抓住产业结构调整这个关键,促进产业结构变"轻"、发展模式变"绿"。
	
	②\textbf{推进各类资源节约集约利用}。生态环境问题,归根到底是资源过度开发、粗放利用、奢侈消费造成的。加快发展方式绿色转型,必须在转变资源利用方式、提高资源利用效率上下功夫。
	
	③\textbf{积极稳妥推进碳达峰碳中和}。推进"双碳"工作是破解资源环境约束突出问题、实现可持续发展的迫切需要,是一场广泛而深刻的经济社会系统性变革。
	
	④\textbf{健全绿色发展的保障体系}。统筹各领域资源,打好法治、市场、科技、政策"组合拳"。
	
	⑤\textbf{坚持把建设美丽中国转化为全体人民自觉行动}。
	
	2.坚持山水林田湖草沙一体化保护和系统治理
	
	(1)加快实施重要生态系统保护和修复重大工程。
	
	强化国土空间规划和用途管控,划定落实\textbf{生态保护红线、永久基本农田、城镇开发边界三条控制线}以及各类海域保护线。\textbf{以国家重点生态功能区、自然保护地等为重点},突出对国家重大战略的生态支撑。
	
	(2)推进自然保护地体系建设。科学划定自然保护地保护范围及功能分区,构建以国家公园为\textbf{主体}、自然保护区为\textbf{基础}、各类自然公园为\textbf{补充}的自然保护地体系。
	
	(3)科学推进荒漠化、石漠化、水土流失综合治理,持续开展大规模国土绿化行动。
	的自然保护地体系。
	
	(4)实施生物多样性保护重大工程。加强国家重点保护和珍稀濒危野生动植物及其栖息地的保护修复,构筑生物多样性保护网络。
	
	(5)推行草原森林河流湖泊湿地休养生息。降低人为活动对草原、森林、河流、湖泊、湿地资源的干扰强度,推动生态保护修复。
	
	3.用最严格制度最严密法治保护生态环境
	
	保护生态环境必须依靠制度、依靠法治。必须把制度建设作为生态文明建设的重中之重,构建\textbf{产权清晰、多元参与、激励约束并重、系统完整}的生态文明制度体系,把生态文明建设纳入制度化、法治化轨道。
	
	(1)实行最严格的生态环境保护制度。
	
	(2)全面建立资源高效利用制度。
	
	(3)严明生态环境保护责任制度。
	
\end{document}