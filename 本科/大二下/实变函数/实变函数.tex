\documentclass[lang=cn,14pt]{elegantbook}
\usepackage{graphicx}
\usepackage{float}
\usepackage{gensymb}
\usepackage{txfonts}
\usepackage{longtable}
\usepackage{algpseudocode}
\usepackage{listings}
\usepackage{paralist}
\usepackage{array}
\RequirePackage{booktabs}  % 三线表宏包
\setmainfont{TeX Gyre Termes}
\title{实变函数}



\author{ Huang}



\setcounter{tocdepth}{3}


\cover{cover.jpg}
% 本文档命令

\newcommand{\ccr}[1]{\makecell{{\color{#1}\rule{1cm}{1cm}}}}

% 修改标题页的橙色带
% \definecolor{customcolor}{RGB}{32,178,170}
% \colorlet{coverlinecolor}{customcolor}

\begin{document}
	
	\maketitle
	\frontmatter
	
	\tableofcontents
	
	\mainmatter
	\chapter{集合论}
	\chapter{测度论}
	\section{可测函数的定义}
		我们知道,对于一个可测函数来说,只需要做到博雷尔集的原像是博雷尔集即可,在简化我们的需求下,可测函数的定义可简化至下述形式
	\begin{definition}[可测函数]
		$f:E\longmapsto \mathbb{R} \cup \left\{ +\infty ,-\infty \right\} \text{可测},\text{如果}\forall a\in \mathbb{R} ,\left\{ x\in E:f\left( x \right) >a \right\} \coloneqq E\left( f>a \right) \text{可测}
		$
	\end{definition}
	\begin{remark}
		$f(x)>a$,事实上$f(x)\in (a,\infty]$,此时意味着$f$可以取到无穷大
	\end{remark}
	类似的,我们可以推出可测的三个充要条件,只需要注意
\begin{equation*}
	E\left( f\le a \right) =\left( E\left( f>a \right) \right) ^c,E\left( f\ge a \right) =\bigcup_{n=1}^{+\infty}{E\left( f\ge a+\frac{1}{n} \right)}
\end{equation*}
同时我们还可以得到一个可测的必要条件
\begin{theorem}
	$\text{若}f\text{可测,则}E\left( a\le f<b \right) \text{可测}$
\end{theorem}
\begin{proof}
	考虑到
	\begin{equation*}
		E\left( a\le f<b \right) =E(f\ge a)-E(f\ge b)
	\end{equation*}
\end{proof}
\begin{remark}
	此时若$f$有界,则该条件也可为充分条件,因为取不到无穷大
\end{remark}
	\section{可测函数的运算}
	在前面那一章节中,我们已经知道了可测函数的定义,现在我们想知道两个可测集之间的运算的结果,如可测集的加减乘除是否还是可测,首先我们要考虑到两个可测函数之间的和是否可测
	\begin{theorem}
		若$f,g:E\longmapsto \mathbb{R} \cup \left\{ +\infty ,-\infty \right\} \text{可测}$,则$f+g$仍可测
	\end{theorem}
	\begin{proof}
		对于这题的证明,我们按照定义,只需要证明$\forall a,E(f+g>a)$即可证明
		
		由于不等号一边存在两个未知量,考虑将其移到右边一个,即命题等价为$\forall a,E(f>a-g)$,此时显然存在一个$r$,使得$f>r>a-g$,那么命题就成立了
		
		不妨记$h=a-g$,有可测集的简单运算,其可测。故我们只需要证明:存在一个$r$,使得$f>r>h$即可。即求证
		\begin{equation*}
			E\left( f>h \right) =\bigcup_{r\in \mathbb{R}}{E\left( f>r \right) \cap E\left( h<r \right)}
		\end{equation*}
		该命题是显然得证的,只需要左边包含右边,右边包含左边即可得证。故定理得证。
	\begin{theorem}
		$\left\{ f_n\left( x \right) \right\} \text{是}E\text{上至多可数个可测函数}$,我们有
		\begin{equation*}
			\mu \left( x \right) =\mathrm{inf} f_n\left( x \right) ,\lambda \left( x \right) =\mathrm{sup} f_n\left( x \right) 
		\end{equation*}
		仍然可测
	\end{theorem}
 	\begin{proof}
 		由题,我们有
 			\begin{equation*}
 				\begin{split}
 						E\left( \mu \ge a \right) =\bigcap_n{E\left( f_n\left( x \right) \ge a \right)}
 					\\
 					E\left( \lambda \le a \right) =\bigcap_n{E\left( f_n\left( x \right) \le a \right)}
 				\end{split}
 			\end{equation*}

 		由于$f$可测,并且可列个可测集的交仍然为可测集,于是命题得证
 	\end{proof}
	\begin{remark}
		连续函数没有这样的性质,如
		\begin{equation*}
			f_n\left( x \right) =\begin{cases}
				0,x\le 0\\
				nx,0<x\le \frac{1}{n}\\
				1,x>\frac{1}{n}\\
			\end{cases}
		\end{equation*}
		显然连续,但不满足上述的条件
	\end{remark}
	
	为了证明可测函数的绝对值可积,我们如下定义可测函数的正部和负部
	\begin{equation*}
		f=f^+ -f^-
	\end{equation*}
	\end{proof}
	\section{用简单函数逼近可测函数}
	这一部分,我们需要用到一个特别重要的定理,即非负可测函数可用非负简单函数来逼近
	\begin{theorem}
		$E$是$\mathbb{R} ^n$中非负可测函数,则存在非负可测简单函数列$\left\{ \varphi _k\left( x \right) \right\} $,满足$\forall x, \varphi _k \le \varphi _{k+1}$,使得
		\begin{equation*}
			\underset{k\rightarrow \infty}{\lim}\varphi _k\left( x \right) =f\left( x \right) 
		\end{equation*}
	\end{theorem}
	\begin{remark}
		此时若$f$,有界,则可以推出是一致收敛而不是点点收敛
	\end{remark}
	\section{几乎处处收敛与叶果洛夫定理}
	\begin{definition}[几乎处处]
		若某个性质在去掉一个零测集之后成立,我们就称其为几乎处处成立
	\end{definition}
	\begin{theorem}[叶果洛夫定理]
		$\left\{ f_k \right\} ,f\left( x \right) \text{是定义在}E\text{上的可测函数且几乎处处有限,}m\left( E \right) <+\infty $,若$f_k\left( x \right) \rightarrow f\left( x \right) ,k\rightarrow \infty \text{在}E\text{上几乎处处成立}$,$\forall \delta >0,\text{存在}E_{\delta}\subset E,m\left( E_{\delta} \right) \le \delta ,\text{使得}\left\{ f_k \right\} \text{在}E-E_{\delta}\text{上一致收敛于}f$		
	\end{theorem}
	\begin{proof}
		不妨设$\left\{ f_k \right\} ,f\left( x \right)$在$E$上有界,否则令$ E_{\delta}$为原来的$ E_{\delta}$并上前二者无界的部分
		
		此时我们翻译以下题目中的几乎处处收敛的含义,即对于$x$属于不收敛的集合,此时该集合为零测集,用对应的语言如下
		\begin{equation*}
			\exists x\in E,\exists \varepsilon >0,\forall N>0,n>N\text{时,}|f_k\left( x \right) -f\left( x \right) |\ge \varepsilon 
		\end{equation*}
		翻译成集合的语言就是
		\begin{equation*}
			\bigcup_{\varepsilon >0}{\bigcap_{N=1}^{+\infty}{\bigcup_{k=N}^{+\infty}{\left\{ x||f_k\left( x \right) -f\left( x \right) |\ge \varepsilon \right\}}}}
		\end{equation*}
		是个零测集
		
		我们知道
		\begin{equation*}
			m(\bigcap_{N=1}^{+\infty}{\bigcup_{k=N}^{+\infty}{\left\{ x||f_k\left( x \right) -f\left( x \right) |\ge \varepsilon \right\}}})\ge 0
		\end{equation*}
		于是乎,我们可得出
		\begin{equation*}
			m(\bigcap_{N=1}^{+\infty}{\bigcup_{k=N}^{+\infty}{\left\{ x||f_k\left( x \right) -f\left( x \right) |\ge \varepsilon \right\}}})= 0
		\end{equation*}
		上述集合的测度的极限等价于
		\begin{equation*}
			\lim_{k\rightarrow +\infty} m\left( \bigcup_{k=N}^{+\infty}{\left\{ x||f_k\left( x \right) -f\left( x \right) |\ge \varepsilon \right\}} \right) =0
		\end{equation*}
		于是乎,$\forall \bar{\varepsilon}>0,\exists N'(\varepsilon,\bar{\varepsilon}),\forall N'>N(\varepsilon,\bar{\varepsilon})$,我们有
		\begin{equation*}
			m\left( \bigcup_{k=N'}^{+\infty}{\left\{ x||f_k\left( x \right) -f\left( x \right) |\ge \varepsilon \right\}} \right)<\bar{\varepsilon}
		\end{equation*}
		考虑集合
		\begin{equation*}
			\bigcup_{\varepsilon>0}{\bigcap_{N'=1}^{+\infty}{\bigcup_{k=N'}^{+\infty}{\left\{ x||f_k\left( x \right) -f\left( x \right) |\ge\varepsilon \right\}}}}
		\end{equation*}
		此时该集合为零测集,取补即为所要的集合
	\end{proof}
	\begin{remark}
		这个定理说明了,几乎处处收敛和一致收敛只差了一个零测集。
	\end{remark}
	\begin{remark}
		若$m(E)=\infty$则结论不成立
	\end{remark}
	\section{依测度收敛}
	\begin{definition}
		$\left\{ f_k \right\} ,f\left( x \right) \text{是定义在}E\text{上的可测函数且几乎处处有限,}$,如果$\forall \sigma>0 $,$\lim_{k\rightarrow +\infty} m\left( E\left( |f_k\left( x \right) -f\left( x \right) |\ge \sigma \right) \right) =0
		$,则称$\left\{ f_k \right\}$依测度收敛于$f$,记为
		\begin{equation*}
			f_k\left( x \right) \Rrightarrow f\left( x \right) 
		\end{equation*}
	\end{definition}
	在一致收敛的时候,我们有一个结论,就是如果一个数列收敛于两个函数,那么这两个函数相等,同样的,在依测度收敛里,我们也有类似的结论
	\begin{theorem}
		若$\left\{ f_k \right\} ,f\left( x \right),g(x) \text{是定义在}E\text{上的可测函数且几乎处处有限,}$,$f_k\left( x \right) \Rrightarrow f\left( x \right)$,$f_k\left( x \right) \Rrightarrow g\left( x \right)$,则$f$几乎处处相等于$g$
	\end{theorem}
	\begin{proof}
		模仿几乎处处收敛的证明方法,我们可以得到
		\begin{equation*}
			|f-g|<|f-f_k|+|g-f_k|
		\end{equation*}
		于是乎,我们可以得到
		\begin{equation*}
			E\left( |f-g|<\varepsilon \right) \supset E\left( |f-f_k|<\frac{\varepsilon}{2} \right) \cap E\left( |g-f_k|<\frac{\varepsilon}{2} \right) 
		\end{equation*}
		取个补集即可得到
	\end{proof}
	\section{依测度收敛和几乎处处收敛的关系}
	在前面,我们讨论了依测度收敛和几乎处处收敛的关系
	
	现在我们再列举一下
	
	$\forall \varepsilon>0$
	几乎处处收敛
    \begin{equation*}
    	mE\left( \lim_{k\rightarrow +\infty} |f_k-f|\ge \varepsilon\right) =0
    \end{equation*}
    
    依测度收敛
   \begin{equation*}
   	\lim_{k\rightarrow +\infty} mE\left( |f_k-f|\ge \varepsilon \right) =0
   \end{equation*}
   我们发现,这两个定理具有一定程度上的相似性,差别在于极限符号的有所不同,那有什么方法可以建立两者之间的关系呢?在我们前面证明叶果洛夫定理的时候,我们就证明了类似于依测度收敛的结论,其中我们用到了几乎处处有界以及$m(E)<+\infty$,于是乎我们就能得到以下定理
   
 	\begin{theorem}
 		如果满足叶果洛夫定理条件,则$	f_k\left( x \right) \Rrightarrow f\left( x \right)$
 	\end{theorem}
 	证明见上
 	
 	同样的,还有一个定理建立起了依测度收敛和几乎处处收敛的关系
 	\begin{theorem}[里斯定理]
 		$\text{如果在}E\text{上}\left\{ f_n \right\} \text{依测度收敛于}f\text{等价于存在子列}\left\{ f_{n_k} \right\} \text{几乎处处收敛于}f
 		$
 	\end{theorem}
   \chapter{积分理论}
\end{document}