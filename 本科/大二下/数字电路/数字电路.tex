\documentclass[lang=cn,14pt]{elegantbook}
\usepackage{graphicx}
\usepackage{float}
\usepackage{gensymb}
\usepackage{txfonts}
\usepackage{longtable}
\usepackage{algpseudocode}
\usepackage{listings}
\usepackage{paralist}
\usepackage{array}
\RequirePackage{booktabs}  % 三线表宏包
\setmainfont{TeX Gyre Termes}
\title{数值分析}



\author{ Huang}



\setcounter{tocdepth}{3}


\cover{cover.jpg}
% 本文档命令

\newcommand{\ccr}[1]{\makecell{{\color{#1}\rule{1cm}{1cm}}}}

% 修改标题页的橙色带
% \definecolor{customcolor}{RGB}{32,178,170}
% \colorlet{coverlinecolor}{customcolor}

\begin{document}
	
	\maketitle
	\frontmatter
	
	\tableofcontents
	
	\mainmatter
	\chapter{引论}
	\section{数值运算的误差估计}
	我们有如下相对误差限估计公式
	\begin{equation*}
		\begin{aligned}
			|e_r^*(y)& =\left|\frac{e^*(y)}{f(x^*)}\right|  \\
			&=\left|\frac{f(x^{*})-f(x)}{x^{*}-x}\cdot\frac{x^{*}}{f(x^{*})}\cdot\frac{x^{*}-x}{x^{*}}\right| \\
			&\approx\left|\frac{x^*\cdot f^{\prime}(x^*)}{f(x^*)}\right|\cdot\left|e_r^*(x)\right|
		\end{aligned}
	\end{equation*}
	若是多变量
	\section{有效数字与绝对误差、相对误差之间的关系}
		\begin{theorem}
		若$x^{*}$ 有$n$位有效数字,则其相对误差$E_{r}^{*}$满足
		\begin{equation*}
			\varepsilon_{r}^{*}\leq\frac{1}{2a_{1}}10^{-(n-1)}
		\end{equation*}
	\end{theorem}
	\chapter{插值多项式}
	\section{拉格朗日插值多项式及其余项}
	\begin{definition}[拉格朗日插值多项式]
			\begin{equation*}
			\sum_{j=0}^n{\prod_{\begin{array}{c}
						i=0\\
						i\ne j\\
				\end{array}}^n{\frac{(x-x_i)}{(x_j-x_i)}}}
		\end{equation*}
	\end{definition}
	\begin{theorem}[误差余项]
			 设$f(x)$在区间$[a,b]$上$n+1$阶可微,$L_n(x)$为$f(x)$在$[a,b]$\text{上的}$n$次插值多项式,$\text{插值节点为}\{x_i\}_{i=0}^n\subset[a,b],\text{则}\forall x\in[a,b],\text{有}$ 
			 \begin{equation*}
			 		R_n(x)=\frac{f^{(n+1)}(\xi)}{(n+1)!}\omega_{n+1}(x) 
			 \end{equation*}
		$	\text{其中 }\omega_{n+1}(x)=\prod_{i=0}^n(x-x_i)\text{,}\xi\in(a,b)\text{,且依赖于}x. $
	\end{theorem}
	\section{牛顿插值多项式}
	\subsection{差商的性质}
	\begin{theorem}[n阶均差与导数的关系]
		\begin{equation*}
			f[x_{0},x_{1},\ldots,x_{n}]=\frac{f^{(n)}(\xi)}{n!}
		\end{equation*}
	\end{theorem}
	\chapter{数值积分与数值微分}
	\section{牛顿-科特斯公式}
	\begin{definition}[梯形公式,代数精度为1 ]
		\begin{equation*}
			\mathrm{T=I_1=\frac12(b-a)[f(a)+f(b)]}
		\end{equation*}
		余项为
		\begin{equation*}
			\mathrm{R_T=-\frac{f^{(2)}(\xi)}{12}(b-a)^3=O(b-a)^3}
		\end{equation*}
	\end{definition}
	\begin{definition}[Simpson公式,代数精度为3]
		\begin{equation*}
			\mathrm{S=I_2=\frac16(b-a)[f(a)+4f(\frac{a+b}2)+f(b)]}
		\end{equation*}
		余项为
		\begin{equation*}
			R_S =-\frac{b-a}{180}(\frac{b-a}2)^4f^{(4)}(\eta)
		\end{equation*}
	\end{definition}
	\begin{definition}[科特斯公式,代数精度为5]
		\begin{equation*}
			\mathrm{C=I_4=\frac1{90}(b-a)[7f(x_0)+32f(x_1)+12f(x_2)+32f(x_3)+7f(x_4)]}
		\end{equation*}
		余项为
		\begin{equation*}
			R_C=R(I_4)=\int_a^bR_4(x)dx=-\frac{2(b-a)}{945}(\frac{b-a}4)^6f^{(6)}(\eta)
		\end{equation*}
	\end{definition}
	\section{复合求积公式}
		\begin{definition}[复合梯形公式,代数精度为1 ]
		\begin{equation*}
		T=I_1=\frac{\mathrm{h}}{2}[\mathrm{f(a)}+2\sum_{\mathrm{k}=1}^{\mathrm{n}-1}{\mathrm{f}}(\mathrm{x}_{\mathrm{k}})+\mathrm{f(b)]}
		\end{equation*}
		余项为
		\begin{equation*}
			\mathrm{R_T=-\frac{f^{(2)}(\xi)}{12}(b-a)(h)^2=O(b-a)^3}
		\end{equation*}
	\end{definition}
	\begin{definition}[Simpson公式,代数精度为3]
		\begin{equation*}
			S=[f(a)+4\sum_{k=0}^{n-1}f(x_{k+\frac12})+2\sum_{k=0}^{n-1}f(x_{k+1})+f(b)]
		\end{equation*}
		余项为
		\begin{equation*}
			R_S =-\frac{b-a}{180}(\frac{h}2)^4f^{(4)}(\eta)
		\end{equation*}
	\end{definition}
	\section{高斯求积公式}
	这个刷题为主
	\chapter{解线性方程组的直接方法}
	\section{高斯消去法及高斯列主元消去法}
	高斯消去法的计算量
	\begin{equation*}
		\frac{n^3}3+n^2-\frac13n
	\end{equation*}
	\section{矩阵的三角分解}
	\begin{theorem}
		A的顺序主子式全都不为零,则A可以分解为单位下三角矩阵$L$与上三角矩阵$U$的乘积,且这种分解是唯一的。
	\end{theorem}
	类似的,我们也有
	\begin{theorem}
		设$A$为对称正定矩阵。则一定存在一个主对角 元全是	
		正数的下三角阵 $L$,使得
		\begin{equation*}
			A=LL^T
		\end{equation*}
		且这种分解是唯一的
	\end{theorem}
	\section{条件数}
	\begin{definition}
		设$A$为非奇异阵,称数$cond(A)_\nu=\left\|A^{-1}\right\|_\nu\left\|A\right\|_\nu\left(\nu=1,2\text{或}\right.$
		$\infty$)为矩阵$A$的条件数。
	\end{definition}
	\chapter{解线性方程组的迭代法}
	\section{三种迭代法的迭代公式}
	\begin{definition}[雅克比迭代法]
		\begin{equation*}
			x_i^{(k+1)}=\frac1{a_{ii}}(-\sum_{j=1}^{i-1}a_{ij}x_j^{(k)}-\sum_{j=i+1}^na_{ij}x_j^{(k)}+b_i)
		\end{equation*}
		用矩阵表示即为
		\begin{equation*}
			X^{(k+1)}=BX^{(k+1)}+f,B=D^{-1}(L+U),f=D^{-1}b
		\end{equation*}
	\end{definition}
	\begin{definition}[高斯-赛德尔迭代法]
		\begin{equation*}
			x_i^{(k+1)}=\frac1{a_{ii}}[b_i-\sum_{j=1}^{i-1}a_{ij}x_j^{(k+1)}-\sum_{j=i+1}^na_{ij}x_j^{(k)}]
		\end{equation*}
		用矩阵表示即为
		\begin{equation*}
			X^{(k+1)}=BX^{(k+1)}+f,B=(D-L)^{-1}U,f=(D-L)^{-1}b
		\end{equation*}
	\end{definition}
	\begin{definition}[SOR迭代法]
		用矩阵表示即为
		\begin{equation*}
			X^{(k+1)}=BX^{(k+1)}+f,B=(D-\omega L)^{-1}((1-\omega)L+\omega U),f=\omega(D-\omega L)^{-1}b
		\end{equation*}
	\end{definition}
	\begin{theorem}
		若线性方程组AX=b的系数矩阵A为严格对角占优矩阵,则解该方程组的Jacobi迭代法和G-S迭代法均收敛
	\end{theorem}
	\begin{theorem}
		迭代格式$X^{(k+1)}=BX^{(k)}+f$ 收敛的充要条件为:$\rho(B)<1$
	\end{theorem}
	\begin{theorem}
		设$A$ 可逆,且$a_{ii}\neq\mathbf{0}$,松弛法从任意 $\bar{x}^{(0)}$出发收敛 $\Rightarrow0<\omega<2$ 。
	\end{theorem}
	\chapter{非线性方程与方程组的数值解法}
	\section{二分法二分次数与预定精度之间的关系}
	\begin{equation*}
		\mid x^*-x_k\mid\leq\frac{b_k-a_k}2=\frac{b-a}{2^{k+1}}
	\end{equation*}
	\section{不动点迭代法收敛发散情况}
	\begin{theorem}[不动点迭代法解的存在唯一性定理]
		$\text{ 设 }\varphi(x)\in\mathbb{C}[a,b]\text{ 满足以下两个条件}:\\(1)\text{ 对任意}x\in[a,b]\text{有 }a\leq\varphi(x)\leq b\\(2)\text{ 存在正数}L<1,\text{ 使对任意的 }x\in[a,b],\text{有}\\$
		\begin{equation*}
			\left|\varphi\left(x\right)-\varphi\left(y\right)\right|\leq L\left|x-y\right|
		\end{equation*}
	$\text{则 }\varphi(x)\text{ 在}[a,b]\text{上存在唯一的不动点 }x^*$
	\end{theorem}
	且存在下述误差估计
	\begin{equation*}
		\left|x^{*}-x_{k}\right|\leq\frac{L^{k}}{1-L}\left|x_{1}-x_{0}\right|
	\end{equation*}
	\begin{theorem}
		设$x^*$为$\varphi(x)$的不动点,$\varphi^\prime(x)$在$x^*$的某个邻域连续,且$|\varphi^\prime(x^*)|<1$,则迭代格式局部收敛。
	\end{theorem}
	\chapter{常微分方程初值问题数值解法}
\end{document}