\section{未命名节}

\subsection{机器翻译简介}

\subsection{9.5.1 机器翻译简介}

\subsection{9.5.1 机器翻译简介}
\begin{center}
\Large \textcolor{blue}{将一种语言翻译成另一种语言}
\end{center}


\paragraph{什么是机器翻译?}
\textbf{定义:}自动将文本从源语言翻译到目标语言

\begin{center}
\begin{tikzpicture}[scale=0.9]
% 源语言
\node[draw,fill=blue!20,rounded corners,text width=4.5cm,align=center] at (0,2) {
\textbf{源语言(英语)}\\
``I love machine learning.''
};

\draw[->,ultra thick] (2.5,2) -- (3.5,2);
\node[above] at (3,2.2) {翻译};

% 目标语言
\node[draw,fill=green!20,rounded corners,text width=4.5cm,align=center] at (6.5,2) {
\textbf{目标语言(法语)}\\
``J'aime l'apprentissage automatique.''
};

\end{tikzpicture}
\end{center}

\begin{definition}[挑战]
\begin{itemize}
    \item 语言结构差异(词序、语法)
    \item 一对多映射(一词多义)
    \item 上下文依赖
    \item 习语和文化差异
\end{itemize}
\end{definition}


\paragraph{机器翻译的历史}
\begin{center}
\begin{tikzpicture}[scale=0.85]
% 时间线
\draw[->,ultra thick] (0,2) -- (12,2);

% 里程碑
\node[draw,fill=yellow!20,rounded corners,text width=2cm,align=center] at (1,2) {
\tiny 1950s\\
\small 规则方法
};

\node[draw,fill=orange!20,rounded corners,text width=2cm,align=center] at (4,2) {
\tiny 1990s\\
\small 统计方法
};

\node[draw,fill=blue!20,rounded corners,text width=2cm,align=center] at (7,2) {
\tiny 2014\\
\small 神经网络
};

\node[draw,fill=green!20,rounded corners,text width=2cm,align=center] at (10,2) {
\tiny 2017+\\
\small Transformer
};

% 说明
\node[below,font=\tiny,text width=2.5cm,align=center] at (1,1) {
人工规则\\
效果差
};

\node[below,font=\tiny,text width=2.5cm,align=center] at (4,1) {
短语表\\
需要对齐
};

\node[below,font=\tiny,text width=2.5cm,align=center] at (7,1) {
Seq2Seq\\
端到端
};

\node[below,font=\tiny,text width=2.5cm,align=center] at (10,1) {
注意力\\
SOTA
};

\end{tikzpicture}
\end{center}

\begin{theorem}[革命性进展]
2014年:神经机器翻译(NMT)出现,性能超越传统方法
\end{theorem}


\paragraph{序列到序列(Seq2Seq)框架}
\textbf{核心思想:}输入序列 $\rightarrow$ 输出序列

\begin{center}
\begin{tikzpicture}[scale=0.85]
% 输入序列
\node[above] at (2,3) {\small 输入序列};
\foreach \t/\w in {1/I,2/love,3/ML} {
    \node[draw,fill=blue!20,minimum size=0.7cm] at (\t*1.2,2.5) {\tiny\w};
}

% 编码器
\node[draw,fill=yellow!20,rounded corners,minimum width=4cm,minimum height=1cm] at (2,1) {
编码器(Encoder)
};

\draw[->,thick] (2,2) -- (2,1.5);

% 中间表示
\node[draw,fill=purple!20,rounded corners,minimum width=2cm] at (2,-0.2) {
上下文向量
};

\draw[->,thick] (2,0.5) -- (2,0.1);

% 解码器
\node[draw,fill=orange!20,rounded corners,minimum width=4cm,minimum height=1cm] at (7,1) {
解码器(Decoder)
};

\draw[->,thick] (3.2,-0.2) -- (5.8,0.5);

% 输出序列
\node[above] at (7,3) {\small 输出序列};
\foreach \t/\w in {1/J',2/aime,3/ML} {
    \node[draw,fill=green!20,minimum size=0.7cm] at (5.5+\t*1.2,2.5) {\tiny\w};
}

\draw[->,thick] (7,1.5) -- (7,2);

\node[below,text width=8cm,align=center] at (4.5,-1.2) {
\small 编码器将源语言压缩成向量,解码器从向量生成目标语言
};

\end{tikzpicture}
\end{center}


\paragraph{机器翻译的特点}


\textbf{与语言模型的区别:}
\begin{itemize}
    \item 语言模型:\\
    $P(x_1, \ldots, x_T)$
    
    \item 机器翻译:\\
    $P(y_1, \ldots, y_{T'} \mid x_1, \ldots, x_T)$
    
    \item 条件生成任务
\end{itemize}


\textbf{挑战:}
\begin{itemize}
    \item 源序列和目标序列\textbf{长度不同}
    \item 需要\textbf{对齐}
    \item 一对多映射
    \item 长距离依赖
\end{itemize}


\vspace{0.5cm}
\begin{example}[例子]
\textbf{英语:}``I love you'' (3个词)\\
\textbf{法语:}``Je t'aime'' (2个词)\\
\textbf{德语:}``Ich liebe dich'' (3个词)

长度不同!
\end{example}



\subsection{数据集下载与预处理}

\subsection{9.5.2 数据集下载与预处理}

\subsection{9.5.2 数据集下载与预处理}
\begin{center}
\Large \textcolor{blue}{准备双语平行语料}
\end{center}


\paragraph{平行语料(Parallel Corpus)}
\textbf{定义:}一一对应的源语言和目标语言句子对

\begin{center}
\begin{tikzpicture}[scale=0.9]
% 表格
\node[draw,fill=blue!20,text width=5cm] at (0,3) {
\textbf{英语(源)}\\
``Go.''\\
``Hi.''\\
``I love you.''\\
``How are you?''
};

\node[draw,fill=green!20,text width=5cm] at (6,3) {
\textbf{法语(目标)}\\
``Va!''\\
``Salut!''\\
``Je t'aime.''\\
``Comment allez-vous?''
};

\draw[<->,thick,red] (2.7,3.5) -- (3.3,3.5);
\draw[<->,thick,red] (2.7,3) -- (3.3,3);
\draw[<->,thick,red] (2.7,2.5) -- (3.3,2.5);
\draw[<->,thick,red] (2.7,2) -- (3.3,2);

\node[red,above] at (3,4) {一一对应};

\end{tikzpicture}
\end{center}

\begin{definition}[来源]
\begin{itemize}
    \item 官方文档翻译(联合国、欧盟)
    \item 电影字幕
    \item 书籍翻译
    \item 网页翻译
\end{itemize}
\end{definition}

\paragraph{本节使用的数据集}
\textbf{数据集:}英语-法语翻译对

\begin{definition}[数据集信息]
\begin{itemize}
    \item \textbf{来源:}Tatoeba项目
    \item \textbf{语言对:}英语(en) $\leftrightarrow$ 法语(fr)
    \item \textbf{规模:}约20万句对
    \item \textbf{特点:}短句为主,适合教学
    \item \textbf{格式:}制表符分隔的文本文件
\end{itemize}
\end{definition}

\begin{example}[数据样例]
\begin{verbatim}
Go.    Va!
Hi.    Salut!
Run!   Cours!
I see. Je vois.
\end{verbatim}
\end{example}


\paragraph{下载和读取数据}
\begin{lstlisting}
import os
import torch

def download_extract(name, cache_dir='./data'):
    """data"""
    # URL
    return os.path.join(cache_dir, name)

def read_data_nmt():
    """读取英语-法语data集"""
    data_dir = download_extract('fra-eng')
    with open(os.path.join(data_dir, 'fra.txt'), 'r', 
              encoding='utf-8') as f:
        return f.read()

raw_text = read_data_nmt()
print(raw_text[:75])
# Go.    Va!
# Hi.    Salut!
# Run!   Cours!
\end{lstlisting}


\paragraph{数据预处理}
\begin{lstlisting}
def preprocess_nmt(text):
    """data"""
    
    def no_space(char, prev_char):
        """判断是否需要在字符前加空格"""
        return char in set(',.!?') and prev_char != ' '
    
    text = text.replace('\u202f', ' ').replace('\xa0', ' ').lower()
    
    out = [' ' + char if i > 0 and no_space(char, text[i - 1]) 
           else char for i, char in enumerate(text)]
    
    return ''.join(out)

text = preprocess_nmt(raw_text)
print(text[:80])
# go .    va !
# hi .    salut !
# run !   cours !
\end{lstlisting}

\begin{theorem}[预处理目的]
\begin{itemize}
    \item 标准化格式(小写)
    \item 分离标点符号(便于词元化)
\end{itemize}
\end{theorem}


\paragraph{词元化(Tokenization)}
\begin{lstlisting}
def tokenize_nmt(text, num_examples=None):
    """ tokensdata Return:tokens """
    source, target = [], []
    
    for i, line in enumerate(text.split('\n')):
        if num_examples and i >= num_examples:
            break
        
        parts = line.split('\t')
        if len(parts) == 2:
            source.append(parts[0].split(' '))
            target.append(parts[1].split(' '))
    
    return source, target

# tokens
source, target = tokenize_nmt(text)
print(source[:3])
# [['go', '.'], ['hi', '.'], ['run', '!']]
print(target[:3])
# [['va', '!'], ['salut', '!'], ['cours', '!']]
\end{lstlisting}


\paragraph{数据统计}
\begin{center}
\begin{tikzpicture}[scale=0.9]
% 柱状图
\draw[->,thick] (0,0) -- (0,4) node[above] {频数};
\draw[->,thick] (0,0) -- (8,0) node[right] {句子长度};

% 英语
\fill[blue!50] (0.5,0) rectangle (1.5,2.5);
\node[below,font=\tiny] at (1,0) {1-5};
\fill[blue!50] (2,0) rectangle (3,3.2);
\node[below,font=\tiny] at (2.5,0) {6-10};
\fill[blue!50] (3.5,0) rectangle (4.5,1.8);
\node[below,font=\tiny] at (4,0) {11-15};
\fill[blue!50] (5,0) rectangle (6,0.8);
\node[below,font=\tiny] at (5.5,0) {$>15$};

\node[blue,above] at (3,3.5) {英语};

% 法语(用虚线)
\draw[green!60!black,thick,dashed] (1,0) -- (1,2.3);
\draw[green!60!black,thick,dashed] (2.5,0) -- (2.5,3.5);
\draw[green!60!black,thick,dashed] (4,0) -- (4,2.0);
\draw[green!60!black,thick,dashed] (5.5,0) -- (5.5,1.0);

\node[green!60!black,above] at (5,3.8) {法语};

\end{tikzpicture}
\end{center}

\begin{definition}[观察]
\begin{itemize}
    \item 大多数句子在10个词以内
    \item 英语和法语长度分布相似
    \item 适合限制最大长度
\end{itemize}
\end{definition}


\paragraph{过滤和截断}
\begin{lstlisting}
def truncate_pad(line, num_steps, padding_token):
    """sequencelength"""
    if len(line) > num_steps:
        return line[:num_steps]  # comment
    return line + [padding_token] * (num_steps - len(line))  # comment

def filter_examples(source, target, max_len=10):
    """过滤过长的句子"""
    source_filtered, target_filtered = [], []
    
    for src, tgt in zip(source, target):
        # lengthmax_len
        if len(src) <= max_len and len(tgt) <= max_len:
            source_filtered.append(src)
            target_filtered.append(tgt)
    
    return source_filtered, target_filtered

source, target = filter_examples(source, target, max_len=10)
print(f"过滤后句子对数量: {len(source)}")
# : 152,777
\end{lstlisting}



\subsection{词表构建}

\subsection{9.5.3 词表构建}

\subsection{9.5.3 词表构建}
\begin{center}
\Large \textcolor{blue}{为源语言和目标语言分别构建词表}
\end{center}


\paragraph{双语词表}
\textbf{关键:}源语言和目标语言需要\textbf{独立的词表}

\begin{center}
\begin{tikzpicture}[scale=0.9]
% 英语词表
\node[draw,fill=blue!20,rounded corners,text width=5cm] at (0,2) {
\textbf{英语词表}\\
\texttt{{'<pad>': 0,}\\
\texttt{ '<bos>': 1,}\\
\texttt{ '<eos>': 2,}\\
\texttt{ 'i': 3,}\\
\texttt{ 'love': 4,}\\
\texttt{ ...}}
};

% 法语词表
\node[draw,fill=green!20,rounded corners,text width=5cm] at (7,2) {
\textbf{法语词表}\\
\texttt{{'<pad>': 0,}\\
\texttt{ '<bos>': 1,}\\
\texttt{ '<eos>': 2,}\\
\texttt{ 'je': 3,}\\
\texttt{ 'aime': 4,}\\
\texttt{ ...}}
};

\node[below,text width=11cm,align=center] at (3.5,0.3) {
\small 两个词表独立,同一个索引对应不同的词
};

\end{tikzpicture}
\end{center}

\begin{theorem}[特殊词元]
\begin{itemize}
    \item \texttt{<pad>}:填充
    \item \texttt{<bos>}:句子开始(Begin of Sequence)
    \item \texttt{<eos>}:句子结束(End of Sequence)
\end{itemize}
\end{theorem}


\paragraph{为什么需要特殊词元?}
\begin{definition}[<pad>(填充)]
批量处理时,需要将不同长度的句子填充到相同长度
\begin{verbatim}
"Go ."       -> [Go, ., <eos>, <pad>, <pad>]
"I love you" -> [I, love, you, ., <eos>]
\end{verbatim}
\end{definition}

\begin{definition}[<bos>(开始)]
解码器的初始输入,告诉模型"开始生成"
\begin{verbatim}
输入:[<bos>] -> 输出:"Je"
输入:[<bos>, Je] -> 输出:"t'"
\end{verbatim}
\end{definition}

\begin{definition}[<eos>(结束)]
标记句子结束,模型知道何时停止生成
\begin{verbatim}
生成序列:[Je, t', aime, ., <eos>] -> 停止
\end{verbatim}
\end{definition}


\paragraph{构建词表}
\begin{lstlisting}
class Vocab:
    """vocabulary"""
    def __init__(self, tokens=None, min_freq=0, reserved_tokens=None):
        if tokens is None:
            tokens = []
        if reserved_tokens is None:
            reserved_tokens = []
        
        counter = count_corpus(tokens)
        self._token_freqs = sorted(counter.items(), 
                                   key=lambda x: x[1], 
                                   reverse=True)
        
        # tokens + 
        self.idx_to_token = ['<unk>'] + reserved_tokens
        self.token_to_idx = {token: idx 
                            for idx, token in enumerate(self.idx_to_token)}
        
        for token, freq in self._token_freqs:
            if freq < min_freq:
                break
            if token not in self.token_to_idx:
                self.idx_to_token.append(token)
                self.token_to_idx[token] = len(self.idx_to_token) - 1
\end{lstlisting}


\paragraph{构建双语词表}
\begin{lstlisting}
def build_array_nmt(lines, vocab, num_steps):
    """sequence"""
    # <eos>
    lines = [vocab[l] + [vocab['<eos>']] for l in lines]
    
    # length
    array = torch.tensor([
        truncate_pad(l, num_steps, vocab['<pad>']) 
        for l in lines
    ])
    
    # lengthpadding
    valid_len = (array != vocab['<pad>']).type(torch.int32).sum(1)
    
    return array, valid_len

# vocabulary
src_vocab = Vocab(source, min_freq=2, 
                  reserved_tokens=['<pad>', '<bos>', '<eos>'])
tgt_vocab = Vocab(target, min_freq=2, 
                  reserved_tokens=['<pad>', '<bos>', '<eos>'])

print(f"英语vocabulary大小: {len(src_vocab)}")  # comment10,000
print(f"法语vocabulary大小: {len(tgt_vocab)}")  # comment17,000
\end{lstlisting}


\paragraph{词表大小对比}
\begin{center}
\begin{tikzpicture}[scale=0.9]
% 柱状图
\draw[->,thick] (0,0) -- (0,4) node[above] {词表大小};
\draw[->,thick] (0,0) -- (8,0) node[right] {语言};

% 英语
\fill[blue!50] (1,0) rectangle (2.5,2.8);
\node[below] at (1.75,0) {英语};
\node[above] at (1.75,2.8) {10,012};

% 法语
\fill[green!50] (4,0) rectangle (5.5,3.5);
\node[below] at (4.75,0) {法语};
\node[above] at (4.75,3.5) {17,851};

% 标注
\node[below,text width=8cm,align=center] at (4,-0.8) {
\small 法语词汇更丰富(动词变位、名词性别等)
};

\end{tikzpicture}
\end{center}

\begin{theorem}[观察]
法语词表比英语大约70\%,因为法语形态变化更丰富
\end{theorem}



\subsection{数据加载器}

\subsection{9.5.4 数据加载器}

\subsection{9.5.4 数据加载器}
\begin{center}
\Large \textcolor{blue}{批量加载句子对}
\end{center}


\paragraph{数据加载的挑战}
\begin{definition}[序列对的特殊性]
\begin{enumerate}
    \item \textbf{两个序列}
    \begin{itemize}
        \item 源序列(英语)
        \item 目标序列(法语)
    \end{itemize}
    
    \item \textbf{长度不同}
    \begin{itemize}
        \item 源序列和目标序列长度可能不同
        \item 需要分别填充
    \end{itemize}
    
    \item \textbf{解码器需要两个目标}
    \begin{itemize}
        \item 输入:$[\text{<bos>}, y_1, \ldots, y_{T-1}]$
        \item 标签:$[y_1, \ldots, y_{T-1}, \text{<eos>}]$
    \end{itemize}
\end{enumerate}
\end{definition}


\paragraph{解码器的输入输出}
\begin{center}
\begin{tikzpicture}[scale=0.85]
% 目标句子
\node[above] at (3,3.5) {\small 目标句子(法语)};
\node at (3,3) {``Je t'aime .''};

\draw[->,thick] (3,2.7) -- (3,2.3);

% 解码器输入
\node[above] at (3,2) {\small 解码器输入};
\foreach \i/\w in {0/<bos>,1/Je,2/t',3/aime,4/.} {
    \node[draw,fill=blue!20,minimum size=0.6cm,font=\tiny] at (\i*1.2,1.5) {\w};
}

% 解码器输出(标签)
\node[above] at (3,0.5) {\small 解码器标签};
\foreach \i/\w in {0/Je,1/t',2/aime,3/.,4/<eos>} {
    \node[draw,fill=green!20,minimum size=0.6cm,font=\tiny] at (\i*1.2,0) {\w};
}

% 标注
\draw[<->,thick,red] (0,-0.8) -- (4.8,-0.8);
\node[red,below] at (2.4,-0.8) {\tiny 错位一个位置};

\end{tikzpicture}
\end{center}

\begin{theorem}[Teacher Forcing]
训练时,解码器的每一步都用\textbf{真实的}前一个词作为输入
\end{theorem}


\paragraph{数据加载函数}
\begin{lstlisting}
def load_data_nmt(batch_size, num_steps, num_examples=600):
    """ data Return:dataiterationvocabulary """
    text = preprocess_nmt(read_data_nmt())
    source, target = tokenize_nmt(text, num_examples)
    
    # vocabulary
    src_vocab = Vocab(source, min_freq=2, 
                      reserved_tokens=['<pad>', '<bos>', '<eos>'])
    tgt_vocab = Vocab(target, min_freq=2, 
                      reserved_tokens=['<pad>', '<bos>', '<eos>'])
    
    src_array, src_valid_len = build_array_nmt(source, src_vocab, num_steps)
    tgt_array, tgt_valid_len = build_array_nmt(target, tgt_vocab, num_steps)
    
    # data
    data_arrays = (src_array, src_valid_len, tgt_array, tgt_valid_len)
    data_iter = load_array(data_arrays, batch_size)
    
    return data_iter, src_vocab, tgt_vocab
\end{lstlisting}


\paragraph{使用数据加载器}
\begin{lstlisting}
# data
train_iter, src_vocab, tgt_vocab = load_data_nmt(
    batch_size=2, 
    num_steps=8,
    num_examples=600
)

# batch
for X, X_valid_len, Y, Y_valid_len in train_iter:
    print('X:', X.type(torch.int32))
    print('X的有效length:', X_valid_len)
    print('Y:', Y.type(torch.int32))
    print('Y的有效length:', Y_valid_len)
    break

# Output
# X: tensor([[  47,   4,    3,    3,    3, 1388,    5,    2],
#            [  33,   81,    4,    3,    3,    5,    2,    0]])
# Xlength: tensor([8, 7])
# Y: tensor([[ 91,   4,   3,   3,  14,   5,   2,   0],
#            [ 24, 115,   4,   3,   3,   5,   2,   0]])
# Ylength: tensor([7, 7])
\end{lstlisting}


\paragraph{批次数据的结构}
\textbf{一个批次包含:}

\begin{center}
\begin{tikzpicture}[scale=0.85]
% 源序列
\node[draw,fill=blue!20,text width=10cm] at (5,3.5) {
\textbf{源序列 X:} $(batch\_size, num\_steps)$\\
编码器的输入,已填充到num\_steps
};

% 源序列有效长度
\node[draw,fill=cyan!20,text width=10cm] at (5,2.4) {
\textbf{源序列有效长度 X\_valid\_len:} $(batch\_size,)$\\
每个序列的实际长度(不含padding)
};

% 目标序列
\node[draw,fill=green!20,text width=10cm] at (5,1.3) {
\textbf{目标序列 Y:} $(batch\_size, num\_steps)$\\
解码器的输入和标签,已填充
};

% 目标序列有效长度
\node[draw,fill=yellow!20,text width=10cm] at (5,0.2) {
\textbf{目标序列有效长度 Y\_valid\_len:} $(batch\_size,)$\\
每个序列的实际长度
};

\end{tikzpicture}
\end{center}


\subsection{9.5节总结}
\begin{definition}[核心内容]
\begin{enumerate}
    \item \textbf{机器翻译}:序列到序列任务
    \begin{itemize}
        \item 编码器-解码器架构
        \item 条件生成:$P(y_{1:T'} \mid x_{1:T})$
    \end{itemize}
    
    \item \textbf{数据预处理}:
    \begin{itemize}
        \item 标准化、词元化
        \item 过滤过长句子
        \item 填充到固定长度
    \end{itemize}
    
    \item \textbf{词表构建}:
    \begin{itemize}
        \item 源语言和目标语言分别构建
        \item 特殊词元:\texttt{<pad>}, \texttt{<bos>}, \texttt{<eos>}
    \end{itemize}
    
    \item \textbf{数据加载}:
    \begin{itemize}
        \item 批量加载句子对
        \item 记录有效长度
    \end{itemize}
\end{enumerate}
\end{definition}


\paragraph{关键要点回顾}
\begin{center}
\begin{tikzpicture}[scale=0.85]
% 要点1
\node[draw,fill=blue!20,rounded corners,text width=10cm] at (5,3.8) {
\textbf{1. 平行语料}\\
源语言和目标语言一一对应的句子对
};

% 要点2
\node[draw,fill=green!20,rounded corners,text width=10cm] at (5,2.6) {
\textbf{2. 特殊词元}\\
\texttt{<bos>}开始 | \texttt{<eos>}结束 | \texttt{<pad>}填充
};

% 要点3
\node[draw,fill=orange!20,rounded corners,text width=10cm] at (5,1.4) {
\textbf{3. 双词表}\\
源语言和目标语言独立的词表
};

% 要点4
\node[draw,fill=purple!20,rounded corners,text width=10cm] at (5,0.2) {
\textbf{4. Teacher Forcing}\\
解码器训练时用真实词作为输入
};

\end{tikzpicture}
\end{center}



