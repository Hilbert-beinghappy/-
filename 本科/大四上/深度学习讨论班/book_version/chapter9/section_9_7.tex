\section{未命名节}

\subsection{训练Seq2Seq模型}

\subsection{9.7.1 训练Seq2Seq模型}

\subsection{9.7.1 训练Seq2Seq模型}
\begin{center}
\Large \textcolor{blue}{英语到法语的翻译}
\end{center}


\paragraph{完整的训练流程}
\begin{lstlisting}
# data
batch_size = 64
num_steps = 10
lr = 0.005
num_epochs = 300

train_iter, src_vocab, tgt_vocab = load_data_nmt(
    batch_size, num_steps)

# modelparameters
embed_size = 32
num_hiddens = 32
num_layers = 2
dropout = 0.1

# model
encoder = Seq2SeqEncoder(len(src_vocab), embed_size, 
                        num_hiddens, num_layers, dropout)
decoder = Seq2SeqDecoder(len(tgt_vocab), embed_size, 
                        num_hiddens, num_layers, dropout)
net = EncoderDecoder(encoder, decoder)

# training
device = torch.device('cuda' if torch.cuda.is_available() else 'cpu')
train_seq2seq(net, train_iter, lr, num_epochs, tgt_vocab, device)
\end{lstlisting}


\paragraph{训练循环详解}
\begin{lstlisting}
def train_seq2seq(net, data_iter, lr, num_epochs, tgt_vocab, device,
                  save_path='seq2seq.pth'):
    """trainingseq2seqmodel"""
    net.to(device)
    optimizer = torch.optim.Adam(net.parameters(), lr=lr)
    animator = Animator(xlabel='epoch', ylabel='loss',
                       xlim=[10, num_epochs])
    
    for epoch in range(num_epochs):
        timer = Timer()
        metric = Accumulator(2)  # loss和, tokens数
        
        for batch in data_iter:
            optimizer.zero_grad()
            X, X_valid_len, Y, Y_valid_len = [x.to(device) for x in batch]
            
            # Y<bos>
            bos = torch.tensor([tgt_vocab['<bos>']] * Y.shape[0], 
                              device=device).reshape(-1, 1)
            dec_input = torch.cat([bos, Y[:, :-1]], 1)
            
            Y_hat, _ = net(X, dec_input)
            
            # loss
            l = masked_softmax_cross_entropy(Y_hat, Y, Y_valid_len)
            l.sum().backward()
            
            grad_clipping(net, 1)
            
            optimizer.step()
            
            with torch.no_grad():
                metric.add(l.sum(), Y_valid_len.sum())
        
        if (epoch + 1) % 10 == 0:
            animator.add(epoch + 1, (metric[0] / metric[1],))
    
    print(f'loss {metric[0] / metric[1]:.3f}, '
          f'{metric[1] / timer.stop():.1f} tokens/sec on {str(device)}')
    
    # model
    torch.save(net.state_dict(), save_path)
\end{lstlisting}


\paragraph{训练过程可视化}
\begin{center}
\begin{tikzpicture}[scale=0.9]
% 坐标轴
\draw[->,thick] (0,0) -- (8,0) node[right] {Epoch};
\draw[->,thick] (0,0) -- (0,4) node[above] {Loss};

% 损失曲线
\draw[blue,thick] (0.5,3.5) -- (1,3) -- (2,2.3) -- (3,1.8) -- (4,1.4) -- (5,1.2) -- (7,1.0);

% 标注关键点
\fill[red] (0.5,3.5) circle (2pt) node[above,font=\tiny] {初始:高};
\fill[green!60!black] (3,1.8) circle (2pt) node[above,font=\tiny] {快速下降};
\fill[blue] (7,1.0) circle (2pt) node[above,font=\tiny] {收敛};

% 阶段标注
\draw[dashed] (0,1.5) -- (8,1.5);
\node[left,font=\tiny] at (0,1.5) {目标};

\node[below,text width=8cm,align=center] at (4,-0.8) {
\small 典型训练曲线:初期快速下降,后期缓慢收敛
};

\end{tikzpicture}
\end{center}

\begin{definition}[训练输出示例]
\begin{verbatim}
loss 0.023, 14523.3 tokens/sec on cuda:0
\end{verbatim}
\end{definition}



\subsection{预测与评估}

\subsection{9.7.2 预测与评估}

\subsection{9.7.2 预测与评估}
\begin{center}
\Large \textcolor{blue}{用训练好的模型翻译}
\end{center}


\paragraph{预测函数}
\begin{lstlisting}
def predict_seq2seq(net, src_sentence, src_vocab, tgt_vocab, 
                   num_steps, device, save_attention_weights=False):
    """prediction"""
    net.eval()
    src_tokens = src_vocab[src_sentence.lower().split(' ')] + [
        src_vocab['<eos>']]
    enc_valid_len = torch.tensor([len(src_tokens)], device=device)
    
    # length
    src_tokens = truncate_pad(src_tokens, num_steps, src_vocab['<pad>'])
    enc_X = torch.unsqueeze(
        torch.tensor(src_tokens, dtype=torch.long, device=device), 
        dim=0)
    
    enc_outputs = net.encoder(enc_X, enc_valid_len)
    dec_state = net.decoder.init_state(enc_outputs, enc_valid_len)
    
    dec_X = torch.unsqueeze(
        torch.tensor([tgt_vocab['<bos>']], dtype=torch.long, 
                    device=device), dim=0)
    
    output_seq, attention_weight_seq = [], []
    
    for _ in range(num_steps):
        Y, dec_state = net.decoder(dec_X, dec_state)
        dec_X = Y.argmax(dim=2)
        pred = dec_X.squeeze(dim=0).type(torch.int32).item()
        
        if pred == tgt_vocab['<eos>']:
            break
        
        output_seq.append(pred)
    
    return ' '.join(tgt_vocab.to_tokens(output_seq))
\end{lstlisting}


\paragraph{翻译示例}
\begin{lstlisting}
# trainingmodel
net.load_state_dict(torch.load('seq2seq.pth'))

# test
engs = ['go .', 'i lost .', 'he\'s calm .', 'i\'m home .']
fras = ['va !', 'j\'ai perdu .', 'il est calme .', 'je suis chez moi .']

for eng, fra in zip(engs, fras):
    translation = predict_seq2seq(
        net, eng, src_vocab, tgt_vocab, num_steps, device)
    print(f'{eng} => {translation}, bleu {bleu(translation, fra, k=2):.3f}')
\end{lstlisting}

\textbf{输出示例:}
\begin{verbatim}
go . => va !, bleu 1.000
i lost . => j'ai perdu ., bleu 1.000
he's calm . => il est calme ., bleu 1.000
i'm home . => je suis chez moi ., bleu 1.000
\end{verbatim}

\begin{theorem}[观察]
简单句子翻译完美,BLEU=1.0
\end{theorem}


\paragraph{翻译质量分析}
\begin{center}
\begin{tikzpicture}[scale=0.85]
% 柱状图
\draw[->,thick] (0,0) -- (0,4) node[above] {BLEU分数};
\draw[->,thick] (0,0) -- (10,0) node[right] {句子类型};

% 简单句
\fill[green!50] (0.5,0) rectangle (1.5,3.8);
\node[below,font=\tiny,text width=1.5cm,align=center] at (1,0) {简单句\\(1-5词)};
\node[above] at (1,3.8) {\small 0.95};

% 中等句
\fill[blue!50] (2.5,0) rectangle (3.5,3.2);
\node[below,font=\tiny,text width=1.5cm,align=center] at (3,0) {中等句\\(6-10词)};
\node[above] at (3,3.2) {\small 0.78};

% 复杂句
\fill[orange!50] (4.5,0) rectangle (5.5,2.5);
\node[below,font=\tiny,text width=1.5cm,align=center] at (5,0) {复杂句\\(11-15词)};
\node[above] at (5,2.5) {\small 0.62};

% 长句
\fill[red!50] (6.5,0) rectangle (7.5,1.8);
\node[below,font=\tiny,text width=1.5cm,align=center] at (7,0) {长句\\($>15$词)};
\node[above] at (7,1.8) {\small 0.45};

% 参考线
\draw[dashed] (0,3) -- (8,3);
\node[left,font=\tiny] at (0,3) {优秀};

\end{tikzpicture}
\end{center}

\begin{definition}[趋势]
句子越长,BLEU越低(信息瓶颈问题)
\end{definition}


\paragraph{常见翻译错误}
\begin{example}[错误类型1:漏词]
\textbf{源:}``I love you very much .''\\
\textbf{参考:}``Je t'aime beaucoup .''\\
\textbf{预测:}``Je t'aime .'' \quad (漏了``beaucoup'')
\end{example}

\begin{example}[错误类型2:词序错误]
\textbf{源:}``The red car .''\\
\textbf{参考:}``La voiture rouge .''\\
\textbf{预测:}``La rouge voiture .'' \quad (形容词位置错)
\end{example}

\begin{example}[错误类型3:重复]
\textbf{源:}``I am happy .''\\
\textbf{参考:}``Je suis heureux .''\\
\textbf{预测:}``Je je suis heureux .'' \quad (重复``je'')
\end{example}



\subsection{性能分析}

\subsection{9.7.3 性能分析}

\subsection{9.7.3 性能分析}
\begin{center}
\Large \textcolor{blue}{Seq2Seq的优缺点}
\end{center}


\paragraph{Seq2Seq的优势}


\textcolor{blue}{\textbf{优点:}}
\begin{itemize}
    \item $\checkmark$ \textbf{端到端}\\
    无需人工对齐
    
    \item $\checkmark$ \textbf{通用}\\
    适用多种任务
    
    \item $\checkmark$ \textbf{可训练}\\
    自动学习特征
    
    \item $\checkmark$ \textbf{灵活}\\
    处理变长序列
\end{itemize}


\textbf{应用场景:}
\begin{itemize}
    \item 机器翻译
    \item 文本摘要
    \item 对话系统
    \item 代码生成
    \item 图像描述
    \item 语音识别
\end{itemize}


\vspace{0.5cm}
\begin{definition}[革命性]
Seq2Seq开启了神经序列到序列学习的新时代
\end{definition}


\paragraph{Seq2Seq的局限}
\begin{theorem}[核心问题:信息瓶颈]
\textbf{固定长度的上下文向量}

\begin{center}
\begin{tikzpicture}[scale=0.85]
% 长源序列
\node[above] at (3,3) {\small 源序列(20词)};
\foreach \i in {1,...,10} {
    \node[draw,fill=blue!20,minimum size=0.4cm] at (\i*0.6,2.5) {};
}
\node at (6.5,2.5) {$\ldots$};

% 压缩
\draw[->,ultra thick,red] (3,2) -- (3,1.5);
\node[right,red] at (3.3,1.75) {\tiny 压缩};

% 上下文向量
\node[draw,fill=purple!20,minimum width=1.5cm] at (3,1) {$\mathbf{c}$};
\node[right,font=\tiny] at (3.8,1) {512维};

% 问题
\node[below,text width=6cm,align=center,red] at (3,0.3) {
\small 所有信息都要塞进这个向量!
};

\end{tikzpicture}
\end{center}
\end{theorem}

\begin{definition}[表现]
\begin{itemize}
    \item 长句翻译质量差
    \item 重要信息丢失
    \item 早期词的信息被"遗忘"
\end{itemize}
\end{definition}


\paragraph{性能随序列长度的变化}
\begin{center}
\begin{tikzpicture}[scale=0.9]
% 坐标轴
\draw[->,thick] (0,0) -- (8,0) node[right] {源序列长度};
\draw[->,thick] (0,0) -- (0,4) node[above] {BLEU};

% Seq2Seq曲线
\draw[red,thick] (0.5,3.5) -- (1.5,3.2) -- (2.5,2.8) -- (3.5,2.3) -- (4.5,1.7) -- (5.5,1.2) -- (7,0.8);
\node[red,right] at (7,0.8) {Seq2Seq};

% 理想曲线
\draw[blue,dashed] (0.5,3.5) -- (1.5,3.4) -- (2.5,3.3) -- (3.5,3.2) -- (4.5,3.1) -- (5.5,3.0) -- (7,2.9);
\node[blue,right] at (7,2.9) {理想};

% 标注
\node[below,font=\tiny] at (1,0) {5};
\node[below,font=\tiny] at (3,0) {10};
\node[below,font=\tiny] at (5,0) {20};
\node[below,font=\tiny] at (7,0) {30};

% 差距
\draw[<->,thick,purple] (5.5,1.2) -- (5.5,3.0);
\node[purple,right] at (5.7,2.1) {\tiny 性能gap};

\node[below,text width=8cm,align=center] at (4,-0.8) {
\small Seq2Seq在长序列上性能下降严重
};

\end{tikzpicture}
\end{center}



\subsection{改进方向}

\subsection{9.7.4 改进方向}

\subsection{9.7.4 改进方向}
\begin{center}
\Large \textcolor{blue}{如何改进Seq2Seq?}
\end{center}


\paragraph{改进方向1:注意力机制}
\textbf{核心思想:}让解码器\textbf{动态}关注源序列的不同部分

\begin{center}
\begin{tikzpicture}[scale=0.75]
% 编码器输出
\node[above] at (3,4) {\small 编码器所有输出};
\foreach \i in {1,...,6} {
    \node[circle,draw,fill=blue!20,minimum size=0.6cm] (h\i) at (\i*1,3.5) {\tiny$h_{\i}$};
}

% 注意力权重
\node[left] at (0,2.5) {\small 注意力};
\foreach \i in {1,...,6} {
    \node[draw,fill=red!20,minimum size=0.4cm] (a\i) at (\i*1,2.5) {\tiny$\alpha_{\i}$};
    \draw[->,thick] (h\i) -- (a\i);
}

% 加权和
\node[draw,fill=purple!20,rounded corners] (c) at (3.5,1.2) {加权和 $\mathbf{c}_t$};
\foreach \i in {1,...,6} {
    \draw[->,thick] (a\i) -- (c);
}

% 解码器
\node[circle,draw,fill=green!20,minimum size=0.8cm] at (3.5,0) {解码器};
\draw[->,thick] (c) -- (3.5,0.4);

\node[below,text width=7cm,align=center] at (3.5,-0.8) {
\small 每个解码步骤关注不同的源词
};

\end{tikzpicture}
\end{center}

\begin{theorem}[优势]
解决信息瓶颈,长序列性能大幅提升
\end{theorem}


\paragraph{改进方向2:束搜索}
\textbf{问题:}贪心解码每步只选最大概率,可能次优

\begin{center}
\begin{tikzpicture}[scale=0.75]
% 时间步1
\node[circle,draw,fill=blue!20] (s1) at (0,2) {$s_1$};

% 候选词
\node[draw,fill=green!20,minimum size=0.5cm] (w11) at (2,3) {\tiny Je (0.7)};
\node[draw,fill=yellow!20,minimum size=0.5cm] (w12) at (2,2) {\tiny J' (0.2)};
\node[draw,fill=gray!20,minimum size=0.5cm] (w13) at (2,1) {\tiny ... (0.1)};

\draw[->,thick] (s1) -- (w11);
\draw[->,thick] (s1) -- (w12);
\draw[->,thick] (s1) -- (w13);

% 贪心
\draw[red,ultra thick] (s1) -- (w11);
\node[red,above] at (1,3.3) {\tiny 贪心:只选最大};

% 束搜索
\draw[blue,thick] (s1) -- (w11);
\draw[blue,thick] (s1) -- (w12);
\node[blue,below] at (1,0.7) {\tiny 束搜索:保留top-k};

\end{tikzpicture}
\end{center}

\begin{definition}[束搜索(Beam Search)]
\begin{itemize}
    \item 每步保留k个最佳候选(beam size)
    \item 权衡搜索质量和计算成本
    \item 通常k=5-10
\end{itemize}
\end{definition}


\paragraph{改进方向3:双向编码器}
\begin{center}
\begin{tikzpicture}[scale=0.8]
% 单向
\node[above] at (2,4) {\small 单向编码器};
\foreach \i in {1,2,3} {
    \node[circle,draw,fill=blue!20,minimum size=0.7cm] (h\i) at (\i*1.2,3.5) {\tiny$\overrightarrow{h}_{\i}$};
}
\draw[->,thick] (h1) -- (h2) -- (h3);
\node[below,font=\tiny] at (2,2.8) {只看过去};

% 双向
\node[above] at (7,4) {\small 双向编码器};
\foreach \i in {1,2,3} {
    \node[circle,draw,fill=blue!20,minimum size=0.6cm] (f\i) at (5.5+\i*1.2,3.8) {\tiny$\overrightarrow{h}_{\i}$};
    \node[circle,draw,fill=green!20,minimum size=0.6cm] (b\i) at (5.5+\i*1.2,3.2) {\tiny$\overleftarrow{h}_{\i}$};
}
\draw[->,thick] (f1) -- (f2) -- (f3);
\draw[<-,thick] (b1) -- (b2) -- (b3);
\node[below,font=\tiny] at (7,2.5) {看过去+未来};

\end{tikzpicture}
\end{center}

\begin{definition}[优势]
编码器利用双向信息,更好理解源序列
\end{definition}


\paragraph{改进方向4:深度模型}
\begin{center}
\begin{tikzpicture}[scale=0.8]
% 单层
\node[above] at (2,4) {\small 单层};
\node[draw,fill=blue!20,minimum width=3cm,minimum height=0.8cm] at (2,3) {Encoder};
\node[draw,fill=green!20,minimum width=3cm,minimum height=0.8cm] at (2,2) {Decoder};

% 多层
\node[above] at (7,4) {\small 多层};
\node[draw,fill=blue!20,minimum width=3cm,minimum height=0.6cm] at (7,3.5) {Encoder 2};
\node[draw,fill=blue!20,minimum width=3cm,minimum height=0.6cm] at (7,2.8) {Encoder 1};
\node[draw,fill=green!20,minimum width=3cm,minimum height=0.6cm] at (7,2.1) {Decoder 1};
\node[draw,fill=green!20,minimum width=3cm,minimum height=0.6cm] at (7,1.4) {Decoder 2};

\draw[->,thick] (7,3.2) -- (7,2.95);
\draw[->,thick] (7,2.5) -- (7,2.25);
\draw[->,thick] (7,1.8) -- (7,1.55);

\end{tikzpicture}
\end{center}

\begin{definition}[配置]
\begin{itemize}
    \item 编码器:2-4层双向LSTM
    \item 解码器:2-4层单向LSTM
    \item 层间Dropout
\end{itemize}
\end{definition}


\paragraph{改进效果对比}
\begin{center}
\begin{tabular}{|l|c|c|}
\hline
\textbf{模型} & \textbf{BLEU(短句)} & \textbf{BLEU(长句)} \\
\hline
基础Seq2Seq & 85.2 & 45.3 \\
\hline
+ 双向编码器 & 87.1 & 52.8 \\
\hline
+ 注意力机制 & 91.3 & 68.5 \\
\hline
+ 束搜索 & 92.8 & 71.2 \\
\hline
+ 深度模型 & 94.1 & 73.9 \\
\hline
全部改进 & \textbf{95.3} & \textbf{76.4} \\
\hline
\end{tabular}
\end{center}

\vspace{0.5cm}
\begin{theorem}[最大提升]
\textbf{注意力机制}对长句的改进最显著(+23\%)
\end{theorem}


\subsection{9.7节总结}
\begin{definition}[核心内容]
\begin{enumerate}
    \item \textbf{完整实现}:英法翻译系统
    \begin{itemize}
        \item 数据准备、模型训练、评估
    \end{itemize}
    
    \item \textbf{性能分析}:
    \begin{itemize}
        \item 短句:BLEU $>$ 0.9
        \item 长句:BLEU下降(信息瓶颈)
    \end{itemize}
    
    \item \textbf{常见错误}:
    \begin{itemize}
        \item 漏词、词序错误、重复
    \end{itemize}
    
    \item \textbf{改进方向}:
    \begin{itemize}
        \item 注意力机制(最重要)
        \item 束搜索、双向编码器、深度模型
    \end{itemize}
\end{enumerate}
\end{definition}


\end{document}


% The lateset FZU-Beamer-Theme is modified by Hanlin Cai.
% The latex template file for FZU can be found here:
% https://github.com/GuangLun2000/FZU-latex-template



