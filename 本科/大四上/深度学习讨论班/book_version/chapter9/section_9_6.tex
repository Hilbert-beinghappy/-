\section{编码器-解码器架构}

\section{9.6 编码器-解码器架构}

\subsection{9.6 编码器-解码器架构}
\begin{center}
\Large \textcolor{blue}{Encoder-Decoder Architecture}\\
\large 序列到序列模型的核心框架
\end{center}


\subsection{9.6节 - 章节导航}
\begin{definition}[本节内容]
\begin{itemize}
    \item 9.6.1 编码器
    \item 9.6.2 解码器
    \item 9.6.3 编码器-解码器模型
    \item 9.6.4 损失函数与训练
\end{itemize}
\end{definition}

\begin{theorem}[学习目标]
\begin{enumerate}
    \item 理解编码器-解码器架构
    \item 实现RNN编码器
    \item 实现RNN解码器
    \item 掌握序列到序列模型的训练
\end{enumerate}
\end{theorem}


\paragraph{编码器-解码器架构概览}
\begin{center}
\begin{tikzpicture}[scale=0.85]
% 源序列
\node[above] at (2,4.5) {\small 源序列(英语)};
\foreach \i/\w in {1/I,2/love,3/you} {
    \node[draw,fill=blue!20,minimum size=0.7cm] at (\i*1.2,4) {\tiny\w};
}

% 编码器
\node[draw,fill=yellow!20,rounded corners,minimum width=4cm,minimum height=1.2cm] at (2.2,2.5) {
\textbf{编码器}(Encoder)\\
\tiny RNN/LSTM/GRU
};

\draw[->,thick] (2.2,3.5) -- (2.2,3.1);

% 上下文向量
\node[draw,fill=purple!20,rounded corners,minimum width=2.5cm] at (2.2,1.3) {
上下文向量 $\mathbf{c}$
};

\draw[->,thick] (2.2,2) -- (2.2,1.6);

% 解码器
\node[draw,fill=orange!20,rounded corners,minimum width=4cm,minimum height=1.2cm,align=center] at (7.5,2.5) {
\textbf{解码器}(Decoder)\\
\tiny RNN/LSTM/GRU
};

\draw[->,ultra thick] (3.5,1.3) -- (5.7,1.8);
\node[above,font=\tiny] at (4.6,1.8) {传递信息};

% 目标序列
\node[above] at (7.5,4.5) {\small 目标序列(法语)};
\foreach \i/\w in {1/Je,2/t',3/aime} {
    \node[draw,fill=green!20,minimum size=0.7cm] at ({5.5+\i*1.2},4) {\tiny\w};
}

\draw[->,thick] (7.5,3.1) -- (7.5,3.5);

\node[below,text width=10cm,align=center] at (5,0.5) {
\small 编码器压缩信息,解码器生成输出
};

\end{tikzpicture}
\end{center}



\subsection{编码器}

\subsection{9.6.1 编码器}

\subsection{9.6.1 编码器}
\begin{center}
\Large \textcolor{blue}{将源序列编码为上下文向量}
\end{center}


\paragraph{编码器的任务}
\textbf{目标:}将变长的源序列压缩成固定长度的向量

\begin{center}
\begin{tikzpicture}[scale=0.9]
% 输入序列
\node[above] at (3,3.5) {\small 输入:变长序列};
\foreach \i in {1,2,3,4,5} {
    \node[draw,fill=blue!20,minimum size=0.6cm] (x\i) at (\i*0.9,3) {\tiny$x_{\i}$};
}
\node at (6,3) {$\ldots$};

% RNN处理
\foreach \i in {1,2,3,4,5} {
    \node[circle,draw,fill=red!20,minimum size=0.7cm] (h\i) at (\i*0.9,1.5) {\tiny$h_{\i}$};
    \draw[->,thick] (x\i) -- (h\i);
}
\draw[->,thick] (h1) -- (h2);
\draw[->,thick] (h2) -- (h3);
\draw[->,thick] (h3) -- (h4);
\draw[->,thick] (h4) -- (h5);

% 上下文向量
\node[draw,fill=purple!20,rounded corners,minimum width=1.5cm] (c) at (9,1.5) {$\mathbf{c}$};
\draw[->,ultra thick] (h5) -- (c);
\node[right,font=\tiny] at (7,1.7) {最后隐状态};

\node[below] at (9,0.8) {\small 输出:固定维度};

\end{tikzpicture}
\end{center}

\begin{definition}[关键点]
\begin{itemize}
    \item 上下文向量 $\mathbf{c} = \mathbf{h}_T$(最后一个隐状态)
    \item 维度固定(如512维)
    \item 压缩了整个源序列的信息
\end{itemize}
\end{definition}


\paragraph{编码器的抽象接口}
\begin{lstlisting}
class Encoder(nn.Module):
    """()"""
    
    def __init__(self, **kwargs):
        super(Encoder, self).__init__(**kwargs)
    
    def forward(self, X, *args):
        """
        前向传播
        X: 输入sequence
        Return:上下文vector(可以是隐状态、Output等)
        """
        raise NotImplementedError
\end{lstlisting}

\begin{theorem}[设计模式]
定义抽象基类,具体的编码器(RNN、LSTM、Transformer等)继承并实现
\end{theorem}


\paragraph{RNN编码器实现}
\begin{lstlisting}
class Seq2SeqEncoder(Encoder):
    """RNN"""
    
    def __init__(self, vocab_size, embed_size, num_hiddens, 
                 num_layers, dropout=0, **kwargs):
        super(Seq2SeqEncoder, self).__init__(**kwargs)
        
        self.embedding = nn.Embedding(vocab_size, embed_size)
        
        # RNNLSTMGRU
        self.rnn = nn.GRU(embed_size, num_hiddens, num_layers,
                         dropout=dropout)
    
    def forward(self, X, *args):
        """
        X: (batch_size, num_steps) 输入sequence(词索引)
        Return:(Output, 隐状态)
        """
        # (batch, steps) -> (batch, steps, embed)
        X = self.embedding(X)
        
        # (batch, steps, embed) -> (steps, batch, embed)
        X = X.permute(1, 0, 2)
        
        # RNN
        output, state = self.rnn(X)
        
        return output, state
\end{lstlisting}


\paragraph{编码器的组件}
\begin{center}
\begin{tikzpicture}[scale=0.85]
% 输入
\node[draw,fill=yellow!20,rounded corners,text width=10cm] at (5,4.5) {
\textbf{1. 嵌入层(Embedding)}\\
词索引 $\rightarrow$ 稠密向量\\
$(batch, steps)$ $\rightarrow$ $(batch, steps, embed\_size)$
};

\draw[->,thick] (5,4) -- (5,3.6);

% RNN
\node[draw,fill=blue!20,rounded corners,text width=10cm] at (5,3.1) {
\textbf{2. RNN层(GRU/LSTM)}\\
处理嵌入序列\\
$(steps, batch, embed)$ $\rightarrow$ $(steps, batch, hidden)$
};

\draw[->,thick] (5,2.6) -- (5,2.2);

% 输出
\node[draw,fill=green!20,rounded corners,text width=10cm] at (5,1.7) {
\textbf{3. 输出}\\
- output: 所有时间步的隐状态\\
- state: 最终隐状态(上下文向量)
};

\end{tikzpicture}
\end{center}


\paragraph{测试编码器}
\begin{lstlisting}
vocab_size = 10  #  commentvocabulary大小
embed_size = 8   #  commentdimension
num_hiddens = 16 #  comment
num_layers = 2   #  comment
batch_size = 4
num_steps = 7

encoder = Seq2SeqEncoder(vocab_size, embed_size, num_hiddens, num_layers)

# test
X = torch.zeros((batch_size, num_steps), dtype=torch.long)
output, state = encoder(X)

print(f"Outputshape: {output.shape}")  # (7, 4, 16)
print(f"状态shape: {state.shape}")   # (2, 4, 16)
\end{lstlisting}

\begin{definition}[输出解释]
\begin{itemize}
    \item output: $(num\_steps, batch, hidden)$ 所有隐状态
    \item state: $(num\_layers, batch, hidden)$ 最终状态
\end{itemize}
\end{definition}



\subsection{解码器}

\subsection{9.6.2 解码器}

\subsection{9.6.2 解码器}
\begin{center}
\Large \textcolor{blue}{从上下文向量生成目标序列}
\end{center}


\paragraph{解码器的任务}
\textbf{目标:}基于上下文向量,逐步生成目标序列

\begin{center}
\begin{tikzpicture}[scale=0.85]
% 上下文
\node[draw,fill=purple!20,rounded corners] (c) at (0,3) {$\mathbf{c}$};

% 解码步骤
\node[circle,draw,fill=blue!20,minimum size=0.8cm] (s1) at (2,3) {\tiny$s_1$};
\node[circle,draw,fill=blue!20,minimum size=0.8cm] (s2) at (4,3) {\tiny$s_2$};
\node[circle,draw,fill=blue!20,minimum size=0.8cm] (s3) at (6,3) {\tiny$s_3$};

\draw[->,thick] (c) -- (s1);
\draw[->,thick] (s1) -- (s2);
\draw[->,thick] (s2) -- (s3);

% 输入
\node[draw,fill=green!20,minimum size=0.6cm] (y0) at (2,1.5) {\tiny<bos>};
\node[draw,fill=green!20,minimum size=0.6cm] (y1) at (4,1.5) {\tiny$y_1$};
\node[draw,fill=green!20,minimum size=0.6cm] (y2) at (6,1.5) {\tiny$y_2$};

\draw[->,thick] (y0) -- (s1);
\draw[->,thick] (y1) -- (s2);
\draw[->,thick] (y2) -- (s3);

% 输出
\node[draw,fill=red!20,minimum size=0.6cm] (o1) at (2,4.5) {\tiny$y_1$};
\node[draw,fill=red!20,minimum size=0.6cm] (o2) at (4,4.5) {\tiny$y_2$};
\node[draw,fill=red!20,minimum size=0.6cm] (o3) at (6,4.5) {\tiny$y_3$};

\draw[->,thick] (s1) -- (o1);
\draw[->,thick] (s2) -- (o2);
\draw[->,thick] (s3) -- (o3);

\node[below,text width=7cm,align=center] at (3.5,0.5) {
\small 每步基于上下文和前一个词预测下一个词
};

\end{tikzpicture}
\end{center}


\paragraph{解码器的输入}
\textbf{训练时(Teacher Forcing):}

\begin{center}
\begin{tikzpicture}[scale=0.9]
% 目标序列
\node[above] at (3,3.5) {\small 目标序列};
\foreach \i/\w in {1/Je,2/t',3/aime,4/<eos>} {
    \node[draw,fill=blue!20,minimum size=0.6cm] at (\i*1.5,3) {\tiny\w};
}

% 解码器输入
\node[left] at (0,1.8) {\small 输入};
\node[draw,fill=green!20,minimum size=0.6cm] at (1.5,1.8) {\tiny<bos>};
\foreach \i/\w in {1/Je,2/t',3/aime} {
    \node[draw,fill=green!20,minimum size=0.6cm] at ({(\i+1)*1.5},1.8) {\tiny\w};
}

% 解码器输出(标签)
\node[left] at (0,0.5) {\small 标签};
\foreach \i/\w in {1/Je,2/t',3/aime,4/<eos>} {
    \node[draw,fill=red!20,minimum size=0.6cm] at (\i*1.5,0.5) {\tiny\w};
}

% 标注
\draw[<->,thick,purple] (1.5,1.3) -- (6,1.3);
\node[purple,below] at (3.75,1.1) {\tiny 错位一个位置};

\end{tikzpicture}
\end{center}

\begin{theorem}[Teacher Forcing]
训练时,每一步都用\textbf{真实的}前一个词作为输入(而非预测的词)
\end{theorem}


\paragraph{解码器的抽象接口}
\begin{lstlisting}
class Decoder(nn.Module):
    """Simple function"""
    
    def __init__(self, **kwargs):
        super(Decoder, self).__init__(**kwargs)
    
    def init_state(self, enc_outputs, *args):
        """
        根据编码器Output初始化解码器状态
        enc_outputs: 编码器的Output
        """
        raise NotImplementedError
    
    def forward(self, X, state):
        """
        前向传播
        X: 目标sequence(解码器输入)
        state: 解码器状态
        """
        raise NotImplementedError
\end{lstlisting}


\paragraph{RNN解码器实现}
\begin{lstlisting}
class Seq2SeqDecoder(Decoder):
    """RNN"""
    
    def __init__(self, vocab_size, embed_size, num_hiddens, 
                 num_layers, dropout=0, **kwargs):
        super(Seq2SeqDecoder, self).__init__(**kwargs)
        
        self.embedding = nn.Embedding(vocab_size, embed_size)
        
        # RNN = embed + context
        self.rnn = nn.GRU(embed_size + num_hiddens, num_hiddens, 
                         num_layers, dropout=dropout)
        
        # Output
        self.dense = nn.Linear(num_hiddens, vocab_size)
    
    def init_state(self, enc_outputs, *args):
        """用编码器的最终状态初始化"""
        return enc_outputs[1]  # enc_outputs = (output, state)
\end{lstlisting}


\paragraph{RNN解码器前向传播}
\begin{lstlisting}
    def forward(self, X, state):
        """ X: (batch_size, num_steps) sequence state: () """
        X = self.embedding(X).permute(1, 0, 2)
        # X: (num_steps, batch, embed_size)
        
        # vector
        context = state[-1].repeat(X.shape[0], 1, 1)
        # context: (num_steps, batch, num_hiddens)
        
        X_and_context = torch.cat((X, context), 2)
        # (num_steps, batch, embed + hidden)
        
        # RNN
        output, state = self.rnn(X_and_context, state)
        
        # Output
        output = self.dense(output).permute(1, 0, 2)
        # (batch, num_steps, vocab_size)
        
        return output, state
\end{lstlisting}


\paragraph{解码器的输入结构}
\begin{center}
\begin{tikzpicture}[scale=0.85]
% 每个时间步
\node[above] at (3,4) {\small 每个时间步的输入};

% 嵌入
\node[draw,fill=blue!20,minimum width=2cm,align=center] at (1,3) {
词嵌入\\
\tiny $(batch, embed)$
};

% 上下文
\node[draw,fill=purple!20,minimum width=2cm,align=center] at (5,3) {
上下文 $\mathbf{c}$\\
\tiny $(batch, hidden)$
};

% 拼接
\node[draw,circle,fill=yellow!20] at (3,1.8) {拼接};

\draw[->,thick] (1,2.6) -- (3,2.1);
\draw[->,thick] (5,2.6) -- (3,2.1);

% 结果
\node[draw,fill=green!20,minimum width=4cm,align=center] at (3,0.8) {
RNN输入\\
\tiny $(batch, embed + hidden)$
};

\draw[->,thick] (3,1.5) -- (3,1.1);

\end{tikzpicture}
\end{center}

\begin{theorem}[关键]
每个时间步都将\textbf{词嵌入}和\textbf{上下文向量}拼接作为RNN输入
\end{theorem}


\paragraph{测试解码器}
\begin{lstlisting}
decoder = Seq2SeqDecoder(vocab_size, embed_size, num_hiddens, num_layers)

state = decoder.init_state(encoder(X))

# sequence
Y = torch.zeros((batch_size, num_steps), dtype=torch.long)

output, state = decoder(Y, state)

print(f"Outputshape: {output.shape}")  # (4, 7, 10)
print(f"状态shape: {state.shape}")   # (2, 4, 16)
\end{lstlisting}

\begin{definition}[输出解释]
\begin{itemize}
    \item output: $(batch, num\_steps, vocab\_size)$ 每步的预测分布
    \item state: $(num\_layers, batch, hidden)$ 最终状态
\end{itemize}
\end{definition}



\subsection{编码器-解码器模型}

\subsection{9.6.3 编码器-解码器模型}

\subsection{9.6.3 编码器-解码器模型}
\begin{center}
\Large \textcolor{blue}{组合编码器和解码器}
\end{center}


\paragraph{EncoderDecoder类}
\begin{lstlisting}
class EncoderDecoder(nn.Module):
    """-"""
    
    def __init__(self, encoder, decoder, **kwargs):
        super(EncoderDecoder, self).__init__(**kwargs)
        self.encoder = encoder
        self.decoder = decoder
    
    def forward(self, enc_X, dec_X, *args):
        """
        enc_X: 编码器输入(源sequence)
        dec_X: 解码器输入(目标sequence,training时)
        """
        enc_outputs = self.encoder(enc_X, *args)
        
        dec_state = self.decoder.init_state(enc_outputs, *args)
        
        return self.decoder(dec_X, dec_state)
\end{lstlisting}


\paragraph{创建完整模型}
\begin{lstlisting}
# parameters
src_vocab_size = 10  #  commentvocabulary
tgt_vocab_size = 10  #  commentvocabulary
embed_size = 8
num_hiddens = 16
num_layers = 2

encoder = Seq2SeqEncoder(src_vocab_size, embed_size, 
                        num_hiddens, num_layers)
decoder = Seq2SeqDecoder(tgt_vocab_size, embed_size, 
                        num_hiddens, num_layers)

# model
net = EncoderDecoder(encoder, decoder)

# test
enc_X = torch.zeros((batch_size, num_steps), dtype=torch.long)
dec_X = torch.zeros((batch_size, num_steps), dtype=torch.long)
output, state = net(enc_X, dec_X)

print(f"Outputshape: {output.shape}")  # (4, 7, 10)
\end{lstlisting}



\subsection{损失函数与训练}

\subsection{9.6.4 损失函数与训练}

\subsection{9.6.4 损失函数与训练}
\begin{center}
\Large \textcolor{blue}{训练序列到序列模型}
\end{center}


\paragraph{序列到序列的损失}
\textbf{目标:}最大化 $P(y_1, \ldots, y_{T'} \mid x_1, \ldots, x_T)$

\begin{definition}[交叉熵损失]
$$\mathcal{L} = -\frac{1}{T'}\sum_{t=1}^{T'}\log P(y_t \mid y_1, \ldots, y_{t-1}, \mathbf{c})$$
\end{definition}

\textbf{问题:}不同序列长度不同,如何处理padding?

\begin{theorem}[Masked Loss]
只计算\textbf{有效位置}的损失,忽略padding部分
$$\mathcal{L} = -\frac{1}{\sum_t \mathbb{1}[y_t \neq \text{<pad>}]}\sum_{t: y_t \neq \text{<pad>}}\log P(y_t \mid \cdots)$$
\end{theorem}


\paragraph{带Mask的损失函数}
\begin{lstlisting}
def sequence_mask(X, valid_len, value=0):
    """ sequencepadding X: (batch, steps, ...) valid_len: (batch,) sequencelength """
    maxlen = X.size(1)
    mask = torch.arange((maxlen), dtype=torch.float32,
                       device=X.device)[None, :] < valid_len[:, None]
    X[~mask] = value
    return X

def masked_softmax_cross_entropy(pred, label, valid_len):
    """
    带mask的交叉熵
    pred: (batch, steps, vocab_size)
    label: (batch, steps)
    valid_len: (batch,) 有效length
    """
    # 1padding0
    weights = torch.ones_like(label)
    weights = sequence_mask(weights, valid_len)
    
    weights = weights.reshape(-1)
    pred = pred.reshape(-1, pred.shape[-1])
    label = label.reshape(-1)
    
    # loss
    unweighted_loss = nn.CrossEntropyLoss(reduction='none')(pred, label)
    
    weighted_loss = (unweighted_loss * weights).sum() / weights.sum()
    return weighted_loss
\end{lstlisting}


\paragraph{Mask损失的可视化}
\begin{center}
\begin{tikzpicture}[scale=0.85]
% 序列
\node[above] at (3,3.5) {\small 目标序列};
\node[draw,fill=green!20,minimum size=0.6cm] at (0.8,3) {\tiny Je};
\node[draw,fill=green!20,minimum size=0.6cm] at (2,3) {\tiny t'};
\node[draw,fill=green!20,minimum size=0.6cm] at (3.2,3) {\tiny aime};
\node[draw,fill=green!20,minimum size=0.6cm] at (4.4,3) {\tiny<eos>};
\node[draw,fill=gray!20,minimum size=0.6cm] at (5.6,3) {\tiny<pad>};

% 权重
\node[left] at (0,1.8) {\small 权重};
\node[draw,fill=blue!20,minimum size=0.6cm] at (0.8,1.8) {\tiny1};
\node[draw,fill=blue!20,minimum size=0.6cm] at (2,1.8) {\tiny1};
\node[draw,fill=blue!20,minimum size=0.6cm] at (3.2,1.8) {\tiny1};
\node[draw,fill=blue!20,minimum size=0.6cm] at (4.4,1.8) {\tiny1};
\node[draw,fill=red!20,minimum size=0.6cm] at (5.6,1.8) {\tiny0};

% 损失
\node[left] at (0,0.5) {\small 损失};
\node[draw,fill=green!20,minimum size=0.6cm] at (0.8,0.5) {\tiny\checkmark};
\node[draw,fill=green!20,minimum size=0.6cm] at (2,0.5) {\tiny\checkmark};
\node[draw,fill=green!20,minimum size=0.6cm] at (3.2,0.5) {\tiny\checkmark};
\node[draw,fill=green!20,minimum size=0.6cm] at (4.4,0.5) {\tiny\checkmark};
\node[draw,fill=red!20,minimum size=0.6cm] at (5.6,0.5) {\tiny$\times$};

\node[below,text width=6cm,align=center] at (3,-0.3) {
\small 只计算有效位置的损失
};

\end{tikzpicture}
\end{center}

\paragraph{训练函数}
\begin{lstlisting}
def train_seq2seq(net, data_iter, lr, num_epochs, device):
    """trainingseq2seqmodel"""
    net.to(device)
    optimizer = torch.optim.Adam(net.parameters(), lr=lr)
    
    for epoch in range(num_epochs):
        net.train()
        total_loss = 0
        num_batches = 0
        
        for batch in data_iter:
            X, X_valid_len, Y, Y_valid_len = [x.to(device) for x in batch]
            
            # bos
            bos = torch.tensor([tgt_vocab['<bos>']] * Y.shape[0], 
                              device=device).reshape(-1, 1)
            dec_input = torch.cat([bos, Y[:, :-1]], 1)  # Teacher forcing
            
            Y_hat, _ = net(X, dec_input)
            
            # loss
            l = masked_softmax_cross_entropy(Y_hat, Y, Y_valid_len)
            
            optimizer.zero_grad()
            l.backward()
            grad_clipping(net, 1)
            optimizer.step()
            
            total_loss += l.item()
            num_batches += 1
        
        print(f'epoch {epoch + 1}, loss {total_loss / num_batches:.3f}')
\end{lstlisting}


\paragraph{训练过程详解}
\begin{center}
\begin{tikzpicture}[scale=0.75]
% 步骤1
\node[draw,fill=yellow!20,rounded corners,text width=11cm] at (5.5,5) {
\textbf{步骤1:}准备解码器输入\\
目标序列:\texttt{[Je, t', aime, <eos>]}\\
解码器输入:\texttt{[<bos>, Je, t', aime]} \quad (前移一位)
};

\draw[->,thick] (5.5,4.5) -- (5.5,4.1);

% 步骤2
\node[draw,fill=blue!20,rounded corners,text width=11cm] at (5.5,3.6) {
\textbf{步骤2:}前向传播\\
编码器处理源序列 $\rightarrow$ 解码器生成预测
};

\draw[->,thick] (5.5,3.1) -- (5.5,2.7);

% 步骤3
\node[draw,fill=green!20,rounded corners,text width=11cm] at (5.5,2.2) {
\textbf{步骤3:}计算Masked损失\\
只在有效位置计算交叉熵(忽略padding)
};

\draw[->,thick] (5.5,1.7) -- (5.5,1.3);

% 步骤4
\node[draw,fill=orange!20,rounded corners,text width=11cm] at (5.5,0.8) {
\textbf{步骤4:}反向传播与更新\\
梯度裁剪 $\rightarrow$ 优化器更新
};

\end{tikzpicture}
\end{center}


\paragraph{预测(推理)}
\begin{lstlisting}
def predict_seq2seq(net, src_sentence, src_vocab, tgt_vocab, 
                   num_steps, device):
    """ trainingmodel src_sentence: () """
    net.eval()
    
    src_tokens = src_vocab[src_sentence.lower().split(' ')] + [
        src_vocab['<eos>']]
    enc_valid_len = torch.tensor([len(src_tokens)], device=device)
    
    src_tokens = truncate_pad(src_tokens, num_steps, src_vocab['<pad>'])
    enc_X = torch.unsqueeze(
        torch.tensor(src_tokens, dtype=torch.long, device=device), dim=0)
    
    enc_outputs = net.encoder(enc_X, enc_valid_len)
    dec_state = net.decoder.init_state(enc_outputs, enc_valid_len)
    
    dec_X = torch.unsqueeze(
        torch.tensor([tgt_vocab['<bos>']], dtype=torch.long, 
                    device=device), dim=0)
    output_seq = []
    
    for _ in range(num_steps):
        Y, dec_state = net.decoder(dec_X, dec_state)
        dec_X = Y.argmax(dim=2)  # comment
        pred = dec_X.squeeze(dim=0).type(torch.int32).item()
        
        if pred == tgt_vocab['<eos>']:
            break
        output_seq.append(pred)
    
    return ' '.join(tgt_vocab.to_tokens(output_seq))
\end{lstlisting}


\paragraph{推理过程(贪心解码)}
\begin{center}
\begin{tikzpicture}[scale=0.8]
% 编码
\node[draw,fill=blue!20,rounded corners,minimum width=3cm,align=center] at (1.5,3) {
编码器\\
\tiny ``I love you''
};

\node[draw,fill=purple!20] (c) at (1.5,1.5) {$\mathbf{c}$};
\draw[->,thick] (1.5,2.6) -- (c);

% 解码步骤
\node[circle,draw,fill=green!20,minimum size=0.7cm] (s1) at (4,1.5) {\tiny$s_1$};
\node[circle,draw,fill=green!20,minimum size=0.7cm] (s2) at (6,1.5) {\tiny$s_2$};
\node[circle,draw,fill=green!20,minimum size=0.7cm] (s3) at (8,1.5) {\tiny$s_3$};
\node at (9.5,1.5) {$\ldots$};

\draw[->,thick] (c) -- (s1);
\draw[->,thick] (s1) -- (s2);
\draw[->,thick] (s2) -- (s3);

% 输入
\node[below,font=\tiny] at (4,0.8) {<bos>};
\node[below,font=\tiny] at (6,0.8) {Je};
\node[below,font=\tiny] at (8,0.8) {t'};

% 输出
\node[above,font=\tiny,fill=red!20] at (4,2.2) {Je};
\node[above,font=\tiny,fill=red!20] at (6,2.2) {t'};
\node[above,font=\tiny,fill=red!20] at (8,2.2) {aime};

% 箭头
\draw[->,thick,red] (4,2) -- (6,1);
\draw[->,thick,red] (6,2) -- (8,1);

\node[below,text width=10cm,align=center] at (5.5,-0.3) {
\small 每步选概率最大的词,作为下一步输入
};

\end{tikzpicture}
\end{center}

\begin{theorem}[贪心解码]
每步选择概率最大的词,简单但可能次优(下节介绍束搜索)
\end{theorem}


\paragraph{BLEU评估}
\begin{lstlisting}
def bleu(pred_seq, label_seq, k):
    """BLEU"""
    pred_tokens, label_tokens = pred_seq.split(' '), label_seq.split(' ')
    len_pred, len_label = len(pred_tokens), len(label_tokens)
    
    score = math.exp(min(0, 1 - len_label / len_pred))
    
    for n in range(1, k + 1):
        num_matches, label_subs = 0, collections.defaultdict(int)
        
        for i in range(len_label - n + 1):
            label_subs[' '.join(label_tokens[i: i + n])] += 1
        
        for i in range(len_pred - n + 1):
            if label_subs[' '.join(pred_tokens[i: i + n])] > 0:
                num_matches += 1
                label_subs[' '.join(pred_tokens[i: i + n])] -= 1
        
        score *= math.pow(num_matches / (len_pred - n + 1), 
                         math.pow(0.5, n))
    
    return score

bleu('je t aime .', 'je t aime .', k=2)  # 1.0 (完美匹配)
bleu('je t aime', 'je t aime .', k=2)    # 0.658 (length惩罚)
\end{lstlisting}


\paragraph{BLEU分数(BiLingual Evaluation Understudy)}
\textbf{评估翻译质量的标准指标}

\begin{definition}[核心思想]
\begin{itemize}
    \item 统计n-gram重叠
    \item 长度惩罚
    \item 综合评分
\end{itemize}
\end{definition}

\begin{example}[例子]
\textbf{参考:}``Je t'aime .''\\
\textbf{预测1:}``Je t'aime .'' $\Rightarrow$ BLEU = 1.0\\
\textbf{预测2:}``Je aime .'' $\Rightarrow$ BLEU $\approx$ 0.7\\
\textbf{预测3:}``aime Je .'' $\Rightarrow$ BLEU $\approx$ 0.4
\end{example}

\begin{theorem}[范围]
BLEU $\in [0, 1]$,越高越好
\end{theorem}


\subsection{9.6节总结}
\begin{definition}[核心内容]
\begin{enumerate}
    \item \textbf{编码器}:将源序列压缩为上下文向量
    $$\mathbf{c} = \text{Encoder}(x_1, \ldots, x_T)$$
    
    \item \textbf{解码器}:从上下文生成目标序列
    $$P(y_t \mid y_{1:t-1}, \mathbf{c})$$
    
    \item \textbf{训练}:Teacher Forcing + Masked Loss
    
    \item \textbf{推理}:贪心解码(每步选最大概率)
    
    \item \textbf{评估}:BLEU分数
\end{enumerate}
\end{definition}


\paragraph{关键要点回顾}
\begin{center}
\begin{tikzpicture}[scale=0.85]
% 要点1
\node[draw,fill=blue!20,rounded corners,text width=10cm,align=left] at (5,3.8) {
\textbf{1. 上下文向量}\\
编码器最后隐状态,压缩整个源序列信息
};

% 要点2
\node[draw,fill=green!20,rounded corners,text width=10cm,align=left] at (5,2.6) {
\textbf{2. Teacher Forcing}\\
训练时用真实词作为解码器输入
};

% 要点3
\node[draw,fill=orange!20,rounded corners,text width=10cm,align=left] at (5,1.4) {
\textbf{3. Masked Loss}\\
只计算有效位置损失,忽略padding
};

% 要点4
\node[draw,fill=purple!20,rounded corners,text width=10cm,align=left] at (5,0.2) {
\textbf{4. 贪心解码}\\
推理时每步选概率最大的词
};

\end{tikzpicture}
\end{center}


\paragraph{编码器-解码器架构的局限}
\begin{theorem}[核心问题:信息瓶颈]
将整个源序列压缩到\textbf{固定长度}的向量
\begin{itemize}
    \item 源序列很长时,信息损失严重
    \item 解码器看不到源序列细节
    \item 所有解码步骤共享同一个上下文
\end{itemize}
\end{theorem}

\begin{example}[例子]
源句子:``The agreement on the European Economic Area was signed in August 1992.''

上下文向量(512维)很难完整保存这么多信息!
\end{example}

\begin{definition}[解决方案]
\textbf{注意力机制}(Attention):让解码器动态关注源序列的不同部分
\end{definition}


\paragraph{实践建议}
\begin{definition}[超参数选择]
\begin{itemize}
    \item \textbf{嵌入维度}:128-512
    \item \textbf{隐藏单元}:256-1024
    \item \textbf{层数}:1-4层
    \item \textbf{Dropout}:0.1-0.5
    \item \textbf{学习率}:0.001(Adam)
    \item \textbf{梯度裁剪}:阈值1-5
\end{itemize}
\end{definition}

\begin{definition}[训练技巧]
\begin{itemize}
    \item 从小数据集开始验证
    \item 监控训练集和验证集BLEU
    \item 使用学习率衰减
    \item 早停(验证集BLEU不再提升)
\end{itemize}
\end{definition}


\paragraph{常见问题与解决}
\begin{definition}[问题1:生成重复]
\textbf{现象:}``Je Je Je ...''

\textbf{原因:}模型陷入局部最优

\textbf{解决:}
\begin{itemize}
    \item 使用束搜索(下一节)
    \item 增加Dropout
    \item 使用Coverage机制
\end{itemize}
\end{definition}

\begin{definition}[问题2:翻译不完整]
\textbf{现象:}提前输出\texttt{<eos>}

\textbf{原因:}长度惩罚不够

\textbf{解决:}
\begin{itemize}
    \item 调整BLEU中的长度惩罚
    \item 限制最小输出长度
\end{itemize}
\end{definition}

\section{9.7 序列到序列学习}

\subsection{9.7 序列到序列学习}
\begin{center}
\Large \textcolor{blue}{Sequence to Sequence Learning}\\
\large 完整实现机器翻译系统
\end{center}


\subsection{9.7节 - 章节导航}
\begin{definition}[本节内容]
\begin{itemize}
    \item 9.7.1 训练Seq2Seq模型
    \item 9.7.2 预测与评估
    \item 9.7.3 性能分析
    \item 9.7.4 改进方向
\end{itemize}
\end{definition}

\begin{theorem}[学习目标]
\begin{enumerate}
    \item 在真实数据集上训练Seq2Seq
    \item 评估翻译质量
    \item 分析模型性能
    \item 了解改进方法
\end{enumerate}
\end{theorem}



