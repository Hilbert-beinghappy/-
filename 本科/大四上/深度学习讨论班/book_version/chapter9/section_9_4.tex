\section{未命名节}

\subsection{双向RNN的动机}

\subsection{9.4.1 双向循环神经网络的动机}

\subsection{9.4.1 双向循环神经网络的动机}
\begin{center}
\Large \textcolor{blue}{为什么需要看"未来"?}
\end{center}


\paragraph{自然语言的双向性}
\textbf{观察:}理解一个词常需要上下文

\begin{example}[例子1:词性标注]
``The \textbf{bank} is closed.'' $\rightarrow$ bank = 银行(名词)\\
``I will \textbf{bank} on you.'' $\rightarrow$ bank = 依靠(动词)

需要看后面的词才能确定词性
\end{example}

\begin{example}[例子2:命名实体识别]
``\textbf{Apple} released a new phone.'' $\rightarrow$ Apple = 公司\\
``I ate an \textbf{apple}.'' $\rightarrow$ apple = 水果

需要看上下文才能判断
\end{example}

\begin{example}[例子3:情感分析]
``The movie was not bad.'' $\rightarrow$ 正面(需要看到``not'')\\
只看``bad''会误判为负面
\end{example}


\paragraph{单向 vs 双向对比}
\begin{center}
\begin{tikzpicture}[scale=0.85]
% 句子
\node[font=\large] at (5,4) {句子:``I love this movie very much''};

% 单向(预测"this")
\node[above] at (2,3) {\small 单向RNN预测``this''};
\node[draw,fill=blue!20] (w1) at (0.5,2.5) {I};
\node[draw,fill=blue!20] (w2) at (1.5,2.5) {love};
\node[draw,fill=red!20] (w3) at (2.5,2.5) {this};
\node[draw,fill=gray!20] (w4) at (3.5,2.5) {?};
\node[draw,fill=gray!20] (w5) at (4.5,2.5) {?};

\draw[->,thick] (w1) -- (w2) -- (w3);
\node[below,text width=5cm,align=center] at (2.5,1.8) {
只能用``I love'' $\rightarrow$ 信息不足
};

% 双向
\node[above] at (8,3) {\small 双向RNN预测``this''};
\node[draw,fill=blue!20] (v1) at (6,2.5) {I};
\node[draw,fill=blue!20] (v2) at (7,2.5) {love};
\node[draw,fill=red!20] (v3) at (8,2.5) {this};
\node[draw,fill=green!20] (v4) at (9,2.5) {movie};
\node[draw,fill=green!20] (v5) at (10,2.5) {much};

\draw[->,thick,blue] (v1) -- (v2) -- (v3);
\draw[<-,thick,green!60!black] (v3) -- (v4) -- (v5);
\node[below,text width=5cm,align=center] at (8,1.8) {
用``I love'' + ``movie much'' $\rightarrow$ 信息充分
};

\end{tikzpicture}
\end{center}


\paragraph{双向RNN适用的任务}
\begin{definition}[完全适合(离线任务)]
\begin{itemize}
    \item \textbf{文本分类}:情感分析、主题分类
    \item \textbf{序列标注}:词性标注、命名实体识别
    \item \textbf{机器翻译}:编码器部分
    \item \textbf{问答系统}:理解问题和文档
    \item \textbf{文本摘要}:理解全文
\end{itemize}
\end{definition}

\begin{theorem}[不适合(在线任务)]
\begin{itemize}
    \item \textbf{语言模型}:预测下一个词(不能看未来)
    \item \textbf{实时语音识别}:需要即时输出
    \item \textbf{在线翻译}:解码器部分(逐词生成)
\end{itemize}
\end{theorem}



\subsection{双向RNN的定义}

\subsection{9.4.2 双向循环神经网络的定义}

\subsection{9.4.2 双向循环神经网络的定义}
\begin{center}
\Large \textcolor{blue}{双向RNN的数学表示}
\end{center}


\paragraph{双向RNN的结构}
\begin{center}
\begin{tikzpicture}[scale=0.8]
% 输入
\foreach \t in {1,2,3,4} {
    \node[circle,draw,fill=yellow!20,minimum size=0.6cm] (x\t) at (\t*2,0) {\tiny$x_{\t}$};
}

% 前向层
\foreach \t in {1,2,3,4} {
    \node[circle,draw,fill=blue!20,minimum size=0.8cm] (f\t) at (\t*2,1.5) {\tiny$\overrightarrow{h}_{\t}$};
    \draw[->,thick] (x\t) -- (f\t);
}
\draw[->,thick,red] (f1) -- (f2) -- (f3) -- (f4);
\node[above,red,font=\small] at (4,2) {前向层};

% 后向层
\foreach \t in {1,2,3,4} {
    \node[circle,draw,fill=green!20,minimum size=0.8cm] (b\t) at (\t*2,3) {\tiny$\overleftarrow{h}_{\t}$};
    \draw[->,thick] (x\t) to[bend left=30] (b\t);
}
\draw[<-,thick,blue] (b1) -- (b2) -- (b3) -- (b4);
\node[above,blue,font=\small] at (4,3.5) {后向层};

% 输出
\foreach \t in {1,2,3,4} {
    \node[circle,draw,fill=purple!20,minimum size=0.8cm] (o\t) at (\t*2,4.5) {\tiny$o_{\t}$};
    \draw[->,thick] (f\t) to[bend right=15] (o\t);
    \draw[->,thick] (b\t) -- (o\t);
}
\node[above,purple,font=\small] at (4,5) {输出(拼接)};

\node[below,text width=8cm,align=center] at (4.5,-0.8) {
\small 每个时间步的输出结合了前向和后向信息
};

\end{tikzpicture}
\end{center}


\paragraph{双向RNN的数学公式}
\begin{definition}[前向层(从左到右)]
$$\overrightarrow{\mathbf{H}}_t = \phi(\mathbf{X}_t\mathbf{W}_{xh}^{(f)} + \overrightarrow{\mathbf{H}}_{t-1}\mathbf{W}_{hh}^{(f)} + \mathbf{b}_h^{(f)})$$
\end{definition}

\begin{definition}[后向层(从右到左)]
$$\overleftarrow{\mathbf{H}}_t = \phi(\mathbf{X}_t\mathbf{W}_{xh}^{(b)} + \overleftarrow{\mathbf{H}}_{t+1}\mathbf{W}_{hh}^{(b)} + \mathbf{b}_h^{(b)})$$
\end{definition}

\begin{definition}[输出(拼接)]
$$\mathbf{H}_t = [\overrightarrow{\mathbf{H}}_t; \overleftarrow{\mathbf{H}}_t]$$
$$\mathbf{O}_t = \mathbf{H}_t\mathbf{W}_{hq} + \mathbf{b}_q$$
\end{definition}

\begin{theorem}[关键]
\begin{itemize}
    \item $\overrightarrow{\mathbf{H}}_t$:前向隐状态,维度 $h$
    \item $\overleftarrow{\mathbf{H}}_t$:后向隐状态,维度 $h$
    \item $\mathbf{H}_t$:拼接后,维度 $2h$
\end{itemize}
\end{theorem}


\paragraph{双向RNN的计算流程}
\begin{center}
\begin{tikzpicture}[scale=0.75]
% 步骤1
\node[draw,fill=yellow!20,rounded corners,text width=11cm] at (5.5,4.5) {
\textbf{步骤1:}前向传播($t=1 \rightarrow T$)\\
$\overrightarrow{\mathbf{H}}_1 = \phi(\mathbf{X}_1\mathbf{W}_{xh}^{(f)} + \mathbf{b}_h^{(f)})$\\
$\overrightarrow{\mathbf{H}}_2 = \phi(\mathbf{X}_2\mathbf{W}_{xh}^{(f)} + \overrightarrow{\mathbf{H}}_1\mathbf{W}_{hh}^{(f)} + \mathbf{b}_h^{(f)})$\\
...
};

\draw[->,thick] (5.5,4) -- (5.5,3.6);

% 步骤2
\node[draw,fill=blue!20,rounded corners,text width=11cm] at (5.5,3.1) {
\textbf{步骤2:}后向传播($t=T \rightarrow 1$)\\
$\overleftarrow{\mathbf{H}}_T = \phi(\mathbf{X}_T\mathbf{W}_{xh}^{(b)} + \mathbf{b}_h^{(b)})$\\
$\overleftarrow{\mathbf{H}}_{T-1} = \phi(\mathbf{X}_{T-1}\mathbf{W}_{xh}^{(b)} + \overleftarrow{\mathbf{H}}_T\mathbf{W}_{hh}^{(b)} + \mathbf{b}_h^{(b)})$\\
...
};

\draw[->,thick] (5.5,2.6) -- (5.5,2.2);

% 步骤3
\node[draw,fill=green!20,rounded corners,text width=11cm] at (5.5,1.7) {
\textbf{步骤3:}拼接隐状态\\
$\mathbf{H}_t = [\overrightarrow{\mathbf{H}}_t; \overleftarrow{\mathbf{H}}_t]$ \quad for $t=1,\ldots,T$
};

\draw[->,thick] (5.5,1.2) -- (5.5,0.8);

% 步骤4
\node[draw,fill=purple!20,rounded corners,text width=11cm] at (5.5,0.3) {
\textbf{步骤4:}计算输出\\
$\mathbf{O}_t = \mathbf{H}_t\mathbf{W}_{hq} + \mathbf{b}_q$
};

\end{tikzpicture}
\end{center}


\paragraph{维度变化分析}
\textbf{假设:}隐藏单元数 $h=256$,词表大小 $V=28$

\begin{center}
\begin{tikzpicture}[scale=0.9]
% 输入
\node[draw,fill=yellow!20,text width=3cm] at (0,3) {
输入 $\mathbf{X}_t$\\
$(batch, V)$\\
$(32, 28)$
};

\draw[->,thick] (1.5,3) -- (2.5,3);

% 前向
\node[draw,fill=blue!20,text width=3cm] at (4,4) {
前向 $\overrightarrow{\mathbf{H}}_t$\\
$(batch, h)$\\
$(32, 256)$
};

% 后向
\node[draw,fill=green!20,text width=3cm] at (4,2) {
后向 $\overleftarrow{\mathbf{H}}_t$\\
$(batch, h)$\\
$(32, 256)$
};

\draw[->,thick] (5.5,4) -- (6.5,3.5);
\draw[->,thick] (5.5,2) -- (6.5,2.5);
\node[right,font=\tiny] at (6,3.8) {拼接};

% 拼接
\node[draw,fill=purple!20,text width=3cm] at (8.5,3) {
拼接 $\mathbf{H}_t$\\
$(batch, 2h)$\\
$(32, 512)$
};

\draw[->,thick] (10,3) -- (11,3);

% 输出
\node[draw,fill=red!20,text width=3cm] at (12.5,3) {
输出 $\mathbf{O}_t$\\
$(batch, V)$\\
$(32, 28)$
};

\end{tikzpicture}
\end{center}

\begin{theorem}[关键]
拼接后维度\textbf{翻倍}:$h \rightarrow 2h$
\end{theorem}


\paragraph{双向RNN的参数分析}
\textbf{参数数量:}

\begin{definition}[单向RNN]
$$\text{params} = (d+h) \times h + h + h \times V + V$$
\end{definition}

\begin{definition}[双向RNN]
\begin{itemize}
    \item 前向层:$(d+h) \times h + h$
    \item 后向层:$(d+h) \times h + h$
    \item 输出层:$2h \times V + V$ (注意:$2h$)
\end{itemize}
$$\text{params} = 2 \times [(d+h) \times h + h] + 2h \times V + V$$
\end{definition}

\begin{example}[具体例子($d=28, h=256, V=28$)]
单向:$(28+256) \times 256 + 256 + 256 \times 28 + 28 \approx 80K$

双向:$2 \times (284 \times 256 + 256) + 512 \times 28 + 28 \approx 160K$

双向参数约是单向的\textbf{2倍}
\end{example}



\subsection{双向RNN的实现}

\subsection{9.4.3 双向循环神经网络的实现}

\subsection{9.4.3 双向循环神经网络的实现}
\begin{center}
\Large \textcolor{blue}{用PyTorch实现双向RNN}
\end{center}


\paragraph{使用PyTorch创建双向RNN}
\begin{lstlisting}
import torch.nn as nn

# RNN
rnn_layer = nn.RNN(
    input_size=vocab_size,
    hidden_size=num_hiddens,
    num_layers=1,
    bidirectional=True  #  commentTrue
)

print(rnn_layer)
# RNN(28, 256, bidirectional=True)

# test
X = torch.randn(num_steps, batch_size, vocab_size)
H = torch.zeros(2, batch_size, num_hiddens)  #  comment2 = 2个方向

Y, H_new = rnn_layer(X, H)

print(f"输入shape: {X.shape}")      # (5, 32, 28)
print(f"Outputshape: {Y.shape}")      # (5, 32, 512) = 2*256
print(f"隐状态shape: {H_new.shape}") # (2, 32, 256)
\end{lstlisting}

\begin{theorem}[注意]
\begin{itemize}
    \item 输出维度:$2 \times hidden\_size$
    \item 隐状态第一维:$2 \times num\_layers$(2个方向)
\end{itemize}
\end{theorem}


\paragraph{双向LSTM/GRU}
\begin{lstlisting}
# LSTM
lstm_layer = nn.LSTM(
    input_size=vocab_size,
    hidden_size=num_hiddens,
    num_layers=1,
    bidirectional=True
)

# LSTMHC
H = torch.zeros(2, batch_size, num_hiddens)  # 2个方向
C = torch.zeros(2, batch_size, num_hiddens)
state = (H, C)

Y, (H_new, C_new) = lstm_layer(X, state)
print(f"Outputshape: {Y.shape}")  # (num_steps, batch, 2*hidden)

# GRU
gru_layer = nn.GRU(
    input_size=vocab_size,
    hidden_size=num_hiddens,
    bidirectional=True
)
\end{lstlisting}


\paragraph{完整的双向RNN模型}
\begin{lstlisting}
class BiRNNModel(nn.Module):
    """RNNmodel"""
    
    def __init__(self, vocab_size, num_hiddens, rnn_type='lstm'):
        super(BiRNNModel, self).__init__()
        self.vocab_size = vocab_size
        self.num_hiddens = num_hiddens
        self.rnn_type = rnn_type
        
        # RNN
        if rnn_type == 'lstm':
            self.rnn = nn.LSTM(vocab_size, num_hiddens, bidirectional=True)
        elif rnn_type == 'gru':
            self.rnn = nn.GRU(vocab_size, num_hiddens, bidirectional=True)
        else:
            self.rnn = nn.RNN(vocab_size, num_hiddens, bidirectional=True)
        
        # Outputdimension2*num_hiddens
        self.linear = nn.Linear(2 * num_hiddens, vocab_size)
    
    def forward(self, inputs, state):
        # one-hot
        X = F.one_hot(inputs.T, self.vocab_size).type(torch.float32)
        
        # RNN
        Y, state = self.rnn(X, state)
        
        # Output
        output = self.linear(Y.reshape(-1, Y.shape[-1]))
        
        return output, state
\end{lstlisting}


\paragraph{完整的双向RNN模型(续)}
\begin{lstlisting}
    def begin_state(self, batch_size, device):
        # 2
        if self.rnn_type == 'lstm':
            # LSTMReturn(H, C)
            return (torch.zeros((2, batch_size, self.num_hiddens), 
                              device=device),
                    torch.zeros((2, batch_size, self.num_hiddens), 
                              device=device))
        else:
            # RNN/GRUReturnH
            return torch.zeros((2, batch_size, self.num_hiddens), 
                             device=device)

# LSTMmodel
net = BiRNNModel(vocab_size=28, num_hiddens=256, rnn_type='lstm')

# parameters
num_params = sum(p.numel() for p in net.parameters())
print(f"parameters数量: {num_params:,}")
# parameters: 1,067,036 (2)
\end{lstlisting}


\paragraph{多层双向RNN}
\begin{lstlisting}
# 2LSTM
lstm_layer = nn.LSTM(
    input_size=vocab_size,
    hidden_size=num_hiddens,
    num_layers=2,          # 2层
    bidirectional=True,    #  comment
    dropout=0.5            #  commentdropout
)

# 2 * 2 = 4
H = torch.zeros(4, batch_size, num_hiddens)
C = torch.zeros(4, batch_size, num_hiddens)
state = (H, C)

Y, (H_new, C_new) = lstm_layer(X, state)
print(f"Outputshape: {Y.shape}")      # (5, 32, 512)
print(f"Hshape: {H_new.shape}")     # (4, 32, 256)
print(f"Cshape: {C_new.shape}")     # (4, 32, 256)
\end{lstlisting}

\begin{theorem}[注意]
隐状态第一维 = $num\_layers \times 2$(方向数)
\end{theorem}


\paragraph{双向RNN的隐状态结构}
\textbf{2层双向LSTM的隐状态排列:}

\begin{center}
\begin{tikzpicture}[scale=0.9]
% 隐状态索引
\node[draw,fill=blue!20,minimum width=2cm,minimum height=0.8cm] at (0,3) {索引0};
\node[draw,fill=blue!20,minimum width=2cm,minimum height=0.8cm] at (0,2) {索引1};
\node[draw,fill=green!20,minimum width=2cm,minimum height=0.8cm] at (0,1) {索引2};
\node[draw,fill=green!20,minimum width=2cm,minimum height=0.8cm] at (0,0) {索引3};

% 说明
\node[right,text width=6cm] at (2.5,3) {第1层前向 $\overrightarrow{\mathbf{H}}^{(1)}$};
\node[right,text width=6cm] at (2.5,2) {第1层后向 $\overleftarrow{\mathbf{H}}^{(1)}$};
\node[right,text width=6cm] at (2.5,1) {第2层前向 $\overrightarrow{\mathbf{H}}^{(2)}$};
\node[right,text width=6cm] at (2.5,0) {第2层后向 $\overleftarrow{\mathbf{H}}^{(2)}$};

% 标注
\draw[decorate,decoration={brace,amplitude=10pt}] (-0.5,3.5) -- (-0.5,1.5);
\node[left] at (-1,2.5) {第1层};

\draw[decorate,decoration={brace,amplitude=10pt}] (-0.5,1.5) -- (-0.5,-0.5);
\node[left] at (-1,0.5) {第2层};

\end{tikzpicture}
\end{center}

\begin{definition}[规律]
对于 $L$ 层双向RNN,隐状态顺序:\\
$[\overrightarrow{H}^{(1)}, \overleftarrow{H}^{(1)}, \overrightarrow{H}^{(2)}, \overleftarrow{H}^{(2)}, \ldots, \overrightarrow{H}^{(L)}, \overleftarrow{H}^{(L)}]$
\end{definition}



\subsection{应用与限制}

\subsection{9.4.4 应用与限制}

\subsection{9.4.4 应用与限制}
\begin{center}
\Large \textcolor{blue}{何时使用双向RNN?}
\end{center}


\paragraph{双向RNN的应用场景}
\begin{definition}[完美适用(批处理任务)]
\begin{enumerate}
    \item \textbf{序列标注}
    \begin{itemize}
        \item 词性标注(POS Tagging)
        \item 命名实体识别(NER)
        \item 语义角色标注
    \end{itemize}
    
    \item \textbf{文本分类}
    \begin{itemize}
        \item 情感分析
        \item 主题分类
        \item 垃圾邮件检测
    \end{itemize}
    
    \item \textbf{机器翻译编码器}
    \begin{itemize}
        \item 理解源语言句子
        \item 提供丰富的上下文表示
    \end{itemize}
    
    \item \textbf{问答系统}
    \begin{itemize}
        \item 理解问题和文档
        \item 提取答案
    \end{itemize}
\end{enumerate}
\end{definition}


\paragraph{双向RNN的限制}
\begin{theorem}[不适用(在线任务)]
\begin{enumerate}
    \item \textbf{语言模型}
    \begin{itemize}
        \item 预测下一个词时不能看未来
        \item 必须用单向RNN
    \end{itemize}
    
    \item \textbf{实时语音识别}
    \begin{itemize}
        \item 需要即时输出
        \item 不能等待整个句子
    \end{itemize}
    
    \item \textbf{在线翻译(解码器)}
    \begin{itemize}
        \item 逐词生成,看不到未来
    \end{itemize}
    
    \item \textbf{流式处理}
    \begin{itemize}
        \item 数据流式到达
        \item 无法回看
    \end{itemize}
\end{enumerate}
\end{theorem}


\paragraph{双向RNN的代价}
\begin{center}
\begin{tikzpicture}[scale=0.9]
% 优点
\node[draw,fill=green!20,rounded corners,text width=5cm] at (0,2.5) {
\textbf{优点:}\\
$\checkmark$ 性能更好\\
$\checkmark$ 更丰富的上下文\\
$\checkmark$ 适合大多数NLP任务
};

% 代价
\node[draw,fill=red!20,rounded corners,text width=5cm] at (6.5,2.5) {
\textbf{代价:}\\
$\times$ 参数翻倍\\
$\times$ 计算翻倍\\
$\times$ 内存翻倍\\
$\times$ 不能在线推理
};

% 权衡
\node[draw,fill=blue!20,text width=11cm] at (3.25,0.5) {
\textbf{权衡:}如果任务允许批处理,双向通常值得(性能提升 $>$ 计算成本)
};

\end{tikzpicture}
\end{center}


\paragraph{单向 vs 双向性能对比}
\textbf{任务:}命名实体识别(NER)

\begin{center}
\begin{tikzpicture}[scale=0.9]
% 柱状图
\draw[->,thick] (0,0) -- (0,4) node[above] {F1分数};
\draw[->,thick] (0,0) -- (8,0) node[right] {模型};

% 单向LSTM
\fill[blue!50] (0.5,0) rectangle (1.5,2.5);
\node[below] at (1,0) {\small 单向};
\node[above] at (1,2.5) {\small 75.2};

% 双向LSTM
\fill[green!50] (2.5,0) rectangle (3.5,3.2);
\node[below] at (3,0) {\small 双向};
\node[above] at (3,3.2) {\small 82.1};

% 提升标注
\draw[<->,thick,red] (1.8,2.5) -- (1.8,3.2);
\node[red,right] at (2,2.85) {\small +6.9\%};

% 单向+深度
\fill[orange!50] (4.5,0) rectangle (5.5,2.8);
\node[below,font=\tiny] at (5,0) {2层单向};
\node[above] at (5,2.8) {\small 77.5};

% 双向+深度
\fill[purple!50] (6.5,0) rectangle (7.5,3.5);
\node[below,font=\tiny] at (7,0) {2层双向};
\node[above] at (7,3.5) {\small 84.3};

\end{tikzpicture}
\end{center}

\begin{definition}[观察]
双向RNN通常比单向提升5-10个百分点
\end{definition}


\paragraph{实际应用案例}
\begin{definition}[经典应用]
\begin{enumerate}
    \item \textbf{Google BERT (2018)}
    \begin{itemize}
        \item 实际是双向Transformer(概念类似)
        \item 革命性地利用了双向信息
    \end{itemize}
    
    \item \textbf{ELMo (2018)}
    \begin{itemize}
        \item 双向LSTM语言模型
        \item 前向和后向独立训练再组合
    \end{itemize}
    
    \item \textbf{BiLSTM-CRF (NER标准架构)}
    \begin{itemize}
        \item 双向LSTM编码
        \item CRF层解码
        \item NER任务的标准选择
    \end{itemize}
    
    \item \textbf{机器翻译的编码器}
    \begin{itemize}
        \item 几乎都用双向RNN
        \item 充分理解源语言
    \end{itemize}
\end{enumerate}
\end{definition}


\paragraph{双向RNN的变体}
\begin{definition}[改进的双向架构]
\begin{enumerate}
    \item \textbf{深度双向RNN}
    \begin{itemize}
        \item 垂直堆叠多层双向RNN
        \item 每层都是双向
    \end{itemize}
    
    \item \textbf{单向+双向混合}
    \begin{itemize}
        \item 底层:双向(捕捉上下文)
        \item 顶层:单向(任务特定)
    \end{itemize}
    
    \item \textbf{不对称双向}
    \begin{itemize}
        \item 前向层:标准LSTM
        \item 后向层:简化版(节省计算)
    \end{itemize}
    
    \item \textbf{延迟后向}
    \begin{itemize}
        \item 前向实时处理
        \item 后向延迟几步(平衡实时性和效果)
    \end{itemize}
\end{enumerate}
\end{definition}


\section{9.5 机器翻译与数据集}

\subsection{9.5 机器翻译与数据集}
\begin{center}
\Large \textcolor{blue}{Machine Translation}\\
\large 序列到序列学习的经典应用
\end{center}


\subsection{9.5节 - 章节导航}
\begin{definition}[本节内容]
\begin{itemize}
    \item 9.5.1 机器翻译简介
    \item 9.5.2 数据集下载与预处理
    \item 9.5.3 词表构建
    \item 9.5.4 数据加载器
\end{itemize}
\end{definition}

\begin{theorem}[学习目标]
\begin{enumerate}
    \item 理解机器翻译的挑战
    \item 掌握双语数据集的处理
    \item 学会构建源语言和目标语言词表
    \item 实现序列对的批量加载
\end{enumerate}
\end{theorem}



