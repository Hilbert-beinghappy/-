\section{长短期记忆网络}

\section{9.2 长短期记忆网络(LSTM)}

\subsection{9.2 长短期记忆网络(LSTM)}
\begin{center}
\Large \textcolor{blue}{Long Short-Term Memory}\\
\large 最经典的门控RNN
\end{center}


\subsection{9.2节 - 章节导航}
\begin{definition}[本节内容]
\begin{itemize}
    \item 9.2.1 长短期记忆网络
    \item 9.2.2 从零开始实现
    \item 9.2.3 简洁实现
\end{itemize}
\end{definition}

\begin{theorem}[学习目标]
\begin{enumerate}
    \item 理解LSTM的三个门
    \item 掌握记忆细胞的概念
    \item 对比LSTM与GRU
    \item 学会实现LSTM
\end{enumerate}
\end{theorem}


\paragraph{为什么需要LSTM?}
\begin{center}
\begin{tikzpicture}[scale=0.9]
% 历史
\node[draw,fill=yellow!20,rounded corners,text width=10cm] at (5,3.8) {
\textbf{历史(1997年,Hochreiter \& Schmidhuber):}\\
RNN无法学习长期依赖,梯度消失严重
};

\draw[->,ultra thick] (5,3.3) -- (5,2.9);

% LSTM的创新
\node[draw,fill=blue!20,rounded corners,text width=10cm] at (5,2.3) {
\textbf{LSTM的创新:}\\
$\checkmark$ 引入\textbf{记忆细胞}(memory cell)\\
$\checkmark$ 3个门精确控制信息流\\
$\checkmark$ 梯度可以无损传播
};

\draw[->,ultra thick] (5,1.8) -- (5,1.4);

% 影响
\node[draw,fill=green!20,rounded corners,text width=10cm] at (5,0.8) {
\textbf{影响:}成为2010年代最流行的RNN架构\\
语音识别、机器翻译、文本生成...
};

\end{tikzpicture}
\end{center}



\subsection{长短期记忆网络}

\subsection{9.2.1 长短期记忆网络}

\subsection{9.2.1 长短期记忆网络}
\begin{center}
\Large \textcolor{blue}{LSTM的核心:记忆细胞}
\end{center}


\paragraph{LSTM的核心思想}
\textbf{关键创新:}分离\textbf{记忆}和\textbf{隐状态}

\begin{center}
\begin{tikzpicture}[scale=0.9]
% 标准RNN
\node[above] at (2,3) {\small 标准RNN};
\node[draw,fill=blue!20,rounded corners,minimum width=2cm,minimum height=1cm,align=center] at (2,2) {
隐状态\\
$\mathbf{h}_t$
};
\node[below,font=\small] at (2,1.2) {既存储又输出};

% LSTM
\node[above] at (7,3) {\small LSTM};
\node[draw,fill=red!20,rounded corners,minimum width=2cm,minimum height=1cm,align=center] at (7,2.5) {
记忆细胞\\
$\mathbf{C}_t$
};
\node[draw,fill=green!20,rounded corners,minimum width=2cm,minimum height=1cm,align=center] at (7,1.2) {
隐状态\\
$\mathbf{H}_t$
};

\draw[->,thick] (7,1.9) -- (7,1.6);
\node[right,font=\tiny] at (7.3,1.75) {过滤};

\node[below,font=\small] at (7,0.4) {分离存储和输出};

\end{tikzpicture}
\end{center}

\begin{definition}[双状态设计]
\begin{itemize}
    \item \textbf{记忆细胞} $\mathbf{C}_t$:长期记忆,内部状态
    \item \textbf{隐状态} $\mathbf{H}_t$:短期记忆,对外输出
\end{itemize}
\end{definition}


\paragraph{LSTM的整体结构}
\begin{center}
\begin{tikzpicture}[scale=0.85]
% 输入
\node[circle,draw,fill=green!20,minimum size=0.7cm] (x) at (0,0) {$\mathbf{X}_t$};
\node[circle,draw,fill=blue!20,minimum size=0.7cm] (h_prev) at (0,4) {$\mathbf{H}_{t-1}$};
\node[circle,draw,fill=red!20,minimum size=0.7cm] (c_prev) at (0,2.5) {$\mathbf{C}_{t-1}$};

% 三个门
\node[draw,fill=yellow!20,rounded corners,minimum width=1.5cm,minimum height=0.8cm,align=center] (forget) at (2.5,3.5) {
遗忘门\\
\tiny $\mathbf{F}_t$
};

\node[draw,fill=orange!20,rounded corners,minimum width=1.5cm,minimum height=0.8cm,align=center] (input) at (2.5,2) {
输入门\\
\tiny $\mathbf{I}_t$
};

\node[draw,fill=purple!20,rounded corners,minimum width=1.5cm,minimum height=0.8cm,align=center] (output) at (2.5,0.5) {
输出门\\
\tiny $\mathbf{O}_t$
};

% 候选
\node[draw,fill=cyan!20,rounded corners,minimum width=1.5cm,minimum height=0.8cm,align=center] (cand) at (5,2) {
候选\\
\tiny $\tilde{\mathbf{C}}_t$
};

% 记忆细胞更新
\node[circle,draw,fill=red!20,minimum size=1cm] (c_t) at (7.5,2.5) {$\mathbf{C}_t$};

% 隐状态
\node[circle,draw,fill=blue!20,minimum size=1cm] (h_t) at (10,2.5) {$\mathbf{H}_t$};

% 连接
\draw[->,thick] (x) -- (forget);
\draw[->,thick] (x) -- (input);
\draw[->,thick] (x) -- (output);
\draw[->,thick] (h_prev) -- (forget);
\draw[->,thick] (h_prev) -- (input);
\draw[->,thick] (h_prev) -- (output);
\draw[->,thick] (input) -- (cand);
\draw[->,thick] (forget) -- (c_t);
\draw[->,thick] (cand) -- (c_t);
\draw[->,thick] (c_prev) -- (c_t);
\draw[->,thick] (c_t) -- (h_t);
\draw[->,thick] (output) -- (h_t);

\node[below,text width=10cm,align=center] at (5,-0.5) {
\small 3个门控制记忆细胞的更新,输出门控制隐状态
};

\end{tikzpicture}
\end{center}


\paragraph{遗忘门(Forget Gate)}
\textbf{作用:}决定从记忆细胞中\textbf{遗忘}多少信息

\begin{definition}[公式]
$$\mathbf{F}_t = \sigma(\mathbf{X}_t\mathbf{W}_{xf} + \mathbf{H}_{t-1}\mathbf{W}_{hf} + \mathbf{b}_f)$$
\end{definition}

\textbf{特点:}
\begin{itemize}
    \item $\sigma$ 是sigmoid,输出 $[0,1]$
    \item $\mathbf{F}_t \approx 1$:保留记忆
    \item $\mathbf{F}_t \approx 0$:遗忘记忆
\end{itemize}

\begin{center}
\begin{tikzpicture}[scale=0.9]
\node[draw,fill=red!20] (c) at (0,0) {$\mathbf{C}_{t-1}$};
\node[draw,fill=yellow!20] (f) at (2.5,0) {$\times \mathbf{F}_t$};
\node[draw,fill=blue!20] (out) at (5,0) {过滤后的记忆};

\draw[->,thick] (c) -- (f);
\draw[->,thick] (f) -- (out);

\node[below,font=\small,align=center] at (2.5,-0.8) {
$\mathbf{F}_t=0$ $\Rightarrow$ 完全遗忘\\
$\mathbf{F}_t=1$ $\Rightarrow$ 完全保留
};
\end{tikzpicture}
\end{center}


\paragraph{输入门(Input Gate)}
\textbf{作用:}决定\textbf{写入}多少新信息到记忆细胞

\begin{definition}[公式]
\begin{align*}
\mathbf{I}_t &= \sigma(\mathbf{X}_t\mathbf{W}_{xi} + \mathbf{H}_{t-1}\mathbf{W}_{hi} + \mathbf{b}_i) \\
\tilde{\mathbf{C}}_t &= \tanh(\mathbf{X}_t\mathbf{W}_{xc} + \mathbf{H}_{t-1}\mathbf{W}_{hc} + \mathbf{b}_c)
\end{align*}
\end{definition}

\textbf{两步骤:}
\begin{enumerate}
    \item 输入门 $\mathbf{I}_t$:控制写入多少
    \item 候选值 $\tilde{\mathbf{C}}_t$:新的候选记忆
\end{enumerate}

\begin{center}
\begin{tikzpicture}[scale=0.9]
\node[draw,fill=cyan!20] (cand) at (0,0) {$\tilde{\mathbf{C}}_t$};
\node[draw,fill=orange!20] (gate) at (2.5,0) {$\times \mathbf{I}_t$};
\node[draw,fill=green!20] (out) at (5,0) {写入的信息};

\draw[->,thick] (cand) -- (gate);
\draw[->,thick] (gate) -- (out);

\node[below,font=\small,align=center] at (2.5,-0.8) {
$\mathbf{I}_t=1$ $\Rightarrow$ 完全写入\\
$\mathbf{I}_t=0$ $\Rightarrow$ 不写入
};
\end{tikzpicture}
\end{center}


\paragraph{记忆细胞更新}
\textbf{核心:}结合遗忘和输入

\begin{definition}[公式]
$$\mathbf{C}_t = \mathbf{F}_t \odot \mathbf{C}_{t-1} + \mathbf{I}_t \odot \tilde{\mathbf{C}}_t$$
\end{definition}

\begin{center}
\begin{tikzpicture}[scale=0.9]
% 旧记忆
\node[draw,fill=red!20] (c_old) at (0,2) {$\mathbf{C}_{t-1}$};
\node[draw,circle,fill=yellow!20] (f) at (2,2) {$\times\mathbf{F}_t$};

% 新信息
\node[draw,fill=cyan!20] (c_new) at (0,0) {$\tilde{\mathbf{C}}_t$};
\node[draw,circle,fill=orange!20] (i) at (2,0) {$\times\mathbf{I}_t$};

% 相加
\node[draw,circle,fill=green!20] (add) at (4,1) {$+$};

% 结果
\node[draw,fill=blue!20] (c_t) at (6,1) {$\mathbf{C}_t$};

\draw[->,thick] (c_old) -- (f);
\draw[->,thick] (c_new) -- (i);
\draw[->,thick] (f) -- (add);
\draw[->,thick] (i) -- (add);
\draw[->,thick] (add) -- (c_t);

\node[below,text width=6cm,align=center] at (3,-1) {
\small 部分遗忘旧记忆 + 部分接收新信息
};
\end{tikzpicture}
\end{center}

\begin{theorem}[关键]
这个加法操作让梯度可以\textbf{无损传播},缓解梯度消失!
\end{theorem}


\paragraph{输出门(Output Gate)}
\textbf{作用:}决定从记忆细胞中\textbf{读出}多少信息到隐状态

\begin{definition}[公式]
\begin{align*}
\mathbf{O}_t &= \sigma(\mathbf{X}_t\mathbf{W}_{xo} + \mathbf{H}_{t-1}\mathbf{W}_{ho} + \mathbf{b}_o) \\
\mathbf{H}_t &= \mathbf{O}_t \odot \tanh(\mathbf{C}_t)
\end{align*}
\end{definition}

\textbf{两步骤:}
\begin{enumerate}
    \item 计算输出门 $\mathbf{O}_t$
    \item 过滤记忆细胞得到隐状态
\end{enumerate}

\begin{center}
\begin{tikzpicture}[scale=0.9]
\node[draw,fill=red!20] (c) at (0,0) {$\mathbf{C}_t$};
\node[draw,circle,fill=cyan!20] (tanh) at (2,0) {$\tanh$};
\node[draw,circle,fill=purple!20] (o) at (4,0) {$\times\mathbf{O}_t$};
\node[draw,fill=blue!20] (h) at (6,0) {$\mathbf{H}_t$};

\draw[->,thick] (c) -- (tanh);
\draw[->,thick] (tanh) -- (o);
\draw[->,thick] (o) -- (h);

\node[below,font=\small] at (3,-0.8) {
先$\tanh$归一化,再用$\mathbf{O}_t$过滤
};
\end{tikzpicture}
\end{center}


\paragraph{LSTM完整公式总结}
\begin{definition}[LSTM的6个公式]
\begin{align*}
\mathbf{F}_t &= \sigma(\mathbf{X}_t\mathbf{W}_{xf} + \mathbf{H}_{t-1}\mathbf{W}_{hf} + \mathbf{b}_f) \quad \text{\textcolor{yellow!60!black}{遗忘门}}\\
\mathbf{I}_t &= \sigma(\mathbf{X}_t\mathbf{W}_{xi} + \mathbf{H}_{t-1}\mathbf{W}_{hi} + \mathbf{b}_i) \quad \text{\textcolor{orange}{输入门}}\\
\mathbf{O}_t &= \sigma(\mathbf{X}_t\mathbf{W}_{xo} + \mathbf{H}_{t-1}\mathbf{W}_{ho} + \mathbf{b}_o) \quad \text{\textcolor{purple}{输出门}}\\
\tilde{\mathbf{C}}_t &= \tanh(\mathbf{X}_t\mathbf{W}_{xc} + \mathbf{H}_{t-1}\mathbf{W}_{hc} + \mathbf{b}_c) \quad \text{\textcolor{cyan}{候选}}\\
\mathbf{C}_t &= \mathbf{F}_t \odot \mathbf{C}_{t-1} + \mathbf{I}_t \odot \tilde{\mathbf{C}}_t \quad \text{\textcolor{red}{细胞更新}}\\
\mathbf{H}_t &= \mathbf{O}_t \odot \tanh(\mathbf{C}_t) \quad \text{\textcolor{blue}{隐状态}}
\end{align*}
\end{definition}

\begin{theorem}[参数]
4组权重和偏置:遗忘门、输入门、输出门、候选
\end{theorem}


\paragraph{LSTM的工作流程}
\begin{center}
\begin{tikzpicture}[scale=0.75]
% 步骤1
\node[draw,fill=yellow!20,rounded corners,text width=11cm] at (5.5,5) {
\textbf{步骤1:}计算三个门\\
$\mathbf{F}_t = \sigma(\mathbf{X}_t\mathbf{W}_{xf} + \mathbf{H}_{t-1}\mathbf{W}_{hf} + \mathbf{b}_f)$ \quad \textcolor{yellow!60!black}{遗忘门}\\
$\mathbf{I}_t = \sigma(\mathbf{X}_t\mathbf{W}_{xi} + \mathbf{H}_{t-1}\mathbf{W}_{hi} + \mathbf{b}_i)$ \quad \textcolor{orange}{输入门}\\
$\mathbf{O}_t = \sigma(\mathbf{X}_t\mathbf{W}_{xo} + \mathbf{H}_{t-1}\mathbf{W}_{ho} + \mathbf{b}_o)$ \quad \textcolor{purple}{输出门}
};

\draw[->,thick] (5.5,4.4) -- (5.5,4);

% 步骤2
\node[draw,fill=cyan!20,rounded corners,text width=11cm] at (5.5,3.5) {
\textbf{步骤2:}计算候选记忆\\
$\tilde{\mathbf{C}}_t = \tanh(\mathbf{X}_t\mathbf{W}_{xc} + \mathbf{H}_{t-1}\mathbf{W}_{hc} + \mathbf{b}_c)$
};

\draw[->,thick] (5.5,3) -- (5.5,2.6);

% 步骤3
\node[draw,fill=red!20,rounded corners,text width=11cm] at (5.5,2.1) {
\textbf{步骤3:}更新记忆细胞\\
$\mathbf{C}_t = \mathbf{F}_t \odot \mathbf{C}_{t-1} + \mathbf{I}_t \odot \tilde{\mathbf{C}}_t$
};

\draw[->,thick] (5.5,1.6) -- (5.5,1.2);

% 步骤4
\node[draw,fill=blue!20,rounded corners,text width=11cm] at (5.5,0.7) {
\textbf{步骤4:}计算隐状态\\
$\mathbf{H}_t = \mathbf{O}_t \odot \tanh(\mathbf{C}_t)$
};

\draw[->,thick] (5.5,0.2) -- (5.5,-0.2);

% 输出
\node[draw,fill=green!20,rounded corners,text width=11cm] at (5.5,-0.7) {
\textbf{输出:}新的记忆细胞 $\mathbf{C}_t$ 和隐状态 $\mathbf{H}_t$
};

\end{tikzpicture}
\end{center}


\paragraph{LSTM vs GRU}
\begin{center}
\begin{tabular}{|l|c|c|}
\hline
\textbf{特性} & \textbf{GRU} & \textbf{LSTM} \\
\hline
门的数量 & 2个 & 3个 \\
\hline
状态数量 & 1个($\mathbf{H}_t$) & 2个($\mathbf{C}_t, \mathbf{H}_t$) \\
\hline
参数量 & 较少 & 较多(约1.3倍) \\
\hline
计算复杂度 & 较低 & 较高 \\
\hline
训练速度 & 较快 & 较慢 \\
\hline
表达能力 & 强 & 更强 \\
\hline
应用广泛度 & 中等 & 很广泛 \\
\hline
\end{tabular}
\end{center}

\vspace{0.5cm}
\begin{definition}[选择建议]
\begin{itemize}
    \item 首选LSTM(更成熟,应用更多)
    \item 追求速度用GRU
    \item 两者性能通常接近
\end{itemize}
\end{definition}


\paragraph{LSTM如何缓解梯度消失?}
\textbf{关键:}记忆细胞的加法更新

\begin{definition}[记忆细胞梯度]
$$\frac{\partial\mathbf{C}_t}{\partial\mathbf{C}_{t-1}} = \mathbf{F}_t$$

当 $\mathbf{F}_t \approx 1$ 时:
$$\mathbf{C}_t \approx \mathbf{C}_{t-1} + \mathbf{I}_t \odot \tilde{\mathbf{C}}_t$$
\end{definition}

\begin{center}
\begin{tikzpicture}[scale=0.9]
% 梯度流
\foreach \t in {1,2,3,4} {
    \node[circle,draw,fill=red!20,minimum size=0.8cm] (c\t) at (\t*2,2) {$\mathbf{C}_{\t}$};
}

% 加法连接(梯度高速公路)
\draw[->,ultra thick,blue] (c4) -- (c3) node[midway,above,font=\tiny] {$\mathbf{F}_t \approx 1$};
\draw[->,ultra thick,blue] (c3) -- (c2);
\draw[->,ultra thick,blue] (c2) -- (c1);

\node[below,text width=8cm,align=center] at (4,0.8) {
梯度沿着记忆细胞的\textbf{加法路径}传播\\
避免了连乘,缓解梯度消失!
};

\end{tikzpicture}
\end{center}

\begin{theorem}[对比RNN]
RNN:梯度连乘 $\prod \mathbf{W}_{hh}$ $\Rightarrow$ 消失/爆炸\\
LSTM:梯度加法 $\mathbf{F}_t$ $\Rightarrow$ 稳定传播
\end{theorem}


\paragraph{三个门的直观理解}
\begin{center}
\begin{tikzpicture}[scale=0.85]
% 遗忘门
\node[draw,fill=yellow!20,rounded corners,text width=3.5cm] at (0,3) {
\textbf{遗忘门 $\mathbf{F}_t$}\\
问:要忘记什么?\\
\\
$\mathbf{F}_t \approx 1$:记住\\
$\mathbf{F}_t \approx 0$:遗忘
};

% 输入门
\node[draw,fill=orange!20,rounded corners,text width=3.5cm] at (5,3) {
\textbf{输入门 $\mathbf{I}_t$}\\
问:要记住什么?\\
\\
$\mathbf{I}_t \approx 1$:接收\\
$\mathbf{I}_t \approx 0$:忽略
};

% 输出门
\node[draw,fill=purple!20,rounded corners,text width=3.5cm] at (10,3) {
\textbf{输出门 $\mathbf{O}_t$}\\
问:要输出什么?\\
\\
$\mathbf{O}_t \approx 1$:全部\\
$\mathbf{O}_t \approx 0$:不输出
};

% 例子
\node[draw,fill=blue!20,text width=13cm,align=left] at (5,0.8) {
\textbf{例子:}``Alice is in Paris. She speaks fluent \_\_\_''\\
预测空格时:\\
- 遗忘门:忘记``Alice'',保留``Paris''($\mathbf{F}_t$对不同位置不同)\\
- 输入门:接收当前词``speaks''的信息\\
- 输出门:输出与语言相关的特征
};

\end{tikzpicture}
\end{center}



\subsection{从零开始实现}

\subsection{9.2.2 从零开始实现}

\subsection{9.2.2 从零开始实现}
\begin{center}
\Large \textcolor{blue}{手写LSTM的每个细节}
\end{center}


\paragraph{初始化LSTM参数}
\begin{lstlisting}
def get_lstm_params(vocab_size, num_hiddens, device):
    """ LSTMparameters LSTM4parameters:、、Output、 """
    num_inputs = num_outputs = vocab_size
    
    def normal(shape):
        return torch.randn(size=shape, device=device) * 0.01
    
    def three():
        """每个门/候选需要3个parameters"""
        return (normal((num_inputs, num_hiddens)),
                normal((num_hiddens, num_hiddens)),
                torch.zeros(num_hiddens, device=device))
    
    # parameters
    W_xf, W_hf, b_f = three()
    # parameters
    W_xi, W_hi, b_i = three()
    # Outputparameters
    W_xo, W_ho, b_o = three()
    # parameters
    W_xc, W_hc, b_c = three()
    
    # Outputparameters
    W_hq = normal((num_hiddens, num_outputs))
    b_q = torch.zeros(num_outputs, device=device)
\end{lstlisting}

\paragraph{初始化LSTM参数(续)}
\begin{lstlisting}
    # parameters
    params = [W_xf, W_hf, b_f,  #  comment
              W_xi, W_hi, b_i,  #  comment
              W_xo, W_ho, b_o,  # Output门
              W_xc, W_hc, b_c,  #  comment
              W_hq, b_q]        # Output
    
    for param in params:
        param.requires_grad_(True)
    
    return params

vocab_size, num_hiddens = 28, 512
device = torch.device('cuda' if torch.cuda.is_available() else 'cpu')
params = get_lstm_params(vocab_size, num_hiddens, device)

print(f"parameters数量: {sum(p.numel() for p in params)}")
# parameters: 1,165,340 (RNN4GRU1.3)
\end{lstlisting}

\begin{theorem}[观察]
LSTM参数最多,但效果最好(通常)
\end{theorem}


\paragraph{初始化LSTM状态}
\begin{lstlisting}
def init_lstm_state(batch_size, num_hiddens, device):
    """ LSTM LSTM:CH """
    return (torch.zeros((batch_size, num_hiddens), device=device),  # H
            torch.zeros((batch_size, num_hiddens), device=device))  # C

batch_size = 32
state = init_lstm_state(batch_size, num_hiddens, device)
H, C = state

print(f"隐状态Hshape: {H.shape}")  # (32, 512)
print(f"记忆细胞Cshape: {C.shape}")  # (32, 512)
\end{lstlisting}

\begin{definition}[双状态]
\begin{itemize}
    \item $\mathbf{H}$:隐状态,对外输出
    \item $\mathbf{C}$:记忆细胞,内部状态
\end{itemize}
\end{definition}


\paragraph{LSTM前向传播}
\begin{lstlisting}
def lstm(inputs, state, params):
    """ LSTM inputs: (num_steps, batch_size, vocab_size) state: (H, C) params: LSTMparameters """
    [W_xf, W_hf, b_f, W_xi, W_hi, b_i, W_xo, W_ho, b_o,
     W_xc, W_hc, b_c, W_hq, b_q] = params
    
    (H, C) = state  # comment
    outputs = []
    
    for X in inputs:  # commenttime step
        F = torch.sigmoid(torch.mm(X, W_xf) + torch.mm(H, W_hf) + b_f)
        
        I = torch.sigmoid(torch.mm(X, W_xi) + torch.mm(H, W_hi) + b_i)
        
        # Output
        O = torch.sigmoid(torch.mm(X, W_xo) + torch.mm(H, W_ho) + b_o)
        
        C_tilde = torch.tanh(torch.mm(X, W_xc) + 
                            torch.mm(H, W_hc) + b_c)
        
        C = F * C + I * C_tilde
        
        H = O * torch.tanh(C)
\end{lstlisting}


\paragraph{LSTM前向传播(续)}
\begin{lstlisting}
        # Output
        Y = torch.mm(H, W_hq) + b_q
        outputs.append(Y)
    
    return torch.cat(outputs, dim=0), (H, C)

# test
vocab_size, num_hiddens = 28, 512
batch_size, num_steps = 2, 5

X = torch.randn(num_steps, batch_size, vocab_size)
state = init_lstm_state(batch_size, num_hiddens, device)

Y, (H_new, C_new) = lstm(X, state, params)

print(f"输入shape: {X.shape}")      # (5, 2, 28)
print(f"Outputshape: {Y.shape}")      # (10, 28)
print(f"新隐状态H: {H_new.shape}") # (2, 512)
print(f"新记忆C: {C_new.shape}")   # (2, 512)
\end{lstlisting}

\begin{definition}[验证]
所有形状正确,可以进行训练!
\end{definition}


\paragraph{创建LSTM模型类}
\begin{lstlisting}
class LSTMModelScratch:
    """LSTMmodel"""
    
    def __init__(self, vocab_size, num_hiddens, device,
                 get_params, init_state, forward_fn):
        self.vocab_size = vocab_size
        self.num_hiddens = num_hiddens
        self.params = get_params(vocab_size, num_hiddens, device)
        self.init_state = init_state
        self.forward_fn = forward_fn
    
    def __call__(self, X, state):
        X = F.one_hot(X.T, self.vocab_size).type(torch.float32)
        return self.forward_fn(X, state, self.params)
    
    def begin_state(self, batch_size, device):
        return self.init_state(batch_size, self.num_hiddens, device)

# model
net = LSTMModelScratch(vocab_size, num_hiddens, device,
                       get_lstm_params, init_lstm_state, lstm)
\end{lstlisting}


\paragraph{训练LSTM模型}
\begin{lstlisting}
# parameters
num_epochs = 500
lr = 1
batch_size = 32
num_steps = 35

# data
train_iter, vocab = load_data_time_machine(batch_size, num_steps)

# training8training
train_ch8(net, train_iter, vocab, lr, num_epochs, device)
\end{lstlisting}

\textbf{训练输出:}
\begin{verbatim}
困惑度 1.0, 9800 词元/秒 on cpu
time traveller smiled round at us then still smiling
traveller for so it will be convenient to speak of him
\end{verbatim}

\begin{definition}[观察]
\begin{itemize}
    \item 困惑度更低(1.0 vs GRU的1.1 vs RNN的1.2)
    \item 生成质量最好
    \item 训练最慢(参数最多)
\end{itemize}
\end{definition}



\subsection{简洁实现}

\subsection{9.2.3 简洁实现}

\subsection{9.2.3 简洁实现}
\begin{center}
\Large \textcolor{blue}{使用PyTorch的\texttt{nn.LSTM}}
\end{center}


\paragraph{使用nn.LSTM}
\begin{lstlisting}
import torch.nn as nn

# LSTM
lstm_layer = nn.LSTM(
    input_size=vocab_size,   #  commentdimension
    hidden_size=num_hiddens, #  comment
    num_layers=1             # LSTM层数
)

print(lstm_layer)
# LSTM(28, 512)

# test
X = torch.randn(num_steps, batch_size, vocab_size)
H = torch.zeros(1, batch_size, num_hiddens)
C = torch.zeros(1, batch_size, num_hiddens)

Y, (H_new, C_new) = lstm_layer(X, (H, C))

print(f"输入shape: {X.shape}")      # (5, 2, 28)
print(f"Outputshape: {Y.shape}")      # (5, 2, 512)
print(f"新隐状态H: {H_new.shape}") # (1, 2, 512)
print(f"新记忆C: {C_new.shape}")   # (1, 2, 512)
\end{lstlisting}


\paragraph{完整的LSTM模型}
\begin{lstlisting}
class LSTMModel(nn.Module):
    """LSTMmodel()"""
    
    def __init__(self, lstm_layer, vocab_size):
        super(LSTMModel, self).__init__()
        self.lstm = lstm_layer
        self.vocab_size = vocab_size
        self.num_hiddens = self.lstm.hidden_size
        
        # Output
        self.linear = nn.Linear(self.num_hiddens, self.vocab_size)
    
    def forward(self, inputs, state):
        # one-hot
        X = F.one_hot(inputs.T, self.vocab_size).type(torch.float32)
        
        # LSTM
        Y, state = self.lstm(X, state)
        
        output = self.linear(Y.reshape(-1, Y.shape[-1]))
        
        return output, state
    
    def begin_state(self, batch_size, device):
        # LSTMReturn(H, C)
        return (torch.zeros((self.lstm.num_layers, batch_size, 
                           self.num_hiddens), device=device),
                torch.zeros((self.lstm.num_layers, batch_size, 
                           self.num_hiddens), device=device))
\end{lstlisting}


\paragraph{创建和训练模型}
\begin{lstlisting}
# parameters
vocab_size = len(vocab)
num_hiddens = 512
device = torch.device('cuda' if torch.cuda.is_available() else 'cpu')

# LSTM
lstm_layer = nn.LSTM(vocab_size, num_hiddens)

# model
net = LSTMModel(lstm_layer, vocab_size)
net = net.to(device)

# training
num_epochs = 500
lr = 1
train_ch8(net, train_iter, vocab, lr, num_epochs, device)
\end{lstlisting}

\textbf{输出:}
\begin{verbatim}
困惑度 1.0, 35000 词元/秒 on cuda:0
time traveller smiled round at us then still smiling
traveller for so it will be convenient to speak of him
\end{verbatim}


\paragraph{三种RNN架构对比}
\begin{center}
\begin{tabular}{|l|c|c|c|}
\hline
\textbf{特性} & \textbf{RNN} & \textbf{GRU} & \textbf{LSTM} \\
\hline
门的数量 & 0 & 2 & 3 \\
\hline
状态数量 & 1 & 1 & 2 \\
\hline
参数量(相对) & 1$\times$ & 3$\times$ & 4$\times$ \\
\hline
计算复杂度 & 低 & 中 & 高 \\
\hline
训练速度 & 快 & 中 & 慢 \\
\hline
梯度消失 & 严重 & 缓解 & 最好 \\
\hline
长期依赖 & 差 & 好 & 最好 \\
\hline
困惑度(字符级) & 1.2 & 1.1 & 1.0 \\
\hline
应用广泛度 & 少 & 中 & 广 \\
\hline
\end{tabular}
\end{center}

\vspace{0.5cm}
\begin{theorem}[权衡]
LSTM:最强但最慢 | GRU:平衡 | RNN:简单但弱
\end{theorem}


\paragraph{LSTM的变体}
\begin{definition}[常见LSTM变体]
\begin{enumerate}
    \item \textbf{标准LSTM}(本节实现)
    \begin{itemize}
        \item 3个门,独立的记忆细胞
    \end{itemize}
    
    \item \textbf{Peephole LSTM}
    \begin{itemize}
        \item 门可以"窥视"记忆细胞
        \item $\mathbf{F}_t = \sigma(\mathbf{X}_t\mathbf{W}_{xf} + \mathbf{H}_{t-1}\mathbf{W}_{hf} + \mathbf{C}_{t-1} \odot \mathbf{W}_{cf})$
    \end{itemize}
    
    \item \textbf{耦合遗忘-输入门}
    \begin{itemize}
        \item $\mathbf{I}_t = 1 - \mathbf{F}_t$(遗忘和输入互补)
    \end{itemize}
    
    \item \textbf{No Output Gate}
    \begin{itemize}
        \item 去掉输出门,直接 $\mathbf{H}_t = \mathbf{C}_t$
    \end{itemize}
\end{enumerate}
\end{definition}


\paragraph{性能对比可视化}
\begin{center}
\begin{tikzpicture}[scale=0.9]
% 坐标轴
\draw[->,thick] (0,0) -- (8,0) node[right] {训练epoch};
\draw[->,thick] (0,0) -- (0,4) node[above] {困惑度};

% RNN曲线
\draw[red,thick] (0.5,3.5) -- (1,2.8) -- (2,2.0) -- (3,1.5) -- (7,1.2);
\node[red,right] at (7,1.2) {RNN};

% GRU曲线
\draw[blue,thick] (0.5,3.5) -- (1,2.5) -- (2,1.6) -- (3,1.2) -- (7,1.05);
\node[blue,right] at (7,1.05) {GRU};

% LSTM曲线
\draw[green!60!black,thick] (0.5,3.5) -- (1,2.3) -- (2,1.5) -- (3,1.1) -- (7,1.0);
\node[green!60!black,right] at (7,1.0) {LSTM};

% 参考线
\draw[dashed] (0,1) -- (8,1);
\node[left,font=\tiny] at (0,1) {PPL=1};

% 标注
\fill[green!60!black] (7,1.0) circle (2pt);
\node[below,text width=8cm,align=center] at (4,-0.8) {
\small LSTM收敛最快,最终困惑度最低
};

\end{tikzpicture}
\end{center}


\paragraph{LSTM的内存和计算开销}
\begin{definition}[内存占用(每个时间步)]
\begin{itemize}
    \item 前向传播:存储 $\mathbf{H}_t, \mathbf{C}_t, \mathbf{F}_t, \mathbf{I}_t, \mathbf{O}_t, \tilde{\mathbf{C}}_t$
    \item 内存:$\sim 6 \times (batch\_size \times num\_hiddens)$
    \item 比RNN多约3倍
\end{itemize}
\end{definition}

\begin{definition}[计算开销(每个时间步)]
\begin{itemize}
    \item 矩阵乘法:8次(4个门,每个门2次)
    \item 激活函数:3次sigmoid + 2次tanh
    \item 逐元素操作:多次
    \item FLOPs:$\sim 8 \times (d \times h + h^2)$
\end{itemize}
\end{definition}

\begin{theorem}[实践]
使用cuDNN优化的LSTM(如PyTorch)可大幅加速
\end{theorem}


\paragraph{LSTM的实际应用}
\begin{definition}[成功应用案例]
\begin{enumerate}
    \item \textbf{机器翻译}(Google Translate 2016)
    \begin{itemize}
        \item LSTM seq2seq架构
    \end{itemize}
    
    \item \textbf{语音识别}(Google Voice Search)
    \begin{itemize}
        \item 双向LSTM + CTC
    \end{itemize}
    
    \item \textbf{图像描述生成}
    \begin{itemize}
        \item CNN提取特征 + LSTM生成描述
    \end{itemize}
    
    \item \textbf{文本生成}
    \begin{itemize}
        \item 字符级LSTM生成莎士比亚风格文本
    \end{itemize}
    
    \item \textbf{情感分析}
    \begin{itemize}
        \item LSTM捕捉长距离依赖
    \end{itemize}
\end{enumerate}
\end{definition}


\subsection{9.2节总结}
\begin{definition}[核心内容]
\begin{enumerate}
    \item \textbf{LSTM动机}:解决RNN梯度消失
    
    \item \textbf{三个门}:
    \begin{itemize}
        \item 遗忘门 $\mathbf{F}_t$:控制遗忘
        \item 输入门 $\mathbf{I}_t$:控制写入
        \item 输出门 $\mathbf{O}_t$:控制输出
    \end{itemize}
    
    \item \textbf{记忆细胞}:
    $$\mathbf{C}_t = \mathbf{F}_t \odot \mathbf{C}_{t-1} + \mathbf{I}_t \odot \tilde{\mathbf{C}}_t$$
    
    \item \textbf{优势}:
    \begin{itemize}
        \item 梯度高速公路
        \item 长期依赖最好
        \item 应用最广泛
    \end{itemize}
\end{enumerate}
\end{definition}


\paragraph{LSTM完整公式回顾}
\begin{center}
\begin{tikzpicture}[scale=0.8]
% 公式框
\node[draw,fill=yellow!20,text width=11cm,rounded corners,font=\small] at (5.5,4.2) {
$\mathbf{F}_t = \sigma(\mathbf{X}_t\mathbf{W}_{xf} + \mathbf{H}_{t-1}\mathbf{W}_{hf} + \mathbf{b}_f)$ \quad \textcolor{yellow!60!black}{遗忘门}
};

\node[draw,fill=orange!20,text width=11cm,rounded corners,font=\small] at (5.5,3.3) {
$\mathbf{I}_t = \sigma(\mathbf{X}_t\mathbf{W}_{xi} + \mathbf{H}_{t-1}\mathbf{W}_{hi} + \mathbf{b}_i)$ \quad \textcolor{orange}{输入门}
};

\node[draw,fill=purple!20,text width=11cm,rounded corners,font=\small] at (5.5,2.4) {
$\mathbf{O}_t = \sigma(\mathbf{X}_t\mathbf{W}_{xo} + \mathbf{H}_{t-1}\mathbf{W}_{ho} + \mathbf{b}_o)$ \quad \textcolor{purple}{输出门}
};

\node[draw,fill=cyan!20,text width=11cm,rounded corners,font=\small] at (5.5,1.5) {
$\tilde{\mathbf{C}}_t = \tanh(\mathbf{X}_t\mathbf{W}_{xc} + \mathbf{H}_{t-1}\mathbf{W}_{hc} + \mathbf{b}_c)$ \quad \textcolor{cyan!60!black}{候选}
};

\node[draw,fill=red!20,text width=11cm,rounded corners,font=\small] at (5.5,0.6) {
$\mathbf{C}_t = \mathbf{F}_t \odot \mathbf{C}_{t-1} + \mathbf{I}_t \odot \tilde{\mathbf{C}}_t$ \quad \textcolor{red}{细胞更新}
};

\node[draw,fill=blue!20,text width=11cm,rounded corners,font=\small] at (5.5,-0.3) {
$\mathbf{H}_t = \mathbf{O}_t \odot \tanh(\mathbf{C}_t)$ \quad \textcolor{blue}{隐状态}
};

\end{tikzpicture}
\end{center}


\paragraph{实践建议}
\begin{definition}[何时使用LSTM]
\textbf{推荐使用LSTM的场景:}
\begin{itemize}
    \item 需要捕捉长期依赖($>100$步)
    \item 任务对准确度要求高
    \item 计算资源充足
    \item 经典任务(翻译、语音识别等)
\end{itemize}
\end{definition}

\begin{definition}[超参数建议]
\begin{itemize}
    \item 隐藏单元数:256-1024
    \item 层数:1-4层(多层LSTM)
    \item Dropout:0.2-0.5
    \item 学习率:0.001-0.01
    \item 梯度裁剪:必须(阈值1-5)
\end{itemize}
\end{definition}


\paragraph{GRU vs LSTM:如何选择?}
\begin{center}
\begin{tikzpicture}[scale=0.9]
% 决策树
\node[draw,fill=yellow!20,rounded corners] (start) at (5,4) {
需要RNN架构
};

\node[draw,fill=blue!20,rounded corners] (q1) at (5,3) {
需要最佳性能?
};

\node[draw,fill=green!20,rounded corners] (lstm) at (2,1.5) {
选择LSTM
};

\node[draw,fill=orange!20,rounded corners] (q2) at (8,1.5) {
追求速度?
};

\node[draw,fill=purple!20,rounded corners] (gru) at (8,0) {
选择GRU
};

% 连接
\draw[->,thick] (start) -- (q1);
\draw[->,thick] (q1) -- (lstm) node[midway,above,sloped] {\tiny 是};
\draw[->,thick] (q1) -- (q2) node[midway,above,sloped] {\tiny 否};
\draw[->,thick] (q2) -- (gru) node[midway,right] {\tiny 是};
\draw[->,thick] (q2) -- (10,1.5) node[right] {\tiny 否,两者都试试};

\end{tikzpicture}
\end{center}

\begin{theorem}[经验法则]
默认选LSTM,如果训练太慢再换GRU
\end{theorem}


\paragraph{准备进入9.3节}
\begin{center}
\Large \textcolor{blue}{深度循环神经网络}
\end{center}

\vspace{0.5cm}
\begin{definition}[9.2节学到了]
\begin{itemize}
    \item $\checkmark$ LSTM的三个门和记忆细胞
    \item $\checkmark$ 梯度高速公路原理
    \item $\checkmark$ 实现LSTM模型
\end{itemize}
\end{definition}

\begin{definition}[9.3节预告]
\begin{itemize}
    \item 为什么要堆叠多层RNN
    \item 深度RNN的架构
    \item 如何实现多层LSTM
    \item 深度vs宽度的权衡
\end{itemize}
\end{definition}


\section{9.3 深度循环神经网络}

\subsection{9.3 深度循环神经网络}
\begin{center}
\Large \textcolor{blue}{Deep Recurrent Neural Networks}\\
\large 堆叠多层RNN
\end{center}


\paragraph{为什么需要深度RNN?}
\begin{center}
\begin{tikzpicture}[scale=0.9]
% 浅层网络
\node[draw,fill=yellow!20,rounded corners,text width=10cm] at (5,3.8) {
\textbf{单层RNN的局限:}\\
$\times$ 表达能力有限\\
$\times$ 特征层次单一\\
$\times$ 复杂模式难以捕捉
};

\draw[->,ultra thick] (5,3.3) -- (5,2.9);
\node[right] at (5.5,3.1) {如何提升?};

% 深度网络的优势
\node[draw,fill=green!20,rounded corners,text width=10cm] at (5,2.3) {
\textbf{深度RNN的优势:}\\
$\checkmark$ 增强表达能力\\
$\checkmark$ 层次化特征学习\\
$\checkmark$ 更好的性能(通常)
};

\draw[->,ultra thick] (5,1.8) -- (5,1.4);

% 结果
\node[draw,fill=blue!20,rounded corners,text width=10cm] at (5,0.8) {
\textbf{深度 = 垂直堆叠多层RNN}
};

\end{tikzpicture}
\end{center}


\paragraph{深度RNN的概念}
\textbf{核心思想:}将多个RNN层垂直堆叠

\begin{center}
\begin{tikzpicture}[scale=0.85]
% 单层RNN
\node[above] at (2,4) {\small 单层RNN};
\foreach \t in {1,2,3} {
    \node[circle,draw,fill=blue!20,minimum size=0.7cm] (h\t) at (\t*1.2,3) {\tiny$\mathbf{h}_{\t}$};
    \node[circle,draw,fill=green!20,minimum size=0.6cm] (x\t) at (\t*1.2,2) {\tiny$\mathbf{x}_{\t}$};
    \draw[->,thick] (x\t) -- (h\t);
}
\draw[->,thick] (h1) -- (h2);
\draw[->,thick] (h2) -- (h3);

% 深度RNN
\node[above] at (7,4) {\small 深度RNN(3层)};
% 第1层
\foreach \t in {1,2,3} {
    \node[circle,draw,fill=red!20,minimum size=0.6cm] (l1\t) at (5+\t*0.8,1) {\tiny$\mathbf{h}^{(1)}_{\t}$};
    \node[circle,draw,fill=green!20,minimum size=0.5cm] (x2\t) at (5+\t*0.8,0) {\tiny$\mathbf{x}_{\t}$};
    \draw[->,thick] (x2\t) -- (l1\t);
}
\draw[->,thick] (l11) -- (l12);
\draw[->,thick] (l12) -- (l13);

% 第2层
\foreach \t in {1,2,3} {
    \node[circle,draw,fill=orange!20,minimum size=0.6cm] (l2\t) at (5+\t*0.8,2) {\tiny$\mathbf{h}^{(2)}_{\t}$};
    \draw[->,thick] (l1\t) -- (l2\t);
}
\draw[->,thick] (l21) -- (l22);
\draw[->,thick] (l22) -- (l23);

% 第3层
\foreach \t in {1,2,3} {
    \node[circle,draw,fill=purple!20,minimum size=0.6cm] (l3\t) at (5+\t*0.8,3) {\tiny$\mathbf{h}^{(3)}_{\t}$};
    \draw[->,thick] (l2\t) -- (l3\t);
}
\draw[->,thick] (l31) -- (l32);
\draw[->,thick] (l32) -- (l33);

\node[below,text width=8cm,align=center] at (5,-0.8) {
\small 每层的输出作为下一层的输入
};

\end{tikzpicture}
\end{center}


\paragraph{深度RNN的数学表示}
\textbf{$L$层深度RNN:}

\begin{definition}[递推公式]
对于第$l$层($l=1,\ldots,L$),时间步$t$:
$$\mathbf{H}_t^{(l)} = \phi(\mathbf{H}_t^{(l-1)}\mathbf{W}_{xh}^{(l)} + \mathbf{H}_{t-1}^{(l)}\mathbf{W}_{hh}^{(l)} + \mathbf{b}_h^{(l)})$$

其中:
\begin{itemize}
    \item $\mathbf{H}_t^{(0)} = \mathbf{X}_t$ (输入)
    \item $\mathbf{H}_t^{(l)}$:第$l$层在时间$t$的隐状态
    \item $\phi$:激活函数(如$\tanh$)
\end{itemize}
\end{definition}

\textbf{输出层:}
$$\mathbf{O}_t = \mathbf{H}_t^{(L)}\mathbf{W}_{hq} + \mathbf{b}_q$$

\begin{theorem}[关键]
每层有自己的参数 $\mathbf{W}_{xh}^{(l)}, \mathbf{W}_{hh}^{(l)}, \mathbf{b}_h^{(l)}$
\end{theorem}


\paragraph{深度RNN的信息流}
\begin{center}
\begin{tikzpicture}[scale=0.85]
% 3层RNN,3个时间步
% 第1层
\foreach \t in {1,2,3} {
    \node[circle,draw,fill=red!20,minimum size=0.8cm] (l1\t) at (\t*2.5,0.5) {\tiny$h^{(1)}_{\t}$};
}

% 第2层
\foreach \t in {1,2,3} {
    \node[circle,draw,fill=orange!20,minimum size=0.8cm] (l2\t) at (\t*2.5,2) {\tiny$h^{(2)}_{\t}$};
}

% 第3层
\foreach \t in {1,2,3} {
    \node[circle,draw,fill=purple!20,minimum size=0.8cm] (l3\t) at (\t*2.5,3.5) {\tiny$h^{(3)}_{\t}$};
}

% 水平连接(时间)
\draw[->,thick,blue] (l11) -- (l12) node[midway,above,font=\tiny] {时间};
\draw[->,thick,blue] (l12) -- (l13);
\draw[->,thick,blue] (l21) -- (l22);
\draw[->,thick,blue] (l22) -- (l23);
\draw[->,thick,blue] (l31) -- (l32);
\draw[->,thick,blue] (l32) -- (l33);

% 垂直连接(层)
\foreach \t in {1,2,3} {
    \draw[->,thick,red] (l1\t) -- (l2\t);
    \draw[->,thick,red] (l2\t) -- (l3\t);
}
\node[right,red] at (8,2) {层};

% 标注
\node[left] at (2,0.5) {第1层};
\node[left] at (2,2) {第2层};
\node[left] at (2,3.5) {第3层};

\node[below,text width=7cm,align=center] at (4,-0.5) {
\small 信息在\textbf{时间}和\textbf{层}两个维度上传播
};

\end{tikzpicture}
\end{center}


\paragraph{深度RNN的架构设计}
\begin{definition}[常见配置]
\begin{enumerate}
    \item \textbf{层数}:通常2-4层
    \begin{itemize}
        \item 1层:单层(baseline)
        \item 2层:常用配置
        \item 3-4层:复杂任务
        \item $>4$层:很少用(易过拟合)
    \end{itemize}
    
    \item \textbf{每层隐藏单元数}:
    \begin{itemize}
        \item 可以相同:如 [512, 512, 512]
        \item 可以递减:如 [1024, 512, 256]
        \item 可以递增:如 [256, 512, 1024]
    \end{itemize}
    
    \item \textbf{层间连接}:
    \begin{itemize}
        \item 标准:上一层输出 $\rightarrow$ 下一层输入
        \item 残差:加上跳跃连接(ResNet思想)
    \end{itemize}
\end{enumerate}
\end{definition}


\paragraph{深度RNN vs 单层RNN}
\begin{center}
\begin{tabular}{|l|c|c|}
\hline
\textbf{特性} & \textbf{单层RNN} & \textbf{深度RNN} \\
\hline
表达能力 & 弱 & 强 \\
\hline
参数数量 & 少 & 多(层数倍) \\
\hline
训练难度 & 简单 & 较难 \\
\hline
过拟合风险 & 低 & 高 \\
\hline
计算速度 & 快 & 慢 \\
\hline
需要数据量 & 少 & 多 \\
\hline
性能上限 & 低 & 高 \\
\hline
\end{tabular}
\end{center}

\vspace{0.5cm}
\begin{theorem}[权衡]
\begin{itemize}
    \item 数据少、任务简单 $\rightarrow$ 单层
    \item 数据多、任务复杂 $\rightarrow$ 深度(2-3层)
\end{itemize}
\end{theorem}


\paragraph{实现深度RNN - 从零开始}
\begin{lstlisting}
def deep_rnn(inputs, states, params, num_layers):
    """ RNN inputs: (num_steps, batch_size, input_size) states: [(H1,), (H2,), ...] params: parameters num_layers: """
    outputs = []
    
    for X in inputs:  # commenttime step
        H_input = X
        
        for l in range(num_layers):
            # lparameters
            W_xh, W_hh, b_h = params[l]
            H, = states[l]
            
            # l
            H = torch.tanh(torch.mm(H_input, W_xh) + 
                          torch.mm(H, W_hh) + b_h)
            
            states[l] = (H,)
            
            # Output
            H_input = H
        
        outputs.append(H_input)  # commentOutput
    
    return torch.cat(outputs, dim=0), states
\end{lstlisting}


\paragraph{使用PyTorch实现深度RNN}
\begin{lstlisting}
import torch.nn as nn

# 2RNN
rnn_layer = nn.RNN(
    input_size=vocab_size,
    hidden_size=num_hiddens,
    num_layers=2  #  comment
)

print(rnn_layer)
# RNN(28, 512, num_layers=2)

# test
X = torch.randn(num_steps, batch_size, vocab_size)
H = torch.zeros(2, batch_size, num_hiddens)  #  comment(num_layers, batch, hidden)

Y, H_new = rnn_layer(X, H)

print(f"输入shape: {X.shape}")      # (5, 2, 28)
print(f"Outputshape: {Y.shape}")      # (5, 2, 512)
print(f"隐状态shape: {H_new.shape}") # (2, 2, 512)
\end{lstlisting}

\begin{theorem}[注意]
隐状态第一维是\textbf{层数}:$(num\_layers, batch\_size, hidden\_size)$
\end{theorem}


\paragraph{深度LSTM/GRU}
\begin{lstlisting}
# 3LSTM
lstm_layer = nn.LSTM(
    input_size=vocab_size,
    hidden_size=num_hiddens,
    num_layers=3  # 3层
)

# 3GRU
gru_layer = nn.GRU(
    input_size=vocab_size,
    hidden_size=num_hiddens,
    num_layers=3  # 3层
)

# LSTMHC
H = torch.zeros(3, batch_size, num_hiddens)
C = torch.zeros(3, batch_size, num_hiddens)
state = (H, C)

# GRUH
H = torch.zeros(3, batch_size, num_hiddens)
state = H
\end{lstlisting}


\paragraph{完整的深度RNN模型}
\begin{lstlisting}
class DeepRNNModel(nn.Module):
    """RNNmodel"""
    
    def __init__(self, vocab_size, num_hiddens, num_layers, rnn_type='lstm'):
        super(DeepRNNModel, self).__init__()
        self.vocab_size = vocab_size
        self.num_hiddens = num_hiddens
        self.num_layers = num_layers
        self.rnn_type = rnn_type
        
        # RNN
        if rnn_type == 'lstm':
            self.rnn = nn.LSTM(vocab_size, num_hiddens, num_layers)
        elif rnn_type == 'gru':
            self.rnn = nn.GRU(vocab_size, num_hiddens, num_layers)
        else:
            self.rnn = nn.RNN(vocab_size, num_hiddens, num_layers)
        
        # Output
        self.linear = nn.Linear(num_hiddens, vocab_size)
    
    def forward(self, inputs, state):
        # one-hot
        X = F.one_hot(inputs.T, self.vocab_size).type(torch.float32)
        
        # RNN
        Y, state = self.rnn(X, state)
        
        # Output
        output = self.linear(Y.reshape(-1, Y.shape[-1]))
        
        return output, state
\end{lstlisting}


\paragraph{深度RNN模型(续)}
\begin{lstlisting}
    def begin_state(self, batch_size, device):
        if self.rnn_type == 'lstm':
            # LSTMReturn(H, C)
            return (torch.zeros((self.num_layers, batch_size, 
                               self.num_hiddens), device=device),
                    torch.zeros((self.num_layers, batch_size, 
                               self.num_hiddens), device=device))
        else:
            # RNN/GRUReturnH
            return torch.zeros((self.num_layers, batch_size, 
                              self.num_hiddens), device=device)

# 3LSTM
net = DeepRNNModel(vocab_size=28, num_hiddens=512, 
                   num_layers=3, rnn_type='lstm')

# parameters
num_params = sum(p.numel() for p in net.parameters())
print(f"parameters数量: {num_params:,}")
# parameters: 3,496,020 (3parameters3)
\end{lstlisting}


\paragraph{训练深度RNN}
\begin{lstlisting}
# parameters
vocab_size = 28
num_hiddens = 256  #  comment
num_layers = 2     # 2层
batch_size = 32
num_steps = 35
num_epochs = 500
lr = 1

# data
train_iter, vocab = load_data_time_machine(batch_size, num_steps)

# model
device = torch.device('cuda' if torch.cuda.is_available() else 'cpu')
net = DeepRNNModel(len(vocab), num_hiddens, num_layers, 'lstm')
net = net.to(device)

# training
train_ch8(net, train_iter, vocab, lr, num_epochs, device)
\end{lstlisting}

\textbf{输出:}
\begin{verbatim}
困惑度 0.95, 28000 词元/秒 on cuda:0
time traveller smiled round at us then still smiling
\end{verbatim}


\paragraph{不同深度的性能对比}
\begin{center}
\begin{tikzpicture}[scale=0.9]
% 坐标轴
\draw[->,thick] (0,0) -- (8,0) node[right] {训练epoch};
\draw[->,thick] (0,0) -- (0,4) node[above] {困惑度};

% 1层
\draw[red,thick] (0.5,3.5) -- (1,2.5) -- (2,1.6) -- (3,1.3) -- (7,1.1);
\node[red,right] at (7,1.1) {1层};

% 2层
\draw[blue,thick] (0.5,3.5) -- (1,2.3) -- (2,1.4) -- (3,1.1) -- (7,0.95);
\node[blue,right] at (7,0.95) {2层};

% 3层
\draw[green!60!black,thick] (0.5,3.5) -- (1,2.2) -- (2,1.3) -- (3,1.05) -- (7,0.92);
\node[green!60!black,right] at (7,0.92) {3层};

% 4层
\draw[purple,thick,dashed] (0.5,3.5) -- (1,2.4) -- (2,1.5) -- (3,1.2) -- (7,1.0);
\node[purple,right] at (7,1.0) {4层(过拟合)};

% 参考线
\draw[dashed] (0,1) -- (8,1);
\node[left,font=\tiny] at (0,1) {PPL=1};

\node[below,text width=8cm,align=center] at (4,-0.8) {
\small 2-3层效果最好,更深可能过拟合
};

\end{tikzpicture}
\end{center}


\paragraph{深度与宽度的权衡}
\begin{definition}[两种增强模型的方式]
\begin{enumerate}
    \item \textbf{增加深度(层数)}
    \begin{itemize}
        \item 更多的非线性变换
        \item 层次化特征学习
        \item 参数效率高(共享结构)
    \end{itemize}
    
    \item \textbf{增加宽度(隐藏单元数)}
    \begin{itemize}
        \item 每层表达能力更强
        \item 更多的记忆容量
        \item 参数增长更快(平方关系)
    \end{itemize}
\end{enumerate}
\end{definition}

\begin{example}[实验对比]
\begin{itemize}
    \item 2层 $\times$ 256单元 ≈ 400K参数,PPL=0.95
    \item 1层 $\times$ 512单元 ≈ 300K参数,PPL=1.0
\end{itemize}
$\Rightarrow$ 相近参数量下,深度通常更好
\end{example}


\paragraph{深度RNN的参数分析}
\textbf{参数数量:}

\begin{definition}[单层LSTM]
$$\text{params} = 4 \times [(d+h) \times h + h] + (h+1) \times V$$
\end{definition}

\begin{definition}[$L$层LSTM]
\begin{itemize}
    \item 第1层:$4 \times [(d+h) \times h + h]$
    \item 第2到L层:每层 $4 \times [(h+h) \times h + h] = 4 \times (2h^2 + h)$
    \item 输出层:$(h+1) \times V$
\end{itemize}
$$\text{params} \approx 4(d+h)h + 4(L-1) \times 2h^2 + (h+1)V$$
\end{definition}

\begin{example}[具体例子($d=28, h=256, V=28, L=2$)]
约 $4 \times 284 \times 256 + 4 \times 2 \times 256^2 + 257 \times 28 \approx 815K$

单层($h=512$):约1.1M

$\Rightarrow$ 深度2层(小宽度)比单层(大宽度)参数少但效果好!
\end{example}


\paragraph{深度RNN的正则化}
\textbf{问题:}深度网络容易过拟合

\begin{definition}[常用正则化技术]
\begin{enumerate}
    \item \textbf{Dropout}
    \begin{itemize}
        \item 在层与层之间应用
        \item \texttt{nn.LSTM(..., dropout=0.5)}
        \item 只在$>1$层时生效
    \end{itemize}
    
    \item \textbf{L2正则化(权重衰减)}
    \begin{itemize}
        \item 优化器中设置:\texttt{weight\_decay=1e-5}
    \end{itemize}
    
    \item \textbf{梯度裁剪}
    \begin{itemize}
        \item 更重要(防止梯度爆炸)
    \end{itemize}
    
    \item \textbf{Early Stopping}
    \begin{itemize}
        \item 监控验证集,防止过拟合
    \end{itemize}
\end{enumerate}
\end{definition}


\paragraph{添加Dropout}
\begin{lstlisting}
# Dropout
lstm_layer = nn.LSTM(
    input_size=vocab_size,
    hidden_size=num_hiddens,
    num_layers=3,
    dropout=0.5  #  comment1-2层后添加dropout
)

# - trainingnet.train()
# - testnet.eval()
# - num_layers > 1

# Dropout
#  -> LSTM1 -> Dropout(0.5) -> LSTM2 -> Dropout(0.5) -> LSTM3 -> Output
\end{lstlisting}

\begin{theorem}[Dropout率建议]
\begin{itemize}
    \item 0.2-0.3:轻度正则化
    \item 0.5:标准配置
    \item $>0.5$:可能欠拟合
\end{itemize}
\end{theorem}


\paragraph{深度RNN的训练技巧}
\begin{definition}[训练深度RNN的建议]
\begin{enumerate}
    \item \textbf{从浅到深}
    \begin{itemize}
        \item 先训练1层,确认有效
        \item 再增加到2层、3层
    \end{itemize}
    
    \item \textbf{调整学习率}
    \begin{itemize}
        \item 深度网络可能需要更小的学习率
        \item 使用学习率衰减
    \end{itemize}
    
    \item \textbf{梯度裁剪必须}
    \begin{itemize}
        \item 深度网络更容易梯度爆炸
        \item 阈值可以稍小(如0.5-1)
    \end{itemize}
    
    \item \textbf{监控梯度}
    \begin{itemize}
        \item 记录每层的梯度范数
        \item 检测梯度消失/爆炸
    \end{itemize}
    
    \item \textbf{批归一化}
    \begin{itemize}
        \item 可以用Layer Norm(RNN中更常用)
    \end{itemize}
\end{enumerate}
\end{definition}


\paragraph{深度RNN的变体}
\begin{definition}[改进的深度架构]
\begin{enumerate}
    \item \textbf{残差连接(Residual Connection)}
    $$\mathbf{H}_t^{(l)} = \text{RNN}^{(l)}(\mathbf{H}_t^{(l-1)}) + \mathbf{H}_t^{(l-1)}$$
    缓解梯度消失,允许更深
    
    \item \textbf{Highway Network}
    $$\mathbf{H}_t^{(l)} = g \odot \text{RNN}^{(l)}(\mathbf{H}_t^{(l-1)}) + (1-g) \odot \mathbf{H}_t^{(l-1)}$$
    学习跳跃连接的权重
    
    \item \textbf{Dense Connection}
    每层连接到所有之前的层(类似DenseNet)
    
    \item \textbf{金字塔结构}
    下层宽、上层窄:[1024, 512, 256]
\end{enumerate}
\end{definition}


\paragraph{实际应用中的深度RNN}
\begin{definition}[典型配置]
\begin{enumerate}
    \item \textbf{Google Neural Machine Translation (2016)}
    \begin{itemize}
        \item 8层LSTM编码器 + 8层LSTM解码器
        \item 残差连接
        \item 注意力机制
    \end{itemize}
    
    \item \textbf{Google Speech Recognition}
    \begin{itemize}
        \item 5层双向LSTM
        \item Dropout + 批归一化
    \end{itemize}
    
    \item \textbf{OpenAI GPT-2 (Transformer前)}
    \begin{itemize}
        \item 研究表明:2-3层LSTM + 注意力 ≈ 浅层Transformer
    \end{itemize}
\end{enumerate}
\end{definition}

\begin{theorem}[趋势]
现在更多用Transformer,但深度RNN在某些任务仍有优势
\end{theorem}


\paragraph{深度RNN vs Transformer}
\begin{center}
\begin{tabular}{|l|c|c|}
\hline
\textbf{特性} & \textbf{深度RNN} & \textbf{Transformer} \\
\hline
并行化 & 困难(顺序处理) & 容易 \\
\hline
训练速度 & 慢 & 快 \\
\hline
长期依赖 & LSTM可以 & 更好 \\
\hline
参数效率 & 高 & 低 \\
\hline
实时推理 & 快(逐步) & 慢(全局) \\
\hline
内存占用 & 低 & 高 \\
\hline
流式处理 & \checkmark 支持 & $\times$不支持 \\
\hline
\end{tabular}
\end{center}

\vspace{0.5cm}
\begin{definition}[深度RNN仍适用的场景]
\begin{itemize}
    \item 实时流式任务(语音识别)
    \item 资源受限环境(移动设备)
    \item 在线学习(持续更新)
    \item 小数据集任务
\end{itemize}
\end{definition}


\paragraph{深度RNN关键要点}
\begin{center}
\begin{tikzpicture}[scale=0.85]
% 要点1
\node[draw,fill=blue!20,rounded corners,text width=10cm] at (5,3.8) {
\textbf{1. 层数选择}\\
1层:基线 | 2层:标准 | 3层:复杂任务 | $>3$层:谨慎
};

% 要点2
\node[draw,fill=green!20,rounded corners,text width=10cm] at (5,2.6) {
\textbf{2. 正则化必须}\\
Dropout(0.5)+ 梯度裁剪 + Early Stopping
};

% 要点3
\node[draw,fill=orange!20,rounded corners,text width=10cm] at (5,1.4) {
\textbf{3. 深度 vs 宽度}\\
相近参数量下,深度(2层256)$>$ 宽度(1层512)
};

% 要点4
\node[draw,fill=purple!20,rounded corners,text width=10cm] at (5,0.2) {
\textbf{4. 训练技巧}\\
从浅到深、监控梯度、调整学习率
};

\end{tikzpicture}
\end{center}


\paragraph{实践建议}
\begin{definition}[如何选择深度]
\begin{enumerate}
    \item \textbf{数据量小($<$10K样本)}
    \begin{itemize}
        \item 用1层,避免过拟合
    \end{itemize}
    
    \item \textbf{数据量中等(10K-100K)}
    \begin{itemize}
        \item 用2层,加Dropout
    \end{itemize}
    
    \item \textbf{数据量大($>$100K)}
    \begin{itemize}
        \item 尝试3层
        \item 监控验证集性能
    \end{itemize}
    
    \item \textbf{调参建议}
    \begin{itemize}
        \item 先固定1层,调好隐藏单元数
        \item 再逐步增加层数
        \item 每次增加层数后重新调学习率
    \end{itemize}
\end{enumerate}
\end{definition}


\paragraph{完整示例:2层LSTM语言模型}
\begin{lstlisting}
import torch
import torch.nn as nn

class TwoLayerLSTM(nn.Module):
    def __init__(self, vocab_size, embed_size=256, hidden_size=512):
        super().__init__()
        self.vocab_size = vocab_size
        
        # one-hot
        self.embedding = nn.Embedding(vocab_size, embed_size)
        
        # 2LSTM
        self.lstm = nn.LSTM(embed_size, hidden_size, 
                           num_layers=2, dropout=0.5)
        
        # Output
        self.fc = nn.Linear(hidden_size, vocab_size)
    
    def forward(self, x, state):
        # x: (batch, seq_len)
        x = self.embedding(x.T)  # (seq_len, batch, embed)
        out, state = self.lstm(x, state)
        out = self.fc(out.reshape(-1, out.shape[-1]))
        return out, state
    
    def init_state(self, batch_size, device):
        h = torch.zeros(2, batch_size, 512, device=device)
        c = torch.zeros(2, batch_size, 512, device=device)
        return (h, c)
\end{lstlisting}


\paragraph{常见错误与解决}
\begin{definition}[问题1:深度网络不收敛]
\textbf{可能原因:}
\begin{itemize}
    \item 学习率太大 $\rightarrow$ 减小10倍
    \item 梯度爆炸 $\rightarrow$ 减小裁剪阈值
    \item 初始化不好 $\rightarrow$ 使用Xavier/He初始化
\end{itemize}
\end{definition}

\begin{definition}[问题2:验证集性能不提升]
\textbf{可能原因:}
\begin{itemize}
    \item 过拟合 $\rightarrow$ 增大Dropout
    \item 层数太多 $\rightarrow$ 减少到2层
    \item 数据太少 $\rightarrow$ 数据增强或减少模型复杂度
\end{itemize}
\end{definition}

\begin{definition}[问题3:训练很慢]
\textbf{解决方案:}
\begin{itemize}
    \item 减少隐藏单元数
    \item 减少层数
    \item 使用cuDNN(PyTorch自动)
    \item 减小批量大小(换取速度)
\end{itemize}
\end{definition}


\paragraph{层数实验对比}
\begin{center}
\begin{tikzpicture}[scale=0.9]
% 柱状图
\draw[->,thick] (0,0) -- (0,4) node[above] {困惑度};
\draw[->,thick] (0,0) -- (8,0) node[right] {层数};

% 1层
\fill[blue!50] (0.5,0) rectangle (1.5,1.1);
\node[below] at (1,0) {1};
\node[above] at (1,1.1) {\small 1.1};

% 2层
\fill[green!50] (2,0) rectangle (3,0.95);
\node[below] at (2.5,0) {2};
\node[above] at (2.5,0.95) {\small 0.95};

% 3层
\fill[orange!50] (3.5,0) rectangle (4.5,0.92);
\node[below] at (4,0) {3};
\node[above] at (4,0.92) {\small 0.92};

% 4层
\fill[red!50] (5,0) rectangle (6,1.05);
\node[below] at (5.5,0) {4};
\node[above] at (5.5,1.05) {\small 1.05};

% 5层
\fill[purple!50] (6.5,0) rectangle (7.5,1.15);
\node[below] at (7,0) {5};
\node[above] at (7,1.15) {\small 1.15};

% 最优标记
\draw[green!60!black,ultra thick] (3.5,0.92) -- (4.5,0.92);
\node[green!60!black,above] at (4,1.3) {最优};

\node[below,text width=8cm,align=center] at (4,-0.8) {
\small 3层最优,更深反而性能下降(过拟合)
};

\end{tikzpicture}
\end{center}


\section{9.4 双向循环神经网络}

\subsection{9.4 双向循环神经网络}
\begin{center}
\Large \textcolor{blue}{Bidirectional RNN}\\
\large 同时利用过去和未来的信息
\end{center}


\subsection{9.4节 - 章节导航}
\begin{definition}[本节内容]
\begin{itemize}
    \item 9.4.1 双向循环神经网络的动机
    \item 9.4.2 双向循环神经网络的定义
    \item 9.4.3 双向循环神经网络的实现
    \item 9.4.4 应用与限制
\end{itemize}
\end{definition}

\begin{theorem}[学习目标]
\begin{enumerate}
    \item 理解为什么需要双向RNN
    \item 掌握双向RNN的结构
    \item 学会实现双向RNN
    \item 了解适用场景
\end{enumerate}
\end{theorem}


\paragraph{为什么需要双向RNN?}
\begin{center}
\begin{tikzpicture}[scale=0.9]
% 问题
\node[draw,fill=red!20,rounded corners,text width=10cm] at (5,3.8) {
\textbf{标准RNN的局限:}\\
只能利用\textbf{过去}的信息,看不到\textbf{未来}
};

\draw[->,ultra thick] (5,3.3) -- (5,2.9);

% 例子
\node[draw,fill=yellow!20,rounded corners,text width=10cm] at (5,2.3) {
\textbf{例子:}填空题\\
``I am \underline{\quad} because I will go to Paris tomorrow.''\\
预测空格需要看到``Paris''(未来信息)才知道是``excited''
};

\draw[->,ultra thick] (5,1.7) -- (5,1.3);

% 解决方案
\node[draw,fill=green!20,rounded corners,text width=10cm] at (5,0.7) {
\textbf{双向RNN:}同时从前向后和从后向前处理
};

\end{tikzpicture}
\end{center}


\paragraph{双向RNN的核心思想}
\textbf{标准RNN:}只看过去

\begin{center}
\begin{tikzpicture}[scale=0.9]
\foreach \t in {1,2,3,4} {
    \node[circle,draw,fill=blue!20,minimum size=0.7cm] (h\t) at (\t*1.8,1) {\tiny$h_{\t}$};
}
\draw[->,thick,red] (h1) -- (h2) -- (h3) -- (h4);
\node[above,red] at (4,1.5) {只看左边(过去)};
\end{tikzpicture}
\end{center}

\textbf{双向RNN:}同时看过去和未来

\begin{center}
\begin{tikzpicture}[scale=0.9]
% 前向
\foreach \t in {1,2,3,4} {
    \node[circle,draw,fill=blue!20,minimum size=0.7cm] (f\t) at (\t*1.8,2) {\tiny$\overrightarrow{h}_{\t}$};
}
\draw[->,thick,red] (f1) -- (f2) -- (f3) -- (f4);
\node[above,red,font=\tiny] at (4,2.3) {前向(看过去)};

% 后向
\foreach \t in {1,2,3,4} {
    \node[circle,draw,fill=green!20,minimum size=0.7cm] (b\t) at (\t*1.8,0.5) {\tiny$\overleftarrow{h}_{\t}$};
}
\draw[<-,thick,blue] (b1) -- (b2) -- (b3) -- (b4);
\node[below,blue,font=\tiny] at (4,0.2) {后向(看未来)};

% 组合
\foreach \t in {1,2,3,4} {
    \draw[->,thick,purple] (f\t) -- (\t*1.8,1.25);
    \draw[->,thick,purple] (b\t) -- (\t*1.8,1.25);
    \node[circle,draw,fill=purple!20,minimum size=0.5cm] at (\t*1.8,1.25) {\tiny$h_{\t}$};
}
\node[right,purple] at (7.5,1.25) {组合};

\end{tikzpicture}
\end{center}



