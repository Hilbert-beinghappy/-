\section{未命名节}

\subsection{学习语言模型}

\subsection{8.3.1 学习语言模型}

\subsection{8.3.1 学习语言模型}
\begin{center}
\Large \textcolor{blue}{语言模型的核心:估计文本序列的概率}
\end{center}


\paragraph{什么是语言模型?}
\textbf{定义:}语言模型对文本序列的概率分布进行建模

\begin{definition}[数学表示]
对于文本序列 $x_1, x_2, \ldots, x_T$,语言模型估计其概率:
$$P(x_1, x_2, \ldots, x_T)$$
\end{definition}

\textbf{链式法则分解:}
$$P(x_1, x_2, \ldots, x_T) = \prod_{t=1}^{T} P(x_t \mid x_1, \ldots, x_{t-1})$$

\begin{example}[直观理解]
给定前面的词,预测下一个词的概率
\begin{itemize}
    \item 已知:``the cat sat on the''
    \item 预测:下一个词是 ``mat'' 的概率?
\end{itemize}
\end{example}


\paragraph{语言模型的应用}


\textbf{1. 文本生成}
\begin{itemize}
    \item 对话系统
    \item 故事创作
    \item 代码补全
\end{itemize}

\textbf{2. 机器翻译}
\begin{itemize}
    \item 评估译文流畅度
    \item 选择最佳翻译
\end{itemize}


\textbf{3. 语音识别}
\begin{itemize}
    \item 纠正识别错误
    \item 消除歧义
\end{itemize}

\textbf{4. 拼写检查}
\begin{itemize}
    \item 检测错误
    \item 提供建议
\end{itemize}


\vspace{0.5cm}
\begin{theorem}[核心思想]
好的语言模型能给\textbf{合理的句子}分配\textbf{高概率},给\textbf{不合理的句子}分配\textbf{低概率}
\end{theorem}


\paragraph{语言模型的挑战}
\textbf{问题:}直接估计 $P(x_1, x_2, \ldots, x_T)$ 非常困难!

\begin{center}
\begin{tikzpicture}[scale=1.0]
% 序列长度
\node[draw,fill=yellow!20,text width=10cm] at (5,3.5) {
\textbf{挑战1:序列长度不固定}\\
不同句子长度不同,如何统一建模?
};

% 参数数量
\node[draw,fill=orange!20,text width=10cm] at (5,2.2) {
\textbf{挑战2:参数爆炸}\\
如果词表大小为 $V$,长度为 $T$ 的序列有 $V^T$ 种可能!
};

% 稀疏性
\node[draw,fill=red!20,text width=10cm] at (5,0.9) {
\textbf{挑战3:数据稀疏}\\
很多序列在训练集中从未出现,如何估计概率?
};

\end{tikzpicture}
\end{center}

\begin{definition}[解决思路]
\begin{itemize}
    \item 使用\textbf{链式法则}分解
    \item 使用\textbf{马尔可夫假设}简化
    \item 使用\textbf{神经网络}建模条件概率
\end{itemize}
\end{definition}


\paragraph{条件概率建模}
\textbf{关键:}将复杂的联合概率分解为条件概率的乘积

$$P(x_1, x_2, x_3, x_4) = P(x_1) \cdot P(x_2 \mid x_1) \cdot P(x_3 \mid x_1, x_2) \cdot P(x_4 \mid x_1, x_2, x_3)$$

\begin{center}
\begin{tikzpicture}[scale=1.1]
% 时间步
\foreach \i in {1,2,3,4} {
    \node[circle,draw,fill=blue!20] (x\i) at (\i*2,0) {$x_{\i}$};
}

% 条件依赖
\draw[->,thick,red] (x1) to[bend left=30] node[above,font=\tiny] {$P(x_2|x_1)$} (x2);
\draw[->,thick,red] (x1) to[bend left=40] (x3);
\draw[->,thick,red] (x2) to[bend left=30] node[above,font=\tiny] {$P(x_3|x_1,x_2)$} (x3);
\draw[->,thick,red] (x1) to[bend left=50] (x4);
\draw[->,thick,red] (x2) to[bend left=40] (x4);
\draw[->,thick,red] (x3) to[bend left=30] node[above,font=\tiny] {$P(x_4|x_1,x_2,x_3)$} (x4);

\node[below,font=\small] at (4,-0.7) {每个词依赖前面所有词};
\end{tikzpicture}
\end{center}

\begin{theorem}[关键问题]
如何高效地估计条件概率 $P(x_t \mid x_1, \ldots, x_{t-1})$?
\end{theorem}


\paragraph{困惑度(Perplexity)}
\textbf{定义:}评估语言模型质量的标准指标

\begin{definition}[数学定义]
$$\text{PPL} = \exp\left(-\frac{1}{n}\sum_{t=1}^{n}\log P(x_t \mid x_1, \ldots, x_{t-1})\right)$$
\end{definition}

\textbf{等价形式:}
$$\text{PPL} = \sqrt[n]{\frac{1}{P(x_1, x_2, \ldots, x_n)}}$$



\textbf{直观理解:}
\begin{itemize}
    \item 平均分支因子
    \item 模型的``困惑程度''
    \item 越低越好
\end{itemize}


\textbf{取值范围:}
\begin{itemize}
    \item 最好:PPL = 1(完全确定)
    \item 随机猜测:PPL = $|V|$
    \item 实际:10-200之间
\end{itemize}



\paragraph{困惑度的直观例子}
\textbf{例子:}预测下一个词

\begin{center}
\begin{tikzpicture}[scale=1.0]
% 好的模型
\node[draw,fill=green!20,text width=5cm] at (0,3) {
\textbf{模型A(好):}\\
``I love''后面是:\\
- ``you'': 0.7\\
- ``it'': 0.2\\
- ``apple'': 0.1\\
\textbf{PPL} $\approx$ \textbf{2.1}
};

% 差的模型
\node[draw,fill=red!20,text width=5cm] at (0,0.5) {
\textbf{模型B(差):}\\
``I love''后面是:\\
- ``you'': 0.01\\
- ``it'': 0.01\\
- 其他998个词均分\\
\textbf{PPL} $\approx$ \textbf{100}
};

% 标注
\node[right,text width=4cm] at (6,3) {
\textcolor{green!60!black}{$\checkmark$ 集中在合理的词上}\\
困惑度低
};

\node[right,text width=4cm] at (6,0.5) {
\textcolor{red}{$\times$ 概率分散}\\
困惑度高
};

\end{tikzpicture}
\end{center}


\paragraph{困惑度计算示例}
\begin{lstlisting}
import torch
import torch.nn.functional as F

def calculate_perplexity(model, data, vocab):
    """model"""
    model.eval()
    total_loss = 0
    total_tokens = 0
    
    with torch.no_grad():
        for X, Y in data:
            # X: (batch_size, seq_len)
            # Y: (batch_size, seq_len)
            output = model(X)  # (batch_size, seq_len, vocab_size)
            
            # loss
            loss = F.cross_entropy(
                output.reshape(-1, len(vocab)),
                Y.reshape(-1),
                reduction='sum'
            )
            
            total_loss += loss.item()
            total_tokens += Y.numel()
    
    #  = exp()
    perplexity = torch.exp(torch.tensor(total_loss / total_tokens))
    return perplexity.item()
\end{lstlisting}



\subsection{马尔可夫模型与元语法}

\subsection{8.3.2 马尔可夫模型与元语法}

\subsection{8.3.2 马尔可夫模型与元语法}
\begin{center}
\Large \textcolor{blue}{简化建模:马尔可夫假设}
\end{center}


\paragraph{马尔可夫假设}
\textbf{问题:}依赖所有历史太复杂

$$P(x_t \mid x_1, x_2, \ldots, x_{t-1})$$

\textbf{马尔可夫假设:}只依赖最近的 $\tau$ 个词

$$P(x_t \mid x_1, x_2, \ldots, x_{t-1}) \approx P(x_t \mid x_{t-\tau}, \ldots, x_{t-1})$$

\begin{center}
\begin{tikzpicture}[scale=1.0]
% 完整依赖
\node[text width=10cm] at (5,3.5) {
\textbf{完整模型:}依赖所有历史
};
\foreach \i in {1,2,3,4,5} {
    \node[circle,draw,fill=blue!20,font=\tiny] (x\i) at (\i*1.5,2.5) {$x_{\i}$};
}
\node[circle,draw,fill=green!30] (x6) at (10,2.5) {$x_6$};
\foreach \i in {1,2,3,4,5} {
    \draw[->,thick,red] (x\i) to[bend left=20] (x6);
}

% 马尔可夫
\node[text width=10cm] at (5,1.2) {
\textbf{马尔可夫模型:}只依赖最近 $\tau=2$ 个
};
\foreach \i in {1,2,3,4,5} {
    \node[circle,draw,fill=blue!20,font=\tiny] (y\i) at (\i*1.5,0) {$x_{\i}$};
}
\node[circle,draw,fill=green!30] (y6) at (10,0) {$x_6$};
\draw[->,thick,red] (y4) to[bend left=20] (y6);
\draw[->,thick,red] (y5) to[bend left=20] (y6);
\end{tikzpicture}
\end{center}


\paragraph{N-gram模型}
\textbf{根据 $\tau$ 的不同,有不同的N-gram模型:}

\begin{center}
\begin{tabular}{|l|l|l|}
\hline
\textbf{模型} & \textbf{$\tau$} & \textbf{条件概率} \\
\hline
Unigram (1-gram) & 0 & $P(x_t)$ \\
\hline
Bigram (2-gram) & 1 & $P(x_t \mid x_{t-1})$ \\
\hline
Trigram (3-gram) & 2 & $P(x_t \mid x_{t-2}, x_{t-1})$ \\
\hline
4-gram & 3 & $P(x_t \mid x_{t-3}, x_{t-2}, x_{t-1})$ \\
\hline
\end{tabular}
\end{center}

\begin{example}[例子:Bigram模型]
\textbf{句子:}``the cat sat on the mat''

\textbf{概率:}
\begin{align*}
P(\text{sentence}) &= P(\text{the}) \cdot P(\text{cat} \mid \text{the}) \cdot P(\text{sat} \mid \text{cat}) \\
&\quad \cdot P(\text{on} \mid \text{sat}) \cdot P(\text{the} \mid \text{on}) \cdot P(\text{mat} \mid \text{the})
\end{align*}
\end{example}


\paragraph{N-gram模型的估计}
\textbf{最大似然估计(MLE):}

$$P(x_t \mid x_{t-\tau}, \ldots, x_{t-1}) = \frac{\text{count}(x_{t-\tau}, \ldots, x_{t-1}, x_t)}{\text{count}(x_{t-\tau}, \ldots, x_{t-1})}$$

\begin{example}[Bigram例子]
训练文本:``I love you I love coding I love AI''

\textbf{计算:}$P(\text{you} \mid \text{love})$

$$P(\text{you} \mid \text{love}) = \frac{\text{count}(\text{love you})}{\text{count}(\text{love})} = \frac{1}{3}$$

其他:
\begin{itemize}
    \item $P(\text{coding} \mid \text{love}) = 1/3$
    \item $P(\text{AI} \mid \text{love}) = 1/3$
\end{itemize}
\end{example}


\paragraph{N-gram模型的问题}


\textbf{优点:}
\begin{itemize}
    \item $\checkmark$ 简单高效
    \item $\checkmark$ 易于实现
    \item $\checkmark$ 可解释性强
\end{itemize}


\textbf{缺点:}
\begin{itemize}
    \item $\times$ 无法捕捉长期依赖
    \item $\times$ 数据稀疏问题严重
    \item $\times$ 参数随 $\tau$ 指数增长
\end{itemize}


\vspace{0.5cm}
\begin{theorem}[数据稀疏问题]
\textbf{例子:}训练集中没见过 ``love machine''

$\Rightarrow$ $P(\text{machine} \mid \text{love}) = 0$

\textbf{解决:}平滑技术(Laplace平滑、Good-Turing平滑等)
\end{theorem}

\begin{definition}[现代方案]
使用\textbf{神经网络语言模型}(如RNN)克服这些问题
\end{definition}


\paragraph{N-gram模型实现}
\begin{lstlisting}
from collections import defaultdict, Counter

class NgramModel:
    def __init__(self, n=2):
        self.n = n
        self.context_counts = defaultdict(Counter)
    
    def train(self, corpus):
        """trainingN-grammodel"""
        for i in range(self.n - 1, len(corpus)):
            context = tuple(corpus[i - self.n + 1:i])
            word = corpus[i]
            self.context_counts[context][word] += 1
    
    def prob(self, context, word):
        """计算条件概率 P(word | context)"""
        context = tuple(context)
        if context not in self.context_counts:
            return 0
        
        total = sum(self.context_counts[context].values())
        return self.context_counts[context][word] / total
\end{lstlisting}



\subsection{自然语言统计}

\subsection{8.3.3 自然语言统计}

\subsection{8.3.3 自然语言统计}
\begin{center}
\Large \textcolor{blue}{真实文本的统计规律}
\end{center}


\paragraph{Zipf定律}
\textbf{定律:}词频与其排名成反比

$$\text{frequency}(k) \propto \frac{1}{k^\alpha}$$

其中 $k$ 是词频排名,$\alpha \approx 1$

\begin{center}
\begin{tikzpicture}[scale=1.0]
% 坐标轴
\draw[->,thick] (0,0) -- (8,0) node[right] {词频排名(对数尺度)};
\draw[->,thick] (0,0) -- (0,4) node[above] {词频(对数尺度)};

% Zipf曲线
\draw[blue,thick,domain=0.5:7.5,samples=100] plot (\x,{3.5-0.5*\x});

% 标注
\node[blue] at (2,3) {the};
\node[blue] at (3,2.3) {of};
\node[blue] at (4,1.8) {and};
\node[blue] at (6,1) {machine};

% 直线拟合
\draw[red,dashed,thick] (0.5,3.2) -- (7.5,0.2);
\node[red,right] at (7.5,0.2) {$y = -\alpha x + c$};

\node[below,text width=8cm,align=center] at (4,-0.8) {
\small 少数高频词 + 大量低频词
};
\end{tikzpicture}
\end{center}


\paragraph{Zipf定律的含义}


\textbf{长尾分布:}
\begin{itemize}
    \item 少数词非常常见
    \item 大部分词很少见
    \item ``the'', ``of'', ``and'' 占比很大
\end{itemize}

\textbf{影响:}
\begin{itemize}
    \item 词表大小选择
    \item 低频词处理
    \item 采样策略
\end{itemize}


\begin{example}[时光机器数据]
\begin{itemize}
    \item 总词数:32,775
    \item 不同词数:4,580
    \item Top 10占比:30\%
    \item Top 100占比:50\%
    \item 出现1次的词:50\%
\end{itemize}
\end{example}


\vspace{0.3cm}
\begin{theorem}[启示]
可以用 \texttt{min\_freq} 过滤低频词,减小词表但保留大部分信息
\end{theorem}


\paragraph{可视化Zipf定律}
\begin{lstlisting}
import matplotlib.pyplot as plt
import numpy as np

counter = count_corpus(tokens)
freqs = sorted(counter.values(), reverse=True)

plt.figure(figsize=(10, 5))

plt.subplot(1, 2, 1)
plt.loglog(range(1, len(freqs) + 1), freqs)
plt.xlabel('Rank')
plt.ylabel('Frequency')
plt.title('Zipf Law (log-log)')

plt.subplot(1, 2, 2)
plt.plot(range(1, 100), freqs[:99])
plt.xlabel('Rank')
plt.ylabel('Frequency')
plt.title('Top 100 words')

plt.tight_layout()
plt.show()
\end{lstlisting}


\paragraph{词频统计可视化}
% 如果有图片可以插入:
% \includegraphics[width=0.8\textwidth]{zipf_plot.png}

\begin{center}
\begin{tikzpicture}[scale=0.85]
% 左图:对数-对数
\begin{scope}[xshift=0cm]
\draw[->,thick] (0,0) -- (5,0) node[right,font=\tiny] {log(rank)};
\draw[->,thick] (0,0) -- (0,4) node[above,font=\tiny] {log(freq)};
\draw[blue,thick] (0.5,3.5) -- (4.5,0.5);
\node[below] at (2.5,-0.5) {\small 对数-对数图(直线)};
\fill[red] (0.5,3.5) circle (2pt) node[left,font=\tiny] {the};
\fill[red] (1.5,2.5) circle (2pt) node[left,font=\tiny] {of};
\fill[red] (2.5,1.5) circle (2pt) node[left,font=\tiny] {and};
\end{scope}

% 右图:普通坐标
\begin{scope}[xshift=7cm]
\draw[->,thick] (0,0) -- (5,0) node[right,font=\tiny] {rank};
\draw[->,thick] (0,0) -- (0,4) node[above,font=\tiny] {frequency};
\draw[blue,thick,domain=0.3:5,samples=50] plot (\x,{3.5/\x});
\node[below] at (2.5,-0.5) {\small 普通坐标(双曲线)};
\end{scope}

\end{tikzpicture}
\end{center}

\begin{definition}[观察]
\begin{itemize}
    \item 双对数图呈直线 $\Rightarrow$ 幂律分布
    \item 少数高频词 + 长长的尾巴
    \item 符合Zipf定律
\end{itemize}
\end{definition}



\subsection{读取长序列数据}

\subsection{8.3.4 读取长序列数据}

\subsection{8.3.4 读取长序列数据}
\begin{center}
\Large \textcolor{blue}{如何高效地训练序列模型?}
\end{center}


\paragraph{训练RNN的挑战}
\textbf{问题:}完整的文本语料库是一个很长的序列

\begin{center}
\begin{tikzpicture}[scale=1.0]
% 长序列
\node[text width=10cm,align=center] at (5,3) {
\textbf{时光机器语料库:}170,580个字符\\
\textcolor{red}{太长了!无法一次性处理}
};

\draw[->,thick] (5,2.5) -- (5,2);

% 需要分割
\node[draw,fill=yellow!20,text width=10cm] at (5,1.2) {
\textbf{解决方案:}将长序列分割成小批量
\begin{itemize}
    \item 每个批量包含多个样本
    \item 每个样本是固定长度的子序列
\end{itemize}
};

\end{tikzpicture}
\end{center}

\begin{theorem}[关键问题]
\begin{enumerate}
    \item 如何分割长序列?
    \item 如何保持序列的连续性?
    \item 如何构建批量数据?
\end{enumerate}
\end{theorem}


\paragraph{两种采样策略}


\textbf{1. 随机采样}
\begin{itemize}
    \item 每个批量随机选起始位置
    \item 批量间相互独立
    \item 打乱了时序
\end{itemize}

\begin{center}
\begin{tikzpicture}[scale=0.6]
\draw (0,0) rectangle (8,0.5);
\fill[blue!30] (1,0) rectangle (3,0.5);
\fill[green!30] (5,0) rectangle (7,0.5);
\node[above] at (2,0.5) {\tiny 样本1};
\node[above] at (6,0.5) {\tiny 样本2};
\node[below] at (4,-0.2) {\tiny 随机位置};
\end{tikzpicture}
\end{center}

\textcolor{blue}{\textbf{优点:}}
\begin{itemize}
    \item 简单
    \item 并行友好
\end{itemize}


\textbf{2. 顺序分区}
\begin{itemize}
    \item 保持批量间的连续性
    \item 按顺序划分
    \item 保留了时序关系
\end{itemize}

\begin{center}
\begin{tikzpicture}[scale=0.6]
\draw (0,0) rectangle (8,0.5);
\fill[blue!30] (0,0) rectangle (2,0.5);
\fill[green!30] (2,0) rectangle (4,0.5);
\fill[red!30] (4,0) rectangle (6,0.5);
\fill[yellow!30] (6,0) rectangle (8,0.5);
\node[above,font=\tiny] at (1,0.5) {批次1};
\node[above,font=\tiny] at (3,0.5) {批次2};
\node[above,font=\tiny] at (5,0.5) {批次3};
\node[above,font=\tiny] at (7,0.5) {批次4};
\end{tikzpicture}
\end{center}

\textcolor{blue}{\textbf{优点:}}
\begin{itemize}
    \item 保持连续性
    \item 更符合实际
\end{itemize}


\vspace{0.5cm}
\begin{theorem}[选择建议]
训练时通常使用\textbf{顺序分区},以更好地利用上下文信息
\end{theorem}


\paragraph{随机采样详解}
\textbf{策略:}从语料库中随机选择起始位置

\begin{center}
\begin{tikzpicture}[scale=0.9]
% 完整语料库
\draw[thick] (0,3) rectangle (10,3.5);
\node[left] at (0,3.25) {语料库};
\node[right] at (10,3.25) {170,580字符};

% 随机采样1
\fill[blue!30] (1,3) rectangle (3,3.5);
\draw[->,thick,blue] (2,3) -- (2,2);
\draw[thick,blue] (1,1.5) rectangle (3,2);
\node[below,blue] at (2,1.5) {样本1};

% 随机采样2
\fill[green!30] (6,3) rectangle (8,3.5);
\draw[->,thick,green!60!black] (7,3) -- (7,1);
\draw[thick,green!60!black] (6,0.5) rectangle (8,1);
\node[below,green!60!black] at (7,0.5) {样本2};

% 随机采样3
\fill[red!30] (4,3) rectangle (6,3.5);
\draw[->,thick,red] (5,3) -- (5,0);
\draw[thick,red] (4,-.5) rectangle (6,0);
\node[below,red] at (5,-0.5) {样本3};

\node[below,text width=10cm,align=center] at (5,-1.2) {
\small 每个样本的起始位置随机选择,批量间无关联
};

\end{tikzpicture}
\end{center}


\paragraph{随机采样实现}
\begin{lstlisting}
def seq_data_iter_random(corpus, batch_size, num_steps):
    """ Random sampling to generate mini-batch subsequences corpus: list of token indices batch_size: batch size num_steps: number of time steps per subsequence """
    corpus = corpus[random.randint(0, num_steps - 1):]
    
    num_subseqs = (len(corpus) - 1) // num_steps
    
    initial_indices = list(range(0, num_subseqs * num_steps, num_steps))
    
    random.shuffle(initial_indices)
    
    def data(pos):
        """Return sequence of length num_steps starting from pos"""
        return corpus[pos: pos + num_steps]
    
    # Generate batches
    num_batches = num_subseqs // batch_size
    for i in range(0, batch_size * num_batches, batch_size):
        # Get batch_size sequences
        initial_indices_per_batch = initial_indices[i: i + batch_size]
        
        X = [data(j) for j in initial_indices_per_batch]
        Y = [data(j + 1) for j in initial_indices_per_batch]
        
        yield torch.tensor(X), torch.tensor(Y)
\end{lstlisting}


\paragraph{顺序分区详解}
\textbf{策略:}将语料库均匀分成 \texttt{batch\_size} 份,顺序读取

\begin{center}
\begin{tikzpicture}[scale=0.9]
% 语料库
\node[left,font=\small] at (-0.5,3) {语料库};
\draw[thick] (0,2.5) rectangle (10,3);

% 分成batch_size份
\foreach \i in {0,1,2,3} {
    \fill[blue!30] (\i*2.5,2.5) rectangle (\i*2.5+2.5,3);
    \draw[thick] (\i*2.5,2.5) rectangle (\i*2.5+2.5,3);
    \node at (\i*2.5+1.25,2.75) {\tiny 分区\i};
}

% 每个分区顺序读取
\draw[->,thick] (1.25,2.5) -- (1.25,1.8);
\draw[thick,fill=blue!20] (0.5,1) rectangle (2,1.5);
\node[below,font=\tiny] at (1.25,1) {批次1-样本0};

\draw[->,thick] (3.75,2.5) -- (3.75,1.8);
\draw[thick,fill=green!20] (3,1) rectangle (4.5,1.5);
\node[below,font=\tiny] at (3.75,1) {批次1-样本1};

\draw[->,thick] (6.25,2.5) -- (6.25,1.8);
\draw[thick,fill=red!20] (5.5,1) rectangle (7,1.5);
\node[below,font=\tiny] at (6.25,1) {批次1-样本2};

\draw[->,thick] (8.75,2.5) -- (8.75,1.8);
\draw[thick,fill=yellow!20] (8,1) rectangle (9.5,1.5);
\node[below,font=\tiny] at (8.75,1) {批次1-样本3};

% 下一批次
\node[below,text width=10cm,align=center] at (5,0.3) {
\small 批次2、批次3...依次从每个分区顺序读取下一段
};

\end{tikzpicture}
\end{center}


\paragraph{顺序分区的优势}
\begin{center}
\begin{tikzpicture}[scale=1.0]
% 时间线
\draw[->,thick] (0,3) -- (11,3) node[right] {时间};

% 分区0
\foreach \i in {0,1,2,3} {
    \fill[blue!30] (\i*2.5,2.5) rectangle (\i*2.5+0.5,3);
    \node[above,font=\tiny,blue] at (\i*2.5+0.25,3) {分区0-\i};
}

% 分区1
\foreach \i in {0,1,2,3} {
    \fill[green!30] (\i*2.5+0.6,2.5) rectangle (\i*2.5+1.1,3);
    \node[above,font=\tiny,green!60!black] at (\i*2.5+0.85,3) {分区1-\i};
}

% 箭头表示连续
\draw[->,thick,blue] (0.5,2.75) -- (2.5,2.75);
\draw[->,thick,blue] (3,2.75) -- (5,2.75);
\draw[->,thick,blue] (5.5,2.75) -- (7.5,2.75);

\draw[->,thick,green!60!black] (1.1,2.85) -- (3.1,2.85);
\draw[->,thick,green!60!black] (3.6,2.85) -- (5.6,2.85);
\draw[->,thick,green!60!black] (6.1,2.85) -- (8.1,2.85);

\node[below,text width=10cm,align=center] at (5,2) {
\textbf{关键:}每个分区内部是\textcolor{red}{连续的}!\\
隐状态可以跨批次传递
};

\end{tikzpicture}
\end{center}

\begin{theorem}[隐状态传递]
\begin{itemize}
    \item 批次$t$的最终隐状态 $\rightarrow$ 批次$t+1$的初始隐状态
    \item 保持了长期依赖关系
    \item 更符合RNN的设计理念
\end{itemize}
\end{theorem}


\paragraph{顺序分区实现}
\begin{lstlisting}
def seq_data_iter_sequential(corpus, batch_size, num_steps):
    """ sequence """
    offset = random.randint(0, num_steps)
    num_tokens = ((len(corpus) - offset - 1) // batch_size) * batch_size
    Xs = torch.tensor(corpus[offset: offset + num_tokens])
    Ys = torch.tensor(corpus[offset + 1: offset + 1 + num_tokens])
    
    # (batch_size, -1)
    Xs = Xs.reshape(batch_size, -1)
    Ys = Ys.reshape(batch_size, -1)
    
    num_batches = Xs.shape[1] // num_steps
    
    for i in range(0, num_steps * num_batches, num_steps):
        X = Xs[:, i: i + num_steps]
        Y = Ys[:, i: i + num_steps]
        yield X, Y
\end{lstlisting}


\paragraph{数据迭代器对比}
\begin{center}
\begin{tabular}{|l|c|c|}
\hline
\textbf{特性} & \textbf{随机采样} & \textbf{顺序分区} \\
\hline
批次间关系 & 独立 & 连续 \\
\hline
隐状态传递 & 不需要 & 需要 \\
\hline
实现复杂度 & 简单 & 中等 \\
\hline
训练效果 & 较好 & 更好 \\
\hline
适用场景 & 调试、快速实验 & 正式训练 \\
\hline
\end{tabular}
\end{center}

\vspace{0.5cm}
\begin{example}[参数设置建议]
\begin{itemize}
    \item \texttt{batch\_size}:32-64(取决于GPU内存)
    \item \texttt{num\_steps}:35-100(序列长度)
    \item 总样本数 = \texttt{batch\_size} $\times$ \texttt{num\_batches}
\end{itemize}
\end{example}


\paragraph{使用示例}
\begin{lstlisting}
# data
corpus, vocab = load_corpus_time_machine()

batch_size, num_steps = 32, 35

random_iter = seq_data_iter_random(corpus, batch_size, num_steps)
for X, Y in random_iter:
    print(f'X shape: {X.shape}, Y shape: {Y.shape}')
    break
# Output: X shape: torch.Size([32, 35]), Y shape: torch.Size([32, 35])

seq_iter = seq_data_iter_sequential(corpus, batch_size, num_steps)
for X, Y in seq_iter:
    print(f'X shape: {X.shape}, Y shape: {Y.shape}')
    # X[i, :]  Y[i, :] 
    # X[i, j]  Y[i, j]
    break
\end{lstlisting}

\begin{definition}[形状说明]
\begin{itemize}
    \item \texttt{X}: (batch\_size, num\_steps) - 输入序列
    \item \texttt{Y}: (batch\_size, num\_steps) - 目标序列
    \item \texttt{X[i, j]} 的目标是 \texttt{Y[i, j]} (即 \texttt{X[i, j+1]})
\end{itemize}
\end{definition}


\paragraph{数据批次的可视化}
\textbf{假设:}语料库 = ``hello world this is a test''

\textbf{参数:}\texttt{batch\_size=2}, \texttt{num\_steps=3}

\begin{center}
\begin{tikzpicture}[scale=0.9]
% 原始序列
\node[left] at (-0.5,3.5) {语料库:};
\foreach \i/\c in {0/h,1/e,2/l,3/l,4/o,5/{ },6/w,7/o,8/r,9/l,10/d,11/{ }} {
    \node[draw,minimum size=0.5cm] at (\i*0.6,3.5) {\tiny \c};
}

% 分成2个分区
\node[left] at (-0.5,2.5) {分区0:};
\foreach \i/\c in {0/h,1/e,2/l,3/l,4/o,5/{ }} {
    \node[draw,fill=blue!20,minimum size=0.5cm] at (\i*0.6,2.5) {\tiny \c};
}

\node[left] at (-0.5,1.8) {分区1:};
\foreach \i/\c in {0/w,1/o,2/r,3/l,4/d,5/{ }} {
    \node[draw,fill=green!20,minimum size=0.5cm] at (\i*0.6,1.8) {\tiny \c};
}

% 第一个批次
\node[left] at (-0.5,0.8) {批次1:};
\node[draw,fill=blue!30,minimum size=0.5cm] at (0,0.8) {\tiny h};
\node[draw,fill=blue!30,minimum size=0.5cm] at (0.6,0.8) {\tiny e};
\node[draw,fill=blue!30,minimum size=0.5cm] at (1.2,0.8) {\tiny l};
\node[right,font=\tiny] at (1.8,0.8) {X[0]};

\node[draw,fill=green!30,minimum size=0.5cm] at (3,0.8) {\tiny w};
\node[draw,fill=green!30,minimum size=0.5cm] at (3.6,0.8) {\tiny o};
\node[draw,fill=green!30,minimum size=0.5cm] at (4.2,0.8) {\tiny r};
\node[right,font=\tiny] at (4.8,0.8) {X[1]};

% 对应的Y
\node[left] at (-0.5,0) {目标:};
\node[draw,fill=blue!30,minimum size=0.5cm] at (0,0) {\tiny e};
\node[draw,fill=blue!30,minimum size=0.5cm] at (0.6,0) {\tiny l};
\node[draw,fill=blue!30,minimum size=0.5cm] at (1.2,0) {\tiny l};
\node[right,font=\tiny] at (1.8,0) {Y[0]};

\node[draw,fill=green!30,minimum size=0.5cm] at (3,0) {\tiny o};
\node[draw,fill=green!30,minimum size=0.5cm] at (3.6,0) {\tiny r};
\node[draw,fill=green!30,minimum size=0.5cm] at (4.2,0) {\tiny l};
\node[right,font=\tiny] at (4.8,0) {Y[1]};

\end{tikzpicture}
\end{center}

\begin{definition}[关键]
X的每个字符的下一个字符就是Y中对应位置的字符
\end{definition}


\paragraph{完整的数据加载函数}
\begin{lstlisting}
class SeqDataLoader:
    """sequencedata"""
    def __init__(self, batch_size, num_steps, use_random_iter, max_tokens):
        if use_random_iter:
            self.data_iter_fn = seq_data_iter_random
        else:
            self.data_iter_fn = seq_data_iter_sequential
        
        self.corpus, self.vocab = load_corpus_time_machine(max_tokens)
        self.batch_size, self.num_steps = batch_size, num_steps
    
    def __iter__(self):
        return self.data_iter_fn(self.corpus, self.batch_size, self.num_steps)

def load_data_time_machine(batch_size, num_steps, 
                           use_random_iter=False, max_tokens=10000):
    """Return时光机器data集的iteration器和vocabulary"""
    data_iter = SeqDataLoader(batch_size, num_steps, 
                              use_random_iter, max_tokens)
    return data_iter, data_iter.vocab

train_iter, vocab = load_data_time_machine(batch_size=32, num_steps=35)
\end{lstlisting}


\subsection{8.3节总结}
\begin{definition}[核心内容]
\begin{enumerate}
    \item \textbf{语言模型}:估计文本序列的概率分布
    $$P(x_1, \ldots, x_T) = \prod_{t=1}^{T} P(x_t \mid x_1, \ldots, x_{t-1})$$
    
    \item \textbf{困惑度}:评估语言模型的标准指标
    $$\text{PPL} = \exp\left(-\frac{1}{n}\sum_{t=1}^{n}\log P(x_t \mid x_{<t})\right)$$
    
    \item \textbf{马尔可夫假设}:简化长期依赖
    $$P(x_t \mid x_{<t}) \approx P(x_t \mid x_{t-\tau}, \ldots, x_{t-1})$$
    
    \item \textbf{数据采样}:随机采样 vs 顺序分区
\end{enumerate}
\end{definition}


\paragraph{关键概念对比}
\begin{center}
\begin{tabular}{|l|p{4cm}|p{4cm}|}
\hline
\textbf{概念} & \textbf{定义} & \textbf{作用} \\
\hline
语言模型 & 文本序列的概率分布 & 生成、评估文本 \\
\hline
困惑度 & $\exp(\text{负对数似然})$ & 评估模型质量 \\
\hline
N-gram & 前$n-1$个词预测 & 简单的语言模型 \\
\hline
Zipf定律 & 词频$\propto$排名$^{-1}$ & 理解词频分布 \\
\hline
随机采样 & 随机起始位置 & 快速训练 \\
\hline
顺序分区 & 保持连续性 & 更好效果 \\
\hline
\end{tabular}
\end{center}


\paragraph{为什么需要RNN?}
\begin{center}
\begin{tikzpicture}[scale=1.0]
% N-gram的局限
\node[draw,fill=red!20,text width=10cm] at (5,3.5) {
\textbf{N-gram模型的问题:}
\begin{itemize}
    \item 固定的上下文窗口
    \item 无法捕捉长期依赖
    \item 参数随$\tau$指数增长
    \item 数据稀疏问题
\end{itemize}
};

\draw[->,ultra thick] (5,2.7) -- (5,2.2);
\node[right] at (5.5,2.45) {需要改进};

% RNN的优势
\node[draw,fill=green!20,text width=10cm] at (5,1.2) {
\textbf{RNN的解决方案:}
\begin{itemize}
    \item 变长的上下文(隐状态)
    \item 能捕捉长期依赖
    \item 参数数量固定
    \item 共享参数,泛化能力强
\end{itemize}
};

\end{tikzpicture}
\end{center}

\begin{theorem}[下一步]
8.4节将详细介绍\textbf{循环神经网络(RNN)}的原理和实现
\end{theorem}


\paragraph{从统计到神经网络}
\begin{center}
\begin{tikzpicture}[scale=0.9]
% 时间线
\draw[->,ultra thick] (0,2) -- (11,2);

% N-gram
\node[draw,fill=yellow!20,text width=2cm,align=center] at (1,2) {
\small N-gram\\
1980s
};
\node[below,font=\tiny,text width=2.5cm,align=center] at (1,1.2) {
统计方法\\
数据稀疏
};

% 神经网络语言模型
\node[draw,fill=orange!20,text width=2cm,align=center] at (4,2) {
\small NNLM\\
2003
};
\node[below,font=\tiny,text width=2.5cm,align=center] at (4,1.2) {
前馈网络\\
固定窗口
};

% RNN
\node[draw,fill=blue!20,text width=2cm,align=center] at (7,2) {
\small RNN\\
2010s
};
\node[below,font=\tiny,text width=2.5cm,align=center] at (7,1.2) {
循环网络\\
变长上下文
};

% Transformer
\node[draw,fill=green!20,text width=2cm,align=center] at (10,2) {
\small Transformer\\
2017+
};
\node[below,font=\tiny,text width=2.5cm,align=center] at (10,1.2) {
注意力机制\\
当前主流
};

% 进化箭头
\foreach \x in {2,5,8} {
    \draw[->,thick,red] (\x,2.3) -- (\x+1.5,2.3);
}

\end{tikzpicture}
\end{center}


\section{8.4 循环神经网络}

\subsection{8.4 循环神经网络 - 章节导航}
\begin{definition}[本节内容]
\begin{itemize}
    \item 8.4.1 无隐状态的神经网络
    \item 8.4.2 有隐状态的循环神经网络
    \item 8.4.3 基于循环神经网络的字符级语言模型
    \item 8.4.4 困惑度(Perplexity)
\end{itemize}
\end{definition}

\begin{theorem}[学习目标]
\begin{enumerate}
    \item 理解为什么需要隐状态
    \item 掌握RNN的前向传播过程
    \item 了解RNN的参数共享机制
    \item 学会用RNN构建语言模型
\end{enumerate}
\end{theorem}



