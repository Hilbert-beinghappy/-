\section{未命名节}

\subsection{定义模型}

\subsection{8.6.1 定义模型}

\subsection{8.6.1 定义模型}
\begin{center}
\Large \textcolor{blue}{使用\texttt{nn.RNN}构建模型}
\end{center}


\paragraph{PyTorch的RNN模块}
\begin{lstlisting}
import torch
import torch.nn as nn

# RNN
rnn_layer = nn.RNN(
    input_size=vocab_size,    #  commentdimension(vocabulary大小)
    hidden_size=num_hiddens,  #  comment
    num_layers=1,             # RNN层数
    nonlinearity='tanh',      #  comment
    batch_first=False         #  commentshape (seq, batch, feature)
)

vocab_size, num_hiddens = 28, 512
rnn_layer = nn.RNN(vocab_size, num_hiddens)

print(rnn_layer)
# RNN(28, 512)
\end{lstlisting}

\begin{definition}[关键参数]
\begin{itemize}
    \item \texttt{input\_size}:每个时间步的输入特征数
    \item \texttt{hidden\_size}:隐藏状态的维度
    \item \texttt{num\_layers}:堆叠的RNN层数(默认1)
\end{itemize}
\end{definition}


\paragraph{RNN层的输入输出}
\begin{lstlisting}
batch_size, num_steps = 2, 5
X = torch.randn(num_steps, batch_size, vocab_size)  #  comment
H = torch.zeros(1, batch_size, num_hiddens)  #  comment

Y, H_new = rnn_layer(X, H)

print(f"输入shape: {X.shape}")      # (5, 2, 28)
print(f"Outputshape: {Y.shape}")      # (5, 2, 512)
print(f"隐状态shape: {H_new.shape}")  # (1, 2, 512)
\end{lstlisting}

\begin{definition}[形状说明]
\begin{itemize}
    \item 输入:$(num\_steps, batch\_size, input\_size)$
    \item 输出:$(num\_steps, batch\_size, hidden\_size)$
    \item 隐状态:$(num\_layers, batch\_size, hidden\_size)$
\end{itemize}
\end{definition}

\begin{theorem}[注意]
\texttt{batch\_first=False}时,第一个维度是时间步!
\end{theorem}


\paragraph{完整的RNN语言模型}
\begin{center}
\begin{tikzpicture}[scale=0.9]
% 模型结构
\node[draw,fill=yellow!20,rounded corners,text width=10cm] at (5,4) {
\textbf{输入层:}独热编码 $(T, B, V)$
};

\draw[->,thick] (5,3.6) -- (5,3.2);

\node[draw,fill=blue!20,rounded corners,text width=10cm] at (5,2.8) {
\textbf{RNN层:}处理序列 $(T, B, V) \rightarrow (T, B, H)$
};

\draw[->,thick] (5,2.4) -- (5,2);

\node[draw,fill=green!20,rounded corners,text width=10cm] at (5,1.6) {
\textbf{全连接层:}映射到词表 $(T \times B, H) \rightarrow (T \times B, V)$
};

\draw[->,thick] (5,1.2) -- (5,0.8);

\node[draw,fill=red!20,rounded corners,text width=10cm] at (5,0.4) {
\textbf{输出:}预测概率分布
};

\end{tikzpicture}
\end{center}


\paragraph{RNN语言模型类定义}
\begin{lstlisting}
class RNNModel(nn.Module):
    """RNNmodel"""
    
    def __init__(self, rnn_layer, vocab_size):
        super(RNNModel, self).__init__()
        self.rnn = rnn_layer
        self.vocab_size = vocab_size
        self.num_hiddens = self.rnn.hidden_size
        
        # Output -> vocabulary
        self.linear = nn.Linear(self.num_hiddens, self.vocab_size)
    
    def forward(self, inputs, state):
        """
        inputs: (batch_size, num_steps) 索引
        state: 隐状态
        """
        X = F.one_hot(inputs.T, self.vocab_size).type(torch.float32)
        # X: (num_steps, batch_size, vocab_size)
        
        # RNN
        Y, state = self.rnn(X, state)
        # Y: (num_steps, batch_size, num_hiddens)
        
        output = self.linear(Y.reshape(-1, Y.shape[-1]))
        # output: (num_steps * batch_size, vocab_size)
        
        return output, state
\end{lstlisting}


\paragraph{初始化隐状态}
\begin{lstlisting}
    def begin_state(self, batch_size, device):
        """Simple function"""
        if not isinstance(self.rnn, nn.LSTM):
            # RNNGRU
            return torch.zeros(
                (self.rnn.num_layers, batch_size, self.num_hiddens),
                device=device
            )
        else:
            # LSTMhc
            return (
                torch.zeros(
                    (self.rnn.num_layers, batch_size, self.num_hiddens),
                    device=device
                ),
                torch.zeros(
                    (self.rnn.num_layers, batch_size, self.num_hiddens),
                    device=device
                )
            )
\end{lstlisting}

\begin{definition}[注意]
\begin{itemize}
    \item 隐状态形状:$(num\_layers, batch\_size, hidden\_size)$
    \item 第一维是层数(支持多层RNN)
\end{itemize}
\end{definition}


\paragraph{创建模型实例}
\begin{lstlisting}
# parameters
vocab_size = 28
num_hiddens = 512
device = torch.device('cuda' if torch.cuda.is_available() else 'cpu')

# RNN
rnn_layer = nn.RNN(vocab_size, num_hiddens)

# model
net = RNNModel(rnn_layer, vocab_size)
net = net.to(device)

# model
print(net)

# Output
# RNNModel(
#   (rnn): RNN(28, 512)
#   (linear): Linear(in_features=512, out_features=28, bias=True)
# )

# parameters
num_params = sum(p.numel() for p in net.parameters())
print(f"parameters数量: {num_params}")
# parameters: 291356
\end{lstlisting}


\paragraph{简洁实现 vs 从零实现对应关系}
\begin{center}
\begin{tikzpicture}[scale=0.85]
% 从零实现
\node[above] at (2,4) {\small 从零实现(8.5)};
\node[draw,fill=yellow!20,text width=4cm,font=\small] at (2,3) {
\texttt{W\_xh, W\_hh, b\_h}\\
\texttt{W\_hq, b\_q}\\
手动初始化参数
};

\node[draw,fill=orange!20,text width=4cm,font=\small] at (2,1.8) {
\texttt{rnn(inputs, state, params)}\\
手写前向传播
};

\node[draw,fill=green!20,text width=4cm,font=\small] at (2,0.6) {
\texttt{init\_rnn\_state()}\\
手动初始化隐状态
};

% 简洁实现
\node[above] at (7.5,4) {\small 简洁实现(8.6)};
\node[draw,fill=blue!20,text width=4cm,font=\small] at (7.5,3) {
\texttt{nn.RNN()}\\
自动管理参数
};
\node[draw,fill=blue!20,text width=4cm,font=\small] at (7.5,1.8) {
\texttt{rnn\_layer(X, state)}\\
自动前向传播
};

\node[draw,fill=blue!20,text width=4cm,font=\small] at (7.5,0.6) {
\texttt{torch.zeros(...)}\\
标准初始化
};

% 对应箭头
\draw[<->,thick,red] (4.2,3) -- (5.3,3);
\draw[<->,thick,red] (4.2,1.8) -- (5.3,1.8);
\draw[<->,thick,red] (4.2,0.6) -- (5.3,0.6);

\node[below,text width=9cm,align=center] at (4.8,-0.5) {
\small 简洁实现封装了从零实现的所有细节
};

\end{tikzpicture}
\end{center}


\paragraph{PyTorch RNN的参数管理}
\textbf{查看RNN层的参数:}

\begin{center}
\begin{tikzpicture}[scale=0.9]
\node[draw,fill=yellow!20,text width=10cm,font=\small] at (5,3) {
\textbf{nn.RNN内部参数(自动管理):}\\
- \texttt{weight\_ih\_l0}: 输入到隐藏 $(input\_size, hidden\_size)$\\
- \texttt{weight\_hh\_l0}: 隐藏到隐藏 $(hidden\_size, hidden\_size)$\\
- \texttt{bias\_ih\_l0}: 输入偏置 $(hidden\_size,)$\\
- \texttt{bias\_hh\_l0}: 隐藏偏置 $(hidden\_size,)$
};

\draw[->,thick] (5,2.4) -- (5,2);

\node[draw,fill=green!20,text width=10cm,font=\small] at (5,1.3) {
\textbf{对应关系:}\\
\texttt{weight\_ih} $\approx$ $\mathbf{W}_{xh}^T$ (转置!)\\
\texttt{weight\_hh} $\approx$ $\mathbf{W}_{hh}^T$\\
\texttt{bias\_ih + bias\_hh} $\approx$ $\mathbf{b}_h$
};

\end{tikzpicture}
\end{center}

\begin{theorem}[注意]
PyTorch使用转置的权重矩阵以优化计算
\end{theorem}


\paragraph{查看模型参数}
\begin{lstlisting}
# RNNparameters
for name, param in net.rnn.named_parameters():
    print(f"{name}: {param.shape}")

# Output
# weight_ih_l0: torch.Size([512, 28])
# weight_hh_l0: torch.Size([512, 512])
# bias_ih_l0: torch.Size([512])
# bias_hh_l0: torch.Size([512])

# parameters
for name, param in net.linear.named_parameters():
    print(f"{name}: {param.shape}")

# Output
# weight: torch.Size([28, 512])
# bias: torch.Size([28])
\end{lstlisting}

\begin{definition}[参数总量]
$(512 \times 28) + (512 \times 512) + 512 + 512 + (28 \times 512) + 28 = 291,356$
\end{definition}


\paragraph{多层RNN}
\textbf{堆叠多个RNN层:}

\begin{center}
\begin{tikzpicture}[scale=0.9]
% 单层
\node[left] at (-0.5,3) {\small 单层};
\foreach \i in {0,1,2,3} {
    \node[circle,draw,fill=blue!20,minimum size=0.6cm] at (\i*1.5,3) {};
}
\node[above,font=\tiny] at (1.5,3.4) {RNN Layer 1};

% 双层
\node[left] at (-0.5,1.5) {\small 双层};
\foreach \i in {0,1,2,3} {
    \node[circle,draw,fill=blue!20,minimum size=0.6cm] (l1\i) at (\i*1.5,1.5) {};
    \node[circle,draw,fill=green!20,minimum size=0.6cm] (l2\i) at (\i*1.5,0.5) {};
}
\node[above,font=\tiny] at (1.5,1.9) {Layer 2};
\node[above,font=\tiny] at (1.5,0.9) {Layer 1};

% 连接
\foreach \i in {0,1,2,3} {
    \draw[->,thick] (l2\i) -- (l1\i);
}
\foreach \i in {0,1,2} {
    \pgfmathtruncatemacro{\next}{\i+1}
    \draw[->,thick,blue] (l1\i) -- (l1\next);
    \draw[->,thick,blue] (l2\i) -- (l2\next);
}

\end{tikzpicture}
\end{center}


\paragraph{创建多层RNN}
\begin{lstlisting}
# RNN
rnn_single = nn.RNN(vocab_size, num_hiddens, num_layers=1)

# RNN
rnn_double = nn.RNN(vocab_size, num_hiddens, num_layers=2)

# RNN
rnn_triple = nn.RNN(vocab_size, num_hiddens, num_layers=3)

# parameters
print(f"单层parameters: {sum(p.numel() for p in rnn_single.parameters())}")
print(f"双层parameters: {sum(p.numel() for p in rnn_double.parameters())}")
print(f"三层parameters: {sum(p.numel() for p in rnn_triple.parameters())}")

# Output
# parameters: 277504
# parameters: 540160  ()
# parameters: 802816
\end{lstlisting}

\begin{theorem}[注意]
层数越多,参数越多,但也更容易过拟合
\end{theorem}



\subsection{训练与预测}

\subsection{8.6.2 训练与预测}

\subsection{8.6.2 训练与预测}
\begin{center}
\Large \textcolor{blue}{使用简洁实现训练RNN}
\end{center}


\paragraph{训练流程对比}
\begin{center}
\begin{tikzpicture}[scale=0.8]
% 从零实现
\node[above] at (2,4.5) {\small 从零实现};
\node[draw,fill=yellow!20,text width=4cm,font=\tiny] at (2,3.5) {
手动独热编码\\
手动矩阵乘法\\
手动梯度裁剪\\
自定义优化器
};

% 简洁实现
\node[above] at (8,4.5) {\small 简洁实现};
\node[draw,fill=green!20,text width=4cm,font=\tiny] at (8,3.5) {
\texttt{F.one\_hot()}自动编码\\
\texttt{rnn\_layer()}自动计算\\
\texttt{nn.utils.clip\_grad}裁剪\\
\texttt{torch.optim}优化器
};

% 共同部分
\node[draw,fill=blue!20,text width=9cm] at (5,2) {
\textbf{共同点:}\\
- 相同的损失函数(交叉熵)\\
- 相同的训练循环\\
- 相同的评估指标(困惑度)
};

% 箭头
\draw[->,thick] (2,3) -- (2,2.5);
\draw[->,thick] (8,3) -- (8,2.5);

\end{tikzpicture}
\end{center}


\paragraph{训练一个epoch(简洁版)}
\begin{lstlisting}
def train_epoch_ch8(net, train_iter, loss, updater, device, 
                    use_random_iter):
    """trainingepoch"""
    state = None
    metric = Accumulator(2)
    
    for X, Y in train_iter:
        if state is None or use_random_iter:
            state = net.begin_state(batch_size=X.shape[0], device=device)
        else:
            if isinstance(net, nn.Module) and not isinstance(state, tuple):
                state.detach_()
            else:
                state = tuple(s.detach() for s in state)
        
        y = Y.T.reshape(-1)
        X, y = X.to(device), y.to(device)
        y_hat, state = net(X, state)
        
        # loss
        l = loss(y_hat, y).mean()
        
        updater.zero_grad()
        l.backward()
        grad_clipping(net, 1)  # comment
        updater.step()
        
        metric.add(l * y.numel(), y.numel())
    
    return math.exp(metric[0] / metric[1])
\end{lstlisting}


\paragraph{梯度裁剪(简洁版)}
\begin{lstlisting}
def grad_clipping(net, theta):
    """PyTorch"""
    if isinstance(net, nn.Module):
        # 1clip_grad_norm_
        params = [p for p in net.parameters() if p.requires_grad]
        norm = torch.nn.utils.clip_grad_norm_(params, theta)
    else:
        # 2
        norm = torch.sqrt(sum(torch.sum((p.grad ** 2)) 
                             for p in net.params))
        if norm > theta:
            for param in net.params:
                param.grad[:] *= theta / norm
    return norm
\end{lstlisting}

\begin{definition}[PyTorch内置方法]
\texttt{torch.nn.utils.clip\_grad\_norm\_}自动处理所有参数的梯度裁剪
\end{definition}


\paragraph{完整训练函数}
\begin{lstlisting}
def train_ch8(net, train_iter, vocab, lr, num_epochs, device,
              use_random_iter=False):
    """trainingmodel()"""
    loss = nn.CrossEntropyLoss()
    
    # optimizer
    if isinstance(net, nn.Module):
        updater = torch.optim.SGD(net.parameters(), lr)
    else:
        updater = lambda batch_size: sgd(net.params, lr, batch_size)
    
    # prediction
    predict = lambda prefix: predict_ch8(prefix, 50, net, vocab, device)
    
    # training
    for epoch in range(num_epochs):
        ppl = train_epoch_ch8(net, train_iter, loss, updater,
                             device, use_random_iter)
        
        if (epoch + 1) % 10 == 0:
            print(f'epoch {epoch + 1}, 困惑度 {ppl:.1f}')
            print(predict('time traveller'))
    
    print(f'困惑度 {ppl:.1f}')
    print(predict('time traveller'))
    print(predict('traveller'))
\end{lstlisting}


\paragraph{运行训练}
\begin{lstlisting}
# parameters
num_epochs = 500
lr = 1
batch_size = 32
num_steps = 35
num_hiddens = 512

# data
train_iter, vocab = load_data_time_machine(batch_size, num_steps)

# model
device = torch.device('cuda' if torch.cuda.is_available() else 'cpu')
rnn_layer = nn.RNN(len(vocab), num_hiddens)
net = RNNModel(rnn_layer, len(vocab))
net = net.to(device)

# training
train_ch8(net, train_iter, vocab, lr, num_epochs, device)
\end{lstlisting}

\textbf{训练输出:}
\begin{verbatim}
epoch 10, 困惑度 15.3
time traveller the the the the the the...
epoch 500, 困惑度 1.2
time traveller smiled round at us then still smiling
\end{verbatim}


\paragraph{简洁实现的优势}
\begin{center}
\begin{tikzpicture}[scale=0.9]
% 优势1
\node[draw,fill=green!20,rounded corners,text width=10cm] at (5,3.8) {
\textbf{1. 代码简洁}\\
RNN定义只需1行:\texttt{nn.RNN(input\_size, hidden\_size)}\\
vs 从零实现需要30+行
};

% 优势2
\node[draw,fill=blue!20,rounded corners,text width=10cm] at (5,2.5) {
\textbf{2. 高度优化}\\
- 使用cuDNN加速(GPU上)\\
- 内存优化的反向传播\\
- 自动混合精度支持
};

% 优势3
\node[draw,fill=orange!20,rounded corners,text width=10cm] at (5,1.2) {
\textbf{3. 功能丰富}\\
- 双向RNN:\texttt{bidirectional=True}\\
- Dropout:\texttt{dropout=0.5}\\
- 多层堆叠:\texttt{num\_layers=3}
};

\end{tikzpicture}
\end{center}


\paragraph{双向RNN}
\begin{lstlisting}
# RNN
rnn_layer = nn.RNN(
    input_size=vocab_size, 
    hidden_size=num_hiddens,
    bidirectional=True  #  comment
)

# Outputdimension 2 * hidden_size
print(f"隐藏层Outputdimension: {rnn_layer.hidden_size * 2}")
# Outputdimension: 1024

# dimension
net.linear = nn.Linear(num_hiddens * 2, vocab_size)
\end{lstlisting}

\begin{definition}[双向RNN原理]
\begin{itemize}
    \item 同时从前向后和从后向前处理序列
    \item 捕捉双向的上下文信息
    \item 适合分类任务,不适合生成任务
\end{itemize}
\end{definition}


\paragraph{添加Dropout}
\begin{lstlisting}
# DropoutRNN
rnn_layer = nn.RNN(
    input_size=vocab_size, 
    hidden_size=num_hiddens,
    num_layers=2,      #  comment
    dropout=0.5        # Dropout比例
)

# Dropout
# 1 -> Dropout -> 2
\end{lstlisting}

\begin{center}
\begin{tikzpicture}[scale=0.8]
\node[draw,fill=blue!20,minimum size=0.8cm] (l1) at (0,2) {Layer 1};
\node[draw,fill=red!20,minimum size=0.8cm] (d) at (0,1) {Dropout};
\node[draw,fill=blue!20,minimum size=0.8cm] (l2) at (0,0) {Layer 2};

\draw[->,thick] (l1) -- (d);
\draw[->,thick] (d) -- (l2);

\node[right,text width=4cm,font=\small] at (1.5,1) {
随机丢弃部分神经元\\
防止过拟合
};
\end{tikzpicture}
\end{center}


\paragraph{性能对比}
\begin{center}
\begin{tabular}{|l|c|c|c|}
\hline
\textbf{指标} & \textbf{从零实现} & \textbf{简洁实现} & \textbf{对比} \\
\hline
代码行数 & $\sim$200 & $\sim$50 & 4倍 \\
\hline
训练速度 & 5000 词/秒 & 12000 词/秒 & 2.4倍 \\
\hline
GPU加速 & 手动优化 & cuDNN自动 & 更快 \\
\hline
内存占用 & 较高 & 优化 & 更少 \\
\hline
易用性 & 复杂 & 简单 & 更易 \\
\hline
可定制性 & 高 & 中等 & 受限 \\
\hline
\end{tabular}
\end{center}

\vspace{0.5cm}
\begin{theorem}[建议]
\begin{itemize}
    \item 学习阶段:从零实现(理解原理)
    \item 实际项目:简洁实现(高效开发)
\end{itemize}
\end{theorem}


\paragraph{预测函数(通用版)}
\begin{lstlisting}
def predict_ch8(prefix, num_preds, net, vocab, device):
    """prediction()"""
    state = net.begin_state(batch_size=1, device=device)
    outputs = [vocab[prefix[0]]]
    
    get_input = lambda: torch.tensor([outputs[-1]], device=device
                                     ).reshape((1, 1))
    
    for y in prefix[1:]:
        _, state = net(get_input(), state)
        outputs.append(vocab[y])
    
    # prediction
    for _ in range(num_preds):
        y, state = net(get_input(), state)
        outputs.append(int(y.argmax(dim=1).reshape(1)))
    
    return ''.join([vocab.idx_to_token[i] for i in outputs])

print(predict_ch8('time traveller', 50, net, vocab, device))
\end{lstlisting}


\paragraph{简洁实现的完整流程}
\begin{center}
\begin{tikzpicture}[scale=0.85]
% 流程
\node[draw,fill=yellow!20,rounded corners,text width=9cm] at (5,4.5) {
\textbf{1. 导入库}\\
\texttt{import torch.nn as nn}
};

\draw[->,thick] (5,4.1) -- (5,3.7);

\node[draw,fill=blue!20,rounded corners,text width=9cm] at (5,3.3) {
\textbf{2. 创建RNN层}\\
\texttt{rnn\_layer = nn.RNN(vocab\_size, num\_hiddens)}
};

\draw[->,thick] (5,2.9) -- (5,2.5);

\node[draw,fill=green!20,rounded corners,text width=9cm] at (5,2.1) {
\textbf{3. 封装为模型}\\
\texttt{net = RNNModel(rnn\_layer, vocab\_size)}
};

\draw[->,thick] (5,1.7) -- (5,1.3);

\node[draw,fill=orange!20,rounded corners,text width=9cm] at (5,0.9) {
\textbf{4. 训练}\\
\texttt{train\_ch8(net, train\_iter, vocab, lr, epochs, device)}
};

\draw[->,thick] (5,0.5) -- (5,0.1);

\node[draw,fill=red!20,rounded corners,text width=9cm] at (5,-0.3) {
\textbf{5. 预测}\\
\texttt{predict\_ch8('prefix', 50, net, vocab, device)}
};

\end{tikzpicture}
\end{center}


\paragraph{实用技巧}
\begin{definition}[1. 模型保存与加载]
\begin{lstlisting}[basicstyle=\ttfamily\tiny]
torch.save(net.state_dict(), 'rnn_model.pth')

net.load_state_dict(torch.load('rnn_model.pth'))
net.eval()  #  comment
\end{lstlisting}
\end{definition}

\begin{definition}[2. 使用更好的优化器]
\begin{lstlisting}[basicstyle=\ttfamily\tiny]
# Adamoptimizer
optimizer = torch.optim.Adam(net.parameters(), lr=0.001)

# learning rate
scheduler = torch.optim.lr_scheduler.StepLR(optimizer, step_size=100, gamma=0.1)
\end{lstlisting}
\end{definition}

\begin{definition}[3. 早停(Early Stopping)]
\begin{lstlisting}[basicstyle=\ttfamily\tiny]
best_ppl = float('inf')
patience = 0

if val_ppl < best_ppl:
    best_ppl = val_ppl
    torch.save(net.state_dict(), 'best_model.pth')
    patience = 0
else:
    patience += 1
    if patience > 20:
        break  #  comment
\end{lstlisting}
\end{definition}

\section{8.7 通过时间反向传播}

\subsection{8.7 通过时间反向传播(BPTT)}
\begin{center}
\Large \textcolor{blue}{深入理解RNN的梯度计算}
\end{center}


\subsection{8.7节 - 章节导航}
\begin{definition}[本节内容]
\begin{itemize}
    \item 8.7.1 循环神经网络的梯度分析
    \item 8.7.2 通过时间反向传播的细节
\end{itemize}
\end{definition}

\begin{theorem}[学习目标]
\begin{enumerate}
    \item 理解BPTT算法的原理
    \item 掌握RNN梯度的数学推导
    \item 理解梯度消失/爆炸的根源
    \item 了解截断BPTT的必要性
\end{enumerate}
\end{theorem}


\paragraph{为什么需要学习BPTT?}
\begin{center}
\begin{tikzpicture}[scale=0.9]
% 问题
\node[draw,fill=red!20,rounded corners,text width=10cm] at (5,3.5) {
\textbf{RNN训练中的问题:}\\
$\times$ 梯度爆炸:损失突然变为NaN\\
$\times$ 梯度消失:长期依赖学不到\\
$\times$ 训练不稳定:需要仔细调参
};

\draw[->,ultra thick] (5,3) -- (5,2.5);
\node[right] at (5.5,2.75) {为什么?};

% 原因
\node[draw,fill=blue!20,rounded corners,text width=10cm] at (5,1.8) {
\textbf{根源:RNN的梯度计算方式}\\
通过时间反向传播(BPTT)导致梯度在长序列上累积或衰减
};

\draw[->,ultra thick] (5,1.3) -- (5,0.8);

% 解决
\node[draw,fill=green!20,rounded corners,text width=10cm] at (5,0.3) {
\textbf{理解BPTT后才能:}\\
$\checkmark$ 知道为何需要梯度裁剪\\
$\checkmark$ 理解LSTM的设计动机
};

\end{tikzpicture}
\end{center}



