\section{从零开始实现循环神经网络}


\section{8.5 从零开始实现循环神经网络}

\paragraph{从零实现的意义}
\begin{center}
\begin{tikzpicture}[scale=0.9]
% 两种方式
\node[draw,fill=yellow!20,rounded corners,text width=5cm,align=center] at (0,2) {
\textbf{使用框架}\\
\texttt{nn.RNN()}\\
简单快速\\
但细节不清楚
};

\node[draw,fill=green!20,rounded corners,text width=5cm,align=center] at (7,2) {
\textbf{从零实现}\\
手写每一行代码\\
理解每个细节\\
知其所以然
};

\draw[->,ultra thick,blue] (2.7,2) -- (4.3,2);
\node[above,blue] at (3.5,2.3) {本节目标};

\end{tikzpicture}
\end{center}

\begin{definition}[你将学会]
\begin{itemize}
    \item RNN的完整工作流程
    \item 梯度计算和反向传播
    \item 序列建模的技巧和tricks
    \item 为学习更复杂的模型打下基础
\end{itemize}
\end{definition}


\paragraph{One-Hot编码的优势}
\begin{definition}[为什么用One-Hot?]
\begin{itemize}
    \item 简单直观
    \item 无大小关系(不会让模型认为'a' < 'b')
    \item 与序列长度无关(参数量与序列长度无关)
\end{itemize}
\end{definition}

\begin{theorem}[与词级别的区别]
\begin{itemize}
    \item 词级别:通常几千到几万个词
    \item 字符级别:通常几十个字符
    \item 字符级别模型参数更少,训练更快
\end{itemize}
\end{theorem}


\paragraph{One-Hot编码实现}
\begin{lstlisting}[language=Python]
import torch.nn.functional as F

def get_one_hot(indices, vocab_size):
    """ one-hotvector indices: shape (N,) vocab_size: """
    one_hot = torch.zeros(len(indices), vocab_size)

    # 1
    one_hot.scatter_(1, indices.unsqueeze(1), 1)

    return one_hot

indices = torch.tensor([0, 2, 4])  # 'a', 'c', 'e'
one_hot = get_one_hot(indices, vocab_size=5)
print(one_hot)
\end{lstlisting}

\begin{definition}[输出结果]
\begin{verbatim}
tensor([[1., 0., 0., 0., 0.],  # 'a'
        [0., 0., 1., 0., 0.],  # 'c'
        [0., 0., 0., 0., 1.]]) # 'e'
\end{verbatim}
\end{definition}

\paragraph{独热编码的维度转换}
\textbf{问题:}字符是离散符号,需要转为向量

\begin{definition}[独热编码:]
输入:形状为 (batch\_size, num\_steps) 的索引 \\
输出:形状为 (num\_steps, batch\_size, vocab\_size) 的one-hot张量
\end{definition}



\begin{center}
\begin{tikzpicture}[scale=0.7]
% 字符表
\node[left] at (0,2) {词表长度为 $V$ 的向量表};

% 示例
\node[draw,fill=blue!20] at (1,1) {hello};
\draw[->,thick] (1.5,1) -- (2.5,1);

% one-hot表示
\foreach \i in {0,...,4} {
    \node[draw,minimum size=0.3cm] at (\i*0.4,0) {\tiny 0};
}
\node[draw,fill=red!20,minimum size=0.3cm] at (0,0) {\tiny 1};

\node[below] at (1,-0.5) {\small 长度为 $V$ 的向量,只有一个位置为1};
\end{tikzpicture}
\end{center}


\begin{itemize}
\item 每个字符对应一个位置
\item 便于神经网络处理
\item 但向量维度很高
\end{itemize}


\paragraph{批处理独热编码}
\begin{lstlisting}[language=Python]
import torch

# data
batch_size, num_steps, vocab_size = 2, 3, 28
X = torch.tensor([[1, 2, 3], [4, 5, 6]])  # (2, 3)

X_one_hot = get_one_hot(X, vocab_size)
print(f"shape: {X_one_hot.shape}")  # (3, 2, 28)
\end{lstlisting}

\begin{theorem}[注意]
转置是为了方便RNN按时间步处理:$(T, N, V)$
\end{theorem}

\begin{center}
\begin{tikzpicture}[scale=0.8]
% 原始数据
\node[draw,fill=blue!20] at (0,2) {$X$ (2,3)};
\node[below] at (0,1.5) {\small 批量索引};

% 转换过程
\draw[->,thick] (0.5,2) -- (2,2);
\node[above] at (1.25,2.1) {\small one-hot};

% 结果
\node[draw,fill=green!20] at (4,2) {$X_{one-hot}$ (3,2,28)};
\node[below] at (4,1.5) {\small 批量向量};
\end{tikzpicture}
\end{center}


\paragraph{RNN语言模型的完整流程}
\begin{itemize}
    \item \textbf{步骤1: 数据准备} - 文本 $\rightarrow$ 字符索引 $\rightarrow$ One-hot向量
    \item \textbf{步骤2: RNN计算} - 循环神经网络处理序列
    \item \textbf{步骤3: 输出预测} - 计算下一个字符的概率分布
    \item \textbf{步骤4: 采样} - 根据概率选择下一个字符
\end{itemize}


\paragraph{步骤3: 输出预测}
\begin{itemize}
    \item 向输出层:$\mathbf{O}_t = \mathbf{H}_t \mathbf{W}_{hq} + \mathbf{b}_q$
    \item 输出维度 = 词表大小
    \item 稀疏表示(未归一化的logits)
\end{itemize}


\paragraph{步骤4: 采样}
根据概率分布选择下一个字符



\paragraph{独热编码实现}
\begin{lstlisting}[language=python]
import torch
import torch.nn.functional as F

def one_hot(indices, vocab_size):
    """ indices: (batch_size, num_steps) Return: (batch_size, num_steps, vocab_size) """
    return F.one_hot(indices, vocab_size).float()

# comment
vocab_size = 28  # 26个字母 + 空格 + 其他
batch_size, num_steps = 2, 5

# comment
X = torch.randint(0, vocab_size, (batch_size, num_steps))
print(f"索引形状: {X.shape}")  # (2, 5)

# comment
X_one_hot = one_hot(X, vocab_size)
print(f"独热编码形状: {X_one_hot.shape}")  # (2, 5, 28)
\end{lstlisting}


\paragraph{完整的RNN模型}
\begin{lstlisting}
class RNNModelScratch:
    def __init__(self, vocab_size, num_hiddens):
        self.vocab_size = vocab_size
        self.num_hiddens = num_hiddens
        
        #  comment
        self.params = self.init_params()
    
    def init_params(self):
        """Simple function"""
        def normal(shape):
            return torch.randn(size=shape) * 0.01
        
        # comment
        W_xh = normal((self.vocab_size, self.num_hiddens))
        # comment
        W_hh = normal((self.num_hiddens, self.num_hiddens))
        # comment
        b_h = torch.zeros(self.num_hiddens)
        # comment
        W_hq = normal((self.num_hiddens, self.vocab_size))
        # comment
        b_q = torch.zeros(self.vocab_size)
        
        # comment
        params = [W_xh, W_hh, b_h, W_hq, b_q]
        for param in params:
            param.requires_grad_(True)
        
        return params
\end{lstlisting}


\paragraph{完整的字符级RNN语言模型}
\begin{center}
\begin{tikzpicture}[scale=0.9]
% RNN流程图
\node[draw,fill=yellow!20,rounded corners,text width=10cm,align=center] at (5,3) {
\textbf{输入:}字符序列 $\to$ 独热编码
};

\draw[->,thick] (5,2.5) -- (5,2);

\node[draw,fill=blue!20,rounded corners,text width=10cm,align=center] at (5,1.5) {
\textbf{RNN层:}循环更新隐状态\\
$\mathbf{H}_t = \tanh(\mathbf{X}_t \mathbf{W}_{xh} + \mathbf{H}_{t-1}\mathbf{W}_{hh} + \mathbf{b}_h)$
};

\draw[->,thick] (5,1) -- (5,0.5);

\node[draw,fill=green!20,rounded corners,text width=10cm,align=center] at (5,0) {
\textbf{输出层:}预测下一个字符\\
$\mathbf{O}_t = \mathbf{H}_t \mathbf{W}_{hq} + \mathbf{b}_q$
};

\end{tikzpicture}
\end{center}

\begin{theorem}[目标]
最大化正确字符的预测概率
\end{theorem}


\paragraph{字符级模型的优缺点}


\textcolor{blue}{\textbf{优点:}}
\begin{itemize}
    \item $\checkmark$ 词表很小($\sim$100)
    \item $\checkmark$ 无OOV问题
    \item $\checkmark$ 可以生成任意词
    \item $\checkmark$ 适合演示和教学
\end{itemize}

\textbf{适用场景:}
\begin{itemize}
    \item 拼写纠错
    \item 文本生成
    \item DNA序列分析
\end{itemize}


\textcolor{red}{\textbf{缺点:}}
\begin{itemize}
    \item $\times$ 序列很长
    \item $\times$ 训练慢
    \item $\times$ 需要更多计算
    \item $\times$ 难以捕捉长期依赖
\end{itemize}

\textbf{实际应用:}
\begin{itemize}
    \item 现代NLP多用词级或子词级
\end{itemize}



\paragraph{训练循环}
\begin{lstlisting}
def train_epoch(model, train_iter, loss_fn, optimizer):
    """epoch"""
    total_loss = 0
    num_batches = 0
    
    for X, Y in train_iter:
        # X, Y: (batch_size, num_steps)
        
        # One-hot编码
        inputs = get_one_hot(X, model.vocab_size)
        # inputs: (num_steps, batch_size, vocab_size)
        
        # comment
        state = model.init_state(X.shape[0])
        
        # comment
        outputs, state = model.forward(inputs, state)
        # outputs: (num_steps * batch_size, vocab_size)
        
        # comment
        Y = Y.T.reshape(-1)  # (num_steps * batch_size,)
        loss = loss_fn(outputs, Y)
        
        # comment
        optimizer.zero_grad()
        loss.backward()
        # comment
        grad_clipping(model.params, 1)
        optimizer.step()
        
        total_loss += loss.item()
        num_batches += 1
    
    return total_loss / num_batches
\end{lstlisting}


\paragraph{梯度裁剪的重要性}
\textbf{问题:}RNN容易出现梯度爆炸

\begin{center}
\begin{tikzpicture}[scale=1.0]
% 梯度爆炸
\draw[->,thick] (0,0) -- (6,0) node[right] {训练步数};
\draw[->,thick] (0,0) -- (0,3) node[above] {梯度范数};

% 爆炸曲线
\draw[red,thick] (0.5,0.5) -- (1,0.6) -- (1.5,0.8) -- (2,1.5) -- (2.5,2.8);
\node[red,above] at (2.5,2.8) {爆炸!};

% 裁剪后
\draw[blue,thick] (3,0.5) -- (3.5,0.6) -- (4,0.8) -- (4.5,1.0) -- (5,1.0) -- (5.5,0.9);
\draw[dashed,green!60!black] (0,1) -- (6,1);
\node[green!60!black,left] at (0,1) {阈值};
\node[blue,above] at (5,1.2) {裁剪后稳定};

\end{tikzpicture}
\end{center}

\begin{definition}[梯度裁剪]
$$\mathbf{g} \leftarrow \min\left(1, \frac{\theta}{\|\mathbf{g}\|}\right) \mathbf{g}$$
其中 $\theta$ 是阈值(通常取1或5)
\end{definition}


\paragraph{困惑度与交叉熵的关系}

\begin{center}
\begin{tikzpicture}[scale=1.0]
% 坐标轴
\draw[->,thick] (0,0) -- (6,0) node[right] {交叉熵};
\draw[->,thick] (0,0) -- (0,4) node[above] {困惑度};

% 指数曲线
\draw[blue,thick,domain=0:3.5,samples=100] plot (\x,{exp(\x*0.6)});

% 标注点
\fill[red] (1,1.82) circle (2pt);
\node[red,above right] at (1,1.82) {CE=1, PPL$\approx$ 2-3};

% 高困惑度
\fill[red] (2,3.32) circle (2pt);
\node[red,above right] at (2,3.32) {CE=2, PPL$=e^2\approx$7.4};

% 公式
\node[below,text width=6cm,align=center] at (3,-0.8) {
$\text{PPL} = \exp(\text{CE})$
};

\end{tikzpicture}
\end{center}

\begin{theorem}[注意]
困惑度对交叉熵的微小差异很敏感(指数关系)
\end{theorem}


\paragraph{困惑度的意义}
\begin{definition}[直观理解]
困惑度告诉我们:在每一步,模型平均在多少个选项之间``困惑''
\end{definition}

\begin{example}[示例]
\begin{itemize}
    \item PPL = 2:模型在2个选项间困惑
    \item PPL = 28:模型在28个字符间困惑(相当于随机猜)
    \item PPL = 1.5:模型比较确定
\end{itemize}
\end{example}


\paragraph{计算困惑度}
\begin{lstlisting}
def evaluate_perplexity(model, data_iter):
    """modeldata"""
    model.eval()
    total_loss = 0
    total_tokens = 0
    
    with torch.no_grad():
        for X, Y in data_iter:
            # One-hot编码
            inputs = get_one_hot(X, model.vocab_size)
            state = model.init_state(X.shape[0])
            
            outputs, state = model.forward(inputs, state)
            # output: (num_steps * batch_size, vocab_size)
            
            # loss
            Y = Y.T.reshape(-1)
            loss = F.cross_entropy(outputs, Y, reduction='sum')
            
            total_loss += loss.item()
            total_tokens += Y.numel()
    
    # comment = exp(平均交叉熵)
    perplexity = math.exp(total_loss / total_tokens)
    return perplexity

# comment
ppl = evaluate_perplexity(model, test_iter)
print(f"test集困惑度: {ppl:.2f}")
\end{lstlisting}


\paragraph{困惑度的实际意义}
\textbf{困惑度 = 平均分支因子}

\begin{center}
\begin{tikzpicture}[scale=0.9]
% PPL = 2
\node[circle,draw,fill=blue!20] at (0,2) {当前};
\draw[->,thick] (0.5,2) -- (1.5,2.5);
\draw[->,thick] (0.5,2) -- (1.5,1.5);
\node[circle,draw,fill=green!20] at (2,2.5) {选项1};
\node[circle,draw,fill=green!20] at (2,1.5) {选项2};
\node[below] at (1,0.5) {PPL = 2:在2个选项间困惑};

% PPL = 4
\node[circle,draw,fill=blue!20] at (5,2) {当前};
\draw[->,thick] (5.5,2) -- (6.5,3);
\draw[->,thick] (5.5,2) -- (6.5,2.3);
\draw[->,thick] (5.5,2) -- (6.5,1.7);
\draw[->,thick] (5.5,2) -- (6.5,1);
\node[circle,draw,fill=orange!20,scale=0.6] at (7,3) {1};
\node[circle,draw,fill=orange!20,scale=0.6] at (7,2.3) {2};
\node[circle,draw,fill=orange!20,scale=0.6] at (7,1.7) {3};
\node[circle,draw,fill=orange!20,scale=0.6] at (7,1) {4};
\node[below] at (6,0.5) {PPL = 4:在4个选项间困惑};
\end{tikzpicture}
\end{center}


\paragraph{困惑度的实际值}
\textbf{不同模型的困惑度范围:}

\begin{center}
\begin{tabular}{|l|c|c|}
\hline
\textbf{模型} & \textbf{数据集} & \textbf{PPL} \\
\hline
随机模型 & 任意 & $\sim$ 词表大小 \\
\hline
N-gram (n=3) & Penn Treebank & $\sim$140 \\
\hline
简单RNN & Penn Treebank & $\sim$120 \\
\hline
LSTM & Penn Treebank & $\sim$80 \\
\hline
LSTM + 正则化 & Penn Treebank & $\sim$60 \\
\hline
Transformer (大) & Penn Treebank & $\sim$20-30 \\
\hline
GPT-3 & 网络文本 & $\sim$20 \\
\hline
\end{tabular}
\end{center}

\vspace{0.3cm}
\begin{theorem}[注意]
\begin{itemize}
    \item 不同数据集的PPL不具可比性
    \item 字符级模型的PPL通常更低(词表小)
    \item PPL是相对指标,越低越好
    \item PPL=4:平均4个选择
\end{itemize}
\end{theorem}


\paragraph{困惑度与准确率的关系}
\begin{definition}[解释]
\begin{center}
\begin{tikzpicture}[scale=1.0]
% 坐标轴
\draw[->,thick] (0,0) -- (8,0) node[right] {训练epoch};
\draw[->,thick] (0,0) -- (0,4) node[above] {PPL};

% 困惑度曲线(下降)
\draw[blue,thick,domain=0:7,samples=50] plot (\x,{3.5*exp(-0.3*\x)+0.5});
\node[blue,right] at (7,1) {困惑度};

\end{tikzpicture}
\end{center}
\end{definition}


\paragraph{训练过程中的困惑度变化}
\begin{center}
\begin{tikzpicture}[scale=1.0]
% 坐标轴
\draw[->,thick] (0,0) -- (8,0) node[right] {Epoch};
\draw[->,thick] (0,0) -- (0,4) node[above] {困惑度};

% 标注区域
\fill[green!20,opacity=0.3] (0,0) rectangle (8,1.5);
\node[green!60!black] at (4,1.2) {目标区域};

% 训练集困惑度
\draw[blue,thick,domain=0.5:7.5,samples=50] plot (\x,{3.8/(\x*0.5+0.5)+0.5});

\fill[red!20,opacity=0.3] (0,2.5) rectangle (8,4);
\node[red!60!black] at (4,3.5) {训练不足};
\node[blue,right] at (7,1.5) {训练集};

% 关键点
\fill[red] (1,3.5) circle (3pt) node[above] {\tiny 开始};
\fill[green!60!black] (5,1.2) circle (3pt) node[above] {\tiny 收敛};

% 验证集困惑度
\draw[red,thick,domain=0.5:7.5,samples=50] plot (\x,{3.8/(\x*0.5+0.5)+0.2});
\node[red,right] at (7,1.5) {验证集};

% 标注
\draw[dashed] (0,1) -- (8,1);
\node[below] at (4,0.8) {\tiny 目标困惑度水平};

\end{tikzpicture}
\end{center}

\begin{definition}[训练过程]
\begin{itemize}
    \item 初期:PPL很高(模型随机猜测)
    \item 中期:PPL快速下降(学习规律)
    \item 后期:PPL缓慢下降(微调,后期趋于平稳)
    \item 最终:PPL稳定(收敛)
\end{itemize}
\end{definition}

\begin{theorem}[过拟合检测]
\begin{itemize}
    \item 训练集PPL继续下降,验证集PPL开始上升 $\to$ 过拟合
    \item 两者的PPL都稳定下降 $\to$ 训练良好
\end{itemize}
\end{theorem}


\subsection{8.4节总结}
\begin{definition}[核心概念]
\begin{enumerate}
    \item \textbf{隐状态}:记忆历史信息的向量
    \item \textbf{RNN}:通过隐状态记住历史
    \item \textbf{参数共享}:所有时间步共享权重
\end{enumerate}
\end{definition}


\subsection{8.4节总结}
\begin{definition}[数学公式]
    \textbf{无隐状态网络}:MLP无法处理变长序列
    $$\mathbf{H}_t = \phi(\mathbf{X}_t \mathbf{W}_{xh} + \mathbf{b})$$

    \textbf{RNN}:通过隐状态记住历史
    $$\mathbf{H}_t = \tanh(\mathbf{X}_t \mathbf{W}_{xh} + \mathbf{H}_{t-1} \mathbf{W}_{hh} + \mathbf{b}_h)$$

    \textbf{参数共享}:所有时间步共享权重
\end{definition}


\subsection{8.4节总结}
\begin{definition}[模型类型]
    \begin{itemize}
        \item \textbf{字符级模型}:用独热编码表示字符
        \item \textbf{词级模型}:用词嵌入表示词语
    \end{itemize}
\end{definition}

\begin{definition}[评估指标]
    \textbf{困惑度}:评估模型质量
    $$\text{PPL} = \exp\left(-\frac{1}{n}\sum_{t=1}^n \log P(x_t|x_{<t})\right)$$
\end{definition}


\paragraph{RNN的关键特性}
\begin{center}
\begin{tikzpicture}[scale=0.9]
% 特性1
\node[draw,fill=blue!20,text width=9cm] at (5,3.5) {
\textbf{1. 参数共享}\\
$\mathbf{W}_{xh}, \mathbf{W}_{hh}, \mathbf{W}_{hy}$ 在所有时间步共享
};
\end{tikzpicture}
\end{center}


\paragraph{RNN的优缺点}


\textcolor{blue}{\textbf{优点:}}
\begin{itemize}
    \item $\checkmark$ 能处理变长序列
    \item $\checkmark$ 参数共享,所有时间步复用泛化好
    \item $\checkmark$ 参数量与序列长度无关
\end{itemize}


\textcolor{red}{\textbf{缺点:}}
\begin{itemize}
    \item $\times$ 梯度消失/爆炸问题
    \item $\times$ 长距离依赖建模困难
    \item $\times$ 无法并行计算(必须按顺序)
\end{itemize}



\paragraph{RNN的关键特性}
\begin{center}
\begin{tikzpicture}[scale=0.9]
% 特性2
\node[draw,fill=green!20,text width=9cm] at (5,2.2) {
\textbf{2. 顺序处理}\\
必须按顺序处理:$\mathbf{H}_1 \rightarrow \mathbf{H}_2 \rightarrow \mathbf{H}_3$\\
$\Rightarrow$ 参数数量固定,与序列长度无关
};

% 特性3
\node[draw,fill=orange!20,text width=9cm] at (5,0.9) {
\textbf{3. 理论上的记忆能力}\\
隐状态 $\mathbf{H}_t$ 可以编码所有历史\\
$\Rightarrow$ 但实际受限于梯度消失/爆炸
};
\end{tikzpicture}
\end{center}

\begin{theorem}[下一步]
8.5节将从零实现RNN,深入理解其工作机制
\end{theorem}


\paragraph{RNN的应用场景}


\textbf{适合RNN的任务:}
\begin{itemize}
    \item 语言建模
    \item 机器翻译
    \item 语音识别
    \item 时间序列预测
    \item 视频分析
\end{itemize}


\textbf{RNN的挑战:}
\begin{itemize}
    \item 梯度消失/爆炸
    \item 长期依赖问题
    \item 训练慢(顺序处理)
    \item 难以并行
\end{itemize}


\vspace{0.5cm}
\begin{definition}[改进方向]
\begin{itemize}
    \item LSTM/GRU:解决梯度问题(第9章)
    \item 注意力机制:捕捉长期依赖
    \item Transformer:完全并行化(后续章节)
\end{itemize}
\end{definition}


latex
\paragraph{从理论到实践}
\begin{center}
\begin{tikzpicture}[scale=1.0]
% 学习路径
\node[draw,fill=yellow!20,rounded corners,text width=2.5cm,align=center] at (0,2) {
8.1-8.3\\
理论基础
};

\draw[->,ultra thick] (1.5,2) -- (2.5,2);

\node[draw,fill=blue!20,rounded corners,text width=2.5cm,align=center] at (4,2) {
8.4\\
RNN原理
};

\draw[->,ultra thick] (5.5,2) -- (6.5,2);

\node[draw,fill=green!20,rounded corners,text width=2.5cm,align=center] at (8,2) {
8.5-8.6\\
RNN实现
};

\draw[->,ultra thick] (9.5,2) -- (10.5,2);

\node[draw,fill=red!20,rounded corners,text width=2.5cm,align=center] at (12,2) {
8.7\\
BPTT
};

% 标注
\node[below,text width=12cm,align=center] at (6,0.5) {
我们已经理解了RNN的\textbf{工作原理}\\
接下来将\textbf{从零实现}RNN(8.5节)
};

\end{tikzpicture}
\end{center}

\begin{definition}[已掌握的知识]
\begin{itemize}
    \item $\checkmark$ 为什么需要隐状态
    \item $\checkmark$ RNN的数学公式
    \item $\checkmark$ 参数共享机制
    \item $\checkmark$ 字符级语言模型
    \item $\checkmark$ 困惑度评估
\end{itemize}
\end{definition}


\subsection{8.4节核心公式总结}
\begin{definition}[RNN的核心公式]
\begin{align*}
\text{\textbf{隐状态更新:}} \quad & \mathbf{H}_t = \phi(\mathbf{X}_t \mathbf{W}_{xh} + \mathbf{H}_{t-1}\mathbf{W}_{hh} + \mathbf{b}_h) \\
\text{\textbf{输出计算:}} \quad & \mathbf{O}_t = \mathbf{H}_t \mathbf{W}_{hq} + \mathbf{b}_q \\
\text{\textbf{概率预测:}} \quad & P(x_{t+1}) = \text{softmax}(\mathbf{O}_t) \\
\text{\textbf{困惑度:}} \quad & \text{PPL} = \exp\left(-\frac{1}{n}\sum_{t=1}^{n}\log P(x_t \mid x_{<t})\right)
\end{align*}
\end{definition}

\begin{theorem}[关键要点]
\begin{itemize}
    \item 隐状态 $\mathbf{H}_t$ 携带历史信息
    \item 参数 $\mathbf{W}_{xh}, \mathbf{W}_{hh}, \mathbf{W}_{hq}$ 在所有时间步共享
    \item 激活函数通常使用 $\tanh$
    \item 困惑度越低,模型越好
\end{itemize}
\end{theorem}


\paragraph{RNN与传统模型对比}
\begin{center}
\begin{tabular}{|l|c|c|c|}
\hline
\textbf{特性} & \textbf{N-gram} & \textbf{MLP} & \textbf{RNN} \\
\hline
处理变长序列 & $\times$ & $\times$ & $\checkmark$ \\
\hline
参数数量 & 随$\tau$指数增长 & 随$\tau$线性增长 & 固定 \\
\hline
长期依赖 & 受限于$\tau$ & 受限于$\tau$ & 理论上可以 \\
\hline
并行化 & $\checkmark$ & $\checkmark$ & $\times$ \\
\hline
参数共享 & $\times$ & $\times$ & $\checkmark$ \\
\hline
训练难度 & 简单 & 中等 & 较难 \\
\hline
\end{tabular}
\end{center}

\vspace{0.5cm}
\begin{definition}[RNN的核心优势]
通过\textbf{隐状态}和\textbf{参数共享},实现了用固定参数处理任意长度序列
\end{definition}


\paragraph{训练RNN的关键技巧}


\textbf{1. 梯度裁剪}
\begin{itemize}
    \item 防止梯度爆炸
    \item 阈值通常取1或5
    \item 必不可少!
\end{itemize}

\textbf{2. 隐状态初始化}
\begin{itemize}
    \item 零初始化
    \item 可学习初始化
    \item 使用前一批次的隐状态
\end{itemize}


\textbf{3. 序列采样}
\begin{itemize}
    \item 随机采样:简单
    \item 顺序分区:效果好
    \item 保持隐状态连续性
\end{itemize}

\textbf{4. 学习率调整}
\begin{itemize}
    \item 从较小值开始
    \item 使用学习率衰减
    \item 监控梯度范数
\end{itemize}


\vspace{0.5cm}
\begin{theorem}[经验]
RNN训练需要仔细调参,梯度裁剪是关键!
\end{theorem}


\paragraph{常见问题与解决方案}
\begin{center}
\begin{tikzpicture}[scale=0.9]
% 问题1
\node[draw,fill=red!20,text width=5cm] at (0,3.5) {
\textbf{问题1:梯度爆炸}\\
梯度突然变得很大
};
\node[draw,fill=green!20,text width=5cm] at (6,3.5) {
\textbf{解决:}\\
梯度裁剪 + 小学习率
};
\draw[->,thick] (2.7,3.5) -- (3.8,3.5);

% 问题2
\node[draw,fill=red!20,text width=5cm] at (0,2.2) {
\textbf{问题2:梯度消失}\\
长期依赖学不到
};
\node[draw,fill=green!20,text width=5cm] at (6,2.2) {
\textbf{解决:}\\
使用LSTM/GRU(第9章)
};
\draw[->,thick] (2.7,2.2) -- (3.8,2.2);

% 问题3
\node[draw,fill=red!20,text width=5cm] at (0,0.9) {
\textbf{问题3:训练慢}\\
顺序处理效率低
};
\node[draw,fill=green!20,text width=5cm] at (6,0.9) {
\textbf{解决:}\\
截断BPTT + GPU加速
};
\draw[->,thick] (2.7,0.9) -- (3.8,0.9);

\end{tikzpicture}
\end{center}


\subsection{8.4节思维导图}
\begin{center}
\begin{tikzpicture}[scale=0.75,
    box/.style={rectangle,draw,fill=blue!20,rounded corners,text width=2.3cm,align=center,minimum height=0.8cm}]

% 中心
\node[box,fill=yellow!30,text width=3cm] (center) at (6,3) {\textbf{8.4 RNN原理}};

% 四个主要分支
\node[box] (no) at (0,5) {8.4.1\\无隐状态\\网络};
\node[box] (yes) at (4,5.5) {8.4.2\\有隐状态\\RNN};
\node[box] (char) at (8,5.5) {8.4.3\\字符级\\语言模型};
\node[box] (ppl) at (12,5) {8.4.4\\困惑度};

% 连接
\draw[->,thick] (center) -- (no);
\draw[->,thick] (center) -- (yes);
\draw[->,thick] (center) -- (char);
\draw[->,thick] (center) -- (ppl);

% 细节
\node[font=\tiny,text width=2.5cm,align=center] at (0,3.8) {MLP无法\\处理变长序列};
\node[font=\tiny,text width=2.5cm,align=center] at (4,3.8) {隐状态记忆\\参数共享};
\node[font=\tiny,text width=2.5cm,align=center] at (8,3.8) {One-hot编码\\预测下一字符};
\node[font=\tiny,text width=2.5cm,align=center] at (12,3.8) {评估指标\\PPL=exp(CE)};

% 关键公式
\node[draw,fill=green!20,text width=10cm,font=\small] at (6,1) {
\textbf{核心公式:}$\mathbf{H}_t = \tanh(\mathbf{X}_t \mathbf{W}_{xh} + \mathbf{H}_{t-1}\mathbf{W}_{hh} + \mathbf{b}_h)$
};

\end{tikzpicture}
\end{center}


\paragraph{知识检查}
\begin{definition}[问题1]
为什么RNN需要隐状态?传统MLP有什么问题?
\end{definition}

\begin{definition}[问题2]
RNN的参数数量与序列长度有关吗?为什么?
\end{definition}

\begin{definition}[问题3]
什么是困惑度?如何解释PPL=10?
\end{definition}

\begin{definition}[问题4]
为什么字符级RNN的困惑度通常比词级低?
\end{definition}

\begin{theorem}[思考]
如何解决RNN的梯度消失问题?(提示:LSTM)
\end{theorem}


\paragraph{准备进入8.5节}
\begin{center}
\Large \textcolor{blue}{从理论到实践:从零实现RNN}
\end{center}

\vspace{0.5cm}
\begin{definition}[8.4节回顾]
\begin{itemize}
    \item 理解了RNN的\textbf{核心思想}:用隐状态记忆历史
    \item 掌握了RNN的\textbf{数学原理}:前向传播公式
    \item 学会了\textbf{评估模型}:困惑度指标
\end{itemize}
\end{definition}

\begin{definition}[8.5节预告]
\begin{itemize}
    \item 从零实现RNN的每个细节
    \item 独热编码、初始化参数
    \item 前向传播、梯度裁剪
    \item 训练完整的字符级语言模型
    \item 文本生成
\end{itemize}
\end{definition}


\paragraph{第8章进度总结}
\begin{center}
\begin{tikzpicture}[scale=0.9]
% 进度条
\draw[thick,fill=gray!20] (0,2) rectangle (12,2.8);

% 已完成部分
\draw[thick,fill=green!40] (0,2) rectangle (6.5,2.8);

% 分节标记
\foreach \x/\label in {0/8.1,3/8.2,4.5/8.3,6/8.4,9/8.5,10.5/8.6,12/8.7} {
    \draw[thick] (\x,2) -- (\x,2.8);
    \node[below,font=\tiny] at (\x,1.8) {\label};
}

% 当前位置
\draw[->,ultra thick,red] (6.5,3.5) -- (6.5,2.9);
\node[above,red] at (6.5,3.6) {当前位置};

% 标注
\node[below,text width=12cm,align=center] at (6,0.8) {
已完成:序列模型、文本预处理、语言模型、RNN原理\\
下一步:RNN从零实现(8.5)
};

\end{tikzpicture}
\end{center}


\section{8.5 循环神经网络的从零开始实现}

\subsection{8.5 循环神经网络的从零开始实现}
\begin{center}
\Large \textcolor{blue}{动手实现:从零开始构建RNN}
\end{center}


\subsection{8.5节 - 章节导航}
\begin{definition}[本节内容]
\begin{itemize}
    \item 8.5.1 独热编码
    \item 8.5.2 初始化模型参数
    \item 8.5.3 循环神经网络模型
    \item 8.5.4 预测
    \item 8.5.5 梯度裁剪
    \item 8.5.6 训练
\end{itemize}
\end{definition}

\begin{theorem}[学习目标]
\begin{enumerate}
    \item 手写独热编码函数
    \item 理解RNN参数初始化
    \item 实现RNN前向传播
    \item 生成文本预测
    \item 掌握梯度裁剪技巧
    \item 训练完整的字符级语言模型
\end{enumerate}
\end{theorem}


\paragraph{从零实现的意义}
\begin{center}
\begin{tikzpicture}[scale=0.9]
% 两种方式
\node[draw,fill=yellow!20,rounded corners,text width=5cm,align=center] at (0,2) {
\textbf{使用框架}\\
\texttt{nn.RNN()}\\
简单快速\\
但细节不清楚
};

\node[draw,fill=green!20,rounded corners,text width=5cm,align=center] at (7,2) {
\textbf{从零实现}\\
手写每一行代码\\
理解每个细节\\
知其所以然
};

\draw[->,ultra thick,blue] (2.7,2) -- (4.3,2);
\node[above,blue] at (3.5,2.3) {本节目标};

\end{tikzpicture}
\end{center}

\begin{definition}[你将学会]
\begin{itemize}
    \item 如何将文本转为张量
    \item RNN的参数有哪些,形状是什么
    \item 前向传播的每一步计算
    \item 如何生成新文本
    \item 训练的完整流程
\end{itemize}
\end{definition}



\subsection{独热编码}

\subsection{8.5.1 独热编码}

\subsection{8.5.1 独热编码}
\begin{center}
\Large \textcolor{blue}{第一步:将字符转为向量}
\end{center}


\paragraph{为什么需要独热编码?}
\textbf{问题:}神经网络只能处理数字

\begin{center}
\begin{tikzpicture}[scale=1.0]
% 字符
\node[draw,fill=yellow!20,rounded corners,text width=2.5cm,align=center] at (0,2) {
字符\\
``a''
};

\draw[->,thick,red] (1.5,2) -- (2.5,2);
\node[red,above] at (2,2.2) {\tiny \text{×}直接用?};

% 神经网络
\node[draw,fill=blue!20,rounded corners,text width=2.5cm,align=center] at (4.5,2) {
RNN\\
需要向量
};

% 解决方案
\draw[->,ultra thick] (0,1.2) -- (0,0.5);
\node[left] at (-0.3,0.85) {\tiny 编码};

\node[draw,fill=green!20,rounded corners,text width=2.5cm,align=center] at (0,0) {
向量\\
$[1,0,0,...,0]$
};

\draw[->,ultra thick,green!60!black] (1.5,0) -- (4.5,1.5);
\node[green!60!black,above] at (3,0.8) {\tiny \checkmark 独热编码};

\end{tikzpicture}
\end{center}

\begin{theorem}[核心思想]
用长度为词表大小的向量表示每个字符,只有一个位置为1,其余为0
\end{theorem}


\paragraph{独热编码详解}
\textbf{假设词表:}\texttt{vocab = {' ':0, 'a':1, 'b':2, 'c':3, 'd':4, 'e':5}}

\begin{center}
\begin{tikzpicture}[scale=0.9]
% 词表
\node[left] at (-0.5,3.5) {词表:};
\foreach \i/\c in {0/{ },1/a,2/b,3/c,4/d,5/e} {
    \node[draw,minimum size=0.5cm] at (\i*1.2,3.5) {\tiny \c};
    \node[below,font=\tiny] at (\i*1.2,3) {\i};
}

% 字符'c'
\node[left] at (-0.5,2) {字符``c''};
\node[right] at (0.5,2) {索引=3};

\draw[->,thick] (1.5,2) -- (1.5,1.5);

% 独热向量
\node[left] at (-0.5,1) {独热编码:};
\foreach \i in {0,1,2,3,4,5} {
    \ifnum\i=3
        \node[draw,fill=red!30,minimum size=0.5cm] at (\i*1.2,1) {\tiny 1};
    \else
        \node[draw,minimum size=0.5cm] at (\i*1.2,1) {\tiny 0};
    \fi
}

% 维度标注
\draw[<->,thick] (0,0.3) -- (6,0.3);
\node[below] at (3,0.1) {向量长度 = 词表大小 = 6};

\end{tikzpicture}
\end{center}

\begin{definition}[特点]
\begin{itemize}
    \item 向量维度 = 词表大小
    \item 稀疏表示:只有一个1,其余全是0
    \item 没有顺序关系:``a''和``b''距离相同
\end{itemize}
\end{definition}


\paragraph{批量独热编码}
\textbf{输入:}批量的索引矩阵

\begin{center}
\begin{tikzpicture}[scale=0.9]
% 索引矩阵
\node[left] at (-1,3) {索引矩阵:};
\node[draw,fill=yellow!20,text width=3cm,font=\small] at (2,3) {
\texttt{X = [[1, 2, 3],}\\
\texttt{      [4, 5, 0]]}\\
形状: $(2, 3)$
};

\node[below,font=\small,align=center] at (2,2.2) {
批量大小=2\\
序列长度=3
};

\draw[->,ultra thick] (2,1.8) -- (2,1.2);
\node[right] at (2.3,1.5) {\tiny 独热编码};

% 独热张量
\node[left] at (-1,0.3) {独热张量:};
\node[draw,fill=green!20,text width=4cm,font=\small,align=center] at (2.5,0.3) {
形状: $(3, 2, 6)$\\
(时间步, 批量, 词表大小)
};

\end{tikzpicture}
\end{center}

\begin{theorem}[维度变化]
$(batch\_size, num\_steps)$ $\rightarrow$ $(num\_steps, batch\_size, vocab\_size)$
\end{theorem}


\paragraph{独热编码实现 - 方法1(手动)}
\begin{lstlisting}
import torch

def one_hot_manual(X, vocab_size):
    """ X: (batch_size, num_steps) vocab_size: vocabulary Return: (num_steps, batch_size, vocab_size) """
    batch_size, num_steps = X.shape
    
    output = torch.zeros((num_steps, batch_size, vocab_size))
    
    # X: (num_steps, batch_size)
    X = X.T
    
    # 1
    for t in range(num_steps):
        for b in range(batch_size):
            idx = X[t, b]
            output[t, b, idx] = 1
    
    return output

X = torch.tensor([[1, 2, 3], [4, 5, 0]])
result = one_hot_manual(X, vocab_size=6)
print(f"shape: {result.shape}")  # (3, 2, 6)
\end{lstlisting}


\paragraph{独热编码实现 - 方法2(PyTorch)}
\begin{lstlisting}
import torch.nn.functional as F

def one_hot(X, vocab_size):
    """ PyTorch X: (batch_size, num_steps) Return: (num_steps, batch_size, vocab_size) """
    # : (num_steps, batch_size)
    X = X.T
    
    output = F.one_hot(X, vocab_size)
    
    # float32neural network
    return output.to(torch.float32)

X = torch.tensor([[1, 2, 3], [4, 5, 0]])
result = one_hot(X, vocab_size=6)
print(f"shape: {result.shape}")  # (3, 2, 6)
print(f"data类型: {result.dtype}")  # torch.float32

print(f"第1个time step,第1个样本:")
print(result[0, 0])  # [0, 1, 0, 0, 0, 0]
\end{lstlisting}

\begin{definition}[推荐]
使用PyTorch内置的\texttt{F.one\_hot},更高效!
\end{definition}


\paragraph{独热编码的优缺点}


\textcolor{blue}{\textbf{优点:}}
\begin{itemize}
    \item $\checkmark$ 简单直观
    \item $\checkmark$ 字符间无顺序假设
    \item $\checkmark$ 易于理解和实现
    \item $\checkmark$ 适合小词表
\end{itemize}

\textbf{适用场景:}
\begin{itemize}
    \item 字符级模型(词表小)
    \item 分类任务
    \item 教学演示
\end{itemize}


\textcolor{red}{\textbf{缺点:}}
\begin{itemize}
    \item $\times$ 高维稀疏
    \item $\times$ 无语义信息
    \item $\times$ 内存占用大
    \item $\times$ 无法表示相似性
\end{itemize}

\textbf{改进方案:}
\begin{itemize}
    \item 词嵌入(Word Embedding)
    \item 降维表示
    \item 预训练向量
\end{itemize}


\vspace{0.5cm}
\begin{theorem}[实际应用]
现代NLP通常用\textbf{词嵌入}代替独热编码(后续章节)
\end{theorem}


\paragraph{完整的数据准备}
\begin{lstlisting}
def prepare_data(batch_size, num_steps):
    """Prepare training data"""
    # data
    train_iter, vocab = load_data_time_machine(
        batch_size, num_steps
    )
    
    # batch
    for X, Y in train_iter:
        print(f"Xshape(索引): {X.shape}")  # (batch_size, num_steps)
        print(f"Yshape(标签): {Y.shape}")  # (batch_size, num_steps)
        
        X_one_hot = one_hot(X, len(vocab))
        print(f"X_one_hotshape: {X_one_hot.shape}")  
        # (num_steps, batch_size, vocab_size)
        
        break
    
    return train_iter, vocab

train_iter, vocab = prepare_data(batch_size=32, num_steps=35)
print(f"vocabulary大小: {len(vocab)}")
\end{lstlisting}



\subsection{初始化模型参数}

\subsection{8.5.2 初始化模型参数}

\subsection{8.5.2 初始化模型参数}
\begin{center}
\Large \textcolor{blue}{第二步:初始化RNN的权重和偏置}
\end{center}


\paragraph{RNN需要哪些参数?}
\textbf{回顾公式:}
\begin{align*}
\mathbf{H}_t &= \tanh(\mathbf{X}_t \mathbf{W}_{xh} + \mathbf{H}_{t-1}\mathbf{W}_{hh} + \mathbf{b}_h) \\
\mathbf{O}_t &= \mathbf{H}_t \mathbf{W}_{hq} + \mathbf{b}_q
\end{align*}

\begin{center}
\begin{tikzpicture}[scale=1.0]
% 参数表格
\node[draw,fill=yellow!20,text width=10cm,align=left] at (5,2) {
\textbf{需要初始化的参数:}\\
1. $\mathbf{W}_{xh}$:输入到隐藏层权重,形状 $(vocab\_size, num\_hiddens)$\\
2. $\mathbf{W}_{hh}$:隐藏层到隐藏层权重,形状 $(num\_hiddens, num\_hiddens)$\\
3. $\mathbf{b}_h$:隐藏层偏置,形状 $(num\_hiddens,)$\\
4. $\mathbf{W}_{hq}$:隐藏层到输出权重,形状 $(num\_hiddens, vocab\_size)$\\
5. $\mathbf{b}_q$:输出层偏置,形状 $(vocab\_size,)$
};

\end{tikzpicture}
\end{center}

\begin{definition}[关键]
这5个参数在\textbf{所有时间步}共享!
\end{definition}


\paragraph{参数形状分析}
\textbf{假设:}词表大小 $V=28$,隐藏单元数 $h=512$

\begin{center}
\begin{tabular}{|l|c|c|}
\hline
\textbf{参数} & \textbf{形状} & \textbf{参数量} \\
\hline
$\mathbf{W}_{xh}$ & $(28, 512)$ & 14,336 \\
\hline
$\mathbf{W}_{hh}$ & $(512, 512)$ & 262,144 \\
\hline
$\mathbf{b}_h$ & $(512,)$ & 512 \\
\hline
$\mathbf{W}_{hq}$ & $(512, 28)$ & 14,336 \\
\hline
$\mathbf{b}_q$ & $(28,)$ & 28 \\
\hline
\textbf{总计} & & \textbf{291,356} \\
\hline
\end{tabular}
\end{center}

\begin{theorem}[观察]
\begin{itemize}
    \item 大部分参数在 $\mathbf{W}_{hh}$(90\%)
    \item 与序列长度\textbf{无关}
    \item 参数量主要由 $num\_hiddens$ 决定
\end{itemize}
\end{theorem}


\paragraph{参数初始化实现}
\begin{lstlisting}
import torch

def get_params(vocab_size, num_hiddens, device):
    """ RNNparameters vocab_size: vocabulary num_hiddens: device: (CPU/GPU) """
    num_inputs = num_outputs = vocab_size
    
    def normal(shape):
        """正态分布初始化"""
        return torch.randn(size=shape, device=device) * 0.01
    
    # parameters
    W_xh = normal((num_inputs, num_hiddens))
    W_hh = normal((num_hiddens, num_hiddens))
    b_h = torch.zeros(num_hiddens, device=device)
    
    # Outputparameters
    W_hq = normal((num_hiddens, num_outputs))
    b_q = torch.zeros(num_outputs, device=device)
    
    params = [W_xh, W_hh, b_h, W_hq, b_q]
    for param in params:
        param.requires_grad_(True)
    
    return params
\end{lstlisting}


\paragraph{初始化策略详解}


\textbf{权重初始化:}
\begin{itemize}
    \item 使用\textbf{正态分布} $\mathcal{N}(0, 0.01^2)$
    \item 小的随机值
    \item 避免对称性
\end{itemize}

\textbf{为什么用0.01?}
\begin{itemize}
    \item 太大:梯度爆炸
    \item 太小:梯度消失
    \item 0.01是经验值
\end{itemize}


\textbf{偏置初始化:}
\begin{itemize}
    \item 使用\textbf{零初始化}
    \item 不影响对称性
    \item 训练中会更新
\end{itemize}

\textbf{更好的初始化:}
\begin{itemize}
    \item Xavier初始化
    \item He初始化
    \item 正交初始化(RNN常用)
\end{itemize}


\vspace{0.5cm}
\begin{theorem}[重要]
RNN对初始化敏感,建议使用\textbf{正交初始化} $\mathbf{W}_{hh}$
\end{theorem}


\paragraph{正交初始化(进阶)}
\begin{lstlisting}
def init_rnn_params_orthogonal(vocab_size, num_hiddens, device):
    """()"""
    num_inputs = num_outputs = vocab_size
    
    W_xh = torch.randn(num_inputs, num_hiddens, device=device) * 0.01
    
    W_hh = torch.eye(num_hiddens, device=device)
    #  torch.nn.init.orthogonal_(W_hh)
    
    b_h = torch.zeros(num_hiddens, device=device)
    
    # Output
    W_hq = torch.randn(num_hiddens, num_outputs, device=device) * 0.01
    b_q = torch.zeros(num_outputs, device=device)
    
    params = [W_xh, W_hh, b_h, W_hq, b_q]
    for param in params:
        param.requires_grad_(True)
    
    return params
\end{lstlisting}

\begin{definition}[优势]
正交矩阵保持梯度范数,缓解梯度消失/爆炸
\end{definition}


\paragraph{初始化隐状态}
\begin{lstlisting}
def init_rnn_state(batch_size, num_hiddens, device):
    """ RNN batch_size: num_hiddens: Return: (batch_size, num_hiddens) """
    return (torch.zeros((batch_size, num_hiddens), device=device),)
    # ReturnLSTM

batch_size, num_hiddens = 32, 512
device = torch.device('cuda' if torch.cuda.is_available() else 'cpu')

state = init_rnn_state(batch_size, num_hiddens, device)
print(f"隐状态shape: {state[0].shape}")  # (32, 512)
print(f"隐状态全为0: {torch.all(state[0] == 0)}")  # True
\end{lstlisting}

\begin{theorem}[注意]
每个新序列开始时需要重新初始化隐状态(或使用上一批次的最终状态)
\end{theorem}



\subsection{循环神经网络模型}

\subsection{8.5.3 循环神经网络模型}

\subsection{8.5.3 循环神经网络模型}
\begin{center}
\Large \textcolor{blue}{第三步:实现RNN前向传播}
\end{center}


\paragraph{前向传播流程}
\begin{center}
\begin{tikzpicture}[scale=0.9]
% 流程图
\node[draw,fill=yellow!20,rounded corners,text width=8cm] at (5,4.5) {
\textbf{输入:}\\
- \texttt{inputs}: 独热编码序列 $(T, B, V)$\\
- \texttt{state}: 初始隐状态 $(B, H)$\\
- \texttt{params}: 模型参数
};

\draw[->,thick] (5,4) -- (5,3.5);

\node[draw,fill=blue!20,rounded corners,text width=8cm] at (5,3) {
\textbf{循环处理每个时间步:}\\
for $t = 1$ to $T$:\\
\quad $\mathbf{H}_t = \tanh(\mathbf{X}_t \mathbf{W}_{xh} + \mathbf{H}_{t-1}\mathbf{W}_{hh} + \mathbf{b}_h)$\\
\quad $\mathbf{O}_t = \mathbf{H}_t \mathbf{W}_{hq} + \mathbf{b}_q$
};

\draw[->,thick] (5,2.5) -- (5,2);

\node[draw,fill=green!20,rounded corners,text width=8cm] at (5,1.5) {
\textbf{输出:}\\
- \texttt{outputs}: 所有时间步的输出 $(T \times B, V)$\\
- \texttt{state}: 最终隐状态 $(B, H)$
};

\end{tikzpicture}
\end{center}


\paragraph{RNN前向传播实现}
\begin{lstlisting}
def rnn(inputs, state, params):
    """ RNN inputs: (num_steps, batch_size, vocab_size) state: (batch_size, num_hiddens) params: [W_xh, W_hh, b_h, W_hq, b_q] Return: outputs (num_steps*batch_size, vocab_size), state (batch_size, num_hiddens) """
    # parameters
    W_xh, W_hh, b_h, W_hq, b_q = params
    H, = state  # comment
    outputs = []
    
    # time step
    for X in inputs:  # X: (batch_size, vocab_size)
        H = torch.tanh(torch.mm(X, W_xh) + 
                       torch.mm(H, W_hh) + b_h)
        # H: (batch_size, num_hiddens)
        
        # Output
        Y = torch.mm(H, W_hq) + b_q
        # Y: (batch_size, vocab_size)
        
        outputs.append(Y)
    
    # Output
    return torch.cat(outputs, dim=0), (H,)
    # (num_steps*batch_size, vocab_size), (batch_size, num_hiddens)
\end{lstlisting}


\paragraph{前向传播的维度变化}
\begin{center}
\begin{tikzpicture}[scale=0.9]
% 时间步1
\node[above] at (2,3.5) {\small 时间步 $t=1$};
\node[draw,fill=green!20,text width=2.5cm,align=center] at (2,2.8) {
$\mathbf{X}_1$\\
$(B, V)$
};
\node[draw,fill=red!20,text width=2.5cm,align=center] at (2,1.8) {
$\mathbf{H}_0$\\
$(B, H)$
};

\draw[->,thick] (2,2.5) -- (2,2.2) node[midway,right,font=\tiny] {$\mathbf{W}_{xh}$};
\draw[->,thick] (2,1.5) -- (2,2.2) node[midway,right,font=\tiny] {$\mathbf{W}_{hh}$};

\node[draw,fill=red!20,text width=2.5cm,align=center] at (2,0.8) {
$\mathbf{H}_1$\\
$(B, H)$
};

\draw[->,thick] (2,0.5) -- (2,0) node[midway,right,font=\tiny] {$\mathbf{W}_{hq}$};

\node[draw,fill=blue!20,text width=2.5cm,align=center] at (2,-0.3) {
$\mathbf{O}_1$\\
$(B, V)$
};

% 时间步2
\node[above] at (6,3.5) {\small 时间步 $t=2$};
\node[draw,fill=green!20,text width=2.5cm,align=center] at (6,2.8) {
$\mathbf{X}_2$\\
$(B, V)$
};
\node[draw,fill=red!20,text width=2.5cm,align=center] at (6,1.8) {
$\mathbf{H}_1$\\
$(B, H)$
};

\draw[->,thick] (6,2.5) -- (6,2.2);
\draw[->,thick] (6,1.5) -- (6,2.2);

\node[draw,fill=red!20,text width=2.5cm,align=center] at (6,0.8) {
$\mathbf{H}_2$\\
$(B, H)$
};

\draw[->,thick] (6,0.5) -- (6,0);

\node[draw,fill=blue!20,text width=2.5cm,align=center] at (6,-0.3) {
$\mathbf{O}_2$\\
$(B, V)$
};

% 隐状态传递
\draw[->,ultra thick,red] (3.3,0.8) -- (4.7,0.8);

% 标注
\node[below,text width=8cm,align=center] at (4,-1.2) {
\small $B$=batch\_size, $V$=vocab\_size, $H$=num\_hiddens
};

\end{tikzpicture}
\end{center}


\paragraph{测试RNN前向传播}
\begin{lstlisting}
vocab_size, num_hiddens = 28, 512
batch_size, num_steps = 2, 5
device = torch.device('cpu')

# parameters
params = get_params(vocab_size, num_hiddens, device)

X = torch.randint(0, vocab_size, (batch_size, num_steps))
X_one_hot = one_hot(X, vocab_size)
print(f"输入shape: {X_one_hot.shape}")  # (5, 2, 28)

state = init_rnn_state(batch_size, num_hiddens, device)

outputs, new_state = rnn(X_one_hot, state, params)

print(f"Outputshape: {outputs.shape}")  # (10, 28) = (5*2, 28)
print(f"新隐状态shape: {new_state[0].shape}")  # (2, 512)
\end{lstlisting}

\textbf{输出解释:}
\begin{itemize}
    \item 输出:$(num\_steps \times batch\_size, vocab\_size) = (10, 28)$
    \item 每个位置是未归一化的logits(预测分数)
\end{itemize}


\paragraph{RNN模型封装成类}
\textbf{为了便于使用,封装成类:}

\begin{center}
\begin{tikzpicture}[scale=0.9]
\node[draw,fill=yellow!20,rounded corners,text width=10cm] at (5,3) {
\textbf{RNNModelScratch类}\\
- \texttt{\_\_init\_\_}: 初始化参数\\
- \texttt{forward}: 前向传播\\
- \texttt{begin\_state}: 创建初始隐状态
};

\draw[->,thick] (5,2.5) -- (5,2);

\node[draw,fill=green!20,rounded corners,text width=10cm] at (5,1.3) {
\textbf{优点:}\\
$\checkmark$ 面向对象,易于管理\\
$\checkmark$ 接口统一\\
$\checkmark$ 便于后续扩展
};

\end{tikzpicture}
\end{center}


\paragraph{RNN类实现 - 初始化}
\begin{lstlisting}
class RNNModelScratch:
    """RNNmodel"""
    
    def __init__(self, vocab_size, num_hiddens, device,
                 get_params, init_state, forward_fn):
        """
        vocab_size: vocabulary大小
        num_hiddens: 隐藏单元数
        device: 设备
        get_params: 获取parameters的函数
        init_state: 初始化状态的函数
        forward_fn: 前向传播函数
        """
        self.vocab_size = vocab_size
        self.num_hiddens = num_hiddens
        self.params = get_params(vocab_size, num_hiddens, device)
        self.init_state = init_state
        self.forward_fn = forward_fn
    
    def __call__(self, X, state):
        """调using向传播"""
        X = one_hot(X, self.vocab_size)
        return self.forward_fn(X, state, self.params)
    
    def begin_state(self, batch_size, device):
        """创建初始隐状态"""
        return self.init_state(batch_size, self.num_hiddens, device)
\end{lstlisting}


\paragraph{创建RNN模型实例}
\begin{lstlisting}
# parameters
vocab_size = 28
num_hiddens = 512
device = torch.device('cuda' if torch.cuda.is_available() else 'cpu')

# model
net = RNNModelScratch(
    vocab_size, num_hiddens, device,
    get_params, init_rnn_state, rnn
)

# test
batch_size, num_steps = 2, 5
X = torch.randint(0, vocab_size, (batch_size, num_steps))
state = net.begin_state(batch_size, device)

Y, new_state = net(X.to(device), state)

print(f"modelOutputshape: {Y.shape}")  # (10, 28)
print(f"modelparameters数量: {sum(p.numel() for p in net.params)}")
\end{lstlisting}

\textbf{输出:}
\begin{verbatim}
模型输出形状: torch.Size([10, 28])
模型参数数量: 291356
\end{verbatim}



\subsection{预测}

\subsection{8.5.4 预测}

\subsection{8.5.4 预测}
\begin{center}
\Large \textcolor{blue}{第四步:使用RNN生成文本}
\end{center}


\paragraph{文本生成的两种方式}


\textbf{1. 给定前缀预测}
\begin{itemize}
    \item 输入:``time trav''
    \item 输出:``eller''
    \item 用于补全
\end{itemize}

\begin{center}
\begin{tikzpicture}[scale=0.7]
\foreach \i/\c in {0/t,1/i,2/m,3/e,4/{ },5/t,6/r,7/a,8/v} {
    \node[draw,fill=blue!20,minimum size=0.4cm,font=\tiny] at (\i*0.5,1) {\c};
}
\node[left,font=\tiny] at (0,1) {输入};

\foreach \i/\c in {0/e,1/l,2/l,3/e,4/r} {
    \node[draw,fill=green!20,minimum size=0.4cm,font=\tiny] at (\i*0.5,0) {\c};
}
\node[left,font=\tiny] at (0,0) {预测};
\end{tikzpicture}
\end{center}


\textbf{2. 随机采样生成}
\begin{itemize}
    \item 输入:``t''
    \item 每步采样下一个字符
    \item 用于创作
\end{itemize}

\begin{center}
\begin{tikzpicture}[scale=0.7]
\node[draw,fill=blue!20,minimum size=0.4cm,font=\tiny] at (0,1) {t};
\node[left,font=\tiny] at (-0.3,1) {输入};

\draw[->,thick] (0.3,1) -- (0.7,1);

\foreach \i/\c in {0/h,1/e,2/{ },3/t,4/i,5/m,6/e} {
    \node[draw,fill=green!20,minimum size=0.4cm,font=\tiny] at (\i*0.5+1,1) {\c};
}
\node[right,font=\tiny] at (4.5,1) {...};

\node[below,font=\tiny,text width=3cm,align=center] at (2,0.3) {
逐步采样生成
};
\end{tikzpicture}
\end{center}


\vspace{0.5cm}
\begin{theorem}[关键]
采样时需要根据概率分布选择,而非总是选最大概率的字符
\end{theorem}


\paragraph{预测函数 - 给定前缀}
\begin{lstlisting}
def predict_ch8(prefix, num_preds, net, vocab, device):
    """ prediction prefix: , "time traveller" num_preds: prediction net: RNNmodel vocab: vocabulary device: """
    state = net.begin_state(batch_size=1, device=device)
    
    outputs = [vocab[prefix[0]]]
    
    get_input = lambda: torch.tensor([outputs[-1]], 
                                     device=device).reshape((1, 1))
    
    # ""using
    for y in prefix[1:]:
        _, state = net(get_input(), state)
        outputs.append(vocab[y])
    
    # predictionnum_preds
    for _ in range(num_preds):
        y, state = net(get_input(), state)
        outputs.append(int(y.argmax(dim=1).reshape(1)))
    
    return ''.join([vocab.idx_to_token[i] for i in outputs])
\end{lstlisting}


\paragraph{预测过程可视化}
\begin{center}
\begin{tikzpicture}[scale=0.9]
% 前缀阶段
\node[above] at (2,3.5) {\small 阶段1: 预热(前缀)};
\foreach \i/\c in {0/t,1/i,2/m,3/e} {
    \node[draw,fill=blue!20,minimum size=0.6cm] at (\i*0.8,3) {\c};
    \node[circle,draw,fill=red!20,minimum size=0.5cm] at (\i*0.8,2) {\tiny$h_{\i}$};
    \draw[->,thick] (\i*0.8,2.6) -- (\i*0.8,2.3);
    \ifnum\i>0
        \pgfmathtruncatemacro{\prev}{\i-1}
        \draw[->,thick,blue] (\prev*0.8+0.3,2) -- (\i*0.8-0.3,2);
    \fi
}

% 预测阶段
\node[above] at (7,3.5) {\small 阶段2: 预测};
\foreach \i/\c in {4/{ },5/t,6/r} {
    \node[draw,fill=green!20,minimum size=0.6cm] at (\i*0.8,3) {\c};
    \node[circle,draw,fill=red!20,minimum size=0.5cm] at (\i*0.8,2) {\tiny$h_{\i}$};
    \draw[->,thick] (\i*0.8,2.6) -- (\i*0.8,2.3);
    \pgfmathtruncatemacro{\prev}{\i-1}
    \draw[->,thick,blue] (\prev*0.8+0.3,2) -- (\i*0.8-0.3,2);
}

% 标注
\node[below,font=\small] at (1.2,1.3) {用前缀更新隐状态};
\node[below,font=\small,green!60!black] at (5.6,1.3) {采样生成新字符};

\end{tikzpicture}
\end{center}

\begin{definition}[两个阶段]
\begin{enumerate}
    \item \textbf{预热}:用前缀字符更新隐状态,不输出
    \item \textbf{生成}:每步采样一个字符,作为下一步输入
\end{enumerate}
\end{definition}


\paragraph{随机采样 vs 贪心采样}


\textbf{贪心采样:}
\begin{lstlisting}[basicstyle=\ttfamily\tiny]
y_hat = net(X, state)
pred = y_hat.argmax(dim=1)
\end{lstlisting}

\textbf{特点:}
\begin{itemize}
    \item 确定性
    \item 可能重复
    \item 缺乏多样性
\end{itemize}


\textbf{随机采样:}
\begin{lstlisting}[basicstyle=\ttfamily\tiny]
y_hat = net(X, state)
probs = F.softmax(y_hat, dim=1)
pred = torch.multinomial(
    probs, num_samples=1
).squeeze()
\end{lstlisting}

\textbf{特点:}
\begin{itemize}
    \item 随机性
    \item 多样化
    \item 更真实
\end{itemize}


\vspace{0.5cm}
\begin{example}[例子]
概率:``a'':0.5, ``b'':0.3, ``c'':0.2\\
贪心:总是选``a'' | 随机:按概率采样
\end{example}


\paragraph{带温度的采样}
\begin{lstlisting}
def predict_with_temperature(prefix, num_preds, net, vocab, 
                            device, temperature=1.0):
    """ parameters temperature: parameters - temperature > 1: () - temperature < 1: () - temperature = 1: """
    state = net.begin_state(batch_size=1, device=device)
    outputs = [vocab[prefix[0]]]
    
    get_input = lambda: torch.tensor([outputs[-1]], 
                                     device=device).reshape((1, 1))
    
    for y in prefix[1:]:
        _, state = net(get_input(), state)
        outputs.append(vocab[y])
    
    # prediction
    for _ in range(num_preds):
        y, state = net(get_input(), state)
        y = y / temperature
        probs = F.softmax(y, dim=1)
        pred = torch.multinomial(probs, num_samples=1).item()
        outputs.append(pred)
    
    return ''.join([vocab.idx_to_token[i] for i in outputs])
\end{lstlisting}


\paragraph{温度参数的作用}
\begin{center}
\begin{tikzpicture}[scale=1.0]
% 坐标轴
\draw[->,thick] (0,0) -- (6,0) node[right] {字符};
\draw[->,thick] (0,0) -- (0,3.5) node[above] {概率};

% 原始分布 (T=1)
\draw[blue,thick] (1,2.5) -- (1,0) node[below,font=\tiny] {a};
\draw[blue,thick] (2,1.5) -- (2,0) node[below,font=\tiny] {b};
\draw[blue,thick] (3,0.8) -- (3,0) node[below,font=\tiny] {c};
\draw[blue,thick] (4,0.4) -- (4,0) node[below,font=\tiny] {d};
\node[blue] at (5.5,2) {$T=1$};

% 低温 (T=0.5)
\draw[red,thick,dashed] (1,3.2) -- (1,0);
\draw[red,thick,dashed] (2,1.0) -- (2,0);
\draw[red,thick,dashed] (3,0.3) -- (3,0);
\draw[red,thick,dashed] (4,0.1) -- (4,0);
\node[red] at (5.5,3.2) {$T=0.5$ (尖锐)};

% 高温 (T=2)
\draw[green!60!black,thick,dotted] (1,1.8) -- (1,0);
\draw[green!60!black,thick,dotted] (2,1.6) -- (2,0);
\draw[green!60!black,thick,dotted] (3,1.3) -- (3,0);
\draw[green!60!black,thick,dotted] (4,1.0) -- (4,0);
\node[green!60!black] at (5.5,0.5) {$T=2$ (平滑)};

\end{tikzpicture}
\end{center}

\begin{definition}[温度效果]
\begin{itemize}
    \item $T \rightarrow 0$:接近贪心,总选最大概率
    \item $T = 1$:原始分布
    \item $T \rightarrow \infty$:接近均匀分布,完全随机
\end{itemize}
\end{definition}



\subsection{梯度裁剪}

\subsection{8.5.5 梯度裁剪}

\subsection{8.5.5 梯度裁剪}
\begin{center}
\Large \textcolor{blue}{第五步:防止梯度爆炸}
\end{center}


\paragraph{为什么需要梯度裁剪?}
\textbf{RNN的梯度问题:}

\begin{center}
\begin{tikzpicture}[scale=1.0]
% 正常梯度
\draw[->,thick] (0,0) -- (0,3.5) node[above] {梯度范数};
\draw[->,thick] (0,0) -- (8,0) node[right] {训练步数};

% 梯度曲线
\draw[blue,thick] (0.5,0.5) -- (1,0.6) -- (1.5,0.7) -- (2,0.8);
\draw[blue,thick] (2,0.8) -- (2.5,1.5) -- (3,3.2);
\node[blue,above] at (3,3.2) {爆炸!};

% 裁剪后
\draw[green!60!black,thick,dashed] (0.5,0.5) -- (1,0.6) -- (1.5,0.7) -- (2,0.8) -- (2.5,1.0) -- (3,1.0) -- (3.5,0.9);

% 阈值线
\draw[red,dashed,thick] (0,1) -- (8,1);
\node[red,left] at (0,1) {阈值};

latex
% 标注
\node[green!60!black,below] at (5,0.5) {裁剪后稳定};

\end{tikzpicture}
\end{center}

\begin{theorem}[梯度爆炸的危害]
\begin{itemize}
    \item 参数更新过大
    \item 损失突然变为NaN
    \item 训练崩溃
\end{itemize}
\end{theorem}


\paragraph{梯度裁剪原理}
\textbf{核心思想:}限制梯度的L2范数不超过阈值

\begin{definition}[算法]
设所有梯度为 $\mathbf{g} = [g_1, g_2, \ldots, g_n]$

1. 计算梯度范数:$\|\mathbf{g}\| = \sqrt{\sum_{i=1}^{n} g_i^2}$

2. 如果 $\|\mathbf{g}\| > \theta$,则缩放:
$$\mathbf{g} \leftarrow \frac{\theta}{\|\mathbf{g}\|} \mathbf{g}$$
\end{definition}

\begin{example}[例子]
\begin{itemize}
    \item 梯度:$\mathbf{g} = [3, 4]$,范数 $\|\mathbf{g}\| = 5$
    \item 阈值:$\theta = 1$
    \item 裁剪后:$\mathbf{g} = \frac{1}{5}[3, 4] = [0.6, 0.8]$,范数=1
\end{itemize}
\end{example}


\paragraph{梯度裁剪实现}
\begin{lstlisting}
def grad_clipping(params, theta):
    """ params: modelparameters theta: """
    # parametersL2
    norm = torch.sqrt(sum(torch.sum((p.grad ** 2)) for p in params))
    
    if norm > theta:
        for param in params:
            param.grad[:] *= theta / norm
    
    return norm  # Return原始范数用于监控

# optimizer.step()
theta = 1.0  # comment
grad_norm = grad_clipping(net.params, theta)
print(f"梯度范数: {grad_norm:.4f}")
\end{lstlisting}

\begin{definition}[关键点]
\begin{itemize}
    \item 在反向传播后、参数更新前调用
    \item 阈值通常取1或5
    \item 保持梯度方向不变,只改变大小
\end{itemize}
\end{definition}


\paragraph{梯度裁剪的效果}
\begin{center}
\begin{tikzpicture}[scale=1.0]
% 对比图
\node[draw,fill=red!20,text width=5cm,align=center] at (0,2) {
\textbf{不裁剪}\\
梯度:$[300, 400]$\\
范数:$500$\\
更新步长:巨大\\
$\Rightarrow$ 参数爆炸
};

\draw[->,ultra thick] (2.7,2) -- (3.3,2);

\node[draw,fill=green!20,text width=5cm,align=center] at (6.5,2) {
\textbf{裁剪后($\theta=1$)}\\
梯度:$[0.6, 0.8]$\\
范数:$1$\\
更新步长:合理\\
$\Rightarrow$ 训练稳定
};

\end{tikzpicture}
\end{center}

\begin{definition}[为什么有效?]
\begin{itemize}
    \item 限制了单步更新的最大幅度
    \item 避免了参数的剧烈变化
    \item 保持了梯度的方向信息
\end{itemize}
\end{definition}


\paragraph{监控梯度范数}
\begin{lstlisting}
def train_epoch_with_grad_monitor(net, train_iter, loss, updater, 
                                  device, theta):
    """trainingepoch"""
    grad_norms = []  # comment
    
    for X, Y in train_iter:
        state = net.begin_state(batch_size=X.shape[0], device=device)
        y, state = net(X.to(device), state)
        
        # loss
        l = loss(y, Y.T.reshape(-1).to(device))
        
        updater.zero_grad()
        l.backward()
        
        grad_norm = grad_clipping(net.params, theta)
        grad_norms.append(grad_norm.item())
        
        # parameters
        updater.step()
    
    return grad_norms

import matplotlib.pyplot as plt
plt.plot(grad_norms)
plt.axhline(y=theta, color='r', linestyle='--', label='阈值')
plt.ylabel('梯度范数')
plt.xlabel('batch')
plt.legend()
\end{lstlisting}


\paragraph{阈值选择}
\begin{center}
\begin{tabular}{|c|l|l|}
\hline
\textbf{阈值} & \textbf{效果} & \textbf{适用场景} \\
\hline
0.1-0.5 & 非常保守,可能训练慢 & 极不稳定的模型 \\
\hline
1.0 & \textcolor{blue}{\textbf{推荐值}} & 大多数RNN \\
\hline
5.0 & 较宽松 & 稳定的模型 \\
\hline
10+ & 几乎不裁剪 & 调试用 \\
\hline
\end{tabular}
\end{center}

\vspace{0.5cm}
\begin{definition}[经验法则]
\begin{itemize}
    \item 从$\theta=1$开始
    \item 观察训练损失曲线
    \item 如果损失突然跳跃 $\Rightarrow$ 减小$\theta$
    \item 如果训练太慢 $\Rightarrow$ 增大$\theta$
\end{itemize}
\end{definition}



\subsection{训练}

\subsection{8.5.6 训练}

\subsection{8.5.6 训练}
\begin{center}
\Large \textcolor{blue}{第六步:训练完整的RNN模型}
\end{center}


\paragraph{训练流程总览}
\begin{center}
\begin{tikzpicture}[scale=0.85]
% 流程图
\node[draw,fill=yellow!20,rounded corners,text width=9cm] at (5,5) {
\textbf{1. 数据准备}\\
加载数据迭代器、词表
};

\draw[->,thick] (5,4.6) -- (5,4.2);

\node[draw,fill=blue!20,rounded corners,text width=9cm] at (5,3.8) {
\textbf{2. 模型初始化}\\
创建RNN模型、初始化参数
};

\draw[->,thick] (5,3.4) -- (5,3);

\node[draw,fill=green!20,rounded corners,text width=9cm] at (5,2.6) {
\textbf{3. 循环训练}\\
for epoch in epochs:\\
\quad for batch in train\_iter:\\
\quad \quad 前向传播 $\rightarrow$ 计算损失 $\rightarrow$ 反向传播\\
\quad \quad 梯度裁剪 $\rightarrow$ 更新参数
};

\draw[->,thick] (5,2) -- (5,1.6);

\node[draw,fill=orange!20,rounded corners,text width=9cm] at (5,1.2) {
\textbf{4. 评估与生成}\\
计算困惑度、生成文本
};

\end{tikzpicture}
\end{center}


\paragraph{训练一个epoch}
\begin{lstlisting}
def train_epoch_ch8(net, train_iter, loss, updater, device, 
                    use_random_iter):
    """trainingepoch"""
    state, timer = None, Timer()
    metric = Accumulator(2)  # comment:(loss和, tokens数)
    
    for X, Y in train_iter:
        if state is None or use_random_iter:
            state = net.begin_state(batch_size=X.shape[0], device=device)
        else:
            if isinstance(state, tuple):
                state = tuple(s.detach() for s in state)
            else:
                state = state.detach()
        
        y = Y.T.reshape(-1).to(device)
        X = X.to(device)
        
        y_hat, state = net(X, state)
        
        # loss
        l = loss(y_hat, y).mean()
        
        updater.zero_grad()
        l.backward()
        grad_clipping(net.params, 1)  # comment
        updater.step()
        
        metric.add(l * y.numel(), y.numel())
    
    return math.exp(metric[0] / metric[1]), metric[1] / timer.stop()
\end{lstlisting}


\paragraph{为什么要分离隐状态?}
\textbf{关键代码:}\texttt{state = state.detach()}

\begin{center}
\begin{tikzpicture}[scale=0.9]
% 不分离
\node[above] at (2,3.5) {\small 不分离(错误)};
\node[draw,fill=red!20,text width=4cm] at (2,2.5) {
批次1 $\rightarrow$ 批次2 $\rightarrow$ 批次3\\
\tiny 梯度一直累积\\
\tiny 内存爆炸!
};

\draw[->,ultra thick,red] (2,2) -- (2,1.5);
\node[red] at (2,1.2) {$\times$};

% 分离
\node[above] at (7,3.5) {\small 分离(正确)};
\node[draw,fill=green!20,text width=4cm] at (7,2.5) {
批次1 | 批次2 | 批次3\\
\tiny 每批次独立计算梯度\\
\tiny 内存稳定
};

\draw[->,ultra thick,green!60!black] (7,2) -- (7,1.5);
\node[green!60!black] at (7,1.2) {$\checkmark$};

\end{tikzpicture}
\end{center}

\begin{definition}[原因]
\begin{itemize}
    \item \texttt{detach()} 切断梯度传播链
    \item 保留隐状态的\textbf{数值},丢弃\textbf{计算图}
    \item 防止BPTT跨越多个批次(内存爆炸)
\end{itemize}
\end{definition}


\paragraph{完整训练函数}
\begin{lstlisting}
def train_ch8(net, train_iter, vocab, lr, num_epochs, device,
              use_random_iter=False):
    """trainingRNNmodel"""
    loss = nn.CrossEntropyLoss()
    # optimizer
    if isinstance(net, nn.Module):
        updater = torch.optim.SGD(net.parameters(), lr)
    else:
        updater = lambda batch_size: sgd(net.params, lr, batch_size)
    
    # prediction
    predict = lambda prefix: predict_ch8(prefix, 50, net, vocab, device)
    
    # training
    for epoch in range(num_epochs):
        ppl, speed = train_epoch_ch8(net, train_iter, loss, updater,
                                     device, use_random_iter)
        
        # 10epoch
        if (epoch + 1) % 10 == 0:
            print(predict('time traveller'))
    
    print(f'困惑度 {ppl:.1f}, {speed:.1f} tokens/秒 on {str(device)}')
    print(predict('time traveller'))
    print(predict('traveller'))
\end{lstlisting}


\paragraph{运行训练}
\begin{lstlisting}
# parameters
num_epochs = 500
lr = 1
batch_size = 32
num_steps = 35
num_hiddens = 512

# data
train_iter, vocab = load_data_time_machine(batch_size, num_steps)

# model
device = torch.device('cuda' if torch.cuda.is_available() else 'cpu')
net = RNNModelScratch(len(vocab), num_hiddens, device,
                      get_params, init_rnn_state, rnn)

# training
train_ch8(net, train_iter, vocab, lr, num_epochs, device)
\end{lstlisting}

\textbf{训练输出示例:}
\begin{verbatim}
困惑度 1.2, 8500.3 词元/秒 on cuda:0
time traveller smiled round at us then still s
traveller smiled round at us then still smiling
\end{verbatim}


\paragraph{训练过程可视化}
\begin{center}
\begin{tikzpicture}[scale=1.0]
% 困惑度曲线
\draw[->,thick] (0,0) -- (8,0) node[right] {Epoch};
\draw[->,thick] (0,0) -- (0,4) node[above] {困惑度};

% 曲线
\draw[blue,thick] (0.5,3.5) -- (1,2.8) -- (1.5,2.2) -- (2,1.8) -- (2.5,1.5) -- (3,1.3) -- (3.5,1.2) -- (7.5,1.1);

% 标注关键点
\fill[red] (0.5,3.5) circle (2pt) node[above,font=\tiny] {初始:高};
\fill[green!60!black] (3,1.3) circle (2pt) node[above,font=\tiny] {快速下降};
\fill[blue] (7.5,1.1) circle (2pt) node[above,font=\tiny] {收敛};

% 分阶段
\draw[dashed] (0,2) -- (8,2);
\node[left,font=\tiny] at (0,2) {目标值};

\node[below,text width=8cm,align=center] at (4,-0.8) {
\small 典型训练曲线:初期快速下降,后期缓慢收敛
};

\end{tikzpicture}
\end{center}


\paragraph{生成文本示例}
\textbf{训练不同阶段的生成效果:}

\begin{center}
\begin{tikzpicture}[scale=0.9]
% Epoch 10
\node[draw,fill=red!20,text width=10cm] at (5,3.5) {
\textbf{Epoch 10 (困惑度 $\sim$10):}\\
\texttt{time travellerkkkkkkkkkk...}\\
\tiny 几乎随机,大量重复
};

% Epoch 100
\node[draw,fill=orange!20,text width=10cm] at (5,2.3) {
\textbf{Epoch 100 (困惑度 $\sim$3):}\\
\texttt{time traveller the the time...}\\
\tiny 开始有简单结构
};

% Epoch 500
\node[draw,fill=green!20,text width=10cm] at (5,1.1) {
\textbf{Epoch 500 (困惑度 $\sim$1.2):}\\
\texttt{time traveller smiled round at us}\\
\tiny 流畅、符合语法
};

\end{tikzpicture}
\end{center}

\begin{theorem}[观察]
困惑度越低,生成文本质量越高
\end{theorem}


\paragraph{调试技巧}
\begin{definition}[常见问题与解决]
\begin{enumerate}
    \item \textbf{损失变NaN}
    \begin{itemize}
        \item 检查学习率(降低)
        \item 检查梯度裁剪(降低阈值)
        \item 检查数据(是否有异常值)
    \end{itemize}
    
    \item \textbf{困惑度不下降}
    \begin{itemize}
        \item 增加隐藏单元数
        \item 增加训练轮数
        \item 调整学习率
    \end{itemize}
    
    \item \textbf{内存不足}
    \begin{itemize}
        \item 减小batch\_size
        \item 减小num\_steps
        \item 减小num\_hiddens
    \end{itemize}
\end{enumerate}
\end{definition}


\paragraph{保存和加载模型}
\begin{lstlisting}
# modelparameters
def save_model(net, path):
    """modelparameters"""
    if isinstance(net, nn.Module):
        torch.save(net.state_dict(), path)
    else:
        # model
        torch.save([p.data for p in net.params], path)

# modelparameters
def load_model(net, path):
    """加载modelparameters"""
    if isinstance(net, nn.Module):
        net.load_state_dict(torch.load(path))
    else:
        params = torch.load(path)
        for p, p_load in zip(net.params, params):
            p.data = p_load

save_model(net, 'rnn_model.pth')
# ...  ...
load_model(net, 'rnn_model.pth')
\end{lstlisting}


\paragraph{超参数调优建议}
\begin{center}
\begin{tabular}{|l|c|l|}
\hline
\textbf{超参数} & \textbf{推荐范围} & \textbf{说明} \\
\hline
学习率 & 0.1-5 & RNN可用较大学习率 \\
\hline
隐藏单元数 & 256-1024 & 越大越好,但内存限制 \\
\hline
批量大小 & 32-128 & 取决于GPU内存 \\
\hline
序列长度 & 35-100 & 太长会梯度消失 \\
\hline
梯度裁剪 & 1-5 & 从1开始 \\
\hline
训练轮数 & 500-2000 & 直到困惑度收敛 \\
\hline
\end{tabular}
\end{center}

\vspace{0.5cm}
\begin{definition}[调优策略]
\begin{enumerate}
    \item 先用小模型(num\_hiddens=128)快速迭代
    \item 确认能训练后,逐步增大模型
    \item 监控困惑度和生成质量
\end{enumerate}
\end{definition}

\section{8.6 循环神经网络的简洁实现}

\subsection{8.6 循环神经网络的简洁实现}
\begin{center}
\Large \textcolor{blue}{使用PyTorch高级API快速实现RNN}
\end{center}


\subsection{8.6节 - 章节导航}
\begin{definition}[本节内容]
\begin{itemize}
    \item 8.6.1 定义模型
    \item 8.6.2 训练与预测
\end{itemize}
\end{definition}

\begin{theorem}[学习目标]
\begin{enumerate}
    \item 掌握PyTorch的\texttt{nn.RNN}模块
    \item 理解简洁实现与从零实现的对应关系
    \item 学会用高级API快速搭建模型
    \item 对比两种实现方式的优缺点
\end{enumerate}
\end{theorem}


\paragraph{从零实现 vs 简洁实现}
\begin{center}
\begin{tikzpicture}[scale=0.9]
% 从零实现
\node[draw,fill=yellow!20,rounded corners,text width=5cm,align=center] at (0,2.5) {
\textbf{从零实现(8.5节)}\\
手写每一行代码
};

\node[draw,fill=orange!20,text width=5cm] at (0,1.3) {
\textbf{优点:}\\
$\checkmark$ 理解原理\\
$\checkmark$ 灵活定制\\
$\checkmark$ 教学价值高
};

\node[draw,fill=red!20,text width=5cm] at (0,-0.2) {
\textbf{缺点:}\\
$\times$ 代码量大\\
$\times$ 容易出错\\
$\times$ 优化不足
};

% 简洁实现
\node[draw,fill=blue!20,rounded corners,text width=5cm,align=center] at (7,2.5) {
\textbf{简洁实现(8.6节)}\\
使用PyTorch API
};

\node[draw,fill=green!20,text width=5cm] at (7,1.3) {
\textbf{优点:}\\
$\checkmark$ 代码简洁\\
$\checkmark$ 高度优化\\
$\checkmark$ 功能丰富
};

\node[draw,fill=orange!20,text width=5cm] at (7,-0.2) {
\textbf{缺点:}\\
$\times$ 细节隐藏\\
$\times$ 定制受限\\
$\times$ 需要理解8.5
};

\end{tikzpicture}
\end{center}



