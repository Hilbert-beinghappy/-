\section{未命名节}

\subsection{读取数据集}

\subsection{8.2.1 读取数据集}

\subsection{8.2.1 读取数据集}
\begin{center}
\Large \textcolor{blue}{第一步:获取原始文本}
\end{center}


\paragraph{为什么需要文本预处理?}
\begin{center}
\begin{tikzpicture}
\node[text width=8cm,draw,fill=yellow!20,rounded corners] at (0,3) {
\textbf{原始文本:}\\
``The Time Machine, by H.G. Wells [1898]''\\
``It was at ten o'clock to-day that...''
};

\draw[->,ultra thick] (0,2.2) -- (0,1.5);
\node at (2,1.8) {需要转换};

\node[text width=8cm,draw,fill=green!20,rounded corners] at (0,0.5) {
\textbf{数字序列:}\\
234, 567, 123, 89, 456, ...\\
计算机才能处理!
};
\end{tikzpicture}
\end{center}

\begin{definition}[文本预处理的挑战]
\begin{itemize}
    \item 文本是\textbf{离散}的符号
    \item 包含大小写、标点、特殊字符
    \item 需要统一的处理流程
\end{itemize}
\end{definition}


\paragraph{时光机器数据集}
\textbf{数据集:}H.G. Wells的科幻小说《时光机器》(The Time Machine, 1895)

\begin{example}[为什么选这个数据集?]
\begin{itemize}
    \item 英文文本,便于处理
    \item 大小适中(约30KB,3万多词)
    \item 语言规范,句子结构清晰
    \item 开源免费
\end{itemize}
\end{example}

\begin{center}
\begin{tikzpicture}
\node[draw,fill=yellow!20,rounded corners,text width=8cm,align=left,font=\small] at (5,0) {
\textbf{原始文本片段:}\\
\textit{``The Time Machine, by H. G. Wells [1898]''}\\
\textit{``I''}\\
\textit{``The Time Traveller (for so it will be convenient to speak of him)''}\\
\textit{``was expounding a recondite matter to us...''}
};
\end{tikzpicture}
\end{center}

\begin{theorem}[特点]
包含标题、章节标记、对话、描述等多种文本形式
\end{theorem}


\paragraph{读取文本文件}
\begin{lstlisting}
import os
import requests

# data
def download_data():
    url = 'http://d2l-data.s3-accelerate.amazonaws.com/timemachine.txt'
    response = requests.get(url)
    with open('timemachine.txt', 'wb') as f:
        f.write(response.content)

def read_time_machine():
    with open('timemachine.txt', 'r', encoding='utf-8') as f:
        lines = f.readlines()
    return lines

lines = read_time_machine()
print(f"Total lines: {len(lines)}")
print(f"First 3 lines:")
for i in range(3):
    print(f"Line {i}: {lines[i]}")
\end{lstlisting}

\textbf{输出:}
\begin{verbatim}
Total lines: 3221
Line 0: The Time Machine, by H. G. Wells [1898]
Line 1: 
Line 2: I
\end{verbatim}


\paragraph{原始文本的"脏数据"}
\textbf{问题:}原始文本包含很多"噪声"



\textbf{常见噪声:}
\begin{itemize}
    \item 大小写混杂\\
    \texttt{"The", "the", "THE"}
    \item 标点符号\\
    \texttt{"Hello!", "Hello?", "Hello."}
    \item 特殊字符\\
    \texttt{"don't", "it's", "---"}
    \item 空行和空格
    \item 数字\\
    \texttt{"year 1898"}
\end{itemize}


\begin{theorem}[为什么要清洗?]
\begin{itemize}
    \item 减少词汇量
    \item 统一表示形式
    \item 避免稀疏性问题
    \item 提高模型泛化能力
\end{itemize}
\end{theorem}

\begin{example}[例子]
不清洗:\\
"The", "the", "THE" $\to$ 3个词

清洗后:\\
"the" $\to$ 1个词
\end{example}



\paragraph{文本清洗}
\begin{lstlisting}
import re

def read_time_machine():
    with open('timemachine.txt', 'r') as f:
        lines = f.readlines()
    
    clean_lines = []
    for line in lines:
        # 1. 
        line = line.lower()
        # 2. 
        line = re.sub('[^a-z]+', ' ', line)
        # 3. 
        line = line.strip()
        # 4. 
        if line:
            clean_lines.append(line)
    
    return clean_lines

lines_raw = read_raw()
lines_clean = read_time_machine()
print("Raw:", lines_raw[0])
print("Clean:", lines_clean[0])
\end{lstlisting}

\textbf{输出:}
\begin{verbatim}
Raw: The Time Machine, by H. G. Wells [1898]
Clean: the time machine by h g wells
\end{verbatim}


\paragraph{正则表达式快速入门}
\textbf{正则表达式(Regex):}强大的文本匹配工具

\begin{center}
\begin{tabular}{|l|l|l|}
\hline
\textbf{模式} & \textbf{含义} & \textbf{示例} \\
\hline
\texttt{[a-z]} & 匹配小写字母 & \texttt{"a", "b", "z"} \\
\texttt{[A-Z]} & 匹配大写字母 & \texttt{"A", "B", "Z"} \\
\texttt{[0-9]} & 匹配数字 & \texttt{"0", "5", "9"} \\
\texttt{[a-zA-Z]} & 匹配所有字母 & \texttt{"a", "B", "z"} \\
\texttt{\^{}} & 取反 & \texttt{[\^{}a-z]} = 非小写字母 \\
\texttt{+} & 一个或多个 & \texttt{[a-z]+} = 多个字母 \\
\hline
\end{tabular}
\end{center}

\begin{example}[应用示例]
\begin{itemize}
    \item 
    把所有\textbf{非小写字母}替换为空格
    
    \item 
    把所有\textbf{数字}替换为特殊标记
\end{itemize}
\end{example}


\paragraph{清洗前后对比}


\textbf{清洗前:}
\begin{lstlisting}[basicstyle=\ttfamily\tiny]
The Time Machine, 
by H. G. Wells [1898]

I

The Time Traveller 
(for so it will be 
convenient to speak of him)
was expounding a 
recondite matter to us.
\end{lstlisting}


\textbf{清洗后:}
\begin{lstlisting}[basicstyle=\ttfamily\tiny]
the time machine by h g wells

i

the time traveller for so 
it will be convenient to 
speak of him was expounding 
a recondite matter to us
\end{lstlisting}


\vspace{0.5cm}
\begin{definition}[观察]
\begin{itemize}
    \item 全部小写
    \item 标点消失
    \item 数字消失
    \item 保留了空格(词与词之间的分隔)
\end{itemize}
\end{definition}

\subsection{8.2.2 词元化}

\subsection{8.2.2 词元化(Tokenization)}
\begin{center}
\Large \textcolor{blue}{把文本拆分成"词元"(token)}
\end{center}


\paragraph{什么是词元?}
\textbf{词元(Token):}文本的基本处理单位

\begin{center}
\begin{tikzpicture}[scale=1.1]
% 原始句子
\node[text width=10cm,align=center,font=\large] at (5,3.5) {
\textbf{原始句子:}"the time traveller was smiling"
};

\draw[->,ultra thick] (5,3) -- (5,2.5);

% 词元化
\node[draw,fill=blue!20,rounded corners] at (1,2) {the};
\node[draw,fill=blue!20,rounded corners] at (2.5,2) {time};
\node[draw,fill=blue!20,rounded corners] at (4.2,2) {traveller};
\node[draw,fill=blue!20,rounded corners] at (6,2) {was};
\node[draw,fill=blue!20,rounded corners] at (7.5,2) {smiling};

\node[below,font=\small] at (4.5,1.3) {5个词元(tokens)};

\end{tikzpicture}
\end{center}

\begin{theorem}[为什么需要词元化?]
\begin{itemize}
    \item 神经网络只能处理\textbf{固定大小}的输入
    \item 需要将连续的文本拆分成\textbf{离散单元}
    \item 词元是\textbf{特征提取}的基础
\end{itemize}
\end{theorem}


\paragraph{词元化的三种粒度}
\begin{center}
\begin{tikzpicture}[scale=1.0]
% 原始句子
\node[font=\large] at (6,4.5) {\textbf{句子:}"natural language processing"};

% 词级别
\node[draw,fill=green!20,text width=10cm,align=left] at (6,3.2) {
\textbf{1. 词级别(Word-level)}\\
\texttt{["natural", "language", "processing"]} \\
$\to$ 3个词元
};

% 字符级别
\node[draw,fill=yellow!20,text width=10cm,align=left] at (6,1.8) {
\textbf{2. 字符级别(Character-level)}\\
\texttt{["n","a","t","u","r","a","l"," ","l","a","n","g",...]} \\
$\to$ 27个词元
};

% 子词级别
\node[draw,fill=orange!20,text width=10cm,align=left] at (6,0.4) {
\textbf{3. 子词级别(Subword-level)}\\
\texttt{["natur", "al", "language", "process", "ing"]} \\
$\to$ 5个词元
};

\end{tikzpicture}
\end{center}


\paragraph{词级别词元化}
\textbf{策略:}按空格或标点分割



\textcolor{blue}{\textbf{优点:}}
\begin{itemize}
    \item \checkmark 语义清晰
    \item \checkmark 符合人类理解
    \item \checkmark 适合大多数任务
\end{itemize}

\textcolor{red}{\textbf{缺点:}}
\begin{itemize}
    \item $\times$ 词表很大
    \item $\times$ 未登录词(OOV)问题
    \item $\times$ 形态变化(如:run/running)
\end{itemize}


\begin{example}[OOV问题]
\textbf{训练集:}\\
\texttt{"dog", "cat", "run"}

\textbf{测试时遇到:}\\
\texttt{"dogs", "running", "bird"}

$\rightarrow$ 模型不认识!
\end{example}


\vspace{0.3cm}
\begin{definition}[词表大小]
英文:常用词汇 5万-10万\\
中文:常用汉字 3千-5千,但词汇量更大
\end{definition}


\paragraph{词级别词元化代码}
\begin{lstlisting}
def tokenize_word_level(text):
    """Simple function"""
    return text.split()

text = "the time traveller was smiling"
tokens = tokenize_word_level(text)
print(f"tokens: {tokens}")
print(f"tokens数: {len(tokens)}")

# Output
# tokens: ['the', 'time', 'traveller', 'was', 'smiling']
# tokens: 5

# data
lines = read_time_machine()
all_tokens = []
for line in lines:
    all_tokens.extend(tokenize_word_level(line))

print(f"总tokens数: {len(all_tokens)}")
print(f"前20个tokens: {all_tokens[:20]}")
\end{lstlisting}

\textbf{输出:}
\begin{lstlisting}[language=Python]
总tokens数: 32775
前20个tokens: ['the', 'time', 'machine', 'by', 'h', 'g', ...]
\end{lstlisting}


\paragraph{字符级别词元化}
\textbf{策略:}把每个字符作为一个词元



\textcolor{blue}{\textbf{优点:}}
\begin{itemize}
    \item \checkmark 词表极小(26个字母)
    \item \checkmark 无OOV问题
    \item \checkmark 能处理任意文本
    \item \checkmark 对拼写错误鲁棒
\end{itemize}

\textcolor{red}{\textbf{缺点:}}
\begin{itemize}
    \item $\times$ 序列变得很长
    \item $\times$ 训练慢
    \item $\times$ 难以捕捉词级别语义
\end{itemize}


\begin{example}[例子]
\textbf{输入:}\\
\texttt{"cat"}

\textbf{词级别:}\\
1个词元:\texttt{["cat"]}

\textbf{字符级别:}\\
3个词元:\texttt{["c","a","t"]}
\end{example}

\begin{theorem}[序列长度]
"hello world" (11个字符)\\
vs \\
"hello world" (2个单词)\\
序列长度差5倍!
\end{theorem}



\paragraph{字符级别词元化代码}
\begin{lstlisting}
def tokenize_char_level(text):
    """tokens"""
    return list(text)

text = "the time"
tokens = tokenize_char_level(text)
print(f"tokens: {tokens}")
print(f"tokens数: {len(tokens)}")

# Output
# tokens: ['t', 'h', 'e', ' ', 't', 'i', 'm', 'e']
# tokens: 8

# Calculate vocabulary size
lines = read_time_machine()
text = ' '.join(lines)
chars = set(text)
print(f"字符集大小: {len(chars)}")
print(f"字符集: {sorted(chars)}")
\end{lstlisting}

\textbf{输出:}
\begin{verbatim}
字符集大小: 27
字符集: [' ', 'a', 'b', 'c', ..., 'z']
\end{verbatim}

\begin{definition}[观察]
词表从5万缩小到27个!但序列长度增加5-10倍
\end{definition}


\paragraph{子词级别词元化}
\textbf{核心思想:}介于词和字符之间的折中方案

\begin{center}
\begin{tikzpicture}[scale=1.0]
% 示意图
\node[font=\large] at (5,3.5) {\textbf{例子:}``unhappiness''};

\draw[->,thick] (5,3.2) -- (5,2.7);

% 词级别
\node[draw,fill=red!20] at (2,2) {``unhappiness''};
\node[below,font=\tiny,text width=3cm,align=center] at (2,1.5) {词级别:1个词元\\但可能OOV};

% 子词级别
\node[draw,fill=green!20] at (5,2) {``un'' + ``happi'' + ``ness''};
\node[below,font=\tiny,text width=3cm,align=center] at (5,1.5) {子词级别:3个词元\\可组合成新词};

% 字符级别
\node[draw,fill=yellow!20,font=\small] at (8,2) {``u'',``n'',``h'',``a'',...};
\node[below,font=\tiny,text width=3cm,align=center] at (8,1.5) {字符级别:11个词元\\序列太长};

\end{tikzpicture}
\end{center}

\begin{definition}[子词的好处]
\begin{itemize}
    \item 可以组合出训练集中没见过的词
    \item 例如:"unhappy" = "un" + "happy"
    \item 即使没见过"unhappy",也能理解它的意思
\end{itemize}
\end{definition}


\paragraph{常见子词算法}
\begin{enumerate}
\item \textbf{BPE (Byte Pair Encoding)}
\begin{itemize}
    \item 从字符开始,迭代合并最频繁的相邻pair
    \item GPT-2, RoBERTa使用
\end{itemize}

\item \textbf{WordPiece}
\begin{itemize}
    \item 类似BPE,但用likelihood而非频率
    \item BERT使用
\end{itemize}

\item \textbf{SentencePiece}
\begin{itemize}
    \item 直接从原始文本训练,不依赖预分词
    \item 支持多语言
    \item T5, XLNet使用
\end{itemize}
\end{enumerate}

\begin{theorem}[现代NLP趋势]
大模型几乎都使用\textbf{子词级别}词元化!\\
平衡了词表大小和序列长度
\end{theorem}


\paragraph{三种粒度对比总结}
\begin{center}
\begin{tabular}{|l|c|c|c|}
\hline
\textbf{特性} & \textbf{词级别} & \textbf{字符级别} & \textbf{子词级别} \\
\hline
词表大小 & 很大(5万+) & 很小($\sim$100) & 中等(3万) \\
\hline
序列长度 & 短 & 很长 & 中等 \\
\hline
OOV问题 & 严重 & 无 & 轻微 \\
\hline
语义保留 & 好 & 差 & 较好 \\
\hline
训练速度 & 快 & 慢 & 中等 \\
\hline
应用场景 & 传统NLP & 生成任务 & 现代大模型 \\
\hline
\end{tabular}
\end{center}

\vspace{0.5cm}
\begin{definition}[本节使用]
为了简单起见,我们使用\textbf{词级别}词元化\\
在8.4节实现RNN时,会用\textbf{字符级别}
\end{definition}


\paragraph{完整词元化函数}
\begin{lstlisting}
def tokenize(lines, token_type='word'):
    """ tokens lines: token_type: 'word' 'char' Return: tokens """
    if token_type == 'word':
        return [line.split() for line in lines]
    elif token_type == 'char':
        return [list(line) for line in lines]
    else:
        raise ValueError(f"Unknown token_type: {token_type}")

lines = read_time_machine()
tokens_word = tokenize(lines, 'word')
tokens_char = tokenize(lines, 'char')

print(f"词级别 - 第1行: {tokens_word[0]}")
print(f"字符级别 - 第1行: {tokens_char[0]}")
\end{lstlisting}

\textbf{输出:}
\begin{verbatim}
词级别 - 第1行: ['the', 'time', 'machine', 'by', 'h', 'g', 'wells']
字符级别 - 第1行: ['t','h','e',' ','t','i','m','e',...]
\end{verbatim}



\subsection{词表}

\subsection{8.2.3 词表}

\subsection{8.2.3 词表(Vocabulary)}
\begin{center}
\Large \textcolor{blue}{建立词元到数字的映射}
\end{center}


\paragraph{为什么需要词表?}
\textbf{问题:}神经网络不能直接处理文本

\begin{center}
\begin{tikzpicture}[scale=1.1]
% 文本
\node[draw,fill=yellow!20,rounded corners,text width=3cm,align=center] at (0,2) {
文本\\
"cat"
};

\draw[->,ultra thick,red] (1.5,2) -- (2.5,2);
\node[red,above] at (2,2.2) {\text{×}不能直接用};

% 神经网络
\node[draw,fill=blue!20,rounded corners,text width=3cm,align=center] at (4.5,2) {
神经网络\\
需要数字
};

% 解决方案
\draw[->,ultra thick,green!60!black] (0,1.2) -- (0,0.5);
\node[draw,fill=green!20,rounded corners,text width=3cm,align=center] at (0,0) {
数字\\
词表["cat"]=42
};

\draw[->,ultra thick,green!60!black] (1.5,0) -- (2.5,2);
\node[green!60!black,below] at (2,1) {$\checkmark$通过词表转换};

\end{tikzpicture}
\end{center}

\begin{theorem}[词表的作用]
建立\textbf{词元 $\leftrightarrow$ 数字}的双向映射
\end{theorem}


\paragraph{词表构建示例}
\textbf{输入文本:}"the cat sat on the mat"

\begin{center}
\begin{tikzpicture}[scale=1.0]
% 步骤1:词元化
\node[draw,fill=yellow!20,text width=8cm] at (5,3.5) {
\textbf{步骤1:词元化}\\
\texttt{["the", "cat", "sat", "on", "the", "mat"]}
};

\draw[->,thick] (5,3) -- (5,2.5);

% 步骤2:统计频率
\node[draw,fill=orange!20,text width=8cm] at (5,2) {
\textbf{步骤2:统计词频}\\
\texttt{"the": 2, "cat": 1, "sat": 1, "on": 1, "mat": 1}
};

\draw[->,thick] (5,1.5) -- (5,1);

% 步骤3:按频率排序+分配ID
\node[draw,fill=green!20,text width=8cm,align=left] at (5,0.2) {
\textbf{步骤3:构建词表(按频率排序)}\\
\texttt{0: <unk>  (未知词)}\\
\texttt{1: the    (freq=2)}\\
\texttt{2: cat    (freq=1)}\\
\texttt{3: mat    (freq=1)}\\
\texttt{4: on     (freq=1)}\\
\texttt{5: sat    (freq=1)}
};

\end{tikzpicture}
\end{center}


\paragraph{特殊词元}
\textbf{除了普通词,词表还包含特殊标记:}

\begin{center}
\begin{tabular}{|l|l|l|}
\hline
\textbf{符号} & \textbf{含义} & \textbf{用途} \\
\hline
\texttt{<unk>} & Unknown & 未知词,词表中没有的词 \\
\hline
\texttt{<pad>} & Padding & 填充,对齐不同长度序列 \\
\hline
\texttt{<bos>} & Begin of Sequence & 句子开始标记 \\
\hline
\texttt{<eos>} & End of Sequence & 句子结束标记 \\
\hline
\texttt{<mask>} & Mask & 掩码,用于BERT等模型 \\
\hline
\end{tabular}
\end{center}

\begin{example}[例子]
\textbf{原始:}"hello world"

\textbf{加特殊标记:}\\
\texttt{[<bos>, "hello", "world", <eos>]}

\textbf{转数字(假设词表):}\\
\texttt{[1, 234, 567, 2]}
\end{example}


\paragraph{词频统计}
\begin{lstlisting}
def count_corpus(tokens):
    """tokens"""
    # tokens1D2D
    if len(tokens) == 0 or isinstance(tokens[0], list):
        # 2D1D
        tokens = [token for line in tokens for token in line]
    
    from collections import Counter
    return Counter(tokens)

lines = read_time_machine()
tokens = tokenize(lines, 'word')
counter = count_corpus(tokens)

print(f"总tokens数: {sum(counter.values())}")
print(f"不同tokens数: {len(counter)}")
print(f"最常见的10个词: {counter.most_common(10)}")
\end{lstlisting}

\textbf{输出:}
\begin{verbatim}
总词元数: 32775
不同词元数: 4580
最常见的10个词: [('the', 2261), ('i', 1267), ('and', 1245), ...]
\end{verbatim}


\paragraph{词表类实现 - 初始化}
\begin{lstlisting}
class Vocab:
    def __init__(self, tokens=None, min_freq=0, 
                 reserved_tokens=None):
        """ tokens: tokens min_freq: reserved_tokens: tokens(<pad>) """
        if tokens is None:
            tokens = []
        if reserved_tokens is None:
            reserved_tokens = []
        
        counter = count_corpus(tokens)
        self._token_freqs = sorted(counter.items(), 
                                   key=lambda x: x[1], 
                                   reverse=True)
        
        # tokens0tokens1
        self.idx_to_token = ['<unk>'] + reserved_tokens
        self.token_to_idx = {token: idx 
                for idx, token in enumerate(self.idx_to_token)}
\end{lstlisting}


\paragraph{词表类实现 - 构建映射}
\begin{lstlisting}
        for token, freq in self._token_freqs:
            if freq < min_freq:
                break
            if token not in self.token_to_idx:
                self.idx_to_token.append(token)
                self.token_to_idx[token] = len(self.idx_to_token) - 1
    
    def __len__(self):
        """vocabulary"""
        return len(self.idx_to_token)
    
    def __getitem__(self, tokens):
        """查询tokens对应的索引"""
        if not isinstance(tokens, (list, tuple)):
            return self.token_to_idx.get(tokens, self.unk)
        return [self.__getitem__(token) for token in tokens]
    
    @property
    def unk(self):
        """未知词的索引,总是0"""
        return 0
    
    def to_tokens(self, indices):
        """索引转回tokens"""
        if not isinstance(indices, (list, tuple)):
            return self.idx_to_token[indices]
        return [self.idx_to_token[index] for index in indices]
\end{lstlisting}


\paragraph{词表使用示例}
\begin{lstlisting}
# vocabulary
lines = read_time_machine()
tokens = tokenize(lines, 'word')
vocab = Vocab(tokens)

print(f"vocabulary大小: {len(vocab)}")
print(f"前10个tokens: {vocab.idx_to_token[:10]}")

# tokens
test_tokens = ['the', 'time', 'machine', 'xyz']
indices = vocab[test_tokens]
print(f"tokens: {test_tokens}")
print(f"索引: {indices}")

# tokens
back_tokens = vocab.to_tokens(indices)
print(f"转回: {back_tokens}")
\end{lstlisting}

\textbf{输出:}
\begin{verbatim}
词表大小: 4581
前10个词元: ['<unk>', 'the', 'i', 'and', 'of', 'a', ...]
词元: ['the', 'time', 'machine', 'xyz']
索引: [1, 19, 50, 0]
转回: ['the', 'time', 'machine', '<unk>']
\end{verbatim}

\begin{theorem}[注意]
'xyz'不在词表中,被映射到 \texttt{<unk>} (索引0)
\end{theorem}


\paragraph{最小频率阈值的作用}
\textbf{问题:}低频词很多,但作用不大

\begin{center}
\begin{tikzpicture}[scale=1.0]
% Zipf定律图示
\draw[->,thick] (0,0) -- (8,0) node[right] {词频排名};
\draw[->,thick] (0,0) -- (0,4) node[above] {词频};

% 曲线
\draw[blue,thick,domain=0.5:7.5,samples=100] plot (\x,{3.5/\x});

% 高频区
\fill[green!20,opacity=0.5] (0,0) rectangle (2,4);
\node[green!60!black] at (1,3.5) {\tiny 高频词};
\node[green!60!black,below] at (1,0) {\tiny 保留};

% 中频区
\fill[yellow!20,opacity=0.5] (2,0) rectangle (5,4);
\node[orange!60!black] at (3.5,2) {\tiny 中频词};

% 低频区
\fill[red!20,opacity=0.5] (5,0) rectangle (8,4);
\node[red!60!black] at (6.5,1) {\tiny 低频词};
\node[red!60!black,below] at (6.5,0) {\tiny 丢弃};

% 阈值线
\draw[red,dashed,thick] (0,0.7) -- (8,0.7);
\node[red,left] at (0,0.7) {\tiny min\_freq};

\end{tikzpicture}
\end{center}

\begin{definition}[策略]
\begin{itemize}
    \item \texttt{min\_freq=0}:保留所有词(词表大)
    \item \texttt{min\_freq=5}:至少出现5次才保留(减小词表)
    \item 低频词通常是拼写错误、人名等,可以用 \texttt{<unk>} 代替
\end{itemize}
\end{definition}


\paragraph{最小频率阈值示例}
\begin{lstlisting}
vocab_all = Vocab(tokens, min_freq=0)
vocab_5 = Vocab(tokens, min_freq=5)
vocab_10 = Vocab(tokens, min_freq=10)

print(f"min_freq=0:  vocabulary大小={len(vocab_all)}")
print(f"min_freq=5:  vocabulary大小={len(vocab_5)}")
print(f"min_freq=10: vocabulary大小={len(vocab_10)}")

counter = count_corpus(tokens)
low_freq = [token for token, freq in counter.items() if freq < 5]
print(f"频率<5的词数: {len(low_freq)}")
print(f"示例: {low_freq[:10]}")
\end{lstlisting}

\textbf{输出:}
\begin{verbatim}
min_freq=0:  词表大小=4581
min_freq=5:  词表大小=2034
min_freq=10: 词表大小=1446
频率<5的词数: 3143
示例: ['1898', 'xiv', 'pg', 'redistribution', ...]
\end{verbatim}

\begin{theorem}[观察]
68\%的词频率$<$5,但它们只占总词数的很小比例!
\end{theorem}



\subsection{整合所有功能}

\subsection{8.2.4 整合所有功能}

\subsection{8.2.4 整合所有功能}
\begin{center}
\Large \textcolor{blue}{完整的文本预处理流程}
\end{center}


\paragraph{完整流程总结}
\begin{center}
\begin{tikzpicture}[scale=0.95]
% 步骤1
\node[draw,fill=yellow!20,rounded corners,text width=10cm] at (5,4.5) {
\textbf{步骤1: 读取数据}\\
\texttt{lines = read\_time\_machine()}
};

\draw[->,thick] (5,4) -- (5,3.5);

% 步骤2
\node[draw,fill=orange!20,rounded corners,text width=10cm] at (5,3) {
\textbf{步骤2: 词元化}\\
\texttt{tokens = tokenize(lines, 'word')}
};

\draw[->,thick] (5,2.5) -- (5,2);

% 步骤3
\node[draw,fill=green!20,rounded corners,text width=10cm] at (5,1.5) {
\textbf{步骤3: 构建词表}\\
\texttt{vocab = Vocab(tokens, min\_freq=5)}
};

\draw[->,thick] (5,1) -- (5,0.5);

% 步骤4
\node[draw,fill=blue!20,rounded corners,text width=10cm] at (5,0) {
\textbf{步骤4: 转换为索引}\\
\texttt{corpus = [vocab[token] for line in tokens for token in line]}
};

\end{tikzpicture}
\end{center}


\paragraph{一站式函数}
\begin{lstlisting}
def load_corpus_time_machine(max_tokens=-1):
    """ Returndatatokensvocabulary max_tokens: tokens,-1 """
    lines = read_time_machine()
    # tokens
    tokens = tokenize(lines, 'char')  # comment
    # vocabulary
    vocab = Vocab(tokens)
    corpus = [vocab[token] for line in tokens for token in line]
    if max_tokens > 0:
        corpus = corpus[:max_tokens]
    
    return corpus, vocab

corpus, vocab = load_corpus_time_machine()
print(f"语料库length: {len(corpus)}")
print(f"vocabulary大小: {len(vocab)}")
print(f"前10个索引: {corpus[:10]}")
print(f"对应tokens: {vocab.to_tokens(corpus[:10])}")
\end{lstlisting}


\paragraph{一站式函数输出}
\begin{lstlisting}[basicstyle=\ttfamily\tiny]
# Output
语料库length: 170580
vocabulary大小: 28
前10个索引: [1, 8, 5, 0, 1, 9, 13, 5, 0, 13]
对应tokens: ['t', 'h', 'e', ' ', 't', 'i', 'm', 'e', ' ', 'm']
\end{lstlisting}

\begin{definition}[观察]
\begin{itemize}
    \item 使用\textbf{字符级别}词元化
    \item 语料库是一个很长的数字序列(17万+)
    \item 词表只有28个字符(26个字母+空格+其他)
    \item 每个索引对应词表中的一个字符
\end{itemize}
\end{definition}

\begin{theorem}[为什么用字符级别?]
\begin{itemize}
    \item RNN训练时序列不宜过长
    \item 字符级别词表小,便于演示
    \item 避免OOV问题
\end{itemize}
\end{theorem}


\paragraph{文本$\to$数字的完整示例}
\begin{center}
\begin{tikzpicture}[scale=1.0]
% 原始文本
\node[draw,fill=yellow!20,rounded corners,text width=10cm,align=center] at (5,4) {
\textbf{原始文本}\\
``the time machine''
};

\draw[->,thick] (5,3.5) -- (5,3.2);
\node[right] at (5.5,3.35) {\tiny 清洗};

% 清洗后
\node[draw,fill=orange!20,rounded corners,text width=10cm,align=center] at (5,2.8) {
\textbf{清洗后}\\
``the time machine''
};

\draw[->,thick] (5,2.4) -- (5,2.1);
\node[right] at (5.5,2.25) {\tiny 词元化(字符级)};

% 词元
\node[draw,fill=green!20,rounded corners,text width=10cm,align=center] at (5,1.7) {
\textbf{词元列表}\\
t, h, e, (space), t, i, m, e, (space), m, a, c, h, i, n, e
};

\draw[->,thick] (5,1.3) -- (5,1.0);
\node[right] at (5.5,1.15) {\tiny 词表映射};

% 索引
\node[draw,fill=blue!20,rounded corners,text width=10cm,align=center] at (5,0.6) {
\textbf{索引序列}\\
20, 8, 5, 0, 20, 9, 13, 5, 0, 13, 1, 3, 8, 9, 14, 5
};

\draw[->,thick] (5,0.2) -- (5,-0.1);

% 神经网络
\node[draw,fill=red!20,rounded corners,text width=10cm,align=center] at (5,-0.5) {
\textbf{输入神经网络} \checkmark
};

\end{tikzpicture}
\end{center}


\paragraph{词表可视化}
\begin{center}
\begin{tikzpicture}[scale=1.0]
% 标题
\node[font=\large\bfseries] at (5,4) {词表结构(字符级别)};

% 词表
\node[draw,fill=yellow!20,text width=10cm,align=left,font=\small] at (5,2) {
\textbf{词表(部分):}\\
\texttt{0: ' '  (空格)}\\
\texttt{1: 'a'}\\
\texttt{2: 'b'}\\
\texttt{3: 'c'}\\
\texttt{4: 'd'}\\
\texttt{5: 'e'}\\
\texttt{...}\\
\texttt{20: 't'}\\
\texttt{...}\\
\texttt{26: 'z'}
};

% 双向映射
\node[draw,fill=green!20,text width=4cm,align=center] at (1,0) {
字符 $\to$ 索引\\
\texttt{'t' $\to$ 20}
};

\draw[<->,ultra thick] (3.5,0) -- (6.5,0);

\node[draw,fill=blue!20,text width=4cm,align=center] at (9,0) {
索引 $\to$ 字符\\
\texttt{20 $\to$ 't'}
};

\end{tikzpicture}
\end{center}


\paragraph{完整数据加载器(带批处理)}
\begin{lstlisting}
def load_data_time_machine(batch_size, num_steps, 
                           use_random_iter=False):
    """ Returndataiterationvocabulary batch_size: num_steps: time step use_random_iter: """
    corpus, vocab = load_corpus_time_machine()
    
    # dataiteration8.3
    if use_random_iter:
        data_iter = seq_data_iter_random(corpus, batch_size, num_steps)
    else:
        data_iter = seq_data_iter_sequential(corpus, batch_size, num_steps)
    
    return data_iter, vocab

batch_size, num_steps = 32, 35
train_iter, vocab = load_data_time_machine(batch_size, num_steps)
\end{lstlisting}


\subsection{8.2节总结}
\begin{definition}[核心流程]
\begin{enumerate}
    \item \textbf{读取数据}:从文件读取原始文本
    \item \textbf{清洗}:转小写、去标点、去数字
    \item \textbf{词元化}:拆分成词元(词/字符/子词)
    \item \textbf{构建词表}:统计词频,建立索引映射
    \item \textbf{数值化}:将文本转为数字序列
\end{enumerate}
\end{definition}

\begin{theorem}[关键概念]
\begin{itemize}
    \item \textbf{词元(Token)}:文本的基本单位
    \item \textbf{词表(Vocabulary)}:词元$\leftrightarrow$索引的映射
    \item \textbf{语料库(Corpus)}:数值化后的文本序列
    \item \textbf{特殊标记}:\texttt{<unk>}, \texttt{<pad>}, \texttt{<bos>}, \texttt{<eos>}
\end{itemize}
\end{theorem}


\paragraph{三种词元化策略对比}
\begin{center}
\begin{tikzpicture}[scale=0.9]
% 表格形式
\node[font=\small] at (5,4) {\textbf{句子:"I love natural language processing"}};

% 词级别
\node[draw,fill=green!20,text width=10cm,align=left] at (5,3) {
\textbf{词级别:}\\
5个词元: \texttt{["I", "love", "natural", "language", "processing"]}\\
词表大: 需要5万+ | 有OOV问题
};

% 子词级别
\node[draw,fill=yellow!20,text width=10cm,align=left] at (5,1.8) {
\textbf{子词级别:}\\
7个词元: \texttt{["I", "love", "natur", "al", "language", "process", "ing"]}\\
词表中: 需要3万左右 | 几乎无OOV
};

% 字符级别
\node[draw,fill=orange!20,text width=10cm,align=left] at (5,0.6) {
\textbf{字符级别:}\\
33个词元: \texttt{["I", " ", "l", "o", "v", "e", " ", "n", ...]}\\
词表小: 只需100左右 | 序列太长
};

\end{tikzpicture}
\end{center}

\begin{definition}[选择建议]
\begin{itemize}
    \item 教学/演示 $\to$ 字符级别(简单)
    \item 传统任务 $\to$ 词级别(快速)
    \item 现代大模型 $\to$ 子词级别(平衡)
\end{itemize}
\end{definition}


\paragraph{实战tips}


\textbf{常见问题:}
\begin{enumerate}
    \item \textbf{词表太大}\\
    $\to$ 提高 \texttt{min\_freq}\\
    $\to$ 使用子词

    \item \textbf{OOV太多}\\
    $\to$ 降低 \texttt{min\_freq}\\
    $\to$ 使用字符/子词
    
    \item \textbf{序列太长}\\
    $\to$ 使用词级别\\
    $\to$ 截断序列
\end{enumerate}


\textbf{优化建议:}
\begin{enumerate}
    \item 根据任务选择粒度
    \item 合理设置频率阈值
    \item 考虑特殊标记需求
    \item 保存词表以便复用
    \item 可视化词频分布
\end{enumerate}


\vspace{0.5cm}
\begin{theorem}[下一步]
有了数值化的文本,接下来学习如何构建\textbf{语言模型}(8.3节)
\end{theorem}


\subsection{8.2节关键代码总结}
\begin{center}
\begin{tikzpicture}[scale=0.9]
% 核心函数
\node[draw,fill=yellow!20,text width=10cm,align=left,font=\small] at (5,3.5) {
\textbf{核心函数:}\\
\texttt{read\_time\_machine()} - 读取并清洗文本\\
\texttt{tokenize(lines, type)} - 词元化\\
\texttt{count\_corpus(tokens)} - 统计词频\\
\texttt{Vocab(tokens, min\_freq)} - 构建词表\\
\texttt{load\_corpus\_time\_machine()} - 一站式加载
};

% Vocab类关键方法
\node[draw,fill=green!20,text width=10cm,align=left,font=\small] at (5,1.5) {
\textbf{Vocab类关键方法:}\\
\texttt{vocab[token]} - 词元$\to$索引\\
\texttt{vocab.to\_tokens(idx)} - 索引$\to$词元\\
\texttt{len(vocab)} - 词表大小\\
\texttt{vocab.unk} - 未知词索引(=0)
};

% 数据流
\node[draw,fill=blue!20,text width=10cm,align=center,font=\small] at (5,0) {
\textbf{数据流:}\\
文本 $\to$ 清洗 $\to$ 词元 $\to$ 词表 $\to$ 索引 $\to$ 神经网络
};

\end{tikzpicture}
\end{center}

\section{8.3 语言模型和数据集}

\subsection{8.3 语言模型和数据集 - 章节导航}
\begin{definition}[本节内容]
\begin{itemize}
    \item 8.3.1 学习语言模型
    \item 8.3.2 马尔可夫模型与元语法
    \item 8.3.3 自然语言统计
    \item 8.3.4 读取长序列数据
\end{itemize}
\end{definition}

\begin{theorem}[学习目标]
\begin{enumerate}
    \item 理解语言模型的数学定义
    \item 掌握困惑度(Perplexity)的计算
    \item 了解马尔可夫假设
    \item 学会构建序列数据迭代器
\end{enumerate}
\end{theorem}



