\section{引言}

    \titlepage
    \begin{figure}[htpb]
        \begin{center}
            \includegraphics[width=0.4\linewidth]{pic/fzu-logo.png}
        \end{center}
    \end{figure}





\subsection{8.1 序列模型 - 完整导航}
\begin{definition}[本节内容结构]
\begin{itemize}
    \item 8.1.1 统计工具
    \item 8.1.2 训练
    \item 8.1.3 预测
    \item 8.1.4 小结

\end{itemize}
\end{definition}




% ==================== 引入 ====================
\paragraph{开场:什么是序列?}


\textbf{生活中的序列无处不在:}
\begin{itemize}
    \item  你说的每句话
    \item  一首歌的旋律
    \item  股票的价格变化
    \item  每天的气温
    \item  视频的每一帧
    \item  聊天记录
\end{itemize}


\begin{theorem}[序列的特点]
\begin{enumerate}
    \item \textbf{有顺序}\\
    "我爱你" $\neq$ "你爱我"
    
    \item \textbf{有依赖}\\
    后面依赖前面的信息
    
    \item \textbf{变长}\\
    长度不固定
\end{enumerate}
\end{theorem}


\vspace{0.5cm}
\begin{center}
\Large \textcolor{red}{问题:如何用模型处理序列数据?}
\end{center}


% ==================== 传统方法的问题 ====================
\paragraph{传统神经网络的局限}
\textbf{标准全连接网络(MLP):}
\begin{center}
\begin{tikzpicture}[scale=1.2]
% 输入层
\foreach \i in {1,2,3} {
    \node[circle,draw,fill=green!30,minimum size=0.6cm] (i\i) at (0,\i*0.8) {};
}
\node[left] at (i2) {输入层};

% 隐藏层
\foreach \i in {1,2,3,4} {
    \node[circle,draw,fill=blue!30,minimum size=0.6cm] (h\i) at (2,\i*0.6+0.2) {};
}
\node at (2,2.8) {隐藏层};

% 输出层
\foreach \i in {1,2} {
    \node[circle,draw,fill=red!30,minimum size=0.6cm] (o\i) at (4,\i+0.5) {};
}
\node[right] at (4,2) {输出层};

% 连接
\foreach \i in {1,2,3} {
    \foreach \j in {1,2,3,4} {
        \draw[->,thin,gray] (i\i) -- (h\j);
    }
}
\foreach \i in {1,2,3,4} {
    \foreach \j in {1,2} {
        \draw[->,thin,gray] (h\i) -- (o\j);
    }
}
\end{tikzpicture}
\end{center}

\textbf{三大问题:}
\begin{enumerate}
    \item \text{×} \textbf{固定输入大小}:必须是固定维度的向量
    \item \text{×} \textbf{无记忆能力}:每次输入独立处理,忘记历史
    \item \text{×} \textbf{参数不共享}:处理序列不同位置用不同参数
\end{enumerate}


\paragraph{问题示例:预测明天的股价}
\begin{center}
\begin{tikzpicture}[scale=1.1]
% 时间轴
\draw[->,thick] (0,0) -- (9,0) node[right] {时间(天)};
\draw[->,thick] (0,0) -- (0,3.5) node[above] {股价};

% 画点和线
\foreach \x/\y in {1/1.2, 2/1.5, 3/1.3, 4/1.8, 5/1.6, 6/2.0, 7/1.9} {
    \fill[blue] (\x,\y) circle (3pt);
    \node[below,font=\tiny] at (\x,0) {$x_{\x}$};
}

% 连线
\draw[blue,thick] (1,1.2) -- (2,1.5) -- (3,1.3) -- (4,1.8) -- (5,1.6) -- (6,2.0) -- (7,1.9);

% 今天
\draw[red,very thick] (7,-0.2) -- (7,2.5);
\node[red,below,font=\large] at (7,-0.3) {\textbf{今天}};

% 明天
\fill[green!60!black] (8,2.2) circle (3pt);
\draw[green!60!black,thick,dashed] (7,1.9) -- (8,2.2);
\node[green!60!black,below,font=\tiny] at (8,0) {$x_8$};
\node[green!60!black,above,font=\large] at (8,2.5) {\textbf{预测?}};

\end{tikzpicture}
\end{center}

\textbf{思考:如何预测 $x_8$(明天的价格)?}

\begin{definition}[三种思路]
\begin{enumerate}
    \item 只看今天:$x_8 = f(x_7)$
    \item 看最近几天:$x_8 = f(x_7, x_6, x_5)$
    \item 看所有历史:$x_8 = f(x_7, x_6, \ldots, x_1)$
\end{enumerate}
\end{definition}


